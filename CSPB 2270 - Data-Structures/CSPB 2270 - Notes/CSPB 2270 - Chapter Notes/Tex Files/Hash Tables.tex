\clearpage

\chapter{Week 9}

\section{Hash Tables}

\horizontalline

\subsection{Activities}

The following are the activities that are planned for Week 9 of this course.

\begin{itemize}
    \item Read zyBooks Chapter 14 and do activities (due Monday of Week 10)
    \item Watch lecture videos
    \item Homework on Hash Tables (due Tuesday of Week 10)
    \item Work on Project Proposal (due Thursday of Week 10)
\end{itemize}

\subsection{Lectures}

Here are the lectures that can be found for this week:

\begin{itemize}
    \item \href{https://applied.cs.colorado.edu/mod/hvp/view.php?id=46014}{Hashes}
    \item \href{https://applied.cs.colorado.edu/mod/hvp/view.php?id=46015}{Hash Implementation}
    \item \href{https://applied.cs.colorado.edu/mod/hvp/view.php?id=46016}{Collision Mitigation}
    \item \href{https://applied.cs.colorado.edu/mod/hvp/view.php?id=46017}{Open Addressing}
    \item \href{https://applied.cs.colorado.edu/mod/hvp/view.php?id=46018}{Hash Tables \& Sets}
\end{itemize}

\subsection{Programming Assignment}

The programming assignment for Week 9 - \href{https://github.com/cu-cspb-2270-Summer-2023/pa7-RelativiBit}{Hash Tables}

\subsection{Chapter 14 - Hash Tables}

The chapter of this week is Chapter 14 - Hash Tables.

\subsection*{Sec. 14.1 - Hash Tables}

\subsubsection{Overview}

A hash table is a data structure that allows efficient storage and retrieval of key-value pairs. It uses a hash function to convert keys into array indices, providing direct access to values in constant 
time. The hash function distributes keys uniformly across the array, minimizing collisions. Collisions occur when multiple keys map to the same index, and they are resolved using techniques like chaining 
or open addressing. Hash tables offer fast insertion, deletion, and retrieval operations, making them ideal for applications that require efficient data lookup, such as symbol tables, caches, and databases.

\subsubsection{Operations}

Hash tables support efficient operations such as insertion, deletion, and retrieval. During insertion, the key is hashed to determine the index where the value is stored, handling collisions if necessary. 
Deletion involves locating the key and removing its associated value. Retrieval utilizes the hash function to find the index and returns the value stored at that location. These operations typically have an 
average time complexity of $\mathcal{O}(1)$, making hash tables effective for fast data storage and retrieval. However, in the worst-case scenario with frequent collisions, the time complexity can degrade to 
$\mathcal{O}(n)$.

\subsubsection{Collisions}

Collisions in hash tables occur when multiple keys map to the same index. To handle collisions, hash tables employ techniques like chaining, where a linked list is used to store multiple key-value pairs at 
the same index. Alternatively, open addressing probes for an alternative location within the array when collisions happen. This can be done using linear probing, quadratic probing, or double hashing. Another 
approach called Robin Hood Hashing reduces clustering by swapping entries to minimize the difference between the desired and actual positions. These collision resolution methods ensure efficient storage and 
retrieval of data in hash tables, accommodating varying trade-offs between memory usage and lookup performance.

\subsection*{Sec. 14.2 - Chaining}

\subsubsection{Overview}

Chaining is a collision resolution technique used in hash tables. In chaining, each bucket or slot in the hash table contains a linked list or another data structure. When a collision occurs and multiple keys 
map to the same index, the new key-value pairs are appended to the linked list at that index. During retrieval, the hash function is used to locate the bucket, and a linear search or other method is employed 
within the linked list to find the desired key-value pair. Chaining provides an efficient way to handle collisions by allowing multiple entries to coexist at the same index, ensuring constant-time average 
performance for insertion, deletion, and retrieval operations in hash tables.

\subsubsection{Searching}

Searching in a hash table involves applying the hash function to the key to determine the index where the value might be stored. Once the index is identified, the search algorithm follows the collision resolution 
mechanism implemented in the hash table. In the case of chaining, a linear search or other suitable method is performed within the linked list at the corresponding index to locate the desired key-value pair. This 
process allows for efficient retrieval of values in constant time, making hash tables an effective data structure for fast searching operations.

\begin{solution}[Hash Table Search Algorithm]

Below is an example of searching in a hash table in the context of C++:

\horizontalline

\begin{verbatim}
// Function to search for a key in the hash table
std::string HashSearch(const std::unordered_map<int, std::list<std::string>>& 
                        hashTable, int key) {
    // Find the bucket list corresponding to the key's hash
    auto bucketList = hashTable.at(Hash(key));
    
    // Search for the key in the bucket list
    auto itemNode = ListSearch(bucketList, key);

    // If the item is found, return its data
    if (itemNode != nullptr) {
        return itemNode->data;
    }
    else {
        return "null";
    }
}
\end{verbatim}

\horizontalline

\end{solution}

\subsubsection{Removing}

Removing items from a hash table involves locating the key in the table and deleting the associated key-value pair. The removal process typically follows the same steps as searching, where the hash function is used 
to find the index or bucket where the key might be stored. Once the bucket is identified, the collision resolution mechanism is applied, such as traversing a linked list or probing in open addressing, to locate the 
specific key-value pair. Once found, the item is removed from the table. Removing items from a hash table ensures efficient deletion operations, with an average time complexity of $\mathcal{O}(1)$.

\begin{solution}[Hash Table Removing Algorithm]

Below is an example of removing in a hash table in the context of C++:

\horizontalline

\begin{verbatim}
// Function to remove an item from the hash table
void HashRemove(std::unordered_map<int, std::list<Item>>& hashTable, 
                    const Item& item) {
    // Find the bucket list corresponding to the item's key's hash
    auto& bucketList = hashTable[Hash(item.key)];

    // Search for the item in the bucket list
    auto itemNode = ListSearch(bucketList, item.key);

    // If the item is found, remove it from the list
    if (itemNode != nullptr) {
        ListRemove(bucketList, itemNode);
    }
}
\end{verbatim}

\horizontalline

\end{solution}

\subsubsection{Inserting}

Inserting items into a hash table involves mapping the key to its corresponding index or bucket using a hash function and then storing the key-value pair at that location. The hash function ensures uniform distribution 
of keys across the hash table, minimizing collisions. If a collision occurs and another key already exists at the computed index, collision resolution techniques like chaining or open addressing are employed to handle it. 
Insertion in a hash table typically has an average time complexity of $\mathcal{O}(1)$, making it an efficient operation for adding new key-value pairs to the table.

\begin{solution}[Hash Table Inserting Algorithm]

Below is an example of inserting in a hash table in the context of C++:

\horizontalline

\begin{verbatim}
// Function to insert an item into the hash table
void HashInsert(std::unordered_map<int, std::list<Item>>& hashTable, 
                    const Item& item) {
    // Check if the item's key already exists in the hash table
    if (HashSearch(hashTable, item.key) == nullptr) {
        // Find the bucket list corresponding to the item's key's hash
        auto& bucketList = hashTable[Hash(item.key)];

        // Allocate a new linked list node
        auto node = new Item;
        node->next = nullptr;
        node->data = item;

        // Append the node to the bucket list
        bucketList.push_back(*node);
    }
}
\end{verbatim}

\horizontalline

\end{solution}

\subsection*{Sec. 14.3 - Linear Probing}

\subsubsection{Overview}

Linear probing is a collision resolution technique used in hash tables. When a collision occurs and a key cannot be inserted at its original hashed index, linear probing sequentially searches for the next available slot in 
the hash table by incrementing the index. The key is inserted at the first unoccupied location found. This process allows for a direct mapping of keys to array indices without the need for additional data structures. However, 
linear probing can lead to clustering, where consecutive occupied slots form groups, potentially impacting search performance. Despite this drawback, linear probing offers simplicity and cache-friendly behavior, and it can be 
an effective strategy for resolving collisions in hash tables.

\subsubsection{Empty Bucket Types}

Empty bucket types, also known as tombstones or placeholders, are used in hash tables as a mechanism to indicate previously occupied slots that have been subsequently emptied. These empty bucket markers serve the purpose of 
distinguishing between genuinely empty slots and those that were previously occupied but have been deleted or removed. By marking these emptied slots with a special value or flag, the hash table can maintain its integrity 
during collision resolution and search operations. Empty bucket types allow for efficient re-use of slots, avoiding unnecessary collisions and improving the performance of hash table operations.

\subsubsection{Searching}

Searching with linear probing is a method used in hash tables to locate a key-value pair. When searching, the hash function is applied to the key to determine the initial index in the array. If the key is not found at that 
index, linear probing is used to sequentially check the subsequent indices until the key is found or an empty slot is encountered. This approach allows for a direct mapping of keys to array indices, enabling efficient search 
operations. However, linear probing can lead to clustering, where consecutive occupied slots form groups, potentially impacting search performance due to longer probe sequences. Nonetheless, searching with linear probing remains 
a simple and widely used technique in hash tables for efficient key retrieval.

\begin{solution}[Hash Table Linear Probe Search Algorithm]

Below is an example of searching in a hash table via linear probing in the context of C++:

\horizontalline

\begin{verbatim}
// Structure for the item in the hash table
struct Item {
    int key;
    std::string data;
};

// Function to search for a key in the hash table using linear probing
const Item* HashSearch(const std::vector<Item>& hashTable, int key) {
    // Hash function determines initial bucket
    size_t bucket = Hash(key);
    size_t bucketsProbed = 0;
    size_t N = hashTable.size();

    while ((hashTable[bucket].status != EmptySinceStart) && (bucketsProbed < N)) {
        if ((hashTable[bucket].status != Empty) && (hashTable[bucket].key == key)) {
            return &hashTable[bucket];
        }

        // Increment bucket index
        bucket = (bucket + 1) % N;

        // Increment number of buckets probed
        ++bucketsProbed;
    }

    return nullptr;  // Item not found
}
\end{verbatim}

\horizontalline

\end{solution}

\subsection*{Removal}

Removing with linear probing is a technique used in hash tables to delete a key-value pair. When removing an item, the hash function is applied to the key to determine the initial index in the array. If the key is not found at that 
index, linear probing is used to sequentially search the subsequent indices until the key is found or an empty slot is encountered. Once the key is located, the corresponding value is removed, and the slot is marked as empty. Linear 
probing ensures that the hash table remains compact and can efficiently handle deletions. However, it can cause clustering and lead to longer probe sequences, potentially impacting the performance of removal operations in hash tables.

\begin{solution}[Hash Table Linear Probe Removal Algorithm]

Below is an example of removing in a hash table via linear probing in the context of C++:

\horizontalline

\begin{verbatim}
// Structure for the item in the hash table
struct Item {
    int key;
    std::string data;
    bool isEmptyAfterRemoval;
};

// Function to remove an item from the hash table using linear probing
void HashRemove(std::vector<Item>& hashTable, int key) {
    // Hash function determines initial bucket
    size_t bucket = Hash(key);
    size_t bucketsProbed = 0;
    size_t N = hashTable.size();

    while ((hashTable[bucket].status != EmptySinceStart) && (bucketsProbed < N)) {
        if ((hashTable[bucket].status != Empty) && (hashTable[bucket].key == key)) {
            hashTable[bucket].isEmptyAfterRemoval = true;
            return;
        }

        // Increment bucket index
        bucket = (bucket + 1) % N;

        // Increment number of buckets probed
        ++bucketsProbed;
    }
}
\end{verbatim}

\horizontalline

\end{solution}

\subsection*{Inserting}

Inserting with linear probing is a technique used in hash tables to add a new key-value pair. When inserting, the hash function is applied to the key to determine the initial index in the array. If the index is already occupied, linear 
probing is used to sequentially search for the next available slot by incrementing the index. The key-value pair is inserted at the first unoccupied location found. Linear probing allows for a direct mapping of keys to array indices and 
avoids the need for additional data structures. However, it can cause clustering and potentially result in longer probe sequences, which may affect the performance of insertion operations in hash tables. Nonetheless, linear probing remains 
a simple and widely used strategy for adding elements to hash tables.

\begin{solution}[Hash Table Linear Probing Insert Algorithm]

Below is an example of inserting in a hash table via linear probing in the context of C++:

\horizontalline

\begin{verbatim}
// Structure for the item in the hash table
struct Item {
    int key;
    std::string data;
};

// Function to insert an item into the hash table using linear probing
bool HashInsert(std::vector<Item>& hashTable, const Item& item) {
    // Hash function determines initial bucket
    size_t bucket = Hash(item.key);
    size_t bucketsProbed = 0;
    size_t N = hashTable.size();

    while (bucketsProbed < N) {
        // Insert item in the next empty bucket
        if (hashTable[bucket].status == Empty) {
            hashTable[bucket] = item;
            return true;
        }

        // Increment bucket index
        bucket = (bucket + 1) % N;

        // Increment number of buckets probed
        ++bucketsProbed;
    }

    return false; // Unable to insert item
}
\end{verbatim}

\horizontalline

\end{solution}

\subsection*{Sec. 14.4 - Quadratic Probing}

\subsubsection{Overview}

Quadratic probing is a collision resolution technique used in hash tables. It addresses collisions by incrementing the index using a quadratic function instead of a linear function. When a collision 
occurs, quadratic probing checks a series of indices using a quadratic formula, such as `$(\text{hash} + c1 * i + c2 * i^2) \% N$', where `hash' is the original hash value, `i' is the probe sequence, 
`c1' and `c2' are constants, and `N' is the size of the hash table. This approach helps to distribute keys more evenly and mitigate clustering. Quadratic probing offers a balance between linear probing 
and other more complex collision resolution techniques, providing a compromise between simplicity and performance for handling collisions in hash tables.

\subsubsection{Searching}

Searching in the context of quadratic probing involves locating a key-value pair in a hash table that uses quadratic probing as its collision resolution technique. Similar to linear probing, the hash 
function is applied to the key to determine the initial index. If the key is not found at that index, quadratic probing is used to incrementally search through subsequent indices based on a quadratic 
formula, such as `$(\text{hash} + c1 * i + c2 * i^2) \% N$'. The search continues until the key is found or an empty slot is encountered. Quadratic probing helps to evenly distribute keys and mitigate 
clustering, enabling efficient search operations in hash tables. However, care must be taken to choose suitable constants `c1' and `c2' to avoid clustering patterns that can impact the performance of 
the search process.

\begin{solution}[Hash Table Quadratic Probe Search Algorithm]

Below is an example of searching in a hash table via quadratic probing in the context of C++:

\horizontalline

\begin{verbatim}
// Structure for the item in the hash table
struct Item {
    int key;
    std::string data;
};

// Function to search for a key in the hash table using quadratic probing
const Item* HashSearch(const std::vector<Item>& hashTable, int key) {
    size_t i = 0;
    size_t bucketsProbed = 0;
    size_t N = hashTable.size();

    // Hash function determines initial bucket
    size_t bucket = Hash(key) % N;

    while ((hashTable[bucket].status != EmptySinceStart) && (bucketsProbed < N)) {
        if ((hashTable[bucket].status == Occupied) 
                && (hashTable[bucket].key == key)) {
            return &hashTable[bucket];
        }

        // Increment i and recompute bucket index using quadratic probing formula
        size_t c1 = 1; // Programmer-defined constant
        size_t c2 = 1; // Programmer-defined constant
        i = i + 1;
        bucket = (Hash(key) + c1 * i + c2 * i * i) % N;

        // Increment number of buckets probed
        bucketsProbed = bucketsProbed + 1;
    }

    return nullptr; // Key not found
}
\end{verbatim}

\horizontalline

\end{solution}

\subsubsection{Removal}

Removing in the context of quadratic probing involves deleting a key-value pair from a hash table that utilizes quadratic probing as its collision resolution technique. Similar to searching, the hash 
function is applied to the key to determine the initial index. If the key is not found at that index, quadratic probing is used to incrementally search through subsequent indices based on a quadratic 
formula, such as `$(\text{hash} + c1 * i + c2 * i^2) \% N$'. Once the key is located, the corresponding value is removed, and the slot is marked as empty. Quadratic probing ensures that the hash table 
remains compact and enables efficient deletion operations. However, it is crucial to use appropriate quadratic probing constants to avoid clustering patterns that can impact the performance of the 
removal process.

\begin{solution}[Hash Table Quadratic Probe Removal Algorithm]

Below is an example of removal in a hash table via quadratic probing in the context of C++:

\horizontalline

\begin{verbatim}
// Structure for the item in the hash table
struct Item {
    int key;
    std::string data;
    bool isEmptyAfterRemoval;
};

// Function to remove an item from the hash table using quadratic probing
bool HashRemove(std::vector<Item>& hashTable, int key) {
    size_t i = 0;
    size_t bucketsProbed = 0;
    size_t N = hashTable.size();

    // Hash function determines initial bucket
    size_t bucket = Hash(key) % N;

    while ((hashTable[bucket].status != EmptySinceStart) && (bucketsProbed < N)) {
        if ((hashTable[bucket].status == Occupied) 
                && (hashTable[bucket].key == key)) {
            hashTable[bucket].isEmptyAfterRemoval = true;
            return true;
        }

        // Increment i and recompute bucket index using quadratic probing formula
        size_t c1 = 1; // Programmer-defined constant
        size_t c2 = 1; // Programmer-defined constant
        i = i + 1;
        bucket = (Hash(key) + c1 * i + c2 * i * i) % N;

        // Increment number of buckets probed
        bucketsProbed = bucketsProbed + 1;
    }

    return false; // Key not found
}
\end{verbatim}

\horizontalline

\end{solution}

\subsubsection{Inserting}

Inserting in the context of quadratic probing involves adding a new key-value pair to a hash table that utilizes quadratic probing as its collision resolution technique. Similar to linear probing, the 
hash function is applied to the key to determine the initial index. If the index is already occupied, quadratic probing is used to sequentially search for the next available slot using a quadratic formula, 
such as `$(\text{hash} + c1 * i + c2 * i^2) \% N$', where `hash' is the original hash value, `i' is the probe sequence, `c1' and `c2' are constants, and `N' is the size of the hash table. The key-value 
pair is inserted at the first unoccupied location found. Quadratic probing helps to evenly distribute keys and mitigate clustering, allowing efficient insertion operations in hash tables. However, it 
is essential to choose appropriate quadratic probing constants to avoid clustering patterns that can impact the performance of the insertion process.

\begin{solution}[Hash Table Quadratic Probe Insert Algorithm]

Below is an example of inserting in a hash table via quadratic probing in the context of C++:

\horizontalline

\begin{verbatim}
// Structure for the item in the hash table
struct Item {
    int key;
    std::string data;
};

// Function to insert an item into the hash table using quadratic probing
bool HashInsert(std::vector<Item>& hashTable, const Item& item) {
    size_t i = 0;
    size_t bucketsProbed = 0;
    size_t N = hashTable.size();

    // Hash function determines initial bucket
    size_t bucket = Hash(item.key) % N;

    while (bucketsProbed < N) {
        // Insert item in the next empty bucket
        if (hashTable[bucket].status == Empty) {
            hashTable[bucket] = item;
            return true;
        }

        // Increment i and recompute bucket index using quadratic probing formula
        size_t c1 = 1; // Programmer-defined constant
        size_t c2 = 1; // Programmer-defined constant
        i = i + 1;
        bucket = (Hash(item.key) + c1 * i + c2 * i * i) % N;

        // Increment number of buckets probed
        bucketsProbed = bucketsProbed + 1;
    }

    return false; // Unable to insert item
}
\end{verbatim}

\horizontalline
    
\end{solution}

\subsection*{Sec. 14.5 - Double Hashing}

\subsubsection{Overview}

Double hashing is a collision resolution technique used in hash tables. It involves using two hash functions to determine the index for key-value pairs when collisions occur. When a collision occurs, 
the secondary hash function is applied to the key to calculate an offset value, which is then used to probe for the next available slot in the hash table. The double hashing technique helps distribute 
keys more evenly and reduces clustering compared to linear probing or quadratic probing. By employing two hash functions, it provides a greater range of possible indices and improves the chances of 
finding an empty slot. Double hashing offers an efficient way to resolve collisions and maintain good performance in hash tables.

\subsubsection{Search}

Searching with double hashing in hash tables involves utilizing two hash functions to handle collisions and locate a specific key-value pair. The primary hash function is applied to calculate the initial 
index, and if the key is not found at that index, the secondary hash function is used to determine the probe offset. The offset is added to the current index, and the search continues until the key is 
found or an empty slot is encountered. By employing two hash functions, double hashing reduces clustering and improves key distribution, leading to efficient and effective search operations in hash tables. 
The use of double hashing allows for a wider range of possible indices, increasing the chances of finding the desired key in a timely manner.

\subsubsection{Removal}

Removing with double hashing in hash tables involves using two hash functions to handle collisions and delete a specific key-value pair. The primary hash function is applied to calculate the initial index, 
and if the key is not found at that index, the secondary hash function is used to determine the probe offset. The offset is added to the current index, and the search continues until the key is found or an 
empty slot is encountered. Once the key is located, the corresponding value is removed, and the slot is marked as empty. By utilizing double hashing, the removal process is efficient and ensures the integrity 
of the hash table, effectively handling collisions and maintaining good performance. The use of two hash functions allows for better key distribution and reduces clustering, providing an effective mechanism 
for removing key-value pairs in hash tables.

\subsubsection{Inserting}

Inserting with double hashing in hash tables involves using two hash functions to handle collisions and add a new key-value pair. The primary hash function is applied to calculate the initial index, and if the 
index is already occupied, the secondary hash function determines the probe offset. The offset is added to the current index, and the search continues until an empty slot is found. Once an empty slot is located, 
the new key-value pair is inserted into that slot, ensuring proper placement within the hash table. The use of double hashing allows for better key distribution and reduces clustering, providing an efficient and 
effective mechanism for inserting key-value pairs in hash tables. By employing two hash functions, the insertion process is optimized, leading to improved performance and maintaining the integrity of the hash table.

\subsection*{Sec. 14.6 - Hash Table Resizing}

\subsubsection{Overview}

Hash table resizing is a dynamic operation that involves increasing or decreasing the size of a hash table to accommodate a changing number of elements. When the number of elements exceeds a certain threshold or 
falls below a certain load factor, resizing is triggered to ensure optimal performance. During resizing, a new hash table with a larger or smaller size is created, and all existing key-value pairs are rehashed and 
inserted into the new table. Resizing allows for better utilization of memory, improved efficiency, and reduced collision probabilities. It helps maintain a balanced and efficient hash table by adjusting its size 
based on the number of elements it needs to store, ensuring optimal performance and avoiding excessive collisions.

\subsubsection{Resizing Requirements}

Resizing a hash table is typically done when certain conditions are met, indicating the need for adjustment in order to maintain efficient performance. One common scenario is when the number of elements stored 
in the hash table exceeds a certain threshold, often referred to as the load factor. Resizing is triggered to increase the size of the hash table, allowing for a larger number of elements and reducing the 
likelihood of collisions. Conversely, if the number of elements decreases significantly, resizing may be performed to reduce the size of the hash table and optimize memory usage. The decision to resize a hash 
table depends on factors such as the desired load factor, the expected number of elements, and the performance requirements. Resizing a hash table ensures that it remains balanced and provides efficient access 
and storage of elements based on the current workload.

\begin{solution}[Hash Table Resizing Example]

Below is an example of resizing a hash table in the context of C++:

\horizontalline

\begin{verbatim}
// Function to resize the hash table
std::vector<int> HashResize(const std::vector<int>& hashTable, int currentSize) {
    int newSize = nextPrime(currentSize * 2);

    std::vector<int> newArray(newSize, EmptySinceStart);

    int bucket = 0;
    while (bucket < currentSize) {
        if (hashTable[bucket] != Empty) {
            int key = hashTable[bucket];
            HashInsert(newArray, key); // Assuming HashInsert function exists
        }
        bucket = bucket + 1;
    }

    return newArray;
}
\end{verbatim}

\horizontalline
    
\end{solution}

\subsection*{Sec. 14.7 - Common Hash Functions}

\subsubsection{Overview}

Common hash functions are algorithms used to convert input data of arbitrary size into a fixed-size output value, typically used for indexing and accessing data in hash tables. Some commonly used hash functions 
include the division method, where the key is divided by the table size and the remainder is used as the index; the multiplication method, where the key is multiplied by a constant fraction and the fractional 
part is used as the index; and the mid-square method, where the key is squared and a portion of the middle bits is extracted as the index. Other popular hash functions include cryptographic hash functions like 
MD5 and SHA-1, which are designed for security applications. Each hash function has its advantages and considerations, such as efficiency, distribution of keys, and collision resistance, and the choice of hash 
function depends on the specific requirements of the application.

\subsubsection{Modulo Hash Function}

The modulo hash function, also known as the division method, is a simple and commonly used hash function. It involves taking the remainder of the key divided by the size of the hash table to determine the index 
for storage or retrieval. The modulo hash function is easy to implement and provides a uniform distribution of keys when the table size is a prime number. However, it may result in clustering and poor performance 
if the table size is not well-chosen, leading to frequent collisions. To mitigate this, the table size is often selected as a prime number to improve the distribution of keys and reduce collisions. The modulo hash 
function is widely used in many applications due to its simplicity and acceptable performance for various scenarios.

\begin{solution}[Modulo Hash Function Example]

Below is an example of the modulo hash function in the context of C++:

\horizontalline

\begin{verbatim}
// Function to calculate the remainder using the modulo hash function
int HashRemainder(int key, int N) {
    return key % N;
}
\end{verbatim}

\horizontalline
    
\end{solution}

\subsubsection{Mid-Square Hash Function}

The mid-square hash function is a technique used to convert a given key into a hash value by squaring the key and extracting a portion of the middle bits from the resulting square. The middle bits are then used as 
the hash index. This method aims to achieve a more uniform distribution of keys in the hash table. However, the effectiveness of the mid-square hash function heavily depends on the choice of the number of bits 
extracted from the square and the properties of the input keys. If the number of bits extracted is not carefully selected or if the input keys exhibit certain patterns, the mid-square hash function can result in 
clustering and an increased likelihood of collisions. Therefore, careful consideration should be given to the selection of the number of bits to extract and the nature of the keys being hashed.

\begin{solution}[Mid-Square Hash Function Example]

Below is an example of the mid-square has function in the context of C++:

\horizontalline

\begin{verbatim}
// Function to calculate the hash using the mid-square hash function
int HashMidSquare(int key, int R, int N) {
    int squaredKey = key * key;

    int lowBitsToRemove = (32 - R) / 2;
    int extractedBits = squaredKey >> lowBitsToRemove;
    extractedBits = extractedBits & (0xFFFFFFFF >> (32 - R));

    return extractedBits % N;
}
\end{verbatim}

\horizontalline
    
\end{solution}

\subsubsection{Multiplicative String Hash Function}

The multiplicative string hash function is a technique used to convert a string into a hash value by multiplying the numeric representation of each character in the string by a constant factor. The constant factor 
is typically a fractional value between 0 and 1, chosen carefully to ensure good distribution and reduce the likelihood of collisions. The multiplication is performed on each character in the string, and the resulting 
values are combined to generate the final hash value. The multiplicative string hash function aims to achieve a balanced distribution of keys and can be efficient for string-based hashing. However, it requires careful 
selection of the constant factor to avoid clustering and ensure a uniform distribution of hash values.

\begin{solution}[Multiplicative String Hash Function Example]

Below is an example of the multiplicative string hash function in the context of C++:

\horizontalline

\begin{verbatim}
// Function to calculate the hash using the multiplicative string hash function
int HashMultiplicative(const std::string& key, int N, 
                        int InitialValue, int HashMultiplier) {
    int stringHash = InitialValue;

    for (char strChar : key) {
        stringHash = (stringHash * HashMultiplier) + static_cast<int>(strChar);
    }

    return stringHash % N;
}
\end{verbatim}

\horizontalline
    
\end{solution}

\subsection*{Sec. 14.8 - Direct Hashing}

\subsubsection{Overview}

Direct hashing, also known as perfect hashing, is a hash function technique that eliminates collisions entirely by ensuring a one-to-one mapping between keys and their corresponding hash values. It achieves this by 
precomputing a hash function specific to the set of keys being hashed. In direct hashing, an initial hash function is used to determine a primary hash value, and then a secondary hash function is applied to resolve 
any potential collisions. The secondary hash function maps each key to a unique index in a small auxiliary table, which stores the key-value pairs. Direct hashing provides constant-time access to the stored values, 
as there are no collisions to handle. However, it requires extra memory to store the auxiliary table and entails additional computation during the construction phase to create the perfect hash function. Direct hashing 
is suitable for applications where collision-free access to a fixed set of keys is essential.

\subsection*{Sec. 14.9 - Cryptography}

\subsubsection{Cryptography}

Cryptographic hashing is a method used to transform input data into a fixed-size, unique, and irreversible hash value. It is commonly employed in various security applications, such as password storage, data integrity 
verification, and digital signatures. Cryptographic hash functions have specific properties, including pre-image resistance, second pre-image resistance, and collision resistance, which ensure that it is computationally 
infeasible to reconstruct the original data or find two different inputs that produce the same hash value. These hash functions generate a unique output for each input, making them suitable for verifying data integrity, 
detecting tampering, and securely storing passwords without revealing the actual passwords themselves. Examples of widely used cryptographic hash functions include MD5, SHA-1, SHA-256, and SHA-3.

\subsubsection{Hashing Functions For Data}

Hashing functions for data are algorithms that convert data of any size or type into a fixed-size hash value. These functions are designed to efficiently map data to a unique representation, enabling quick data retrieval, 
identification, and comparison. Hashing functions generate a hash value based on the content of the data, ensuring that even a small change in the input data produces a significantly different output. This property makes 
hashing functions ideal for tasks such as data indexing, data integrity verification, duplicate detection, and password storage. Well-designed hashing functions exhibit desirable properties, including uniform distribution, 
minimal collisions, and resistance to pre-image attacks, ensuring the integrity and security of data in various applications and systems.