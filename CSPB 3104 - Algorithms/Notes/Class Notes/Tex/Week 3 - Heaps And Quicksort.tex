\clearpage

\renewcommand{\ChapTitle}{Heaps And Quicksort}
\renewcommand{\SectionTitle}{Heaps And Quicksort}

\chapter{\ChapTitle}

\section{\SectionTitle}
\horizontalline{0}{0}

\subsection{Assigned Reading}

The reading assignment for this week is from, \Textbook:

\begin{itemize}
    \item \textbf{Chapter 5.1 - The Hiring Problem}
    \item \textbf{Chapter 5.2 - Indicator Random Variables}
    \item \textbf{Chapter 5.3 - Randomized Algorithms}
    \item \textbf{Chapter 5.4 - Probabilistic Analysis And Further Uses Of Indicator Random Variables}
    \item \textbf{Chapter 6.1 - Heaps}
    \item \textbf{Chapter 6.2 - Maintaining The Heap Property}
    \item \textbf{Chapter 6.3 - Building A Heap}
    \item \textbf{Chapter 6.4 - The Heapsort Algorithm}
    \item \textbf{Chapter 6.5 - Priority Queues}
    \item \textbf{Chapter 7.1 - Description Of Quicksort}
    \item \textbf{Chapter 7.2 - Performance Of Quicksort}
    \item \textbf{Chapter 7.3 - A Randomized Version Of Quicksort}
    \item \textbf{Chapter 7.4 - Analysis Of Quicksort}
\end{itemize}

\subsection{Lectures}

The lecture videos for this week are:

\begin{itemize}
    \item \lecture{https://www.youtube.com/watch?v=XTtAMIg1ZlE}{Heaps: Introduction And Basic Properties}{24}
    \item \lecture{https://www.youtube.com/watch?v=ALXdoVCRqAo}{Heap Operations: Bubble Up/Down}{29}
    \item \lecture{https://www.youtube.com/watch?v=uFd5WysEo28}{Heap Operations: Insert, Delete And Heapsort}{29}
    \item \lecture{https://www.youtube.com/watch?v=ESmAiz0eAVI}{Overview of Quicksort}{14}
    \item \lecture{https://www.youtube.com/watch?v=MZu94lyy8Wo}{Quicksort: Lomuto's Partitioning Algorithm}{25}
    \item \lecture{https://www.youtube.com/watch?v=AGAII5LahQk}{Quicksort: Worst Case And Average Case Analysis}{28}
\end{itemize}

\subsection{Assignments}

The assignment for this week is:

\begin{itemize}
    \item \href{https://github.com/QuantumCompiler/CU/tree/main/CSPB%203104%20-%20Algorithms/Assignments/Problem%20Sets/Problem%20Set%203%20-%20Heaps%20And%20Quicksort}{Problem Set 3 - Heaps And Quicksort}
    \item \href{https://github.com/QuantumCompiler/CU/tree/main/CSPB%203104%20-%20Algorithms/Assignments/Programming%20Assignments/Programming%20Assignment%201%20-%20Heaps%20And%20Quicksort}{Programming Assignment 1 - Heaps And Quicksort}
\end{itemize}

\subsection{Quiz}

The quizzes for this week are:

\begin{itemize}
    \item \pdflink{\QuizDir Quiz 3 - Heaps And Quicksort.pdf}{Quiz 3 - Heaps And Quicksort}
\end{itemize}

\subsection{Chapter Summary}

The first chapter that we are covering this week is \textbf{Chapter 5: Probabilistic Analysis and Randomized Algorithms}. The first section that I we are covering from this section is \textbf{Section 5.1: The hiring problem}.

\begin{notes}{Section 5.1: The hiring problem}
    \subsubsection*{The Hiring Problem}

    The hiring problem is a classic example in the study of probabilistic analysis and randomized algorithms, illustrating the challenges and strategies involved in decision-making under uncertainty. 
    It models a situation where an employer must decide whether to hire a candidate immediately after an interview, without the ability to recall previous candidates. The decision is based solely on 
    the relative ranking of the current candidate compared to those interviewed so far. The employer aims to minimize the cost associated with hiring while maximizing the quality of the hire. This 
    problem showcases the use of expected values to determine the best strategy and introduces randomized algorithms as a means to optimize hiring decisions in an uncertain environment.
    
    \begin{itemize}
        \item \textbf{Problem Description}
        \begin{itemize}
            \item In the hiring problem, an employer interviews $n$ applicants one by one in a random sequence and must decide immediately after each interview whether to hire the candidate. The 
            goal is to hire the best candidate with minimal costs, assuming each hiring incurs a fixed cost. The challenge arises because the decision must be made without knowledge of future candidates.
            \item Example: Consider interviewing 5 candidates randomly, where the decision to hire is based on whether a candidate is superior to all previously interviewed ones, potentially 
            leading to multiple hires and associated costs.
        \end{itemize}
        
        \item \textbf{Probabilistic Analysis}
        \begin{itemize}
            \item Probabilistic analysis of the hiring problem assumes equal likelihood for any order of candidate quality, aiming to calculate the expected number of hires. This analysis reveals 
            that, depending on the hiring strategy, the expected number of hires can significantly vary, with an optimal strategy minimizing this expectation.
            \item Example: While the worst-case scenario results in hiring after every interview ($n$ hires), an optimal strategy could significantly reduce the expected number of hires, often 
            related to the logarithm of the number of candidates ($\log n$).
        \end{itemize}
        
        \item \textbf{Randomized Algorithms}
        \begin{itemize}
            \item To improve decision-making under uncertainty, randomized algorithms may be employed in the hiring process. These algorithms can optimize the expected outcome by introducing randomness 
            into the hiring criteria, thereby potentially reducing the total hiring cost while maintaining a high probability of hiring the best candidate.
            \item Example: Implementing a randomized threshold for hiring decisions that adjusts based on the proportion of candidates already interviewed to those remaining can optimize hiring costs 
            and outcomes.
        \end{itemize}
    \end{itemize}
\end{notes}

The next section that will be covered from this chapter this week is \textbf{Section 5.2: Indicator Random Variables}.

\begin{notes}{Section 5.2: Indicator Random Variables}
    \subsubsection*{Indicator Random Variables}

    Indicator random variables are a fundamental concept in probability theory and statistical analysis, particularly useful in the study of algorithms. They simplify the analysis of complex random 
    processes by breaking them down into simpler, binary outcomes. An indicator random variable, denoted as $I_A$, takes the value 1 if event $A$ occurs and 0 otherwise. This binary nature makes 
    it incredibly powerful for probabilistic analysis and expected value calculations, as it allows for the straightforward aggregation of probabilities across multiple events.
    
    \begin{itemize}
        \item \textbf{Definition}
        \begin{itemize}
            \item An indicator random variable $I_A$ is defined for an event $A$, such that $I_A = 1$ if $A$ occurs, and $I_A = 0$ if $A$ does not occur. This simple mechanism provides 
            a clear way to quantify the occurrence of events in probabilistic terms.
            \item Example: In a card game, let $A$ be the event that a drawn card is an ace. The indicator random variable $I_A$ would be 1 if the drawn card is an ace, and 0 otherwise.
        \end{itemize}
        
        \item \textbf{Expected Value of an Indicator Random Variable}
        \begin{itemize}
            \item The expected value, $E[I_A]$, of an indicator random variable is equal to the probability of the event $A$ occurring, denoted by $P(A)$. This is because $E[I_A] = 1 \cdot P(A) + 0 \cdot P(\neg A) = P(A)$, 
            where $\neg A$ represents the complement of $A$.
            \item Example: If the probability of drawing an ace from a standard deck of cards is $\frac{1}{13}$, then $E[I_A] = \frac{1}{13}$.
        \end{itemize}
        
        \item \textbf{Using Indicator Random Variables in Algorithms}
        \begin{itemize}
            \item Indicator random variables can significantly simplify the analysis of algorithms, especially when calculating the expected number of occurrences of an event. By summing the indicator 
            variables for each instance of the event, we can easily compute the overall expected number of occurrences.
            \item Example: Consider an algorithm that processes $n$ items, where each item has a probability $p$ of triggering a specific action (event $A$). The expected number of triggered 
            actions is the sum of $n$ indicator random variables, each representing the occurrence of $A$ for an item, resulting in an expected total of $n \cdot p$.
        \end{itemize}
    \end{itemize}    
\end{notes}

The next section that will be covered in this chapter \textbf{Section 5.3: Randomized Algorithms}.

\begin{notes}{Section 5.3: Randomized Algorithms}
    \subsubsection*{Randomized Algorithms}

    Randomized algorithms incorporate randomness as a part of their logic, making decisions not solely based on input but also on random values. This approach can lead to simpler, faster, or more 
    efficient algorithms compared to their deterministic counterparts. The use of randomness can help in overcoming obstacles such as worst-case scenarios or unknown inputs. Randomized algorithms are 
    widely applied in fields such as cryptography, algorithm design, and computational complexity.
    
    \begin{itemize}
        \item \textbf{Advantages of Randomized Algorithms}
        \begin{itemize}
            \item Randomized algorithms often have simpler implementations and can achieve better performance on average compared to deterministic algorithms. They are particularly useful in dealing 
            with adversarial inputs or in scenarios where the input distribution is unknown.
            \item Example: QuickSort algorithm, where the pivot is chosen randomly. This randomization ensures that the average time complexity remains $O(n \log n)$, even in the worst-case input 
            scenarios.
        \end{itemize}
        
        \item \textbf{Types of Randomized Algorithms}
        \begin{itemize}
            \item \emph{Monte Carlo algorithms} provide a probabilistic solution that may be incorrect with a certain probability. These algorithms are useful when a fast approximate solution is 
            acceptable.
            \item \emph{Las Vegas algorithms} always produce a correct solution, but their running time is variable and depends on the random choices made during execution. The algorithm terminates 
            once a correct solution is found.
            \item Example: The ZPP (Zero-error Probabilistic Polynomial time) class includes algorithms that are Las Vegas type, ensuring correctness with variable execution time.
        \end{itemize}
    \end{itemize}

    \begin{highlight}[Sample Python Code for Randomized QuickSort]
        The Randomized QuickSort algorithm selects a pivot element randomly from the array to be sorted, partitioning the array around this pivot, and recursively sorting the subarrays. This random 
        selection helps to ensure that the algorithm achieves good average-case performance.
    \begin{code}[Python]
    import random

    def randomized_partition(arr, low, high):
        pivot_index = random.randint(low, high)
        arr[pivot_index], arr[high] = arr[high], arr[pivot_index]
        pivot = arr[high]
        i = low - 1
        for j in range(low, high):
            if arr[j] < pivot:
                i += 1
                arr[i], arr[j] = arr[j], arr[i]
        arr[i+1], arr[high] = arr[high], arr[i+1]
        return i+1

    def randomized_quicksort(arr, low, high):
        if low < high:
            pi = randomized_partition(arr, low, high)
            randomized_quicksort(arr, low, pi-1)
            randomized_quicksort(arr, pi+1, high)

    # Example usage
    arr = [10, 7, 8, 9, 1, 5]
    randomized_quicksort(arr, 0, len(arr)-1)
    print("Sorted array:", arr)
    \end{code}

        This Python code snippet demonstrates how Randomized QuickSort works by selecting a pivot element randomly and using it to partition the array. The use of randomness in selecting the pivot 
        reduces the likelihood of encountering the worst-case performance scenario, which is typical in the standard QuickSort when the smallest or largest element is always chosen as the pivot.
    \end{highlight}
\end{notes}

The last section from this chapter is \textbf{Section 5.4: Probabilistic Analysis And Further Uses Of Indicator Random Variables}.

\begin{notes}{Section 5.4: Probabilistic Analysis And Further Uses Of Indicator Random Variables}
    \subsubsection*{Probabilistic Analysis and Further Uses of Indicator Random Variables}

    Probabilistic analysis leverages the concept of probability to analyze the behavior of algorithms under the assumption that the input is determined randomly. This analysis often uses indicator 
    random variables to simplify the computation of expected values, particularly in complex algorithms. Indicator random variables serve as a powerful tool for breaking down the analysis into 
    manageable parts, each representing a basic event that can either occur or not.
    
    \begin{itemize}
        \item \textbf{Basics of Probabilistic Analysis}
        \begin{itemize}
            \item Probabilistic analysis calculates the average-case complexity of algorithms, assuming a probabilistic distribution of all possible inputs. It helps in understanding the behavior of 
            algorithms under typical conditions.
            \item Example: Analyzing the average number of comparisons in QuickSort or the average-case running time of a hashing algorithm.
        \end{itemize}
        
        \item \textbf{Further Uses of Indicator Random Variables}
        \begin{itemize}
            \item Indicator random variables are extensively used to simplify the analysis of the expected value of complex random variables by breaking the problem down into simpler, binary outcomes.
            \item Example: The number of collisions in a hashing algorithm can be analyzed using indicator random variables representing whether each pair of elements collides.
        \end{itemize}
    \end{itemize}
    
    \begin{highlight}[Mathematical Formulae For Expected Value Calculation]
        The expected value of a sum of indicator random variables can be calculated using the linearity of expectation. If $X = \sum_{i=1}^{n} I_{A_i}$, where $I_{A_i}$ is the indicator random variable 
        for event $A_i$, then the expected value of $X$, $E[X]$, is the sum of the expected values of the indicator random variables:
    
        \begin{equation*}
            E[X] = E\left[\sum_{i=1}^{n} I_{A_i}\right] = \sum_{i=1}^{n} E[I_{A_i}] = \sum_{i=1}^{n} P(A_i)
        \end{equation*}
    
        This formula is particularly useful because it allows the expected value of $X$ to be computed directly from the probabilities of the individual events $A_i$ without needing to consider the 
        dependencies between them.
    \end{highlight}
    
    This section highlights the significance of probabilistic analysis in understanding the average-case performance of algorithms and how indicator random variables facilitate this analysis by enabling 
    the straightforward calculation of expected values. Mathematical formulae within the highlight environment elucidate the method for calculating the expected value of a sum of indicator random 
    variables, showcasing the elegance and power of this approach in algorithmic analysis.
\end{notes}

The next chapter that we will cover this week is \textbf{Chapter 6: Heapsort}. The first section that will be covered from this section is \textbf{Section 6.1: Heaps}.

\begin{notes}{Section 6.1: Heaps}
    \subsubsection*{Heaps}

    Heaps are a specialized tree-based data structure that satisfies the heap property. In a heap, for any given node $C$ with parent $P$, the key (the value) of $P$ is either greater than or equal 
    to $C$ for a max heap, or less than or equal to $C$ for a min heap. This property makes heaps useful for implementing priority queues and for the heapsort algorithm.
    
    \begin{itemize}
        \item \textbf{Heap Properties}
        \begin{itemize}
            \item A heap is a complete binary tree, meaning it is completely filled at all levels except possibly the lowest, which is filled from the left up to a point.
            \item In a max heap, for every node $i$ other than the root, the value of node $i$ is less than or equal to the value of its parent. Conversely, in a min heap, every node $i$'s value is 
            greater than or equal to the value of its parent. These properties can be succinctly represented as $A[\text{parent}(i)] \geq A[i]$ for a max heap and $A[\text{parent}(i)] \leq A[i]$ 
            for a min heap, where $A$ is the array representation of the heap.
        \end{itemize}
        
        \item \textbf{Basic Heap Operations}
        \begin{itemize}
            \item \emph{Insertion} - Adding a new element to the heap while maintaining the heap property. This operation is typically $O(\log n)$ due to the need for potentially "bubbling up" the 
            new element.
            \item \emph{Deletion} - Removing the root element from the heap (the maximum element in a max heap or the minimum in a min heap) and then restructuring the heap to maintain the heap 
            property, usually through a process called "heapify," which is also $O(\log n)$.
        \end{itemize}
    \end{itemize}
    
    \begin{highlight}[Sample Python Code for Heapsort]
        Heapsort is an efficient comparison-based sorting algorithm that builds a heap from the input data and then repeatedly extracts the maximum element (in a max heap) or the minimum element 
        (in a min heap) to build the sorted list.
    \begin{code}[Python]
    def heapify(arr, n, i):
        largest = i
        left = 2 * i + 1
        right = 2 * i + 2
        
        if left < n and arr[largest] < arr[left]:
            largest = left
        if right < n and arr[largest] < arr[right]:
            largest = right
        if largest != i:
            arr[i], arr[largest] = arr[largest], arr[i]
            heapify(arr, n, largest)
    
    def heapsort(arr):
        n = len(arr)
        for i in range(n // 2 - 1, -1, -1):
            heapify(arr, n, i)
        for i in range(n-1, 0, -1):
            arr[i], arr[0] = arr[0], arr[i]
            heapify(arr, i, 0)
    
    # Example usage
    arr = [12, 11, 13, 5, 6, 7]
    heapsort(arr)
    print("Sorted array is:", arr)
    \end{code}
        
        This code first builds a max heap from the input array. Then, it repeatedly removes the largest element from the heap (the root of the heap), placing it at the end of the array, and calls 
        heapify to restore the max heap property for the remaining elements.
    \end{highlight}    
\end{notes}

The next section that will be covered this week is \textbf{Section 6.2: Maintaining The Heap Property}.

\begin{notes}{Section 6.2: Maintaining The Heap Property}
    \subsubsection*{Maintaining the Heap Property}

    Maintaining the heap property is crucial in heap-based data structures, particularly in operations such as insertions, deletions, and the Heapsort algorithm. The Max-Heapify operation is a key 
    procedure used to maintain the max heap property in a heap after an element has been added or removed.
    
    \begin{itemize}
        \item \textbf{Max-Heapify Operation}
        \begin{itemize}
            \item The Max-Heapify operation ensures that the heap property is maintained by comparing a node with its children and swapping it with one of them if necessary so that the node's value 
            is not less than the values of the children. This process is applied recursively down the heap.
            \item For a node at index $i$, its left child is at index $2i + 1$ and its right child at $2i + 2$ in the array representation of the heap. The Max-Heapify operation is applied when the 
            subtree rooted at $i$ violates the heap property.
        \end{itemize}
    \end{itemize}
    
    \begin{highlight}[Sample Python Code for Max-Heapify]
        The Max-Heapify function is a cornerstone of heap operations, ensuring the max heap property is satisfied throughout the heap. This function is most commonly used in the Heapsort algorithm 
        and when inserting or removing elements from a heap.
    \begin{code}[Python]
    def max_heapify(arr, n, i):
        largest = i
        left = 2 * i + 1
        right = 2 * i + 2
        
        if left < n and arr[left] > arr[largest]:
            largest = left
        if right < n and arr[right] > arr[largest]:
            largest = right
        if largest != i:
            arr[i], arr[largest] = arr[largest], arr[i]
            max_heapify(arr, n, largest)
    
    # Example usage
    arr = [3, 2, 15, 5, 4, 45]
    n = len(arr)
    max_heapify(arr, n, 0)
    print("Max-Heapified array:", arr)
    \end{code}
        
        This Python code snippet demonstrates the Max-Heapify operation applied to an array that represents a heap. The function ensures that, for any given node, if the children's values are greater 
        than the node's value, the largest value among the node and its children becomes the parent node, thus maintaining the max heap property.
    \end{highlight}
\end{notes}

The next section that will be covered this week is \textbf{Section 6.3: Building A Heap}.

\begin{notes}{Section 6.3: Building A Heap}
    \subsubsection*{Building a Heap}

    Building a heap from an unsorted array is an essential step in heap operations, particularly before performing the Heapsort algorithm. This process transforms an arbitrary array into a heap by 
    applying the Max-Heapify operation in a bottom-up manner. The algorithm efficiently produces a max heap (or min heap, accordingly) by ensuring that all subtrees satisfy the heap property.
    
    \begin{itemize}
        \item \textbf{Heap Construction Algorithm}
        \begin{itemize}
            \item The process starts from the lowest non-leaf nodes and applies the Max-Heapify operation to each, moving upwards to the root of the heap. Since leaves are trivially heaps, starting 
            from the first non-leaf node ensures that each application of Max-Heapify makes the subtree rooted at the current node a valid heap.
            \item This bottom-up approach guarantees that the heap property is maintained for all nodes. The efficiency of this method lies in its $O(n)$ complexity, contrary to the intuitive 
            $O(n \log n)$ if Max-Heapify were applied naively to each node starting from the root.
        \end{itemize}
    \end{itemize}
    
    \begin{highlight}[Sample Python Code for Building a Max Heap]
        Building a max heap involves organizing the elements of an array so that the value of each parent node is greater than or equal to the values of its children, satisfying the max heap property 
        for the entire array.
    \begin{code}[Python]
    def heapify(arr, n, i):
        largest = i
        left = 2 * i + 1
        right = 2 * i + 2
    
        if left < n and arr[left] > arr[largest]:
            largest = left
        if right < n and arr[right] > arr[largest]:
            largest = right
        if largest != i:
            arr[i], arr[largest] = arr[largest], arr[i]
            heapify(arr, n, largest)
    
    def build_max_heap(arr):
        n = len(arr)
        for i in range(n // 2 - 1, -1, -1):
            heapify(arr, n, i)
    
    # Example usage
    arr = [1, 12, 9, 5, 6, 10]
    build_max_heap(arr)
    print("Max Heap:", arr)
    \end{code}
        
        This code snippet demonstrates how to transform an unsorted array into a max heap. The `build\_max\_heap' function iterates over each non-leaf node, applying `heapify` to ensure that the subtree 
        rooted at each node satisfies the max heap property, ultimately building a max heap from the entire array.
    \end{highlight}
\end{notes}

The next section that is covered from this chapter is \textbf{Section 6.4: The Heapsort Algorithm}.

\begin{notes}{Section 6.4: The Heapsort Algorithm}
    \subsubsection*{The Heapsort Algorithm}

    The Heapsort algorithm is a comparison-based sorting technique based on the binary heap data structure. It's similar to selection sort where we first find the maximum element and place the maximum 
    at the end. We repeat the same process for the remaining elements. Heapsort is particularly efficient for its ability to sort in-place, requiring no additional storage beyond what is needed for the 
    list.
    
    \begin{itemize}
        \item \textbf{Heapsort Procedure}
        \begin{itemize}
            \item The algorithm begins by building a max heap from the input array.
            \item It then repeatedly removes the maximum element from the heap (the root of the heap), and moves it to the end of the array. After removing the maximum element, it must re-heapify the 
            remaining elements to ensure the heap property is maintained.
            \item This process continues until all elements are removed from the heap and placed into the array in sorted order.
            \item The beauty of Heapsort lies in its $O(n \log n)$ time complexity for all cases: worst, average, and best.
        \end{itemize}
    \end{itemize}
    
    \begin{highlight}[Sample Python Code for Heapsort]
        Heapsort utilizes the heap data structure to sort an array in an efficient manner. The key process in Heapsort is to first transform the list of elements into a heap and then sort the elements 
        using the heap properties.
    \begin{code}[Python]
    def heapify(arr, n, i):
        largest = i
        left = 2 * i + 1
        right = 2 * i + 2
    
        if left < n and arr[left] > arr[largest]:
            largest = left
        if right < n and arr[right] > arr[largest]:
            largest = right
        if largest != i:
            arr[i], arr[largest] = arr[largest], arr[i]
            heapify(arr, n, largest)
    
    def heapsort(arr):
        n = len(arr)
        for i in range(n // 2 - 1, -1, -1):
            heapify(arr, n, i)
        for i in range(n-1, 0, -1):
            arr[i], arr[0] = arr[0], arr[i]  # swap
            heapify(arr, i, 0)
    
    # Example usage
    arr = [12, 11, 13, 5, 6, 7]
    heapsort(arr)
    print("Sorted array is:", arr)
    \end{code}
        
        This code showcases the Heapsort algorithm in action. The initial step is to build a max heap from the unsorted input array. Following this, the algorithm repeatedly swaps the first element of 
        the array (the largest value in the heap) with the last element of the heap, reduces the heap size by one, and then heapifies the root element to ensure the max heap property. This process is 
        repeated until the array is sorted.
    \end{highlight}
\end{notes}

The last section from this chapter is \textbf{Section 6.5: Priority Queues}.

\begin{notes}{Section 6.5: Priority Queues}
    \subsubsection*{Priority Queues}

    Priority queues are an abstract data type that operate similarly to regular queues or stacks, but with an added feature: each element has a "priority" associated with it. In a priority queue, an 
    element with high priority is served before an element with low priority. If two elements have the same priority, they are served according to their order in the queue. Priority queues are typically 
    implemented using heaps, as heaps provide an efficient way to maintain the order of the elements by their priority, allowing for quick insertion and removal of the highest or lowest priority item.
    
    \begin{itemize}
        \item \textbf{Applications of Priority Queues}
        \begin{itemize}
            \item Priority queues are used in many areas of computer science, including scheduling processes in operating systems, pathfinding algorithms like A* for AI and games, and in the 
            implementation of algorithms like Dijkstra's shortest path algorithm.
        \end{itemize}
        
        \item \textbf{Operations}
        \begin{itemize}
            \item The primary operations of a priority queue include \emph{insert} (or \emph{enqueue}), which adds an element to the queue with a given priority; \emph{peek} (or \emph{find-min} / \emph{find-max}), 
            which returns the highest (or lowest) priority element without removing it; and \emph{remove} (or \emph{dequeue}), which removes and returns the element with the highest (or lowest) priority.
        \end{itemize}
    \end{itemize}
    
    \begin{highlight}[Sample Python Code for Priority Queue using Heap]
        This Python code example demonstrates a simple priority queue implemented with a binary heap. The heapq module in Python provides an easy way to maintain a heap, allowing the priority queue to 
        efficiently perform its operations.
    \begin{code}[Python]
    import heapq
    
    class PriorityQueue:
        def __init__(self):
            self.heap = []
        
        def insert(self, item, priority):
            heapq.heappush(self.heap, (priority, item))
        
        def peek(self):
            return self.heap[0][1] if self.heap else None
        
        def remove(self):
            return heapq.heappop(self.heap)[1] if self.heap else None
    
    # Example usage
    pq = PriorityQueue()
    pq.insert("Task 1", 3)
    pq.insert("Task 2", 1)
    pq.insert("Task 3", 2)
    print("Peek at highest priority item:", pq.peek())
    print("Remove highest priority item:", pq.remove())
    print("Remove next highest priority item:", pq.remove())
    \end{code}
        
        In this implementation, the `PriorityQueue' class utilizes a min heap to keep track of items by priority, where a smaller number indicates a higher priority. The `insert' method adds a new item 
        with its priority to the heap, the `peek' method retrieves the highest priority item without removing it, and the `remove' method removes and returns the highest priority item from the queue.
    \end{highlight}    
\end{notes}

The last chapter this week is \textbf{Chapter 7: Quicksort}. The first section from this chapter is \textbf{Section 7.1: Description Of Quicksort}.

\begin{notes}{Section 7.1: Description Of Quicksort}
    \subsubsection*{Description of Quicksort}

    Quicksort is a highly efficient sorting algorithm that employs a divide-and-conquer approach to sort elements in an array. It was developed by Tony Hoare in 1959 and has since become one of the 
    most widely utilized sorting algorithms. Quicksort works by selecting a 'pivot' element from the array and partitioning the other elements into two sub-arrays, based on whether they are less than 
    or greater than the pivot. These sub-arrays are then sorted recursively, leading to a fully sorted array.
    
    \begin{itemize}
        \item \textbf{Algorithm Steps}
        \begin{itemize}
            \item \emph{Choosing a Pivot}: The pivot can be selected through various strategies, such as picking the first element, the last element, the median, or a random element. The choice of 
            pivot significantly affects the algorithm's performance.
            \item \emph{Partitioning}: The array is rearranged so all elements less than the pivot come before it, while all elements greater than the pivot come after it. The pivot then occupies its 
            correct position in the sorted array.
            \item \emph{Recursively Sorting Sub-arrays}: The process is applied recursively to the sub-array of elements with smaller values and the sub-array of elements with larger values.
        \end{itemize}
        
        \item \textbf{Performance}
        \begin{itemize}
            \item The average and best-case time complexity of Quicksort is $O(n \log n)$, making it highly efficient for large datasets. In the worst case, its time complexity can degrade to $O(n^2)$, 
            especially when the smallest or largest element is consistently chosen as the pivot.
            \item Its practical efficiency comes from the low overhead and cache efficiency, making it faster in practice than other $O(n \log n)$ sorting algorithms.
        \end{itemize}
        
        \item \textbf{In-place Sorting}
        \begin{itemize}
            \item Quicksort sorts the array in place, requiring no additional storage space except for the stack space used by recursion, enhancing its memory efficiency.
        \end{itemize}
    \end{itemize}
    
    \begin{highlight}[Sample Python Code for Quicksort]
        The following Python code illustrates the Quicksort algorithm, demonstrating the process of pivot selection, partitioning, and recursive sorting.
    \begin{code}[Python]
    def quicksort(arr, low, high):
        if low < high:
            pi = partition(arr, low, high)
            quicksort(arr, low, pi-1)
            quicksort(arr, pi+1, high)
    
    def partition(arr, low, high):
        pivot = arr[high]
        i = low - 1
        for j in range(low, high):
            if arr[j] < pivot:
                i += 1
                arr[i], arr[j] = arr[j], arr[i]
        arr[i+1], arr[high] = arr[high], arr[i+1]
        return i + 1
    
    # Example usage
    arr = [10, 7, 8, 9, 1, 5]
    quicksort(arr, 0, len(arr)-1)
    print("Sorted array:", arr)
    \end{code}
        
        This implementation selects the last element as the pivot and uses the partitioning method to place the pivot element in its correct position in the sorted array. It then recursively sorts the 
        sub-arrays before and after the pivot to achieve the sorted array.
    \end{highlight}    
\end{notes}

The next section from this chapter is \textbf{Section 7.2: Performance Of Quicksort}.

\begin{notes}{Section 7.2: Performance Of Quicksort}
    \subsubsection*{Performance of Quicksort}

    The performance of Quicksort is one of the key reasons for its popularity in practical applications. It is known for its efficiency in sorting large datasets, with a performance that can vary 
    significantly based on the choice of pivot and the arrangement of the input data.
    
    \begin{itemize}
        \item \textbf{Time Complexity}
        \begin{itemize}
            \item The average-case time complexity of Quicksort is $O(n \log n)$, where $n$ is the number of elements in the array. This optimal performance is achieved through the effective division 
            of the array into partitions and the efficient sorting of these partitions.
            \item The worst-case time complexity occurs when the smallest or largest element is consistently chosen as the pivot, leading to partitions of sizes $n-1$ and $1$, resulting in a time 
            complexity of $O(n^2)$. However, this scenario is rare, especially with good pivot selection strategies.
            \item The best-case scenario, with a time complexity of $O(n \log n)$, occurs when the partition process divides the array into two equal parts, thereby minimizing the depth of the recursion 
            tree.
        \end{itemize}
        
        \item \textbf{Space Complexity}
        \begin{itemize}
            \item Quicksort is an in-place sorting algorithm, but it requires additional space for the recursive function calls. The space complexity is $O(\log n)$ in the best case, which corresponds 
            to the height of a balanced recursion tree.
            \item In the worst case, with unbalanced partitions, the space complexity can degrade to $O(n)$ due to the depth of the recursion stack.
        \end{itemize}
        
        \item \textbf{Optimizations and Practical Performance}
        \begin{itemize}
            \item Several strategies can optimize Quicksort's performance, including choosing a pivot using methods like the median-of-three, which selects the pivot as the median of the first, middle, 
            and last elements of the partition.
            \item The use of a hybrid algorithm, such as switching to insertion sort for small partitions, can significantly reduce the overhead of recursive calls, thereby improving performance.
            \item The dual-pivot Quicksort, where two pivots are used to divide the array into three parts, has also been shown to offer improvements in sorting efficiency.
        \end{itemize}
    \end{itemize}
    
    In summary, the performance of Quicksort is generally very good and often surpasses other sorting algorithms, especially for large arrays. With intelligent pivot selection and optimizations for 
    small arrays, Quicksort can achieve near-optimal efficiency, making it a versatile choice for a wide range of sorting applications.    
\end{notes}

The next section for this chapter that is covered this week is \textbf{Section 7.3: A Randomized Version Of Quicksort}.

\begin{notes}{Section 7.3: A Randomized Version Of Quicksort}
    \subsubsection*{A Randomized Version of Quicksort}

    A randomized version of Quicksort enhances the traditional Quicksort algorithm by selecting the pivot in a random manner. This approach helps to optimize the performance of Quicksort by reducing 
    the likelihood of encountering the worst-case time complexity of $O(n^2)$, regardless of the input distribution. The randomized selection of pivots ensures that the algorithm's average-case time 
    complexity of $O(n \log n)$ is achieved more consistently.
    
    \begin{itemize}
        \item \textbf{Advantages of Randomization}
        \begin{itemize}
            \item Randomization in Quicksort mitigates the risk of poor performance on already sorted or nearly sorted arrays, which are problematic for deterministic pivot selection methods.
            \item It provides a probabilistic guarantee on the algorithm's performance, making its time complexity predictable and generally close to the average-case scenario.
        \end{itemize}
        
        \item \textbf{Implementation Considerations}
        \begin{itemize}
            \item The primary change in a randomized Quicksort algorithm is the method of pivot selection. Instead of choosing a fixed position, the pivot is chosen randomly from the sub-array that 
            is currently being sorted.
            \item This random pivot selection can be efficiently implemented by swapping the randomly chosen element with the first or last element of the sub-array and then proceeding with the 
            standard partitioning process.
        \end{itemize}
    \end{itemize}
    
    \begin{highlight}[Sample Python Code for Randomized Quicksort]
        The following Python code demonstrates the implementation of a randomized version of Quicksort, highlighting the randomized pivot selection.
    \begin{code}[Python]
    import random
    
    def randomized_partition(arr, low, high):
        pivot_index = random.randint(low, high)
        arr[pivot_index], arr[high] = arr[high], arr[pivot_index]  # Swap pivot with the end element
        return partition(arr, low, high)
    
    def partition(arr, low, high):
        pivot = arr[high]
        i = low - 1
        for j in range(low, high):
            if arr[j] < pivot:
                i += 1
                arr[i], arr[j] = arr[j], arr[i]
        arr[i+1], arr[high] = arr[high], arr[i+1]
        return i + 1
    
    def randomized_quicksort(arr, low, high):
        if low < high:
            pi = randomized_partition(arr, low, high)
            randomized_quicksort(arr, low, pi-1)
            randomized_quicksort(arr, pi+1, high)
    
    # Example usage
    arr = [10, 7, 8, 9, 1, 5]
    randomized_quicksort(arr, 0, len(arr)-1)
    print("Sorted array using randomized Quicksort:", arr)
    \end{code}
        
        This implementation introduces randomness to the pivot selection process in Quicksort, effectively randomizing the algorithm. By selecting a random pivot, the algorithm minimizes the chance 
        of experiencing the worst-case scenario, thus maintaining efficient performance across various input arrays.
    \end{highlight}    
\end{notes}

The last section from this chapter is \textbf{Section 7.4: Analysis Of Quicksort}.

\begin{notes}{Section 7.4: Analysis Of Quicksort}
    \subsubsection*{Analysis of Quicksort}

    The analysis of Quicksort focuses on understanding its time complexity in various scenarios and the factors that influence its performance. Despite its worst-case time complexity of $O(n^2)$, 
    Quicksort is often faster in practice than other sorting algorithms with the same average-case complexity of $O(n \log n)$. This efficiency is largely due to the constant factors hidden in the 
    big-O notation, the way it benefits from cache memory, and its ability to sort in place with minimal additional memory requirements.
    
    \begin{itemize}
        \item \textbf{Average-Case Analysis}
        \begin{itemize}
            \item The average-case time complexity of Quicksort is $O(n \log n)$, which is derived under the assumption that all permutations of the input array are equally probable. This scenario 
            occurs when the pivot divides the array into parts of reasonably proportional sizes, allowing the depth of the recursion to be logarithmic relative to the size of the array.
        \end{itemize}
        
        \item \textbf{Worst-Case Analysis}
        \begin{itemize}
            \item The worst-case scenario happens when the partitioning routine produces one sub-array with $n-1$ elements and one with $0$ elements, leading to a recursion depth of $n$. This scenario 
            typically occurs when the smallest or largest element is always chosen as the pivot. The time complexity in this case is $O(n^2)$.
        \end{itemize}
        
        \item \textbf{Best-Case Analysis}
        \begin{itemize}
            \item The best-case scenario for Quicksort occurs when the partition process divides the array into two equal parts at each step. This ideal situation also results in a time complexity of 
            $O(n \log n)$, but with a smaller constant factor than in the average case.
        \end{itemize}
        
        \item \textbf{Space Complexity}
        \begin{equation*}
            O(\log n)
        \end{equation*}
        \begin{itemize}
            \item The space complexity of Quicksort is $O(\log n)$ in the best case, corresponding to the height of a balanced recursion tree. In the worst case, the space complexity can increase to 
            $O(n)$ due to the stack space required for the recursive calls.
        \end{itemize}
        
        \item \textbf{Optimizations}
        \begin{itemize}
            \item Various optimizations can improve Quicksort's performance and reduce the likelihood of hitting the worst-case scenario. These include choosing the pivot via more sophisticated methods 
            (such as the median of three or random selection), switching to a different sorting algorithm like insertion sort for small arrays, and using tail recursion optimization.
        \end{itemize}
    \end{itemize}
    
    In summary, the efficiency of Quicksort in practice is attributed to its $O(n \log n)$ average-case time complexity and the various optimizations that can be applied to minimize the chances of 
    encountering the worst-case scenario. Despite the theoretical $O(n^2)$ worst-case time complexity, these optimizations ensure that Quicksort remains one of the fastest sorting algorithms for practical 
    applications.    
\end{notes}