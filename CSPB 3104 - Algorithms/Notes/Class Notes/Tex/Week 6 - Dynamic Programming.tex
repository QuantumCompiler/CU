\clearpage

\renewcommand{\ChapTitle}{Dynamic Programming}
\renewcommand{\SectionTitle}{Dynamic Programming}

\chapter{\ChapTitle}
\section{\SectionTitle}
\horizontalline{0}{0}

\subsection{Assigned Reading}

The reading assignment for this week is from, \Textbook:

\begin{itemize}
    \item \textbf{Chapter 15.1 - Rod Cutting}
    \item \textbf{Chapter 15.2 - Matrix-Chain Multiplication}
    \item \textbf{Chapter 15.3 - Elements Of Dynamic Programming}
    \item \textbf{Chapter 15.4 - Longest Common Subsequence}
    \item \textbf{Chapter 15.5 - Optimal Binary Search Trees}
\end{itemize}

\subsection{Lectures}

The lecture videos for this week are:

\begin{itemize}
    \item \lecture{https://www.youtube.com/watch?v=JiaVGtMWENE}{Rod Cutting Problem - First Steps}{31}
    \item \lecture{https://www.youtube.com/watch?v=ncI7_ytzw3Y}{Rod Cutting Problem: Memoization And Solution Recovery}{24}
    \item \lecture{https://www.youtube.com/watch?v=5Y0WnXkE9dE}{Dynamic Programming: Coin Changing Problem}{12}
    \item \lecture{https://www.youtube.com/watch?v=ZXiAzZ6jnSY}{Knapsack Problem}{29}
    \item \lecture{https://www.youtube.com/watch?v=gsVzzpVA9CA}{Failure of Optimal Substructure}{10}
    \item \lecture{https://www.youtube.com/watch?v=9Fpp_JpDIHo}{Longest Common Subsequence Problem}{25}
\end{itemize}

\subsection{Assignments}

The assignment for this week is:

\begin{itemize}
    \item \href{https://github.com/QuantumCompiler/CU/tree/main/CSPB%203104%20-%20Algorithms/Assignments/Problem%20Sets/Problem%20Set%206%20-%20Dynamic%20Programming}{Problem Set 6 - Dynamic Programming}
\end{itemize}

\subsection{Quiz}

The quizzes for this week are:

\begin{itemize}
    \item \pdflink{\QuizDir Quiz 6 - Dynamic Programming.pdf}{Quiz 6 - Dynamic Programming}
\end{itemize}

\subsection{Chapter Summary}

The chapter that is being covered this week is \textbf{Chapter 15: Dynamic Programming}. The first section that is being covered this week is \textbf{Section 15.1: Rod Cutting}.

\begin{notes}{Section 15.1: Rod Cutting}
    \subsubsection*{Rod Cutting}

    The Rod Cutting problem is a classic example of dynamic programming, where the goal is to maximize profit from cutting a rod into smaller lengths and selling the pieces based on a given price table. 
    This problem exemplifies the technique of solving complex problems by breaking them down into simpler subproblems, solving each subproblem just once, and storing their solutions - often in an array 
    or hashtable - to avoid the computational cost of solving the same subproblem multiple times.
    
    \begin{itemize}
        \item \textbf{Problem Statement}
        \begin{itemize}
            \item Given a rod of length $n$ and a table of prices $p_i$ for $i = 1, 2, ..., n$, where $p_i$ is the price of a rod of length $i$, determine the maximum revenue $r_n$ obtainable by cutting 
            up the rod and selling the pieces.
        \end{itemize}
        
        \item \textbf{Approach}
        \begin{itemize}
            \item The problem can be approached recursively, considering the revenue gained by making the first cut at different lengths $k$, and recursively solving the problem for a rod of length $n-k$.
            \item To avoid the exponential time complexity due to the recursive structure of the problem, dynamic programming techniques are employed. These include:
                \begin{enumerate}
                    \item \textbf{Top-Down with Memoization}: Where the algorithm solves the problem by recursively breaking it down into smaller subproblems, caching the result of each subproblem to 
                    ensure that each subproblem is solved only once.
                    \item \textbf{Bottom-Up Approach}: Where the algorithm iteratively solves all subproblems from the smallest to the largest, using the solutions of smaller problems to solve larger ones.
                \end{enumerate}
        \end{itemize}
        
        \item \textbf{Runtime Complexity}
        \begin{itemize}
            \item The naive recursive solution has a runtime complexity of $O(2^n)$, which is significantly reduced to $O(n^2)$ with the use of dynamic programming, either top-down with memoization or bottom-up.
        \end{itemize}
    \end{itemize}
    
    \begin{highlight}[Sample Python Code for Rod Cutting using Dynamic Programming]
        This Python code snippet demonstrates the bottom-up dynamic programming approach to solve the Rod Cutting problem.
    \begin{code}[Python]
    def rod_cutting(prices, n):
        # Initialize revenue array
        revenue = [0] * (n + 1)
        
        for j in range(1, n + 1):
            max_rev = float('-inf')
            for i in range(1, j + 1):
                max_rev = max(max_rev, prices[i] + revenue[j - i])
            revenue[j] = max_rev
        
        return revenue[n]
    \end{code}
        In this implementation, `prices` is a dictionary where the keys are the lengths of the rod pieces and the values are the corresponding prices. The function `rod\_cutting` calculates the maximum 
        revenue that can be obtained for a rod of length `n` using the bottom-up dynamic programming approach.
    \end{highlight}    
\end{notes}

The next section that is being covered from this chapter is \textbf{Section 15.2: Matrix-Chain Multiplication}.

\begin{notes}{Section 15.2: Matrix-Chain Multiplication}
    \subsubsection*{Matrix-Chain Multiplication}

    Matrix-Chain Multiplication is a classic optimization problem addressed by dynamic programming. It seeks to find the most efficient way to multiply a sequence of matrices. The challenge lies in the 
    fact that matrix multiplication is associative, meaning the order in which the multiplications are performed can significantly affect the total number of scalar operations required, without changing 
    the product. The goal is to determine the optimal parenthesization of the matrix product that minimizes the number of scalar multiplications.
    
    \begin{itemize}
        \item \textbf{Problem Statement}
        \begin{itemize}
            \item Given a chain $\langle A_1, A_2, ..., A_n \rangle$ of $n$ matrices, where for $i = 1, 2, ..., n$, matrix $A_i$ has dimension $p_{i-1} \times p_i$, fully parenthesize the product 
            $A_1A_2...A_n$ in a way that minimizes the number of scalar multiplications.
        \end{itemize}
        
        \item \textbf{Dynamic Programming Approach}
        \begin{itemize}
            \item The solution involves a bottom-up approach that constructs a table $m[i,j]$ for the minimum number of scalar multiplications needed to compute the matrix $A_iA_{i+1}...A_j$, for 
            $1 \leq i, j \leq n$. The optimal solution to the entire problem is then found in $m[1,n]$.
            \item The algorithm also maintains a table $s[i,j]$ to record the index $k$ at which the optimal split of the product sequence occurs, facilitating the construction of the optimal parenthesization.
        \end{itemize}
        
        \item \textbf{Runtime Complexity}
        \begin{itemize}
            \item The time complexity of the dynamic programming solution for the Matrix-Chain Multiplication problem is $O(n^3)$, where $n$ is the number of matrices in the chain. The space complexity 
            for storing the $m$ and $s$ tables is $O(n^2)$.
        \end{itemize}
    \end{itemize}
    
    \begin{highlight}[Sample Python Code for Matrix-Chain Multiplication]
        The following Python code demonstrates the dynamic programming solution to the Matrix-Chain Multiplication problem.
    \begin{code}[Python]
    def matrix_chain_order(p):
        n = len(p) - 1
        m = [[0 for _ in range(n)] for _ in range(n)]
        s = [[0 for _ in range(n)] for _ in range(n)]
    
        for length in range(2, n + 1):
            for i in range(n - length + 1):
                j = i + length - 1
                m[i][j] = float('inf')
                for k in range(i, j):
                    q = m[i][k] + m[k + 1][j] + p[i] * p[k + 1] * p[j + 1]
                    if q < m[i][j]:
                        m[i][j] = q
                        s[i][j] = k
        return m, s
    
    def print_optimal_parens(s, i, j):
        if i == j:
            print(f"A{i+1}", end="")
        else:
            print("(", end="")
            print_optimal_parens(s, i, s[i][j])
            print_optimal_parens(s, s[i][j] + 1, j)
            print(")", end="")
    \end{code}
        This implementation calculates the minimum number of scalar multiplications needed to multiply a chain of matrices and prints the optimal way to parenthesize the product. The `matrix\_chain\_order` 
        function builds the $m$ and $s$ tables, and the `print\_optimal\_parens` function prints the optimal parenthesization.
    \end{highlight}      
\end{notes}

The next section that is being covered from this chapter this week is \textbf{Section 15.3: Elements Of Dynamic Programming}.

\begin{notes}{Section 15.3: Elements Of Dynamic Programming}
    \subsubsection*{Elements of Dynamic Programming}

    Dynamic Programming (DP) is a method for solving complex problems by breaking them down into simpler subproblems. It is particularly useful for problems exhibiting the properties of overlapping 
    subproblems and optimal substructure. Understanding the key elements of dynamic programming is crucial for effectively applying it to a wide range of problems.
    
    \begin{itemize}
        \item \textbf{Overlapping Subproblems}
        \begin{itemize}
            \item This property indicates that the space of subproblems is small, meaning that the same subproblems are solved repeatedly during the computation. Dynamic programming exploits this by 
            solving each subproblem only once and storing its solution in a table, thereby avoiding the work of recomputing the answer every time the subproblem is encountered.
        \end{itemize}
        
        \item \textbf{Optimal Substructure}
        \begin{itemize}
            \item A problem exhibits optimal substructure if an optimal solution to the problem contains within it optimal solutions to the subproblems. This property allows a problem to be solved by 
            combining the solutions of its subproblems. Dynamic programming algorithms typically build up solutions to larger and larger subproblems using the solutions to smaller subproblems.
        \end{itemize}
        
        \item \textbf{Memoization vs. Tabulation}
        \begin{itemize}
            \item \textbf{Memoization} (Top-Down Approach): This technique involves writing the recursive algorithm and using a table to store the results (memo) of subproblems to avoid recomputing 
            them. The algorithm is typically expressed in a natural manner but modified to save the results of intermediate subproblems.
            \item \textbf{Tabulation} (Bottom-Up Approach): This technique involves filling up a DP table by solving subproblems in a specific order, ensuring that all subproblems needed to solve a 
            problem are solved first. This approach iteratively builds up the solution to larger problems based on the solutions to smaller problems.
        \end{itemize}
        
        \item \textbf{Constructing Solutions}
        \begin{itemize}
            \item Beyond calculating the optimal value that a dynamic programming solution seeks, many problems also require constructing the solution itself. This typically involves additional 
            bookkeeping to trace back the decisions that led to the optimal value, often through a separate table or by augmenting the table used to calculate the optimal values.
        \end{itemize}
        
        \item \textbf{Runtime Complexity}
        \begin{itemize}
            \item The time complexity of a dynamic programming algorithm is typically determined by the number of subproblems multiplied by the time taken to solve each subproblem. Since each subproblem 
            is solved only once and stored for future reference, dynamic programming can significantly reduce the time complexity from exponential to polynomial in many cases.
        \end{itemize}
    \end{itemize}
    
    \begin{highlight}[Applying Dynamic Programming]
        Successfully applying dynamic programming involves identifying that the problem has overlapping subproblems and optimal substructure, choosing between memoization and tabulation based on the 
        problem and personal preference, and carefully defining the structure of the table(s) used for storing the solutions of subproblems. Understanding these elements is fundamental to designing 
        efficient dynamic programming solutions.
    \end{highlight}    
\end{notes}

The next section that is being covered from this chapter this week is \textbf{Section 15.4: Longest Common Subsequence}.

\begin{notes}{Section 15.4: Longest Common Subsequence}
    \subsubsection*{Longest Common Subsequence}

    The Longest Common Subsequence (LCS) problem is a classic example in the field of dynamic programming and computer science, focusing on finding the longest subsequence present in two sequences 
    (not necessarily contiguous) in the same order. This problem is fundamental in the domains of bioinformatics, text processing, and diff tools, among others.
    
    \begin{itemize}
        \item \textbf{Problem Statement}
        \begin{itemize}
            \item Given two sequences $X = \{x_1, x_2, ..., x_m\}$ and $Y = \{y_1, y_2, ..., y_n\}$, find the length of their longest common subsequence, which is the longest sequence that can be 
            derived from both $X$ and $Y$ by deleting some items without changing the order of the remaining elements.
        \end{itemize}
        
        \item \textbf{Dynamic Programming Formulation}
        \begin{itemize}
            \item The problem can be solved by considering smaller instances of the problem, namely finding the LCS of prefixes of $X$ and $Y$. Let $c[i, j]$ be the length of the LCS of $X_i$ and 
            $Y_j$, where $X_i$ and $Y_j$ are the first $i$ and $j$ elements of $X$ and $Y$, respectively.
            \item The solution is built up using the following recurrence:
            \begin{equation*}
                c[i, j] = 
                \begin{cases} 
                0 & \text{if } i = 0 \text{ or } j = 0 \\
                c[i-1, j-1] + 1 & \text{if } i, j > 0 \text{ and } x_i = y_j \\
                \max(c[i, j-1], c[i-1, j]) & \text{if } i, j > 0 \text{ and } x_i \neq y_j
                \end{cases}
            \end{equation*}
            \item This approach ensures that each subproblem is solved only once, with the solution being stored in a table from which the solution to the problem can be constructed.
        \end{itemize}
        
        \item \textbf{Runtime Complexity}
        \begin{itemize}
            \item The runtime complexity of solving the LCS problem using dynamic programming is $O(mn)$, where $m$ and $n$ are the lengths of the two sequences. This complexity arises from filling 
            out a table of size $m \times n$ based on the recurrence relation.
        \end{itemize}
    \end{itemize}
    
    \begin{highlight}[Sample Python Code for Longest Common Subsequence]
        This Python code snippet demonstrates the dynamic programming solution to the LCS problem.
    \begin{code}[Python]
    def lcs(X, Y):
        m, n = len(X), len(Y)
        c = [[0] * (n+1) for _ in range(m+1)]
        for i in range(1, m+1):
            for j in range(1, n+1):
                if X[i-1] == Y[j-1]:
                    c[i][j] = c[i-1][j-1] + 1
                else:
                    c[i][j] = max(c[i][j-1], c[i-1][j])
        return c[m][n]
    \end{code}
        The `lcs` function computes the length of the longest common subsequence between two strings `X` and `Y`, illustrating the application of dynamic programming to efficiently solve this problem.
    \end{highlight}      
\end{notes}

The next section that is being covered from this chapter this week is \textbf{Section 15.5: Optimal Binary Search Trees}.

\begin{notes}{Section 15.5: Optimal Binary Search Trees}
    \subsubsection*{Optimal Binary Search Trees}

    Optimal Binary Search Trees (BSTs) are a specialized form of binary search trees that are structured to minimize the cost of searching for a set of keys. The cost is typically measured in terms of 
    the number of comparisons needed to find a key in the tree. The concept of optimality here relates to the minimization of the expected search cost, given a set of keys with known probabilities of access.
    
    \begin{itemize}
        \item \textbf{Problem Statement}
        \begin{itemize}
            \item Given a set of $n$ keys with their probabilities of being searched for, construct a binary search tree that has the minimum expected search cost. The search cost of a key is proportional 
            to the depth of the key in the tree.
        \end{itemize}
        
        \item \textbf{Dynamic Programming Approach}
        \begin{itemize}
            \item The solution to constructing an optimal BST involves a dynamic programming approach that examines all possible trees and selects the one with the lowest cost. This includes not only 
            the given keys but also the probabilities of searching for values between these keys (often treated as "dummy keys" representing searches for values not in the set).
            \item The algorithm calculates the cost of each subtree and uses these costs to construct the optimal solution for larger trees. It maintains a table $e[i,j]$ for the cost of the optimal 
            BST for keys $k_i$ through $k_j$, and a table $root[i,j]$ to store the root of the optimal subtree for $k_i$ through $k_j$.
        \end{itemize}
        
        \item \textbf{Runtime Complexity}
        \begin{itemize}
            \item The time complexity of constructing an optimal BST using dynamic programming is $O(n^3)$, where $n$ is the number of keys. This complexity arises from the need to examine each possible 
            subtree for each interval of keys.
        \end{itemize}
    \end{itemize}
    
    \begin{highlight}[Sample Python Code for Optimal Binary Search Trees]
        The following Python code snippet provides an implementation of the dynamic programming approach to construct optimal binary search trees.
    \begin{code}[Python]
    def optimal_bst(keys, p, q, n):
        e = [[0 for j in range(n + 1)] for i in range(n + 2)]
        w = [[0 for j in range(n + 1)] for i in range(n + 2)]
        root = [[0 for j in range(n + 1)] for i in range(n + 1)]
    
        for i in range(1, n + 2):
            e[i][i - 1] = q[i - 1]
            w[i][i - 1] = q[i - 1]
    
        for l in range(1, n + 1):
            for i in range(1, n - l + 2):
                j = i + l - 1
                e[i][j] = float('inf')
                w[i][j] = w[i][j - 1] + p[j] + q[j]
                for r in range(i, j + 1):
                    t = e[i][r - 1] + e[r + 1][j] + w[i][j]
                    if t < e[i][j]:
                        e[i][j] = t
                        root[i][j] = r
        return e, root
    \end{code}
        This implementation calculates the cost and structure of the optimal BST for a given set of keys and their search probabilities. The `optimal\_bst` function computes the cost $e$ and the root 
        table, which can be used to construct the tree itself.
    \end{highlight}    
\end{notes}