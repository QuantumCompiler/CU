\clearpage

\renewcommand{\ChapTitle}{Graphs And Basic Algorithms On Graphs}
\renewcommand{\SectionTitle}{Graphs And Basic Algorithms On Graphs}

\chapter{\ChapTitle}
\section{\SectionTitle}
\horizontalline{0}{0}

\subsection{Assigned Reading}

The reading assignment for this week is from, \Textbook:

\begin{itemize}
    \item \pdflink{\ReadMatDir Chapter 22.1 - Representation Of Graphs.pdf}{Chapter 22.1 - Representation Of Graphs}
    \item \pdflink{\ReadMatDir Chapter 22.2 - Breadth-First Search.pdf}{Chapter 22.2 - Breadth-First Search}
    \item \pdflink{\ReadMatDir Chapter 22.3 - Depth-First Search.pdf}{Chapter 22.3 - Depth-First Search}
    \item \pdflink{\ReadMatDir Chapter 22.4 - Topological Sort.pdf}{Chapter 22.4 - Topological Sort}
\end{itemize}

\subsection{Lectures}

The lecture videos for this week are:

\begin{itemize}
    \item \lecture{https://www.youtube.com/watch?v=-o0x1uXtPHM}{Introduction To Graphs}{14}
    \item \lecture{https://www.youtube.com/watch?v=tYk2vvm09D0}{Breadth First Search}{17}
    \item \lecture{https://www.youtube.com/watch?v=Ig25N9h33Zg}{Depth First Search}{33}
    \item \lecture{https://www.youtube.com/watch?v=NSjjPtcrz3g}{Topological Sorting}{11}
\end{itemize}

\subsection{Assignments}

The assignment for this week is:

\begin{itemize}
    \item \href{https://github.com/QuantumCompiler/CU/tree/main/CSPB%203104%20-%20Algorithms/CSPB%203104%20-%20Assignments/CSPB%203104%20-%20Problem%20Sets/CSPB%203104%20-%20Problem%20Set%207%20-%20Graphs}{Problem Set 7 - Graphs}
\end{itemize}

\subsection{Quiz}

The quizzes for this week are:

\begin{itemize}
    \item \pdflink{\QuizDir Quiz 8.pdf}{Quiz 8}
\end{itemize}

\subsection{Chapter Summary}

The chapter that is being covered this week is \textbf{Chapter 22: Elementary Graph Algorithms}. The first section that is being covered from this chapter this week is \textbf{Section 22.1: Representations Of Graphs}.

\begin{notes}{Section 22.1: Representations Of Graphs}
    \subsubsection*{Representations of Graphs}

    Graphs can be represented in various ways, each with its own advantages and disadvantages. The two most common representations are adjacency lists and adjacency matrices. These representations 
    are crucial for designing and analyzing algorithms that operate on graphs. \vspace*{1em}
    
    \subsubsection*{Adjacency Lists}
    
    An adjacency list represents a graph as an array of lists. Each list corresponds to a vertex in the graph and contains all vertices adjacent to it. This representation is efficient in terms of 
    space, especially for sparse graphs where the number of edges is much less than the square of the number of vertices. 
    
    \begin{itemize}
        \item \textbf{Advantages:} Space-efficient for sparse graphs; easy to iterate over the neighbors of a vertex.
        \item \textbf{Disadvantages:} Checking whether an edge exists between two vertices can be less efficient than with an adjacency matrix.
    \end{itemize}
    
    \subsubsection*{Adjacency Matrices}
    
    An adjacency matrix represents a graph as a 2D array of size $V \times V$ where $V$ is the number of vertices in the graph. Each cell at position $(i, j)$ in the matrix indicates whether there is 
    an edge from vertex $i$ to vertex $j$, typically with a 1 for the presence of an edge and a 0 for its absence.
    
    \begin{itemize}
        \item \textbf{Advantages:} Efficient for checking the existence of an edge between any two vertices; easy to add or remove edges.
        \item \textbf{Disadvantages:} Space-inefficient for sparse graphs, as it requires storing $V^2$ elements regardless of the number of edges.
    \end{itemize}
    
    \subsubsection*{Choosing the Right Representation}
    
    The choice between adjacency lists and matrices largely depends on the specific requirements of the algorithm and the nature of the graph being dealt with. Adjacency lists are generally preferred 
    for sparse graphs due to their space efficiency, while adjacency matrices may be more suitable for dense graphs or when frequent edge existence checks are necessary.
\end{notes}

The next section that is being covered from this chapter this week is \textbf{Section 22.2: Breadth-First Search}.

\begin{notes}{Section 22.2: Breadth-First Search}
    \subsubsection*{Breadth-first Search (BFS)}

    Breadth-first Search (BFS) is an algorithm for traversing or searching tree or graph data structures. It starts at the tree root (or some arbitrary node of a graph, sometimes referred to as a 
    'search key') and explores the neighbor nodes at the present depth prior to moving on to the nodes at the next depth level. \vspace*{1em}
    
    \subsubsection*{Algorithm Overview:}
    \begin{itemize}
        \item BFS explores a graph in the broadest manner by visiting all neighbors of a starting vertex before moving to their children.
        \item It uses a queue to keep track of the next location to visit.
        \item BFS can be used to find the shortest path between two nodes in an unweighted graph.
    \end{itemize}
    
    \subsubsection*{Steps:}
    \begin{enumerate}
        \item Start by putting any one of the graph's vertices at the back of a queue.
        \item Take the front item of the queue and add it to the visited list.
        \item Create a list of that vertex's adjacent nodes. Add the ones which aren't in the visited list to the back of the queue.
        \item Keep repeating steps 2 and 3 until the queue is empty.
    \end{enumerate}
    
    \subsubsection*{Applications:}
    \begin{itemize}
        \item Finding shortest paths in unweighted graphs.
        \item Crawling web pages from a given URL.
        \item Social networking applications where you want to find people within a certain distance 'k' from a person.
    \end{itemize}
    
    \begin{highlight}[Python Example: Breadth-first Search]
        Below is a simple Python example of BFS algorithm implementation to traverse graphs.
    \begin{code}[Python]
    from collections import deque
    
    def bfs(graph, start):
        visited = set()
        queue = deque([start])
        
        while queue:
            vertex = queue.popleft()
            if vertex not in visited:
                visited.add(vertex)
                print(vertex, end=" ")
                
                # Add all unvisited neighbours to the queue.
                queue.extend(neighbour for neighbour in graph[vertex] if neighbour not in visited)
    \end{code}
        
        This Python function demonstrates performing a BFS on a graph represented as an adjacency list. Starting from a given node, it visits all nodes of the graph in the breadth-first manner.
    \end{highlight}
\end{notes}

The next section that is being covered from this chapter this week is \textbf{Section 22.3: Depth-First Search}.

\begin{notes}{Section 22.3: Depth-First Search}
    \subsubsection*{Depth-first Search (DFS)}

    Depth-first Search (DFS) is a fundamental algorithm used in graph theory to explore vertices and edges of a graph. It emphasizes diving as deep as possible into the graph's branches before 
    backtracking. This strategy makes DFS an excellent tool for exploring maze structures, solving puzzles, and analyzing networks. \vspace*{1em}
    
    \subsubsection*{Algorithm Overview:}
    \begin{itemize}
        \item DFS explores a graph by starting at the root (or any arbitrary node) and explores as far as possible along each branch before backtracking.
        \item It uses a stack (either the function call stack through recursion or an explicit stack data structure) to keep track of the path it's exploring.
    \end{itemize}
    
    \subsubsection*{Steps:}
    \begin{enumerate}
        \item Mark the current vertex as visited and print or record it.
        \item For each adjacent vertex connected to the current vertex, if it has not been visited, recursively visit that vertex.
        \item Continue until all vertices that are reachable from the original source vertex have been explored.
        \item If there are still unvisited vertices, select one as a new source and repeat the process, ensuring all vertices in the graph are explored.
    \end{enumerate}
    
    \subsubsection*{Applications:}
    \begin{itemize}
        \item Pathfinding and checking for connected components in a graph.
        \item Topological sorting.
        \item Finding strongly connected components.
    \end{itemize}
    
    \begin{highlight}[Python Example: Depth-first Search]
        Below is a Python example demonstrating the DFS algorithm.
    \begin{code}[Python]
    def dfs(graph, node, visited=set()):
        if node not in visited:
            print(node, end=" ")
            visited.add(node)
            for neighbour in graph[node]:
                dfs(graph, neighbour, visited)
    \end{code}
    
        This Python function showcases how to perform a DFS on a graph represented as an adjacency list. It emphasizes a recursive approach to traverse all vertices of the graph, marking each as visited 
        and exploring as far as possible before backtracking.
    \end{highlight}    
\end{notes}

The last section that is being covered from this chapter this week is \textbf{Section 22.4: Topological Sort}.

\begin{notes}{Section 22.4: Topological Sort}
    \subsubsection*{Topological Sort}

    Topological sort is an algorithm for linearly ordering the vertices of a directed acyclic graph (DAG) such that for every directed edge $uv$ from vertex $u$ to vertex $v$, $u$ comes before $v$ in 
    the ordering. \vspace*{1em}
    
    \subsubsection*{Algorithm Overview:}
    \begin{itemize}
        \item Topological sorting is only possible if the graph has no directed cycles, i.e., it must be a DAG.
        \item A common approach to find a topological sort is to use Depth-first Search (DFS).
    \end{itemize}
    
    \subsubsection*{Steps:}
    \begin{enumerate}
        \item Call DFS to compute the finishing times for each vertex.
        \item As each vertex is finished, insert it onto the front of a linked list.
        \item After all vertices have been processed, return the linked list of vertices.
    \end{enumerate}
    
    \subsubsection*{Applications:}
    \begin{itemize}
        \item Scheduling tasks under precedence constraints.
        \item Course scheduling.
        \item Determining the order of compilation tasks in programming projects.
    \end{itemize}
    
    \begin{highlight}[Python Example: Topological Sort]
        Below is an example of how to implement topological sorting in Python using DFS.
    \begin{code}[Python]
    def dfs(graph, node, visited, stack):
        visited.add(node)
        for neighbour in graph[node]:
            if neighbour not in visited:
                dfs(graph, neighbour, visited, stack)
        stack.insert(0, node)
    
    def topological_sort(graph):
        visited = set()
        stack = []
        for node in graph:
            if node not in visited:
                dfs(graph, node, visited, stack)
        return stack
    \end{code}
    
        This Python function demonstrates a DFS-based approach to achieve a topological sort of a directed acyclic graph. The vertices are pushed onto a stack in the reverse order of their finishing 
        times, ensuring that all directed edges point from a vertex earlier in the order to a vertex later in the order.
    \end{highlight}    
\end{notes}