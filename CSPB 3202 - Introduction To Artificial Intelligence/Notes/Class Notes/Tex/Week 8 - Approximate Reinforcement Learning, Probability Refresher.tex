\clearpage

\renewcommand{\ChapTitle}{Approximate Reinforcement Learning, Probability Refresher}
\renewcommand{\SectionTitle}{Approximate Reinforcement Learning, Probability Refresher}

\chapter{\ChapTitle}

\section{\SectionTitle}
\horizontalline{0}{0}

\subsection{Assigned Reading}

The reading for this week is from, \AITextbook \hspace*{1pt} and \RLTextbook.

\begin{itemize}
    \item \textbf{Artificial Intelligence - A Modern Approach - Chapter 12.1 - Acting Under Uncertainty}
    \item \textbf{Artificial Intelligence - A Modern Approach - Chapter 12.2 - Basic Probability Notation}
    \item \textbf{Artificial Intelligence - A Modern Approach - Chapter 12.3 - Interference Using Full Joint Distributions}
    \item \textbf{Artificial Intelligence - A Modern Approach - Chapter 12.4 - Independence}
    \item \textbf{Artificial Intelligence - A Modern Approach - Chapter 12.5 - Bayes' Rule And Its Use}
    \item \textbf{Artificial Intelligence - A Modern Approach - Chapter 12.6 - Naive Bayes Models}
    \item \textbf{Artificial Intelligence - A Modern Approach - Chapter 12.7 - The Wumpus World Revisited}
    \item \textbf{Artificial Intelligence - A Modern Approach - Chapter 22.4 - Generalization In Reinforcement Learning}
\end{itemize}

\subsection{Piazza}

Must post at least \textbf{three} times this week to Piazza.

\subsection{Lectures}

The lectures for this week are:

\begin{itemize}
    \item \lecture{https://www.youtube.com/watch?v=ljnVJCowWJg}{Approximate Reinforcement Learning}{73}
    \item \lecture{https://www.youtube.com/watch?v=ayGGK5TcoqY}{Probabilistic Reasoning 1 - Probability Review}{69}
\end{itemize}

\noindent The lecture notes for this week are:

\begin{itemize}
    \item \pdflink{\LecNoteDir/Notes/Approximate Reinforcement Learning Lecture Notes.pdf}{Approximate Reinforcement Learning Lecture Notes}
    \item \pdflink{\LecNoteDir/Notes/Approximate Reinforcement Learning Review Lecture Notes.pdf}{Approximate Reinforcement Learning Review Lecture Notes}
    \item \pdflink{\LecNoteDir/Notes/Probabilistic Reasoning 1 - Probability Review Lecture Notes.pdf}{Probabilistic Reasoning 1 - Probability Review Lecture Notes}
\end{itemize}

\subsection{Quiz}

The quiz for this week is:

\begin{itemize}
    \item \pdflink{\QuizDir Quiz 8 - Approximate Reinforcement Learning And Probability.pdf}{Quiz 8 - Approximate Reinforcement Learning, Probability Refresher}
\end{itemize}

\newpage

\subsection{Chapter Summary}

The first chapter that is being covered this week is \textbf{Chapter 12: Quantifying Uncertainty}. The first section that is being covered from this chapter this week is \textbf{Section 12.1: Acting Under Uncertainty}.

\begin{notes}{Section 12.1: Acting Under Uncertainty}
    \subsection*{Overview}

    This section explores how agents can act rationally under uncertainty. It highlights the challenges agents face when dealing with incomplete information, unpredictable environments, and the need 
    for decision-making under such conditions. The section emphasizes the importance of quantifying uncertainty to enable rational decision-making.
    
    \subsubsection*{Uncertainty in Agent Actions}
    
    Agents in real-world environments must handle uncertainty caused by factors such as partial observability, nondeterminism, and adversaries. Traditional logical approaches, which rely on belief 
    states and contingency plans, are often inadequate for dealing with such uncertainty.
    
    \begin{highlight}[Uncertainty in Agent Actions]
    
        \begin{itemize}
            \item \textbf{Belief State}: Represents all possible states the agent might be in, given its knowledge and sensor data.
            \item \textbf{Contingency Planning}: Generating plans that handle every possible outcome can be inefficient and infeasible.
            \item \textbf{Example}: An automated taxi planning to reach the airport must consider various uncertainties such as traffic, breakdowns, and accidents.
        \end{itemize}
    
    \end{highlight}
    
    \subsubsection*{Summarizing Uncertainty}
    
    The logical approach to handling uncertainty, such as using propositional logic for diagnosis, often fails due to practical limitations. Agents need a way to represent and reason with degrees of 
    belief instead of relying solely on logic.
    
    \begin{highlight}[Summarizing Uncertainty]
    
        \begin{itemize}
            \item \textbf{Degree of Belief}: Representing uncertainty using numeric values (probabilities) rather than binary true/false logic.
            \item \textbf{Example}: Diagnosing a toothache might involve multiple possible causes, each with a certain probability.
            \item \textbf{Challenges}: Laziness (incomplete rules), theoretical ignorance (lack of complete knowledge), and practical ignorance (incomplete tests).
        \end{itemize}
    
    \end{highlight}
    
    \subsubsection*{Uncertainty and Rational Decisions}
    
    Making rational decisions under uncertainty involves considering both the likelihood of various outcomes and their relative importance. Utility theory is introduced to represent preferences and 
    make quantitative decisions based on expected utility.
    
    \begin{highlight}[Uncertainty and Rational Decisions]
    
        \begin{itemize}
            \item \textbf{Utility Theory}: Assigns a utility value to each possible outcome, representing the agent's preferences.
            \item \textbf{Decision Theory}: Combines probability theory and utility theory to make rational decisions that maximize expected utility.
            \item \textbf{Example}: Choosing between different plans to get to the airport, considering trade-offs between the likelihood of success and the cost (e.g., waiting time).
        \end{itemize}
    
    \end{highlight}
    
    \subsubsection*{Agent Architecture for Decision Making}
    
    The section concludes with an outline of a decision-theoretic agent architecture, which maintains probabilistic beliefs about the state of the world and selects actions that maximize expected utility.
    
    \begin{highlight}[Agent Architecture for Decision Making]
    
        \begin{itemize}
            \item \textbf{Belief State}: Updated based on actions and perceptions.
            \item \textbf{Outcome Probabilities}: Calculated for potential actions.
            \item \textbf{Action Selection}: Chooses the action with the highest expected utility.
        \end{itemize}
    
    \end{highlight}
    
    \begin{highlight}[Summary of Key Concepts]
    
        \begin{itemize}
            \item \textbf{Belief State}: A representation of all possible states an agent might be in.
            \item \textbf{Degree of Belief}: Using probabilities to represent uncertainty.
            \item \textbf{Utility Theory}: Quantifying preferences to guide decision-making.
            \item \textbf{Decision Theory}: Combining probability and utility to make rational decisions.
            \item \textbf{Agent Architecture}: A framework for decision-theoretic agents to act under uncertainty.
        \end{itemize}
    
        Understanding these concepts is crucial for developing agents that can operate effectively in uncertain environments, balancing the need for accurate beliefs with practical decision-making.
    
    \end{highlight}
\end{notes}

The next section from this chapter that is being covered this week is \textbf{Section 12.2: Basic Probability Notation}.

\begin{notes}{Section 12.2: Basic Probability Notation}
    \subsection*{Overview}

    This section introduces the basic probability notation necessary for understanding and working with uncertainty in artificial intelligence. It covers foundational concepts such as random variables, 
    probability distributions, and key axioms and properties of probability theory. This notation forms the basis for more advanced probabilistic reasoning techniques.
    
    \subsubsection*{Random Variables}
    
    Random variables are used to represent uncertain quantities in a probabilistic framework. They can take on a range of possible values, each associated with a probability.
    
    \begin{highlight}[Random Variables]
    
        \begin{itemize}
            \item \textbf{Definition}: A random variable $X$ is a function that assigns a real number to each outcome in a sample space.
            \item \textbf{Example}: In a coin toss, the random variable $X$ could represent the outcome, with $X = 1$ for heads and $X = 0$ for tails.
        \end{itemize}
    
    \end{highlight}
    
    \subsubsection*{Probability Distributions}
    
    Probability distributions describe how probabilities are assigned to the values of random variables. They can be defined for both discrete and continuous variables.
    
    \begin{highlight}[Probability Distributions]
    
        \begin{itemize}
            \item \textbf{Discrete Random Variables}: The probability distribution is specified by a probability mass function $P(X = x)$, which gives the probability that $X$ takes on the value $x$.
            \item \textbf{Example}: For a fair die roll, $P(X = x) = \frac{1}{6}$ for $x \in \{1, 2, 3, 4, 5, 6\}$.
            \item \textbf{Continuous Random Variables}: The probability distribution is specified by a probability density function $f_X(x)$, where the probability that $X$ falls within an interval $[a, b]$ is given by:
            \[
            P(a \leq X \leq b) = \int_a^b f_X(x) \, dx
            \]
        \end{itemize}
    
    \end{highlight}
    
    \subsubsection*{Axioms of Probability}
    
    The axioms of probability form the foundation of probability theory, providing the rules for how probabilities are assigned and manipulated.
    
    \begin{highlight}[Axioms of Probability]
    
        \begin{itemize}
            \item \textbf{Non-negativity}: For any event $A$, $P(A) \geq 0$.
            \item \textbf{Normalization}: The probability of the sample space $\Omega$ is 1, $P(\Omega) = 1$.
            \item \textbf{Additivity}: For any two mutually exclusive events $A$ and $B$,
            \[
            P(A \cup B) = P(A) + P(B)
            \]
        \end{itemize}
    
    \end{highlight}
    
    \subsubsection*{Conditional Probability}
    
    Conditional probability quantifies the probability of one event given that another event has occurred. It is a key concept for reasoning about dependencies between events.
    
    \begin{highlight}[Conditional Probability]
    
        \begin{itemize}
            \item \textbf{Definition}: The conditional probability of $A$ given $B$ is defined as:
            \[
            P(A | B) = \frac{P(A \cap B)}{P(B)} \quad \text{for } P(B) > 0
            \]
            \item \textbf{Example}: If $A$ is the event of drawing an ace from a deck of cards and $B$ is the event of drawing a spade, then:
            \[
            P(A | B) = \frac{P(A \cap B)}{P(B)} = \frac{\frac{1}{52}}{\frac{13}{52}} = \frac{1}{13}
            \]
        \end{itemize}
    
    \end{highlight}
    
    \subsubsection*{Independence}
    
    Two events are independent if the occurrence of one does not affect the probability of the other. Independence simplifies the computation of joint probabilities.
    
    \begin{highlight}[Independence]
    
        \begin{itemize}
            \item \textbf{Definition}: Events $A$ and $B$ are independent if:
            \[
            P(A \cap B) = P(A)P(B)
            \]
            \item \textbf{Example}: Rolling two fair dice, the outcome of one die does not affect the outcome of the other.
        \end{itemize}
    
    \end{highlight}
    
    \begin{highlight}[Summary of Key Concepts]
    
        \begin{itemize}
            \item \textbf{Random Variables}: Represent uncertain quantities, taking on various values with associated probabilities.
            \item \textbf{Probability Distributions}: Describe how probabilities are assigned to random variables, through probability mass functions for discrete variables and density functions for 
            continuous variables.
            \item \textbf{Axioms of Probability}: The foundational rules for probability, including non-negativity, normalization, and additivity.
            \item \textbf{Conditional Probability}: The probability of an event given that another event has occurred, quantifying dependencies.
            \item \textbf{Independence}: Events that do not affect each other's probabilities, simplifying joint probability computations.
        \end{itemize}
    
        Understanding these basic probability notations is essential for modeling and reasoning about uncertainty in intelligent systems, forming the basis for more advanced probabilistic methods and 
        algorithms.
    
    \end{highlight}
\end{notes}

The next section from this chapter that is being covered this week is \textbf{Section 12.3: Interference Using Full Joint Distributions}.

\begin{notes}{Section 12.3: Interference Using Full Joint Distributions}
    \subsection*{Overview}

    This section delves into inference using full joint probability distributions, which provide a comprehensive way to represent the probabilities of all possible combinations of variable values in a 
    domain. The full joint distribution serves as the foundation for deriving probabilities of interest through marginalization and conditioning.
    
    \subsubsection*{Full Joint Probability Distributions}
    
    A full joint probability distribution specifies the probability of every possible combination of values for a set of random variables. It is fundamental to probabilistic reasoning but often impractical 
    to use directly due to its size.
    
    \begin{highlight}[Full Joint Probability Distributions]
    
        \begin{itemize}
            \item \textbf{Definition}: For a set of random variables $X_1, X_2, \ldots, X_n$, the full joint probability distribution $P(X_1, X_2, \ldots, X_n)$ assigns a probability to each possible 
            combination of values $(x_1, x_2, \ldots, x_n)$.
            \item \textbf{Example}: In a domain with two binary variables, $Rain$ and $Sprinkler$, the full joint distribution $P(Rain, Sprinkler)$ includes probabilities for all four combinations of 
            true/false values.
        \end{itemize}
    
    \end{highlight}
    
    \subsubsection*{Marginalization}
    
    Marginalization involves summing the probabilities of the full joint distribution over the values of one or more variables to obtain the marginal probability of a subset of variables.
    
    \begin{highlight}[Marginalization]
    
        \begin{itemize}
            \item \textbf{Definition}: The marginal probability $P(X)$ of a variable $X$ is obtained by summing over the probabilities of all other variables:
            \[
            P(X) = \sum_{Y} P(X, Y)
            \]
            \item \textbf{Example}: To find $P(Rain)$ from the joint distribution $P(Rain, Sprinkler)$, sum over all values of $Sprinkler$:
            \[
            P(Rain = true) = P(Rain = true, Sprinkler = true) + P(Rain = true, Sprinkler = false)
            \]
        \end{itemize}
    
    \end{highlight}
    
    \subsubsection*{Conditional Probability and Independence}
    
    Conditional probabilities can be derived from the full joint distribution using the definition of conditional probability. Independence assumptions simplify the joint distribution by reducing the number 
    of probabilities that must be specified.
    
    \begin{highlight}[Conditional Probability and Independence]
    
        \begin{itemize}
            \item \textbf{Conditional Probability}: The conditional probability $P(X | Y)$ is derived from the joint distribution:
            \[
            P(X | Y) = \frac{P(X, Y)}{P(Y)}
            \]
            \item \textbf{Independence}: Two variables $X$ and $Y$ are independent if:
            \[
            P(X, Y) = P(X)P(Y)
            \]
            Independence allows for simplifying the joint distribution into the product of marginal probabilities.
        \end{itemize}
    
    \end{highlight}
    
    \subsubsection*{Inference by Enumeration}
    
    Inference by enumeration involves computing desired probabilities by summing over the appropriate entries in the full joint distribution. This method, while straightforward, can be computationally 
    expensive for large domains.
    
    \begin{highlight}[Inference by Enumeration]
    
        \begin{itemize}
            \item \textbf{Process}: To compute $P(X)$, sum over all entries in the joint distribution that are consistent with $X$:
            \[
            P(X) = \sum_{y_1, y_2, \ldots, y_n} P(X, Y_1 = y_1, Y_2 = y_2, \ldots, Y_n = y_n)
            \]
            \item \textbf{Example}: To compute $P(Rain | Sprinkler = true)$, sum over all values of $Rain$ and $Sprinkler$ consistent with $Sprinkler = true$:
            \begin{align*}
                P(Rain | Sprinkler = true) & = \frac{P(Rain = true, Sprinkler = true)}{P(Sprinkler = true)} \\
                & = \frac{P(Rain = true, Sprinkler = true)}{P(Rain = true, Sprinkler = true) + P(Rain = false, Sprinkler = true)}
            \end{align*}
        \end{itemize}
    
    \end{highlight}
    
    \begin{highlight}[Summary of Key Concepts]
    
        \begin{itemize}
            \item \textbf{Full Joint Probability Distributions}: Represent the probabilities of all possible combinations of variable values.
            \item \textbf{Marginalization}: Summing over the joint distribution to obtain marginal probabilities.
            \item \textbf{Conditional Probability}: Deriving conditional probabilities from the joint distribution.
            \item \textbf{Independence}: Simplifying joint distributions using independence assumptions.
            \item \textbf{Inference by Enumeration}: Computing probabilities by summing entries in the joint distribution, though computationally expensive for large domains.
        \end{itemize}
    
        Understanding full joint distributions and their manipulations is essential for probabilistic reasoning, forming the basis for more sophisticated inference methods in uncertain environments.
    
    \end{highlight}
\end{notes}

The next section from this chapter that is being covered this week is \textbf{Section 12.4: Independence}.

\begin{notes}{Section 12.4: Independence}
    \subsection*{Overview}

    This section discusses the concept of independence, which plays a crucial role in simplifying probabilistic models. Independence between variables allows for more efficient representation and 
    computation by reducing the complexity of joint probability distributions.
    
    \subsubsection*{Types of Independence}
    
    There are different types of independence that can be leveraged in probabilistic models: absolute (marginal) independence and conditional independence.
    
    \begin{highlight}[Types of Independence]
    
        \begin{itemize}
            \item \textbf{Absolute (Marginal) Independence}: Two variables $X$ and $Y$ are absolutely independent if:
            \[
            P(X, Y) = P(X)P(Y)
            \]
            This implies that knowing the value of one variable provides no information about the other.
            \item \textbf{Conditional Independence}: Two variables $X$ and $Y$ are conditionally independent given a third variable $Z$ if:
            \[
            P(X, Y | Z) = P(X | Z)P(Y | Z)
            \]
            This implies that given $Z$, knowing $X$ provides no additional information about $Y$ and vice versa.
        \end{itemize}
    
    \end{highlight}
    
    \subsubsection*{Benefits of Independence}
    
    Independence assumptions simplify the specification and computation of joint probability distributions, making it feasible to handle large and complex probabilistic models.
    
    \begin{highlight}[Benefits of Independence]
    
        \begin{itemize}
            \item \textbf{Simplification}: Independence reduces the number of parameters needed to specify a joint distribution. For $n$ variables, the full joint distribution requires $2^n$ parameters, 
            but with independence assumptions, this number can be significantly reduced.
            \item \textbf{Efficient Computation}: Independence allows for decomposing joint probability distributions into products of smaller, marginal or conditional distributions, enabling more efficient 
            computation.
            \item \textbf{Example}: In a medical diagnosis model, symptoms might be conditionally independent given the disease. This reduces the complexity of the model and makes inference more tractable.
        \end{itemize}
    
    \end{highlight}
    
    \subsubsection*{Graphical Models and Independence}
    
    Graphical models, such as Bayesian networks, leverage independence assumptions to represent complex joint distributions compactly and efficiently.
    
    \begin{highlight}[Graphical Models and Independence]
    
        \begin{itemize}
            \item \textbf{Bayesian Networks}: Use directed acyclic graphs (DAGs) to represent variables and their conditional dependencies. Each node represents a variable, and edges represent direct 
            dependencies.
            \item \textbf{Markov Assumptions}: Each variable is conditionally independent of its non-descendants given its parents in the network.
            \item \textbf{Example}: In a Bayesian network for a medical diagnosis, nodes might represent diseases and symptoms, with edges indicating which symptoms are directly influenced by which diseases.
        \end{itemize}
    
    \end{highlight}
    
    \begin{highlight}[Summary of Key Concepts]
    
        \begin{itemize}
            \item \textbf{Absolute (Marginal) Independence}: Simplifies joint distributions by assuming no direct influence between variables.
            \item \textbf{Conditional Independence}: Further simplifies models by assuming independence given some other variables.
            \item \textbf{Benefits of Independence}: Reduces the number of parameters and computational complexity in probabilistic models.
            \item \textbf{Graphical Models}: Leverage independence assumptions to efficiently represent and compute joint distributions.
        \end{itemize}
    
        Understanding and utilizing independence is crucial for building scalable and efficient probabilistic models, enabling more effective reasoning under uncertainty.
    
    \end{highlight}
\end{notes}

The next section from this chapter that is being covered this week is \textbf{Section 12.5: Bayes' Rule And Its Use}.

\begin{notes}{Section 12.5: Bayes' Rule And Its Use}
    \subsection*{Overview}

    This section introduces Bayes' Rule, a fundamental theorem in probability theory that allows for updating probabilities based on new evidence. Bayes' Rule is crucial for probabilistic reasoning and 
    is extensively used in various AI applications, such as diagnostics, decision-making, and machine learning.
    
    \subsubsection*{Bayes' Rule}
    
    Bayes' Rule provides a way to update the probability of a hypothesis $H$ given new evidence $E$. It relates the posterior probability $P(H | E)$ to the prior probability $P(H)$, the likelihood 
    $P(E | H)$, and the marginal likelihood $P(E)$.
    
    \begin{highlight}[Bayes' Rule]
    
        \begin{itemize}
            \item \textbf{Formula}: Bayes' Rule is expressed as:
            \[
            P(H | E) = \frac{P(E | H) P(H)}{P(E)}
            \]
            where:
            \begin{itemize}
                \item $P(H | E)$ is the posterior probability of the hypothesis $H$ given evidence $E$.
                \item $P(E | H)$ is the likelihood of the evidence $E$ given that the hypothesis $H$ is true.
                \item $P(H)$ is the prior probability of the hypothesis $H$.
                \item $P(E)$ is the marginal likelihood of the evidence $E$, calculated as:
                \[
                P(E) = \sum_{i} P(E | H_i) P(H_i)
                \]
                summing over all possible hypotheses.
            \end{itemize}
            \item \textbf{Example}: For medical diagnosis, let $H$ be a disease and $E$ be a symptom. Bayes' Rule updates the probability of the disease given the symptom.
        \end{itemize}
    
    \end{highlight}
    
    \subsubsection*{Applications of Bayes' Rule}
    
    Bayes' Rule is applied in numerous fields within AI, providing a systematic way to update beliefs and make decisions based on evidence.
    
    \begin{highlight}[Applications of Bayes' Rule]
    
        \begin{itemize}
            \item \textbf{Medical Diagnosis}: Updating the probability of a disease based on observed symptoms and test results.
            \item \textbf{Spam Filtering}: Classifying emails as spam or not spam based on the presence of certain words.
            \item \textbf{Machine Learning}: Bayesian inference methods use Bayes' Rule for updating model parameters based on data.
            \item \textbf{Robotics}: Updating the robot's belief about its location based on sensor readings.
        \end{itemize}
    
    \end{highlight}
    
    \subsubsection*{Deriving Bayes' Rule}
    
    Bayes' Rule can be derived from the definition of conditional probability and the product rule of probability.
    
    \begin{highlight}[Deriving Bayes' Rule]
    
        \begin{itemize}
            \item \textbf{Conditional Probability}: The definition of conditional probability is:
            \[
            P(H | E) = \frac{P(H \cap E)}{P(E)} \quad \text{and} \quad P(E | H) = \frac{P(H \cap E)}{P(H)}
            \]
            \item \textbf{Product Rule}: Rearranging the product rule gives:
            \[
            P(H \cap E) = P(E | H) P(H) = P(H | E) P(E)
            \]
            \item \textbf{Bayes' Rule}: Solving for $P(H | E)$ gives:
            \[
            P(H | E) = \frac{P(E | H) P(H)}{P(E)}
            \]
        \end{itemize}
    
    \end{highlight}
    
    \subsubsection*{Normalization}
    
    The denominator $P(E)$ in Bayes' Rule ensures that the posterior probabilities sum to 1, making it a normalization factor.
    
    \begin{highlight}[Normalization]
    
        \begin{itemize}
            \item \textbf{Calculating $P(E)$}: The marginal likelihood $P(E)$ is computed by summing the joint probabilities over all hypotheses:
            \[
            P(E) = \sum_{i} P(E | H_i) P(H_i)
            \]
            \item \textbf{Normalization Factor}: Ensures that the posterior probabilities are properly scaled:
            \[
            P(H | E) = \frac{P(E | H) P(H)}{\sum_{i} P(E | H_i) P(H_i)}
            \]
        \end{itemize}
    
    \end{highlight}
    
    \begin{highlight}[Summary of Key Concepts]
    
        \begin{itemize}
            \item \textbf{Bayes' Rule}: Provides a method for updating probabilities based on new evidence.
            \item \textbf{Formula}: $P(H | E) = \frac{P(E | H) P(H)}{P(E)}$.
            \item \textbf{Applications}: Used in medical diagnosis, spam filtering, machine learning, and robotics.
            \item \textbf{Derivation}: Derived from the definition of conditional probability and the product rule.
            \item \textbf{Normalization}: Ensures that posterior probabilities sum to 1, calculated using the marginal likelihood.
        \end{itemize}
    
        Understanding Bayes' Rule is essential for probabilistic reasoning and decision-making in uncertain environments, enabling systematic updates to beliefs based on observed evidence.
    
    \end{highlight}
\end{notes}

The next section from this chapter that is being covered this week is \textbf{Section 12.6: Naive Bayes Models}.

\begin{notes}{Section 12.6: Naive Bayes Models}
    \subsection*{Overview}

    This section introduces Naive Bayes models, a simple yet powerful probabilistic classifier based on Bayes' Rule. The "naive" assumption is that features are conditionally independent given the class label. 
    Despite this simplification, Naive Bayes models perform well in various real-world applications, particularly in text classification and spam filtering.
    
    \subsubsection*{The Naive Bayes Assumption}
    
    The Naive Bayes assumption simplifies the computation of the joint probability of the features given the class by assuming conditional independence among the features.
    
    \begin{highlight}[The Naive Bayes Assumption]
    
        \begin{itemize}
            \item \textbf{Conditional Independence}: Given a class $C$, the features $X_1, X_2, \ldots, X_n$ are conditionally independent:
            \[
            P(X_1, X_2, \ldots, X_n | C) = \prod_{i=1}^n P(X_i | C)
            \]
            \item \textbf{Simplification}: This assumption reduces the complexity of calculating the joint probability from exponential to linear in the number of features.
            \item \textbf{Example}: In spam filtering, words in an email are considered independent given the email is spam or not spam.
        \end{itemize}
    
    \end{highlight}
    
    \subsubsection*{Classification with Naive Bayes}
    
    Naive Bayes classifiers use Bayes' Rule and the Naive Bayes assumption to compute the posterior probability of each class given the observed features and to classify new instances.
    
    \begin{highlight}[Classification with Naive Bayes]
    
        \begin{itemize}
            \item \textbf{Posterior Probability}: For a given instance with features $X_1, X_2, \ldots, X_n$, the posterior probability of class $C_k$ is:
            \[
            P(C_k | X_1, X_2, \ldots, X_n) = \frac{P(C_k) \prod_{i=1}^n P(X_i | C_k)}{P(X_1, X_2, \ldots, X_n)}
            \]
            \item \textbf{Classification Rule}: The instance is classified into the class with the highest posterior probability:
            \[
            \hat{C} = \arg\max_{C_k} P(C_k) \prod_{i=1}^n P(X_i | C_k)
            \]
            \item \textbf{Example}: In spam filtering, classify an email as spam if $P(\text{spam} | \text{words}) > P(\text{not spam} | \text{words})$.
        \end{itemize}
    
    \end{highlight}
    
    \subsubsection*{Parameter Estimation}
    
    The parameters of a Naive Bayes model, namely the prior probabilities of the classes and the conditional probabilities of the features given the classes, can be estimated from training data.
    
    \begin{highlight}[Parameter Estimation]
    
        \begin{itemize}
            \item \textbf{Class Priors}: Estimated as the relative frequencies of the classes in the training set:
            \[
            P(C_k) = \frac{N(C_k)}{N}
            \]
            where $N(C_k)$ is the number of instances of class $C_k$ and $N$ is the total number of instances.
            \item \textbf{Conditional Probabilities}: Estimated as the relative frequencies of the feature values given the class:
            \[
            P(X_i = x_i | C_k) = \frac{N(X_i = x_i, C_k)}{N(C_k)}
            \]
            where $N(X_i = x_i, C_k)$ is the number of instances where feature $X_i$ takes value $x_i$ in class $C_k$.
        \end{itemize}
    
    \end{highlight}
    
    \subsubsection*{Applications of Naive Bayes}
    
    Naive Bayes models are widely used in various domains due to their simplicity, efficiency, and effectiveness in classification tasks.
    
    \begin{highlight}[Applications of Naive Bayes]
    
        \begin{itemize}
            \item \textbf{Text Classification}: Categorizing documents or emails into predefined categories based on word frequencies.
            \item \textbf{Spam Filtering}: Classifying emails as spam or not spam using the presence of certain keywords.
            \item \textbf{Sentiment Analysis}: Determining the sentiment of a text (positive, negative, neutral) based on word usage.
            \item \textbf{Medical Diagnosis}: Diagnosing diseases based on symptoms and patient history.
        \end{itemize}
    
    \end{highlight}
    
    \begin{highlight}[Summary of Key Concepts]
    
        \begin{itemize}
            \item \textbf{Naive Bayes Assumption}: Features are conditionally independent given the class.
            \item \textbf{Classification}: Uses Bayes' Rule to compute posterior probabilities and classify instances into the most probable class.
            \item \textbf{Parameter Estimation}: Class priors and conditional probabilities are estimated from training data.
            \item \textbf{Applications}: Widely used in text classification, spam filtering, sentiment analysis, and medical diagnosis.
        \end{itemize}
    
        Understanding Naive Bayes models provides a foundation for applying probabilistic reasoning to practical classification problems, leveraging simplicity and efficiency to achieve effective results.
    
    \end{highlight}
\end{notes}

The last section from this chapter that is being covered this week is \textbf{Section 12.7: The Wumpus World Revisited}.

\begin{notes}{Section 12.7: The Wumpus World Revisited}
    \subsection*{Overview}

    This section revisits the Wumpus World, a classic AI problem, to illustrate how probabilistic reasoning can be applied to complex, uncertain environments. The Wumpus World is a grid-based environment 
    where an agent must navigate to find gold while avoiding pits and the Wumpus, a dangerous creature. This section demonstrates how to use probability to handle the inherent uncertainty in the Wumpus World.
    
    \subsubsection*{The Wumpus World Environment}
    
    The Wumpus World consists of a grid of squares, some of which contain pits, and one contains the Wumpus. The agent must find the gold without falling into a pit or encountering the Wumpus.
    
    \begin{highlight}[The Wumpus World Environment]
    
        \begin{itemize}
            \item \textbf{Grid Layout}: The environment is a 4x4 grid with each square potentially containing a pit, the Wumpus, or being empty.
            \item \textbf{Percepts}: The agent receives percepts indicating nearby dangers:
                \begin{itemize}
                    \item \textbf{Breeze}: Indicates a pit in an adjacent square.
                    \item \textbf{Stench}: Indicates the Wumpus is in an adjacent square.
                    \item \textbf{Glitter}: Indicates the gold is in the current square.
                \end{itemize}
            \item \textbf{Actions}: The agent can move forward, turn left, turn right, grab the gold, or shoot an arrow to kill the Wumpus.
        \end{itemize}
    
    \end{highlight}
    
    \subsubsection*{Using Probabilities in the Wumpus World}
    
    Probabilistic reasoning is used to handle the uncertainty in the agent's knowledge about the Wumpus World's state, helping the agent to make informed decisions.
    
    \begin{highlight}[Using Probabilities in the Wumpus World]
    
        \begin{itemize}
            \item \textbf{Belief State}: Represents the agent's knowledge about the environment, updated using Bayesian inference based on percepts.
            \item \textbf{Probability Distribution}: The agent maintains a probability distribution over possible states of the world, reflecting the likelihood of pits, the Wumpus, and the gold being 
            in each square.
            \item \textbf{Example}: If the agent perceives a breeze, it updates the probability of pits in adjacent squares.
            \item \textbf{Update Rule}: Bayes' Rule is used to update the belief state:
            \[
            P(X | E) = \frac{P(E | X) P(X)}{P(E)}
            \]
            where $X$ represents the state of the world and $E$ represents the percepts.
        \end{itemize}
    
    \end{highlight}
    
    \subsubsection*{Inference and Decision Making}
    
    The agent uses its probabilistic model to infer the most likely state of the world and make decisions that maximize its chances of success.
    
    \begin{highlight}[Inference and Decision Making]
    
        \begin{itemize}
            \item \textbf{Risk Assessment}: The agent assesses the risks associated with different actions by considering the probabilities of encountering pits or the Wumpus.
            \item \textbf{Action Selection}: Actions are chosen to maximize expected utility, considering both the likelihood of success and potential dangers.
            \item \textbf{Example}: If the agent perceives a stench but no breeze, it might infer that moving into a square with a high probability of containing the Wumpus is too risky and choose an 
            alternative path.
        \end{itemize}
    
    \end{highlight}
    
    \subsubsection*{Performance in the Wumpus World}
    
    The probabilistic approach allows the agent to make more informed decisions, improving its performance in navigating the Wumpus World.
    
    \begin{highlight}[Performance in the Wumpus World]
    
        \begin{itemize}
            \item \textbf{Improved Decision Making}: The agent uses probabilities to evaluate the safety and potential rewards of different actions, leading to better overall performance.
            \item \textbf{Flexibility}: The probabilistic model can handle new percepts and update the belief state dynamically, allowing the agent to adapt to changing information.
            \item \textbf{Example}: The agent successfully finds the gold while avoiding pits and the Wumpus by continually updating its belief state and making decisions based on the most likely scenarios.
        \end{itemize}
    
    \end{highlight}
    
    \begin{highlight}[Summary of Key Concepts]
    
        \begin{itemize}
            \item \textbf{Wumpus World Environment}: A grid-based environment with pits, the Wumpus, and gold.
            \item \textbf{Probabilistic Reasoning}: Using probabilities to handle uncertainty and update the belief state based on percepts.
            \item \textbf{Inference and Decision Making}: Making decisions that maximize expected utility by assessing risks and updating beliefs.
            \item \textbf{Performance}: Improved decision-making and flexibility in adapting to new information.
        \end{itemize}
    
        Revisiting the Wumpus World with probabilistic reasoning highlights the power of probability in handling uncertainty and making rational decisions in complex environments.
    
    \end{highlight}
\end{notes}

The next chapter that is being covered this week is \textbf{Chapter 22: Reinforcement Learning}. The section that is being covered from this chapter this week is \textbf{Section 22.4: Generalization In Reinforcement Learning}.

\begin{notes}{Section 22.4: Generalization In Reinforcement Learning}
    \subsection*{Overview}

    This section addresses generalization in reinforcement learning, highlighting the limitations of tabular representations of utility and Q-functions in large state spaces. It introduces function 
    approximation techniques, which allow reinforcement learning algorithms to generalize from observed states to unvisited states, thereby improving learning efficiency and performance in complex environments.
    
    \subsubsection*{Function Approximation}
    
    Function approximation methods construct compact representations of utility or Q-functions, enabling efficient learning in large state spaces. This is essential for practical applications where the 
    number of states is vast.
    
    \begin{highlight}[Function Approximation]
    
        \begin{itemize}
            \item \textbf{Utility Function Approximation}: Approximates the utility function $U(s)$ using a parameterized function $\hat{U}_\theta(s)$, such as a weighted linear combination of features:
            \[
            \hat{U}_\theta(s) = \theta_0 + \theta_1 f_1(s) + \theta_2 f_2(s) + \ldots + \theta_n f_n(s)
            \]
            \item \textbf{Example}: In a 4x3 grid world, the utility function can be approximated using features like the coordinates of the grid squares.
            \item \textbf{Gradient Descent}: Parameters $\theta$ are updated using gradient descent to minimize the error between the predicted and actual utilities.
            \item \textbf{Widrow-Hoff Rule}: Also known as the delta rule, used for online least-squares updates:
            \[
            \theta_i \leftarrow \theta_i + \alpha [u_j(s) - \hat{U}_\theta(s)] \frac{\partial \hat{U}_\theta(s)}{\partial \theta_i}
            \]
        \end{itemize}
    
    \end{highlight}
    
    \subsubsection*{Approximating Direct Utility Estimation}
    
    Direct utility estimation generates trajectories and uses the sum of rewards as training examples for supervised learning algorithms, enabling approximation of utility functions for large state spaces.
    
    \begin{highlight}[Approximating Direct Utility Estimation]
    
        \begin{itemize}
            \item \textbf{Linear Function Approximator}: Uses features such as coordinates in a grid world to approximate utility:
            \[
            \hat{U}_\theta(x, y) = \theta_0 + \theta_1 x + \theta_2 y
            \]
            \item \textbf{Online Learning}: Parameters are updated after each trial based on observed rewards, using gradient descent to reduce prediction error.
            \item \textbf{Example}: If the total reward obtained from state $(1,1)$ is 0.4 and the current utility estimate is 0.8, the parameters are adjusted to reduce the utility estimate.
        \end{itemize}
    
    \end{highlight}
    
    \subsubsection*{Approximating Temporal-Difference Learning}
    
    Temporal-Difference (TD) learning methods can also be adapted to use function approximation, adjusting parameters to minimize temporal differences between successive states.
    
    \begin{highlight}[Approximating Temporal-Difference Learning]
    
        \begin{itemize}
            \item \textbf{TD Learning Update Rule}: Parameters are updated to reduce the temporal difference:
            \[
            \theta_i \leftarrow \theta_i + \alpha [R(s, a, s') + \gamma \hat{U}_\theta(s') - \hat{U}_\theta(s)] \frac{\partial \hat{U}_\theta(s)}{\partial \theta_i}
            \]
            \item \textbf{Q-learning Update Rule}: Similar update rule for Q-learning with function approximation:
            \[
            \theta_i \leftarrow \theta_i + \alpha [R(s, a, s') + \gamma \max_{a'} \hat{Q}_\theta(s', a') - \hat{Q}_\theta(s, a)] \frac{\partial \hat{Q}_\theta(s, a)}{\partial \theta_i}
            \]
            \item \textbf{Challenges}: With nonlinear function approximators like neural networks, parameters may diverge, leading to instability.
        \end{itemize}
    
    \end{highlight}
    
    \subsubsection*{Experience Replay}
    
    Experience replay addresses catastrophic forgetting, where the agent may forget important experiences as it learns, by reusing past experiences to reinforce learning.
    
    \begin{highlight}[Experience Replay]
    
        \begin{itemize}
            \item \textbf{Mechanism}: The agent stores trajectories and periodically replays them to ensure the value function remains accurate for all visited states.
            \item \textbf{Benefit}: Prevents the agent from forgetting important past experiences and maintains stability in learning.
            \item \textbf{Example}: A self-driving car replays experiences where it narrowly avoided obstacles to reinforce safe driving behaviors.
        \end{itemize}
    
    \end{highlight}
    
    \subsubsection*{Deep Reinforcement Learning}
    
    Deep reinforcement learning combines deep neural networks with reinforcement learning algorithms to handle high-dimensional state spaces and automatically extract useful features.
    
    \begin{highlight}[Deep Reinforcement Learning]
    
        \begin{itemize}
            \item \textbf{Deep Neural Networks}: Used as function approximators to learn complex state representations from raw inputs.
            \item \textbf{Backpropagation}: Gradients required for parameter updates are computed using backpropagation.
            \item \textbf{Applications}: Achieved significant results in playing video games, Go, and robotic tasks.
            \item \textbf{Challenges}: Deep RL systems can behave unpredictably in environments that differ from training data.
        \end{itemize}
    
    \end{highlight}
    
    \subsubsection*{Reward Shaping}
    
    Reward shaping involves providing additional rewards, or pseudorewards, to guide the agent's learning process, addressing the credit assignment problem in environments with sparse rewards.
    
    \begin{highlight}[Reward Shaping]
    
        \begin{itemize}
            \item \textbf{Pseudorewards}: Additional rewards for intermediate progress towards the goal, speeding up learning.
            \item \textbf{Risk}: Agent might learn to maximize pseudorewards rather than true rewards.
            \item \textbf{Example}: A soccer robot receives pseudorewards for making contact with the ball or advancing it towards the goal.
        \end{itemize}
    
    \end{highlight}
    
    \subsubsection*{Hierarchical Reinforcement Learning}
    
    Hierarchical reinforcement learning (HRL) breaks down complex tasks into smaller sub-tasks, simplifying learning by decomposing the overall task hierarchy.
    
    \begin{highlight}[Hierarchical Reinforcement Learning]
    
        \begin{itemize}
            \item \textbf{Hierarchical Structure}: Tasks are decomposed into sub-tasks, each with its own policies and value functions.
            \item \textbf{Joint State Space}: Combines physical state and machine state (program counter, arguments, variables) to form a joint state space.
            \item \textbf{Example}: In a soccer game, higher-level actions like passing and shooting are broken down into lower-level motor behaviors.
            \item \textbf{Benefit}: Learning becomes more efficient by focusing on smaller sub-problems within the task hierarchy.
        \end{itemize}
    
    \end{highlight}
    
    \begin{highlight}[Summary of Key Concepts]
    
        \begin{itemize}
            \item \textbf{Function Approximation}: Enables efficient learning by representing utility and Q-functions compactly.
            \item \textbf{Experience Replay}: Prevents forgetting by reusing past experiences.
            \item \textbf{Deep Reinforcement Learning}: Combines deep neural networks with RL for high-dimensional state spaces.
            \item \textbf{Reward Shaping}: Uses pseudorewards to guide learning in sparse reward environments.
            \item \textbf{Hierarchical Reinforcement Learning}: Decomposes tasks into sub-tasks for more efficient learning.
        \end{itemize}
    
        Generalization in reinforcement learning is crucial for handling large state spaces, enabling agents to learn efficiently and perform effectively in complex environments.
    
    \end{highlight}
\end{notes}