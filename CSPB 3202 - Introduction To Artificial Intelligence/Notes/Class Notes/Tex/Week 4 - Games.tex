\clearpage

\renewcommand{\ChapTitle}{Games}
\renewcommand{\SectionTitle}{Games}

\chapter{\ChapTitle}

\section{\SectionTitle}
\horizontalline{0}{0}

\subsection{Assigned Reading}

The reading for this week is from, \AITextbook \hspace*{1pt} and \RLTextbook.

\begin{itemize}
    \item \textbf{Artificial Intelligence - A Modern Approach - Chapter 5.1 - Game Theory}
    \item \textbf{Artificial Intelligence - A Modern Approach - Chapter 5.2 - Optimal Decisions In Games}
    \item \textbf{Artificial Intelligence - A Modern Approach - Chapter 5.3 - Heuristic Alpha-Beta Tree Search}
    \item \textbf{Artificial Intelligence - A Modern Approach - Chapter 5.4 - Monte Carlo Tree Search}
    \item \textbf{Artificial Intelligence - A Modern Approach - Chapter 5.5 - Partially Observable Games}
\end{itemize}

\subsection{Piazza}

Must post at least \textbf{three} times this week to Piazza.

\subsection{Lectures}

The lectures for this week are:

\begin{itemize}
    \item \lecture{https://www.youtube.com/watch?v=XvGNnxQ4jkM}{Advesarial Search Intro, Minimax Algorith}{27}
    \item \lecture{https://www.youtube.com/watch?v=vOFAPSts6Hk}{Improving Minimax: Alpha-Beta Pruning And Depth-Limited Approach}{22}
    \item \lecture{https://www.youtube.com/watch?v=ZQIYTRwjl9g}{Expectimax Search}{28}
\end{itemize}

\noindent The lecture notes for this week are:

\begin{itemize}
    \item \pdflink{\LecNoteDir/Notes/Advesarial Search Intro, Minimax Algorith Lecture Notes.pdf}{Advesarial Search Intro, Minimax Algorith Lecture Notes}
    \item \pdflink{\LecNoteDir/Notes/Expectimax Search Lecture Notes.pdf}{Expectimax Search Lecture Notes}
    \item \pdflink{\LecNoteDir/Notes/Improving Minimax - Alpha-Beta Pruning And Depth-Limited Approach Lecture Notes.pdf}{Improving Minimax - Alpha-Beta Pruning And Depth-Limited Approach Lecture Notes}
\end{itemize}

\subsection{Assignments}

The assignment(s) for this week are:

\begin{itemize}
    \item \pdflink{\AssDir/Assignment 4 - Games/Assignment 4 - Games.pdf}{Assignment 4 - Games}
\end{itemize}

\subsection{Quiz}

The quiz for this week is:

\begin{itemize}
    \item \pdflink{\QuizDir/Quiz 4 - Games.pdf}{Quiz 4 - Games}
\end{itemize}

\subsection{Chapter Summary}

The chapter that is being covered this week is \textbf{Chapter 5: Adversarial Search And Games}. The first section that is being covered from this chapter this week is \textbf{Section 5.1: Game Theory}.

\begin{notes}{Section 5.1: Game Theory}
    \subsection*{Overview}

    Key topics include two-player zero-sum games, game trees, minimax search, and heuristic evaluation functions. These concepts are crucial for understanding how to model and solve competitive environments 
    where agents have conflicting goals.
    
    \subsubsection*{Two-Player Zero-Sum Games}
    
    Two-player zero-sum games are a fundamental concept in game theory where two players take turns making moves, and one player's gain is the other player's loss. Examples include chess, Go, and tic-tac-toe.
    
    \begin{highlight}[Two-Player Zero-Sum Games]
        In two-player zero-sum games, each player's goal is to maximize their own payoff while minimizing their opponent's payoff.
        
        \begin{itemize}
            \item \textbf{Perfect Information}: Both players have complete knowledge of the game state at all times.
            \item \textbf{Zero-Sum}: The total payoff to all players is zero, meaning one player's gain is exactly the other player's loss.
            \item \textbf{Game Representation}: The game can be represented using states, actions, results, terminal tests, and utility functions.
        \end{itemize}
    \end{highlight}
    
    \subsubsection*{Game Trees}
    
    A game tree is a theoretical construct that represents all possible moves in a game, starting from the initial state and expanding through all possible sequences of moves to terminal states.
    
    \begin{highlight}[Game Trees]
        Game trees illustrate the decision-making process in adversarial games by representing all possible game states and actions.
        
        \begin{itemize}
            \item \textbf{Initial State}: The configuration of the game at the beginning.
            \item \textbf{Actions}: The set of legal moves available to the player whose turn it is to move.
            \item \textbf{Result}: The state resulting from a specific action taken in a given state.
            \item \textbf{Terminal Test}: A condition that determines whether the game has ended.
            \item \textbf{Utility Function}: A function that assigns a numeric value to each terminal state, indicating the payoff for the players.
        \end{itemize}
    \end{highlight}
    
    \subsubsection*{Minimax Search}
    
    Minimax search is an algorithm used to determine the optimal strategy in two-player zero-sum games by assuming both players play optimally. It recursively evaluates the minimax value of each state in the game tree.
    
    \begin{highlight}[Minimax Search]
        Minimax search computes the best move by evaluating the minimax values of states, assuming optimal play by both players.
        
        \begin{itemize}
            \item \textbf{Minimax Value}: The utility of a state assuming both players play optimally.
            \item \textbf{MAX Player}: Seeks to maximize the minimax value.
            \item \textbf{MIN Player}: Seeks to minimize the minimax value.
            \item \textbf{Ply}: One level of moves by each player.
        \end{itemize}
    \end{highlight}
    
    \subsubsection*{Heuristic Evaluation Functions}
    
    Heuristic evaluation functions estimate the value of a game state when the game tree is too large to be fully explored. These functions approximate the minimax value based on features of the state.
    
    \begin{highlight}[Heuristic Evaluation Functions]
        Heuristic evaluation functions provide an estimate of the game state's value, allowing for effective decision-making without exhaustive search.
        
        \begin{itemize}
            \item \textbf{State Features}: Characteristics of the game state used to compute the heuristic value.
            \item \textbf{Approximation}: Provides an estimate of the true minimax value.
            \item \textbf{Efficiency}: Enables decision-making in complex games where full search is impractical.
        \end{itemize}
    \end{highlight}
    
    \subsubsection*{Key Concepts}
    
    \begin{highlight}[Key Concepts]
        This section summarizes the key concepts related to game theory and adversarial search, emphasizing their definitions, properties, and applications in AI.
        
        \begin{itemize}
            \item \textbf{Two-Player Zero-Sum Games}
                \begin{itemize}
                    \item \textbf{Perfect Information}: Complete knowledge of the game state.
                    \item \textbf{Zero-Sum}: One player's gain is the other player's loss.
                    \item \textbf{Game Representation}: States, actions, results, terminal tests, and utility functions.
                \end{itemize}
            \item \textbf{Game Trees}
                \begin{itemize}
                    \item \textbf{Initial State}: Configuration at the game's start.
                    \item \textbf{Actions}: Legal moves available to players.
                    \item \textbf{Result}: Outcome of a specific action.
                    \item \textbf{Terminal Test}: Determines if the game has ended.
                    \item \textbf{Utility Function}: Numeric value assigned to terminal states.
                \end{itemize}
            \item \textbf{Minimax Search}
                \begin{itemize}
                    \item \textbf{Minimax Value}: Utility assuming optimal play.
                    \item \textbf{MAX Player}: Maximizes the minimax value.
                    \item \textbf{MIN Player}: Minimizes the minimax value.
                    \item \textbf{Ply}: One level of moves by each player.
                \end{itemize}
            \item \textbf{Heuristic Evaluation Functions}
                \begin{itemize}
                    \item \textbf{State Features}: Characteristics used to compute heuristic value.
                    \item \textbf{Approximation}: Estimate of the true minimax value.
                    \item \textbf{Efficiency}: Facilitates decision-making in complex games.
                \end{itemize}
        \end{itemize}
    \end{highlight}
\end{notes}

The next section that is being covered from this chapter this week is \textbf{Section 5.2: Optimal Decisions In Games}.

\begin{notes}{Section 5.2: Optimal Decisions In Games}    
    \subsection*{Overview}
    
    Key topics include the minimax search algorithm, alpha-beta pruning, and optimal decisions in multiplayer games. These concepts are crucial for understanding how to model and solve competitive 
    environments where agents have conflicting goals.
    
    \subsubsection*{Minimax Search Algorithm}
    
    The minimax search algorithm is used to determine the optimal strategy in two-player zero-sum games by assuming both players play optimally. It evaluates the minimax value of each state in the game tree.
    
    \begin{highlight}[Minimax Search Algorithm]
        Minimax search computes the best move by evaluating the minimax values of states, assuming optimal play by both players.
        
        \begin{itemize}
            \item \textbf{Minimax Value}: The utility of a state assuming both players play optimally.
            \item \textbf{MAX Player}: Seeks to maximize the minimax value.
            \item \textbf{MIN Player}: Seeks to minimize the minimax value.
            \item \textbf{Ply}: One level of moves by each player.
            \item \textbf{Algorithm}: Recursively explores the game tree and backs up values to determine the optimal move.
        \end{itemize}
    \end{highlight}
    
    \subsubsection*{Alpha-Beta Pruning}
    
    Alpha-beta pruning is a technique used to improve the efficiency of the minimax algorithm by eliminating branches in the game tree that do not influence the final decision.
    
    \begin{highlight}[Alpha-Beta Pruning]
        Alpha-beta pruning reduces the number of nodes evaluated in the game tree by pruning branches that cannot affect the final decision.
        
        \begin{itemize}
            \item \textbf{Pruning Condition}: Stops evaluation of a move when at least one possibility has been found that proves the move to be worse than a previously examined move.
            \item \textbf{Alpha Value}: The best value that the maximizer currently can guarantee at that level or above.
            \item \textbf{Beta Value}: The best value that the minimizer currently can guarantee at that level or above.
            \item \textbf{Efficiency}: Reduces the effective branching factor, allowing deeper search in the same amount of time.
        \end{itemize}
    \end{highlight}
    
    \subsubsection*{Optimal Decisions in Multiplayer Games}
    
    Extending the minimax approach to multiplayer games involves considering a vector of utilities for each player and recognizing that optimal strategies may involve forming and breaking alliances.
    
    \begin{highlight}[Optimal Decisions in Multiplayer Games]
        Multiplayer games require a more complex strategy as they involve multiple agents with potentially conflicting goals.
        
        \begin{itemize}
            \item \textbf{Utility Vector}: Represents the utility of a state for each player.
            \item \textbf{Alliances}: Players may form alliances to improve their individual outcomes, although these alliances may be temporary.
            \item \textbf{Non-Zero-Sum}: In non-zero-sum games, collaboration can occur to achieve mutually beneficial outcomes.
        \end{itemize}
    \end{highlight}
    
    \subsubsection*{Move Ordering}
    
    The effectiveness of alpha-beta pruning is highly dependent on the order in which nodes are examined. Good move ordering can significantly enhance the performance of the search.
    
    \begin{highlight}[Move Ordering]
        Effective move ordering improves the performance of alpha-beta pruning by exploring the most promising moves first.
        
        \begin{itemize}
            \item \textbf{Killer Moves}: Moves that have been successful in the past are tried first.
            \item \textbf{Dynamic Ordering}: Uses iterative deepening to refine move ordering based on earlier searches.
            \item \textbf{Transposition Tables}: Cache heuristic values of previously evaluated states to avoid redundant calculations.
        \end{itemize}
    \end{highlight}
    
    \subsubsection*{Key Concepts}
    
    \begin{highlight}[Key Concepts]
        This section summarizes the key concepts related to making optimal decisions in games, emphasizing their definitions, properties, and applications in AI.
        
        \begin{itemize}
            \item \textbf{Minimax Search Algorithm}
                \begin{itemize}
                    \item \textbf{Minimax Value}: Utility assuming optimal play.
                    \item \textbf{MAX Player}: Maximizes the minimax value.
                    \item \textbf{MIN Player}: Minimizes the minimax value.
                    \item \textbf{Ply}: One level of moves by each player.
                    \item \textbf{Algorithm}: Recursively explores the game tree.
                \end{itemize}
            \item \textbf{Alpha-Beta Pruning}
                \begin{itemize}
                    \item \textbf{Pruning Condition}: Stops evaluation when a move is proven worse.
                    \item \textbf{Alpha Value}: Best value maximizer can guarantee.
                    \item \textbf{Beta Value}: Best value minimizer can guarantee.
                    \item \textbf{Efficiency}: Reduces effective branching factor.
                \end{itemize}
            \item \textbf{Optimal Decisions in Multiplayer Games}
                \begin{itemize}
                    \item \textbf{Utility Vector}: Represents utility for each player.
                    \item \textbf{Alliances}: Temporary partnerships to improve outcomes.
                    \item \textbf{Non-Zero-Sum}: Collaboration for mutual benefit.
                \end{itemize}
            \item \textbf{Move Ordering}
                \begin{itemize}
                    \item \textbf{Killer Moves}: Successful moves tried first.
                    \item \textbf{Dynamic Ordering}: Refines move order through iterative deepening.
                    \item \textbf{Transposition Tables}: Cache heuristic values to avoid redundant calculations.
                \end{itemize}
        \end{itemize}
    \end{highlight}
\end{notes}

The next section that is being covered from this chapter this week is \textbf{Section 5.3: Heuristic Alpha-Beta Tree Search}.

\begin{notes}{Section 5.3: Heuristic Alpha-Beta Tree Search}    
    \subsection*{Overview}
    
    Key topics include heuristic evaluation functions, cutting off search, quiescence search, the horizon effect, forward pruning, and the use of lookup tables. These concepts are crucial for optimizing 
    search strategies in complex games and making efficient use of computational resources.
    
    \subsubsection*{Heuristic Evaluation Functions}
    
    Heuristic evaluation functions estimate the utility of a game state when the search tree is too large to fully explore. These functions approximate the minimax value based on features of the state.
    
    \begin{highlight}[Heuristic Evaluation Functions]
        Heuristic evaluation functions provide an estimate of a game state's value, allowing effective decision-making without exhaustive search.
        
        \begin{itemize}
            \item \textbf{EVAL Function}: Replaces the UTILITY function for nonterminal nodes.
            \item \textbf{Features}: State characteristics (e.g., number of pawns, knights) used to compute the evaluation.
            \item \textbf{Weighted Linear Function}: Combines feature values with weights to estimate utility.
            \item \textbf{Correlation}: Should be strongly correlated with the actual chances of winning.
        \end{itemize}
    \end{highlight}
    
    \subsubsection*{Cutting Off Search}
    
    Cutting off search involves stopping the search at a certain depth or when a certain condition is met, and applying the heuristic evaluation function to estimate the state's value.
    
    \begin{highlight}[Cutting Off Search]
        The search is cut off early to save computation time, and the evaluation function is applied to nonterminal states.
        
        \begin{itemize}
            \item \textbf{Cutoff Test}: Determines when to stop the search based on depth or state properties.
            \item \textbf{Fixed Depth Limit}: A simple approach to control search amount by setting a maximum depth.
            \item \textbf{Iterative Deepening}: Searches incrementally deeper until time runs out, using results to improve move ordering.
        \end{itemize}
    \end{highlight}
    
    \subsubsection*{Quiescence Search}
    
    Quiescence search continues the search until a stable position (quiescent) is reached, avoiding misleading evaluations due to volatile positions.
    
    \begin{highlight}[Quiescence Search]
        Quiescence search extends the search to stable positions, avoiding drastic evaluation changes due to immediate threats or opportunities.
        
        \begin{itemize}
            \item \textbf{Quiescent Positions}: Stable states where no dramatic changes in evaluation are expected.
            \item \textbf{Extra Search}: Searches beyond the cutoff for non-quiescent positions to resolve uncertainties.
        \end{itemize}
    \end{highlight}
    
    \subsubsection*{Horizon Effect}
    
    The horizon effect occurs when a program cannot see beyond a certain depth, causing it to miss future significant events or threats.
    
    \begin{highlight}[Horizon Effect]
        The horizon effect arises when the search is unable to see beyond a certain depth, leading to inaccurate evaluations.
        
        \begin{itemize}
            \item \textbf{Delaying Tactics}: Moves that postpone the inevitable, pushing significant events beyond the search horizon.
            \item \textbf{Singular Extensions}: Extending search for clearly superior moves to mitigate the horizon effect.
        \end{itemize}
    \end{highlight}
    
    \subsubsection*{Forward Pruning}
    
    Forward pruning selectively prunes moves that appear to be poor, saving computation time at the risk of missing the best move.
    
    \begin{highlight}[Forward Pruning]
        Forward pruning saves computation time by pruning moves that are unlikely to be good, at the risk of making errors.
        
        \begin{itemize}
            \item \textbf{Beam Search}: Considers only a "beam" of the best moves at each ply.
            \item \textbf{PROBCUT}: Prunes moves that are probably outside the current window based on statistical estimates.
            \item \textbf{Late Move Reduction}: Reduces the search depth for moves considered less promising.
        \end{itemize}
    \end{highlight}
    
    \subsubsection*{Search vs. Lookup}
    
    Using table lookup for well-known positions in the opening and endgame stages can significantly enhance performance by relying on precomputed solutions.
    
    \begin{highlight}[Search vs. Lookup]
        Table lookup is used for known positions, especially in the opening and endgame stages, to enhance performance.
        
        \begin{itemize}
            \item \textbf{Opening Books}: Precomputed sequences of optimal moves in the opening stage.
            \item \textbf{Endgame Tables}: Complete solutions for endgames, allowing perfect play by looking up the best move.
            \item \textbf{Retrograde Minimax Search}: Constructs endgame tables by reversing the rules to determine winning moves.
        \end{itemize}
    \end{highlight}
    
    \subsubsection*{Key Concepts}
    
    \begin{highlight}[Key Concepts]
        This section summarizes the key concepts related to heuristic alpha-beta tree search, emphasizing their definitions, properties, and applications in AI.
        
        \begin{itemize}
            \item \textbf{Heuristic Evaluation Functions}
                \begin{itemize}
                    \item \textbf{EVAL Function}: Estimates utility for nonterminal nodes.
                    \item \textbf{Features}: State characteristics used in evaluation.
                    \item \textbf{Weighted Linear Function}: Combines feature values with weights.
                    \item \textbf{Correlation}: Should align with actual chances of winning.
                \end{itemize}
            \item \textbf{Cutting Off Search}
                \begin{itemize}
                    \item \textbf{Cutoff Test}: Decides when to stop search.
                    \item \textbf{Fixed Depth Limit}: Simple approach to control search depth.
                    \item \textbf{Iterative Deepening}: Incremental search for better move ordering.
                \end{itemize}
            \item \textbf{Quiescence Search}
                \begin{itemize}
                    \item \textbf{Quiescent Positions}: Stable states for reliable evaluation.
                    \item \textbf{Extra Search}: Extends search to avoid volatile evaluations.
                \end{itemize}
            \item \textbf{Horizon Effect}
                \begin{itemize}
                    \item \textbf{Delaying Tactics}: Postponing significant events.
                    \item \textbf{Singular Extensions}: Extends search for superior moves.
                \end{itemize}
            \item \textbf{Forward Pruning}
                \begin{itemize}
                    \item \textbf{Beam Search}: Considers top moves at each ply.
                    \item \textbf{PROBCUT}: Uses statistics to prune unlikely moves.
                    \item \textbf{Late Move Reduction}: Reduces depth for less promising moves.
                \end{itemize}
            \item \textbf{Search vs. Lookup}
                \begin{itemize}
                    \item \textbf{Opening Books}: Precomputed opening sequences.
                    \item \textbf{Endgame Tables}: Perfect play solutions for endgames.
                    \item \textbf{Retrograde Minimax Search}: Constructs endgame tables by reversing rules.
                \end{itemize}
        \end{itemize}
    \end{highlight}
\end{notes}

The next section that is being covered from this chapter this week is \textbf{Section 5.4: Monte Carlo Tree Search}.

\begin{notes}{Section 5.4: Monte Carlo Tree Search}
    \subsection*{Overview}

    Key topics include the basics of Monte Carlo Tree Search (MCTS), the selection, expansion, simulation, and back-propagation steps, upper confidence bounds applied to trees (UCT), and the comparison 
    of MCTS with alpha-beta search. These concepts are essential for understanding how MCTS can be used to make decisions in complex games with high branching factors and uncertain evaluation functions.
    
    \subsubsection*{Basics of Monte Carlo Tree Search (MCTS)}
    
    Monte Carlo Tree Search (MCTS) is a strategy that estimates the value of a state by averaging the results of numerous simulated games played from that state to termination.
    
    \begin{highlight}[Basics of Monte Carlo Tree Search (MCTS)]
        MCTS estimates state values through simulations, providing a robust method for decision-making in complex games.
        
        \begin{itemize}
            \item \textbf{Simulation}: Runs complete games from a given state to terminal positions, using random or biased moves.
            \item \textbf{Average Utility}: The state value is estimated by the average utility (win percentage) from simulations.
            \item \textbf{Playout Policy}: Guides move selection during simulations to improve accuracy.
        \end{itemize}
    \end{highlight}
    
    \subsubsection*{Selection, Expansion, Simulation, and Back-Propagation}
    
    MCTS involves four main steps: selection, expansion, simulation, and back-propagation. These steps iteratively build the search tree and update the estimated values.
    
    \begin{highlight}[Selection, Expansion, Simulation, and Back-Propagation]
        The four steps of MCTS iteratively build the search tree and refine state value estimates.
        
        \begin{itemize}
            \item \textbf{Selection}: Navigates the tree from the root to a leaf node, guided by a selection policy.
            \item \textbf{Expansion}: Adds new child nodes to the tree from the selected node.
            \item \textbf{Simulation}: Runs a playout from the newly added node to a terminal state.
            \item \textbf{Back-Propagation}: Updates the values of nodes along the path from the new node to the root.
        \end{itemize}
    \end{highlight}
    
    \subsubsection*{Upper Confidence Bounds Applied to Trees (UCT)}
    
    UCT is a popular selection policy for MCTS that balances exploration and exploitation using a mathematical formula.
    
    \begin{highlight}[Upper Confidence Bounds Applied to Trees (UCT)]
        UCT guides the selection step in MCTS by balancing the need to explore new moves and exploit known good moves.
        
        \begin{itemize}
            \item \textbf{Exploitation Term}: \( \frac{U(n)}{N(n)} \) represents the average utility of node \( n \).
            \item \textbf{Exploration Term}: \( C \times \sqrt{\frac{\log N(\text{PARENT}(n))}{N(n)}} \) encourages exploring less-visited nodes.
            \item \textbf{Balancing Constant \( C \)}: Adjusts the balance between exploration and exploitation.
        \end{itemize}
    \end{highlight}
    
    \subsubsection*{Comparison with Alpha-Beta Search}
    
    MCTS and alpha-beta search have different strengths and weaknesses, particularly in games with high branching factors and complex evaluation functions.
    
    \begin{highlight}[Comparison with Alpha-Beta Search]
        MCTS and alpha-beta search each have advantages depending on the game's characteristics and evaluation function accuracy.
        
        \begin{itemize}
            \item \textbf{Branching Factor}: MCTS handles high branching factors better than alpha-beta search.
            \item \textbf{Evaluation Function}: MCTS does not rely on an accurate heuristic evaluation function.
            \item \textbf{Robustness}: MCTS is less susceptible to single-point evaluation errors.
            \item \textbf{Hybrid Approaches}: Combining aspects of MCTS and alpha-beta can leverage the strengths of both.
        \end{itemize}
    \end{highlight}
    
    \subsubsection*{Key Concepts}
    
    \begin{highlight}[Key Concepts]
        This section summarizes the key concepts related to Monte Carlo Tree Search, emphasizing their definitions, properties, and applications in AI.
        
        \begin{itemize}
            \item \textbf{Basics of Monte Carlo Tree Search (MCTS)}
                \begin{itemize}
                    \item \textbf{Simulation}: Runs complete games to terminal positions.
                    \item \textbf{Average Utility}: State value estimated by simulation results.
                    \item \textbf{Playout Policy}: Guides move selection during simulations.
                \end{itemize}
            \item \textbf{Selection, Expansion, Simulation, and Back-Propagation}
                \begin{itemize}
                    \item \textbf{Selection}: Navigates the tree to a leaf node.
                    \item \textbf{Expansion}: Adds new child nodes.
                    \item \textbf{Simulation}: Runs a playout to a terminal state.
                    \item \textbf{Back-Propagation}: Updates node values along the path.
                \end{itemize}
            \item \textbf{Upper Confidence Bounds Applied to Trees (UCT)}
                \begin{itemize}
                    \item \textbf{Exploitation Term}: Represents average utility.
                    \item \textbf{Exploration Term}: Encourages exploring less-visited nodes.
                    \item \textbf{Balancing Constant \( C \)}: Adjusts exploration-exploitation balance.
                \end{itemize}
            \item \textbf{Comparison with Alpha-Beta Search}
                \begin{itemize}
                    \item \textbf{Branching Factor}: MCTS handles high branching factors well.
                    \item \textbf{Evaluation Function}: MCTS does not rely on heuristic evaluation.
                    \item \textbf{Robustness}: Less susceptible to single-point errors.
                    \item \textbf{Hybrid Approaches}: Combines MCTS and alpha-beta strengths.
                \end{itemize}
        \end{itemize}
    \end{highlight}
\end{notes}

The last section that is being covered from this chapter this week is \textbf{Section 5.5: Partially Observable Games}.

\begin{notes}{Section 5.5: Partially Observable Games}
    \subsection*{Overview}

    Key topics include the nature of partially observable games, the example of Kriegspiel (partially observable chess), belief states, and strategies for handling partial observability in games. These 
    concepts are essential for understanding how to model and solve games where players have limited information about the game state.
    
    \subsubsection*{Nature of Partially Observable Games}
    
    Partially observable games are characterized by the players' incomplete information about the game state, leading to uncertainty and the need for strategies that account for this lack of information.
    
    \begin{highlight}[Nature of Partially Observable Games]
        Partially observable games involve uncertainty due to players' incomplete information about the game state, requiring strategies that manage this uncertainty.
        
        \begin{itemize}
            \item \textbf{Partial Observability}: Players do not have full information about the game state.
            \item \textbf{Information Gathering}: Use of scouts and spies to gather information about the opponent's state.
            \item \textbf{Concealment and Bluffing}: Techniques to hide information and mislead the opponent.
        \end{itemize}
    \end{highlight}
    
    \subsubsection*{Kriegspiel: Partially Observable Chess}
    
    Kriegspiel is a variant of chess where each player can only see their own pieces. A referee, who sees all pieces, adjudicates the game and provides limited information to the players.
    
    \begin{highlight}[Kriegspiel: Partially Observable Chess]
        Kriegspiel is a partially observable variant of chess where players only see their own pieces, and a referee provides limited information about the opponent's moves.
        
        \begin{itemize}
            \item \textbf{Referee Announcements}: The referee announces information such as captures and checks.
            \item \textbf{Move Proposals}: Players propose moves to the referee, who indicates if the move is legal or not.
            \item \textbf{Information Gained}: Players gain information about the opponent's pieces based on the legality of proposed moves.
        \end{itemize}
    \end{highlight}
    
    \subsubsection*{Belief States}
    
    In partially observable games, belief states represent the set of all possible configurations of the game state that are consistent with the player's observations and actions so far.
    
    \begin{highlight}[Belief States]
        Belief states encapsulate all possible game states that are consistent with the player's observations and actions, allowing for decision-making under uncertainty.
        
        \begin{itemize}
            \item \textbf{Initial Belief State}: The starting point, often a single known state.
            \item \textbf{Update Mechanism}: Belief states are updated based on new observations and actions.
            \item \textbf{Representation}: Belief states can be represented as a set or a probability distribution over possible states.
        \end{itemize}
    \end{highlight}
    
    \subsubsection*{Strategies for Partial Observability}
    
    Strategies in partially observable games must account for the uncertainty and use information-gathering techniques to improve decision-making.
    
    \begin{highlight}[Strategies for Partial Observability]
        Effective strategies in partially observable games focus on managing uncertainty and gathering information to improve decision-making.
        
        \begin{itemize}
            \item \textbf{Information Gathering}: Actively seeking information about the opponent's state through probing actions.
            \item \textbf{Conservative Play}: Avoiding high-risk moves that rely on uncertain information.
            \item \textbf{Belief State Management}: Continuously updating and refining belief states based on new information.
        \end{itemize}
    \end{highlight}
    
    \subsubsection*{Key Concepts}
    
    \begin{highlight}[Key Concepts]
        This section summarizes the key concepts related to partially observable games, emphasizing their definitions, properties, and applications in AI.
        
        \begin{itemize}
            \item \textbf{Nature of Partially Observable Games}
                \begin{itemize}
                    \item \textbf{Partial Observability}: Incomplete information about the game state.
                    \item \textbf{Information Gathering}: Use of scouts and spies.
                    \item \textbf{Concealment and Bluffing}: Techniques to mislead the opponent.
                \end{itemize}
            \item \textbf{Kriegspiel: Partially Observable Chess}
                \begin{itemize}
                    \item \textbf{Referee Announcements}: Provides limited information about moves.
                    \item \textbf{Move Proposals}: Players propose moves to the referee.
                    \item \textbf{Information Gained}: Based on the legality of proposed moves.
                \end{itemize}
            \item \textbf{Belief States}
                \begin{itemize}
                    \item \textbf{Initial Belief State}: Starting point of known states.
                    \item \textbf{Update Mechanism}: Updating beliefs with new observations.
                    \item \textbf{Representation}: Set or probability distribution of states.
                \end{itemize}
            \item \textbf{Strategies for Partial Observability}
                \begin{itemize}
                    \item \textbf{Information Gathering}: Probing actions to gather information.
                    \item \textbf{Conservative Play}: Avoiding risky moves.
                    \item \textbf{Belief State Management}: Refining beliefs with new information.
                \end{itemize}
        \end{itemize}
    \end{highlight}
\end{notes}