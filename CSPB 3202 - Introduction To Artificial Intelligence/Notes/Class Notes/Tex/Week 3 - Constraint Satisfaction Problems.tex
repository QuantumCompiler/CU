\clearpage

\renewcommand{\ChapTitle}{Constraint Satisfaction Problems}
\renewcommand{\SectionTitle}{Constraint Satisfaction Problems}

\chapter{\ChapTitle}

\section{\SectionTitle}
\horizontalline{0}{0}

\subsection{Assigned Reading}

The reading for this week is from, \AITextbook \hspace*{1pt} and \RLTextbook.

\begin{itemize}
    \item \textbf{Artificial Intelligence - A Modern Approach - Chapter 6.1 - Defining Constraint Satisfaction Problems}
    \item \textbf{Artificial Intelligence - A Modern Approach - Chapter 6.2 - Constraint Propagation - Inference In CSPs}
    \item \textbf{Artificial Intelligence - A Modern Approach - Chapter 6.3 - Backtracking Search For CSPs}
    \item \textbf{Artificial Intelligence - A Modern Approach - Chapter 6.4 - Local Search For CSPs}
    \item \textbf{Artificial Intelligence - A Modern Approach - Chapter 6.5 - The Structure Of Problems}
\end{itemize}

\subsection{Piazza}

Must post at least \textbf{three} times this week to Piazza.

\subsection{Lectures}

The lectures for this week are:

\begin{itemize}
    \item \lecture{https://www.youtube.com/watch?v=NbiN7av-qVo}{CSP Intro}{20}
    \item \lecture{https://www.youtube.com/watch?v=xqh__WX3FDE}{CSP Solving 1: Backtracking \& Arc Consistency}{21}
    \item \lecture{https://www.youtube.com/watch?v=JgbHedekhJw}{CSP Ordering \& Structure}{28}
\end{itemize}

\noindent The lecture notes for this week are:

\begin{itemize}
    \item \pdflink{\LecNoteDir/Notes/CSP Intro Lecture Notes.pdf}{CSP Intro Lecture Notes}
    \item \pdflink{\LecNoteDir/Notes/CSP Solving 1 - Backtracking & Arc Consistency Lecture Notes.pdf}{CSP Solving 1 - Backtracking & Arc Consistency Lecture Notes}
    \item \pdflink{\LecNoteDir/Notes/CSP Solving 2 - Backtracking & Arc Consistency Lecture Notes.pdf}{CSP Solving 2 - Backtracking & Arc Consistency Lecture Notes}
\end{itemize}

\subsection{Assignments}

The assignment(s) for this week are:

\begin{itemize}
    \item \pdflink{\AssDir/Assignment 3 - Constraint Satisfaction Problems/Assignment 3 - Constraint Satisfaction Problems.pdf}{Assignment 3 - Constraint Satisfaction Problems}
\end{itemize}

\subsection{Quiz}

The quiz for this week is:

\begin{itemize}
    \item \pdflink{\QuizDir Quiz 3 - Constraint Satisfaction Problems.pdf}{Quiz 3 - Constraint Satisfaction Problems}
\end{itemize}

\subsection{Chapter Summary}

The chapter that is being covered this week is \textbf{Chapter 6: Constraint Satisfaction Problems}. The first section that is covered from this chapter this week is \textbf{Section 6.1: Defining Constraint Satisfaction Problems}.

\begin{notes}{Section 6.1: Defining Constraint Satisfaction Problems}
    \subsection*{Overview}

    In this section, we delve into constraint satisfaction problems (CSPs), which are mathematical problems defined by a set of objects whose state must satisfy a number of constraints or limitations. 
    This approach treats states as more than just indivisible entities by using a factored representation, allowing us to use general heuristics to solve complex problems efficiently.
    
    \subsubsection*{Defining Constraint Satisfaction Problems}
    
    A constraint satisfaction problem consists of three main components: variables, domains, and constraints.
    
    \begin{highlight}[Components of a CSP]
    
        \begin{itemize}
            \item \textbf{Variables (X)}: A set of variables $\{X_1, X_2, \ldots, X_n\}$.
            \item \textbf{Domains (D)}: A set of domains $\{D_1, D_2, \ldots, D_n\}$, one for each variable, where each domain $D_i$ consists of a set of allowable values $\{v_1, v_2, \ldots, v_k\}$ for variable $X_i$.
            \item \textbf{Constraints (C)}: A set of constraints specifying allowable combinations of values. Each constraint $C_j$ consists of a pair $\langle \text{scope}, \text{rel} \rangle$, 
            where scope is a tuple of variables involved in the constraint and rel is a relation defining the permissible values.
        \end{itemize}
    
    \end{highlight}
    
    CSPs deal with assignments of values to variables. An assignment that does not violate any constraints is called a consistent or legal assignment. A complete assignment assigns values to all 
    variables, and a solution to a CSP is a consistent, complete assignment. Partial assignments leave some variables unassigned, and a partial solution is a consistent partial assignment.
    
    \subsubsection*{Example Problem: Map Coloring}
    
    To illustrate a CSP, consider the problem of coloring a map of Australia such that no two neighboring regions have the same color.
    
    \begin{highlight}[Map Coloring CSP]
    
        \begin{itemize}
            \item \textbf{Variables (X)}: The regions $\{WA, NT, Q, NSW, V, SA, T\}$.
            \item \textbf{Domains (D)}: The colors $\{\text{red}, \text{green}, \text{blue}\}$ for each variable.
            \item \textbf{Constraints (C)}: Adjacent regions must have different colors. This results in constraints like $SA \neq WA$, $SA \neq NT$, etc.
        \end{itemize}
    
    \end{highlight}
    
    Visualizing a CSP as a constraint graph, where nodes represent variables and edges represent constraints, can help understand the problem structure and facilitate efficient problem solving.
    
    \subsubsection*{Example Problem: Job-Shop Scheduling}
    
    In job-shop scheduling, the task is to schedule a series of jobs subject to constraints such as task precedence and resource limitations.
    
    \begin{highlight}[Job-Shop Scheduling CSP]
    
        \begin{itemize}
            \item \textbf{Variables (X)}: Tasks involved in assembling a car, such as $\{\text{AxleF}, \text{AxleB}, \text{WheelRF}, \ldots, \text{Inspect}\}$.
            \item \textbf{Domains (D)}: Start times for each task, within a specified range (e.g., 0 to 30 minutes).
            \item \textbf{Constraints (C)}: Precedence constraints (e.g., $\text{AxleF} + 10 \leq \text{WheelRF}$), resource constraints (e.g., only one tool for axle installation), and overall completion time.
        \end{itemize}
    
    \end{highlight}
    
    CSPs are particularly effective in such scheduling problems due to their ability to prune large portions of the search space by identifying variable assignments that violate constraints.
    
    \subsubsection*{Variations on the CSP Formalism}
    
    CSPs can vary based on the types of variables and constraints involved.
    
    \begin{highlight}[CSP Variations]
    
        \begin{itemize}
            \item \textbf{Discrete and Finite Domains}: Variables with a limited set of possible values, such as colors in the map-coloring problem.
            \item \textbf{Infinite Domains}: Variables with an infinite set of possible values, such as start times without a deadline.
            \item \textbf{Continuous Domains}: Variables with continuous values, common in operations research (e.g., scheduling telescope observations).
            \item \textbf{Unary Constraints}: Constraints on a single variable (e.g., $SA \neq \text{green}$).
            \item \textbf{Binary Constraints}: Constraints between two variables (e.g., $SA \neq NSW$).
            \item \textbf{Global Constraints}: Constraints involving multiple variables (e.g., \texttt{Alldiff} constraints in Sudoku).
            \item \textbf{Preference Constraints}: Indicating preferred solutions (e.g., scheduling preferences for professors).
        \end{itemize}
    
    \end{highlight}
    
    \begin{highlight}[Summary of Key Concepts]
    
        \begin{itemize}
            \item \textbf{CSPs}: Problems defined by variables, domains, and constraints, where the goal is to find a consistent and complete assignment.
            \item \textbf{Map Coloring Example}: Illustrates CSP with variables as regions, domains as colors, and constraints ensuring adjacent regions have different colors.
            \item \textbf{Job-Shop Scheduling Example}: Demonstrates CSP in scheduling tasks with precedence and resource constraints.
            \item \textbf{CSP Variations}: Covers discrete, finite, infinite, and continuous domains, as well as different types of constraints (unary, binary, global, and preference).
        \end{itemize}
    
        CSPs provide a powerful framework for solving a wide range of problems efficiently by leveraging the structure of the problem to prune the search space.
    
    \end{highlight}
\end{notes}

The next section that is covered from this chapter this week is \textbf{Section 6.2: Constraint Propagation - Inference In CSPs}.

\begin{notes}{Section 6.2: Constraint Propagation - Inference In CSPs}
    \subsection*{Overview}

    This section explores constraint propagation, a technique used in constraint satisfaction problems (CSPs) to reduce the number of legal values for variables by enforcing local consistency. Constraint 
    propagation can significantly enhance the efficiency of solving CSPs by reducing the search space and sometimes solving the problem entirely without further search.
    
    \subsubsection*{Constraint Propagation}
    
    Constraint propagation uses the constraints of a CSP to reduce the domains of variables, making it easier to find a solution. This process can be integrated with search or performed as a preprocessing step.
    
    \begin{highlight}[Constraint Propagation]
    
        \begin{itemize}
            \item \textbf{Purpose}: Reduce the number of legal values for variables, thereby simplifying the problem.
            \item \textbf{Integration}: Can be intertwined with search or done as preprocessing.
            \item \textbf{Effectiveness}: Sometimes solves the entire problem without further search.
        \end{itemize}
    
    \end{highlight}
    
    The key idea behind constraint propagation is local consistency. Enforcing local consistency helps eliminate inconsistent values, reducing the domains of variables and making the problem easier to solve.
    
    \subsubsection*{Node Consistency}
    
    A variable is node-consistent if all values in its domain satisfy its unary constraints. Ensuring node consistency involves removing values that violate these constraints.
    
    \begin{highlight}[Node Consistency]
    
        \begin{itemize}
            \item \textbf{Definition}: A variable is node-consistent if all values in its domain satisfy its unary constraints.
            \item \textbf{Example}: In the Australia map-coloring problem, if South Australians dislike green, remove green from SA's domain.
            \item \textbf{Implementation}: Reduce the domain of variables with unary constraints at the start of the solving process.
        \end{itemize}
    
    \end{highlight}
    
    \subsubsection*{Arc Consistency}
    
    A variable is arc-consistent with another variable if every value in its domain satisfies the binary constraint between them. Enforcing arc consistency ensures that all binary constraints are satisfied.
    
    \begin{highlight}[Arc Consistency]
    
        \begin{itemize}
            \item \textbf{Definition}: A variable $X_i$ is arc-consistent with respect to $X_j$ if for every value in $D_i$, there is some value in $D_j$ that satisfies the binary constraint between them.
            \item \textbf{Example}: In a CSP where $Y = X^2$ with domains {0, 1, 2, 3}, $X$ is reduced to {0, 1, 2} and $Y$ to {0, 1, 4}.
            \item \textbf{Algorithm (AC-3)}: Maintains a queue of arcs and makes each arc consistent, updating domains and rechecking affected arcs.
            \item \textbf{Complexity}: Worst-case time complexity is $O(cd^3)$ for a CSP with $n$ variables, domain size $d$, and $c$ binary constraints.
        \end{itemize}
    
    \end{highlight}
    
    \subsubsection*{Path Consistency}
    
    Path consistency extends the idea of arc consistency by considering triples of variables. It ensures that if two variables are consistent, then there exists a consistent value for a third variable.
    
    \begin{highlight}[Path Consistency]
    
        \begin{itemize}
            \item \textbf{Definition}: A set $\{X_i, X_j\}$ is path-consistent with $X_m$ if for every consistent assignment $\{X_i = a, X_j = b\}$, there is an assignment to $X_m$ that satisfies all constraints.
            \item \textbf{Example}: In the Australia map-coloring problem with two colors, path consistency shows no solution is possible because three regions touching each other need at least three colors.
            \item \textbf{Purpose}: Helps solve problems that cannot be addressed by arc consistency alone.
        \end{itemize}
    
    \end{highlight}
    
    \subsubsection*{K-Consistency}
    
    K-consistency generalizes the concept of consistency to any number of variables. A CSP is k-consistent if for any set of $k-1$ variables, a consistent value can always be assigned to the $k$-th variable.
    
    \begin{highlight}[K-Consistency]
    
        \begin{itemize}
            \item \textbf{Definition}: A CSP is k-consistent if for any set of $k-1$ variables, a consistent value can always be assigned to any $k$-th variable.
            \item \textbf{Levels}: 
                \begin{itemize}
                    \item 1-consistency (node consistency)
                    \item 2-consistency (arc consistency)
                    \item 3-consistency (path consistency)
                \end{itemize}
            \item \textbf{Strongly K-Consistent}: A CSP is strongly k-consistent if it is k-consistent, (k-1)-consistent, ..., down to 1-consistent.
            \item \textbf{Trade-Off}: Ensuring higher levels of consistency can be computationally expensive and requires more space.
        \end{itemize}
    
    \end{highlight}
    
    \subsubsection*{Global Constraints}
    
    Global constraints involve multiple variables and are common in real-world CSPs. Special-purpose algorithms can handle these efficiently.
    
    \begin{highlight}[Global Constraints]
    
        \begin{itemize}
            \item \textbf{Definition}: Constraints involving an arbitrary number of variables.
            \item \textbf{Examples}:
                \begin{itemize}
                    \item \textbf{Alldiff Constraint}: Ensures all variables involved have distinct values.
                    \item \textbf{Resource Constraint (Atmost)}: Limits the total resources used by several tasks.
                \end{itemize}
            \item \textbf{Detection and Enforcement}: Efficient algorithms can detect inconsistencies and enforce constraints, often more effectively than binary constraints.
            \item \textbf{Bounds Propagation}: Manages large integer domains by maintaining and propagating upper and lower bounds.
        \end{itemize}
    
    \end{highlight}
    
    \subsubsection*{Sudoku}
    
    Sudoku puzzles are a popular example of CSPs, where the objective is to fill a 9x9 grid so that each row, column, and 3x3 box contains the digits 1 to 9 without repetition.
    
    \begin{highlight}[Sudoku]
    
        \begin{itemize}
            \item \textbf{Structure}: 81 variables with domains {1, 2, ..., 9} and 27 Alldiff constraints (one for each row, column, and box).
            \item \textbf{Arc Consistency}: Can simplify the puzzle by reducing domains of variables, but often not sufficient alone.
            \item \textbf{Advanced Strategies}: Techniques like "naked triples" enforce stronger consistency and help solve more complex puzzles.
        \end{itemize}
    
    \end{highlight}
    
    \begin{highlight}[Summary of Key Concepts]
    
        \begin{itemize}
            \item \textbf{Constraint Propagation}: Reduces the number of legal values for variables by enforcing local consistency.
            \item \textbf{Node Consistency}: Ensures all values in a variable's domain satisfy unary constraints.
            \item \textbf{Arc Consistency}: Ensures binary constraints are satisfied for all values in variable domains.
            \item \textbf{Path Consistency}: Extends arc consistency to triples of variables.
            \item \textbf{K-Consistency}: Generalizes consistency to any set of $k$ variables, with strong k-consistency ensuring all lower levels of consistency.
            \item \textbf{Global Constraints}: Involves multiple variables, handled by specialized algorithms for efficiency.
            \item \textbf{Sudoku}: A well-known example of CSP demonstrating the application of various consistency techniques.
        \end{itemize}
    
        Constraint propagation techniques enhance the efficiency of solving CSPs by reducing the search space and enforcing various levels of consistency.
    
    \end{highlight}
\end{notes}

The next section that is covered from this chapter this week is \textbf{Section 6.3: Backtracking Search For CSPs}.

\begin{notes}{Section 6.3: Backtracking Search For CSPs}
    \subsection*{Overview}

    This section covers backtracking search algorithms for solving constraint satisfaction problems (CSPs). When constraint propagation cannot reduce the domains of all variables to single values, we 
    need to search for a solution. Backtracking search works on partial assignments and is enhanced by various heuristics and inference techniques to improve efficiency and reduce the search space.
    
    \subsubsection*{Backtracking Search}
    
    Backtracking search is a recursive, depth-first search method that extends partial assignments one variable at a time and backtracks when a constraint violation occurs.
    
    \begin{highlight}[Backtracking Search]
    
        \begin{itemize}
            \item \textbf{Algorithm}: Extends a partial assignment by selecting an unassigned variable and trying all possible values. If a value leads to a consistent assignment, the search continues; 
            otherwise, it backtracks.
            \item \textbf{Implementation}: Uses a recursive approach similar to depth-first search.
            \item \textbf{Efficiency}: Enhanced by using variable and value ordering heuristics, as well as inference techniques to prune the search space.
        \end{itemize}

    \begin{code}[Pseudo]
    function BACKTRACKING-SEARCH(csp) returns a solution or failure
        return BACKTRACK(csp, {})
    
    function BACKTRACK(csp, assignment) returns a solution or failure
        if assignment is complete then return assignment
        var ← SELECT-UNASSIGNED-VARIABLE(csp, assignment)
        for each value in ORDER-DOMAIN-VALUES(csp, var, assignment) do
            if value is consistent with assignment then
                add {var = value} to assignment
                inferences ← INFERENCE(csp, var, assignment)
                if inferences ≠ failure then
                    add inferences to csp
                    result ← BACKTRACK(csp, assignment)
                    if result ≠ failure then return result
                    remove inferences from csp
                remove {var = value} from assignment
        return failure
    \end{code}
    
    \end{highlight}
    
    
    \subsubsection*{Variable and Value Ordering}
    
    Choosing the right variable and value ordering can significantly reduce the search space and improve efficiency.
    
    \begin{highlight}[Variable and Value Ordering]
    
        \begin{itemize}
            \item \textbf{Minimum-Remaining-Values (MRV)}: Selects the variable with the fewest legal values remaining, reducing the likelihood of future conflicts.
            \item \textbf{Degree Heuristic}: Chooses the variable involved in the largest number of constraints with other unassigned variables, aiming to reduce the branching factor.
            \item \textbf{Least-Constraining-Value}: Prefers the value that leaves the most options open for neighboring variables, maximizing future flexibility.
        \end{itemize}
    
    \end{highlight}
    
    \subsubsection*{Interleaving Search and Inference}
    
    Combining search with inference can detect inconsistencies earlier and reduce the search space more effectively.
    
    \begin{highlight}[Interleaving Search and Inference]
    
        \begin{itemize}
            \item \textbf{Forward Checking}: After assigning a value to a variable, it removes inconsistent values from the domains of neighboring unassigned variables.
            \item \textbf{Maintaining Arc Consistency (MAC)}: Calls AC-3 algorithm during the search to enforce arc consistency and propagate constraints more thoroughly than forward checking.
        \end{itemize}
    
    \end{highlight}
    
    \subsubsection*{Intelligent Backtracking: Looking Backward}
    
    When a branch fails, intelligent backtracking methods can backtrack to more appropriate points in the search tree, avoiding unnecessary re-evaluations.
    
    \begin{highlight}[Intelligent Backtracking]
    
        \begin{itemize}
            \item \textbf{Chronological Backtracking}: Reverts to the most recent variable and tries a different value.
            \item \textbf{Conflict-Directed Backjumping}: Backtracks to the most recent variable in the conflict set that caused the failure, skipping irrelevant variables.
            \item \textbf{Conflict Sets}: Track variables responsible for conflicts, allowing the algorithm to backtrack more effectively.
        \end{itemize}
    
    \end{highlight}
    
    \subsubsection*{Constraint Learning}
    
    Constraint learning involves identifying and recording sets of assignments that lead to conflicts, known as no-goods, to prevent the algorithm from repeating the same mistakes.
    
    \begin{highlight}[Constraint Learning]
    
        \begin{itemize}
            \item \textbf{No-Goods}: Sets of variable assignments that are inconsistent, stored to avoid re-exploring the same conflicting paths.
            \item \textbf{Efficiency}: Reduces redundant searches and improves the overall efficiency of the backtracking algorithm.
        \end{itemize}
    
    \end{highlight}
    
    \begin{highlight}[Summary of Key Concepts]
    
        \begin{itemize}
            \item \textbf{Backtracking Search}: Recursive search method for extending partial assignments and backtracking on conflicts.
            \item \textbf{Variable and Value Ordering}: Heuristics like MRV, degree heuristic, and least-constraining-value to improve search efficiency.
            \item \textbf{Interleaving Search and Inference}: Techniques like forward checking and MAC to enforce consistency during the search.
            \item \textbf{Intelligent Backtracking}: Methods like conflict-directed backjumping to backtrack more effectively based on conflict sets.
            \item \textbf{Constraint Learning}: Recording no-goods to prevent redundant searches and enhance efficiency.
        \end{itemize}
    
        Backtracking search, enhanced by intelligent heuristics and inference techniques, is a powerful method for solving CSPs efficiently by reducing the search space and avoiding redundant work.
    
    \end{highlight}
\end{notes}

The next section that is covered from this chapter this week is \textbf{Section 6.4: Local Search For CSPs}.

\begin{notes}{Section 6.4: Local Search For CSPs}
    \subsection*{Overview}

    This section explores local search algorithms for solving constraint satisfaction problems (CSPs). Unlike backtracking search, which works on partial assignments, local search operates on complete 
    assignments and iteratively improves them by making local changes. Local search is particularly effective for large-scale CSPs and can handle problems dynamically as they evolve.
    
    \subsubsection*{Local Search for CSPs}
    
    Local search algorithms begin with a complete assignment of values to variables and attempt to find a solution by iteratively improving the assignment.
    
    \begin{highlight}[Local Search for CSPs]
    
        \begin{itemize}
            \item \textbf{Complete-State Formulation}: Starts with a complete assignment, even if it violates constraints.
            \item \textbf{Local Changes}: Iteratively changes the value of one variable at a time to reduce the number of constraint violations.
            \item \textbf{Dynamic Adaptation}: Can handle changes in the problem dynamically, making it suitable for real-time applications.
        \end{itemize}
    
    \end{highlight}
    
    \subsubsection*{Min-Conflicts Heuristic}
    
    The min-conflicts heuristic is a simple yet powerful technique for local search. It selects a variable with conflicts and assigns it a value that minimizes the number of conflicts.
    
    \begin{highlight}[Min-Conflicts Heuristic]
    
        \begin{itemize}
            \item \textbf{Selection}: Choose a variable that is currently in conflict.
            \item \textbf{Assignment}: Assign the variable a value that results in the fewest conflicts with other variables.
            \item \textbf{Effectiveness}: Particularly effective for large CSPs and problems with densely distributed solutions.
        \end{itemize}

    \begin{code}[Pseudo]
    function MIN-CONFLICTS(csp, max_steps) returns a solution or failure
        inputs: csp, a constraint satisfaction problem
                max_steps, the number of steps allowed before giving up
        current ← an initial complete assignment for csp
        for i = 1 to max_steps do
            if current is a solution for csp then return current
            var ← a randomly chosen conflicted variable from csp.VARIABLES
            value ← the value for var that minimizes CONFLICTS(csp, var, value, current)
            set var = value in current
        return failure
    \end{code}

    \end{highlight}
    
    \subsubsection*{Example: 8-Queens Problem}
    
    The 8-queens problem is a classic example of a CSP that can be effectively solved using local search with the min-conflicts heuristic.
    
    \begin{highlight}[8-Queens Problem]
    
        \begin{itemize}
            \item \textbf{Initial State}: Start with a complete assignment where queens are placed on the board, even if they attack each other.
            \item \textbf{Min-Conflicts Heuristic}: Move a queen to a position that minimizes the number of conflicts with other queens.
            \item \textbf{Efficiency}: The algorithm quickly converges to a solution, even for large instances like the million-queens problem.
        \end{itemize}
    
    \end{highlight}
    
    \subsubsection*{Plateau Search and Techniques for Escaping Local Optima}
    
    Local search can encounter plateaus where many assignments have the same number of conflicts. Several techniques help escape these local optima.
    
    \begin{highlight}[Escaping Local Optima]
    
        \begin{itemize}
            \item \textbf{Plateau Search}: Allows sideways moves to another state with the same score.
            \item \textbf{Tabu Search}: Maintains a list of recently visited states and forbids the algorithm from returning to them.
            \item \textbf{Simulated Annealing}: Uses probabilistic moves to escape local optima by occasionally accepting worse states.
            \item \textbf{Constraint Weighting}: Increases the weight of violated constraints to guide the search towards more critical constraints, adding topography to plateaus and enabling learning over time.
        \end{itemize}
    
    \end{highlight}
    
    \subsubsection*{Applications in Dynamic Environments}
    
    Local search is particularly well-suited for dynamic environments where the problem constraints can change, such as scheduling problems.
    
    \begin{highlight}[Dynamic Environments]
    
        \begin{itemize}
            \item \textbf{Adaptability}: Can quickly adapt to changes in constraints, making minimal adjustments to existing solutions.
            \item \textbf{Example}: Airline scheduling where bad weather disrupts the schedule; local search can reassign flights with minimal disruption.
            \item \textbf{Efficiency}: More efficient than re-running a complete backtracking search from scratch.
        \end{itemize}
    
    \end{highlight}
    
    \begin{highlight}[Summary of Key Concepts]
    
        \begin{itemize}
            \item \textbf{Local Search for CSPs}: Uses a complete-state formulation and iteratively improves assignments.
            \item \textbf{Min-Conflicts Heuristic}: Selects variables in conflict and assigns values to minimize conflicts, effective for large CSPs.
            \item \textbf{Plateau Search and Techniques}: Methods like tabu search, simulated annealing, and constraint weighting help escape local optima.
            \item \textbf{Applications in Dynamic Environments}: Local search's adaptability makes it ideal for real-time and changing CSPs, such as scheduling.
        \end{itemize}
    
        Local search algorithms are powerful tools for solving CSPs, especially in dynamic and large-scale problems, due to their ability to iteratively improve solutions and adapt to changes.
    
    \end{highlight}
\end{notes}

The last section that is covered from this chapter this week is \textbf{Section 6.5: The Structure Of Problems}.

\begin{notes}{Section 6.5: The Structure Of Problems}
    \subsection*{Overview}

    This section examines how the structure of a problem, as represented by the constraint graph, can be leveraged to find solutions more efficiently. By understanding and exploiting the structural 
    properties of CSPs, we can often decompose large problems into smaller, more manageable subproblems.
    
    \subsubsection*{Independent Subproblems}
    
    One of the key insights is that many CSPs can be decomposed into independent subproblems, which can be solved separately.
    
    \begin{highlight}[Independent Subproblems]
    
        \begin{itemize}
            \item \textbf{Connected Components}: Identifying connected components in the constraint graph allows for decomposition into independent subproblems.
            \item \textbf{Example}: In the Australia map-coloring problem, Tasmania is not connected to the mainland, making it an independent subproblem.
            \item \textbf{Efficiency}: Solving each subproblem separately can drastically reduce the overall complexity from exponential to linear in the number of variables.
        \end{itemize}
    
    \end{highlight}
    
    \subsubsection*{Tree-Structured CSPs}
    
    Tree-structured CSPs can be solved efficiently using specialized algorithms that take advantage of their hierarchical structure.
    
    \begin{highlight}[Tree-Structured CSPs]
    
        \begin{itemize}
            \item \textbf{Definition}: A CSP where the constraint graph forms a tree.
            \item \textbf{Directional Arc Consistency (DAC)}: Ensuring that every variable is arc-consistent with its descendants in a given ordering.
            \item \textbf{Algorithm (TREE-CSP-SOLVER)}: Solves tree-structured CSPs in linear time by first making the tree directed arc-consistent and then assigning values in a topologically sorted order.
            \item \textbf{Complexity}: Time complexity is $O(nd^2)$, where $n$ is the number of variables and $d$ is the domain size.
        \end{itemize}
    
    \begin{code}[Pseudo]
    function TREE-CSP-SOLVER(csp) returns a solution, or failure
        inputs: csp, a CSP with components X, D, C
        n ← number of variables in X
        assignment ← an empty assignment
        root ← any variable in X
        X ← TOPOLOGICALSORT(X, root)
        for j = n down to 2 do
            MAKE-ARC-CONSISTENT(PARENT(Xj), Xj)
            if it cannot be made consistent then return failure
        for i = 1 to n do
            assignment[Xi] ← any consistent value from Di
            if there is no consistent value then return failure
        return assignment
    \end{code}

    \end{highlight}
    
    
    \subsubsection*{Cutset Conditioning}
    
    Cutset conditioning is a technique to transform a general constraint graph into a tree by fixing some variables, turning the remaining graph into a tree.
    
    \begin{highlight}[Cutset Conditioning]
    
        \begin{itemize}
            \item \textbf{Cycle Cutset}: A subset of variables whose removal makes the constraint graph a tree.
            \item \textbf{Procedure}: Assign values to variables in the cycle cutset and solve the remaining tree-structured CSP.
            \item \textbf{Complexity}: Time complexity is $O(d^c \cdot (n-c)d^2)$, where $c$ is the size of the cycle cutset.
            \item \textbf{Efficiency}: Effective if the cycle cutset is small, significantly reducing the problem complexity.
        \end{itemize}
    
    \end{highlight}
    
    \subsubsection*{Tree Decomposition}
    
    Tree decomposition transforms the constraint graph into a tree of clusters, each containing a subset of variables, facilitating efficient problem solving.
    
    \begin{highlight}[Tree Decomposition]
    
        \begin{itemize}
            \item \textbf{Definition}: Converts the original graph into a tree where each node represents a cluster of variables.
            \item \textbf{Conditions}:
                \begin{itemize}
                    \item Every variable appears in at least one cluster.
                    \item If two variables are connected by a constraint, they appear together in at least one cluster.
                    \item If a variable appears in two clusters, it must appear in all clusters along the path connecting them.
                \end{itemize}
            \item \textbf{Algorithm}: Solve the tree-structured CSP formed by the clusters.
            \item \textbf{Complexity}: Time complexity is $O(nd^w)$, where $w$ is the tree width.
            \item \textbf{Tree Width}: The minimum width among all tree decompositions of the graph.
        \end{itemize}
    
    \end{highlight}
    
    \subsubsection*{Value Symmetry}
    
    Value symmetry occurs when multiple equivalent solutions exist due to interchangeable values, and breaking this symmetry can reduce the search space.
    
    \begin{highlight}[Value Symmetry]
    
        \begin{itemize}
            \item \textbf{Definition}: Permutations of values that result in equivalent solutions.
            \item \textbf{Example}: In the map-coloring problem with $d$ colors, there are $d!$ equivalent solutions for permuting colors.
            \item \textbf{Symmetry-Breaking Constraint}: Introduces constraints to reduce equivalent solutions and thus the search space.
        \end{itemize}
    
    \end{highlight}
    
    \begin{highlight}[Summary of Key Concepts]
    
        \begin{itemize}
            \item \textbf{Independent Subproblems}: Decompose CSPs into smaller subproblems using connected components.
            \item \textbf{Tree-Structured CSPs}: Solved efficiently using directional arc consistency and topological sorting.
            \item \textbf{Cutset Conditioning}: Reduces general graphs to trees by fixing a subset of variables, solving the simplified problem.
            \item \textbf{Tree Decomposition}: Converts the graph into a tree of clusters, simplifying the problem solving process.
            \item \textbf{Value Symmetry}: Reduces the search space by breaking value symmetries with additional constraints.
        \end{itemize}
    
        Understanding the structure of CSPs and leveraging these techniques can greatly enhance the efficiency of finding solutions by exploiting problem decomposition and structural properties.
    
    \end{highlight}
\end{notes}