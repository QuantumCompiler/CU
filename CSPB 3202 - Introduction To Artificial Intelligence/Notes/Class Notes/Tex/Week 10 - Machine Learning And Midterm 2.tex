\clearpage

\renewcommand{\ChapTitle}{Machine Learning And Midterm 2}
\renewcommand{\SectionTitle}{Machine Learning And Midterm 2}

\chapter{\ChapTitle}

\section{\SectionTitle}
\horizontalline{0}{0}

\subsection{Assigned Reading}

The reading for this week is from, \AITextbook \hspace*{1pt} and \RLTextbook.

\begin{itemize}
    \item \textbf{Artificial Intelligence - A Modern Approach - Chapter 19.1 - Forms Of Learning}
    \item \textbf{Artificial Intelligence - A Modern Approach - Chapter 19.2 - Supervised Learning}
    \item \textbf{Artificial Intelligence - A Modern Approach - Chapter 19.3 - Learning Decision Trees}
    \item \textbf{Artificial Intelligence - A Modern Approach - Chapter 19.4 - Model Selection And Optimization}
    \item \textbf{Artificial Intelligence - A Modern Approach - Chapter 19.5 - The Theory Of Learning}
    \item \textbf{Artificial Intelligence - A Modern Approach - Chapter 19.6 - Linear Regression And Classification}
    \item \textbf{Artificial Intelligence - A Modern Approach - Chapter 19.7 - Nonparametric Models}
    \item \textbf{Artificial Intelligence - A Modern Approach - Chapter 19.8 - Ensemble Learning}
\end{itemize}

\subsection{Piazza}

Must post at least \textbf{three} times this week to Piazza.

\subsection{Lectures}

The lectures for this week are:

\begin{itemize}
    \item \lecture{https://www.youtube.com/watch?v=g7iayGZGA0o}{Machine Learning Intro}{32}
    \item \lecture{https://www.youtube.com/watch?v=90VhDOIjzbI}{Linear Regression Refresher}{37}
    \item \lecture{https://www.youtube.com/watch?v=PtNBgUu08so}{Logistic Regression (Binary-Class Classification) Refresher}{53}
    \item \lecture{https://www.youtube.com/watch?v=pt_44hs6dD4}{Regularization And Cross Validation}{27}
    \item \lecture{https://www.youtube.com/watch?v=EDm5CuiTphs}{Non-Parametric Models}{64}
\end{itemize}

\noindent The lecture notes for this week are:

\begin{itemize}
    \item \pdflink{\LecNoteDir/Notes/Improving Training Lecture Notes.pdf}{Improving Training Lecture Notes}
    \item \pdflink{\LecNoteDir/Notes/Linear Regression Lecture Notes.pdf}{Linear Regression Lecture Notes}
    \item \pdflink{\LecNoteDir/Notes/Logistic Regression Lecture Notes.pdf}{Logistic Regression Lecture Notes}
    \item \pdflink{\LecNoteDir/Notes/Machine Learning Intro Lecture Notes.pdf}{Machine Learning Intro Lecture Notes}
    \item \pdflink{\LecNoteDir/Notes/Non-Parametric Models Lecture Notes.pdf}{Non-Parametric Models Lecture Notes}
\end{itemize}

\subsection{Assignments}

The assignment(s) for this week are:

\begin{itemize}
    \item \href{https://github.com/QuantumCompiler/CU/tree/main/CSPB%203202%20-%20Introduction%20To%20Artificial%20Intelligence/Assignments/Assignment%206%20-%20Kaggle%20Cancer%20Competition}{Assignment 6 - Kaggle Cancer Competition}
\end{itemize}

\subsection{Quiz}

The quiz for this week is:

\begin{itemize}
    \item \href{https://applied.cs.colorado.edu/mod/quiz/view.php?id=60014}{Quiz 9 - Basic Machine Learning}
\end{itemize}

\subsection{Exam}

The exam for this week is:

\begin{itemize}
    \item \pdflink{\ExamNotesDir Exam 2 Notes.pdf}{Exam 2 Notes}
    \item \pdflink{\ExamsDir Exam 2.pdf}{Exam 2}
\end{itemize}

\newpage

\subsection{Chapter Summary}

The chapter that is being covered this week is \textbf{Chapter 19: Learning From Examples}. The first section that is being covered from this chapter this week is \textbf{Section 19.1: Forms Of Learning}.

\begin{notes}{Section 19.1: Forms Of Learning}
    \subsection*{Overview}

    This section introduces the various forms of learning that an agent can undergo to improve its performance. Learning involves observing the environment, gathering data, and building models that help 
    predict outcomes or guide actions. It covers the different components of agent programs that can be enhanced through learning, the role of prior knowledge, and the different types of feedback that 
    can guide the learning process.
    
    \subsubsection*{Components of an Agent that Can Learn}
    
    Learning can enhance various components of an agent's architecture, leading to improved decision-making and adaptability.
    
    \begin{highlight}[Components of an Agent that Can Learn]
    
        \begin{itemize}
            \item \textbf{Direct Mapping from States to Actions}: The agent can learn condition-action rules, improving its response to specific situations.
            \item \textbf{Perceptual Processing}: The agent can learn to infer relevant properties of the environment from sensory inputs.
            \item \textbf{World Model}: The agent can learn about the effects of its actions and how the world evolves.
            \item \textbf{Utility Information}: Learning can enhance the agent's utility function, refining its understanding of the desirability of different states.
            \item \textbf{Action-Value Information}: The agent can learn the value of different actions in various states, helping to prioritize certain behaviors.
            \item \textbf{Goals and Problem-Solving Capabilities}: The agent can refine its goals and problem-solving strategies through learning.
        \end{itemize}
    
    \end{highlight}
    
    \subsubsection*{Prior Knowledge and Learning}
    
    The learning process is influenced by the agent's prior knowledge and the type of model it uses to represent this knowledge.
    
    \begin{highlight}[Prior Knowledge and Learning]
    
        \begin{itemize}
            \item \textbf{Influence on Model Building}: Prior knowledge can shape the model framework chosen by the agent, facilitating faster and more accurate learning.
            \item \textbf{Learning from Scratch}: The section also discusses scenarios where the agent starts with minimal prior knowledge and builds understanding entirely from observed data.
            \item \textbf{Transfer Learning}: Mentioned as a method where knowledge from one domain is applied to a new domain, enhancing learning efficiency.
        \end{itemize}
    
    \end{highlight}
    
    \subsubsection*{Types of Learning}
    
    Learning can be categorized based on the type of feedback provided to the agent, leading to different approaches and techniques.
    
    \begin{highlight}[Types of Learning]
    
        \begin{itemize}
            \item \textbf{Supervised Learning}: The agent learns a mapping from inputs to outputs using labeled data. This involves observing pairs of inputs and their corresponding outputs and developing 
            a function $h$ that approximates the underlying function $f$.
            \item \textbf{Unsupervised Learning}: The agent identifies patterns or structures in the input data without explicit output labels. A common task is clustering, where the agent groups similar 
            inputs together.
            \item \textbf{Reinforcement Learning}: The agent learns from a series of reinforcements—rewards and punishments—associated with actions taken. The agent aims to maximize cumulative rewards by 
            discovering which actions yield the best outcomes.
        \end{itemize}
    
    \end{highlight}
    
    \subsubsection*{Applications of Learning in Software Systems}
    
    Learning technologies are widely applicable in modern software engineering, providing significant improvements in performance and efficiency.
    
    \begin{highlight}[Applications of Learning in Software Systems]
    
        \begin{itemize}
            \item \textbf{Image Analysis}: Machine learning models can significantly speed up image analysis tasks, such as those used in astrophysics.
            \item \textbf{Energy Efficiency}: Machine learning can optimize energy usage, such as reducing the cooling costs of data centers.
            \item \textbf{Broad Impact}: The integration of learning algorithms into various software systems marks a transformative phase in computer science and AI, often referred to as the "Golden Age" 
            of machine learning.
        \end{itemize}
    
    \end{highlight}
    
    \begin{highlight}[Summary of Key Concepts]
    
        \begin{itemize}
            \item \textbf{Learning Components}: Various components of an agent, such as direct mapping, perceptual processing, and utility functions, can be improved through learning.
            \item \textbf{Prior Knowledge}: Influences the model-building process and can accelerate learning.
            \item \textbf{Types of Learning}: Supervised, unsupervised, and reinforcement learning cater to different types of feedback and learning objectives.
            \item \textbf{Applications}: Machine learning is broadly applied across different domains, enhancing the capabilities and efficiency of software systems.
        \end{itemize}
    
        Understanding these forms of learning and their applications is essential for developing intelligent systems capable of improving over time through experience.
    
    \end{highlight}
\end{notes}

The next section that is being covered from this chapter this week is \textbf{Section 19.2: Supervised Learning}.

\begin{notes}{Section 19.2: Supervised Learning}
    \subsection*{Overview}

    This section provides an in-depth look at supervised learning, a type of machine learning where an agent learns a function that maps inputs to outputs using labeled training data. It discusses the key 
    concepts involved, such as hypothesis spaces, model selection, and the trade-offs between bias and variance.
    
    \subsubsection*{Supervised Learning Defined}
    
    In supervised learning, the agent observes input-output pairs and learns a function that maps inputs to outputs. This function, often called a hypothesis, is trained using a labeled dataset.
    
    \begin{highlight}[Supervised Learning Defined]
    
        \begin{itemize}
            \item \textbf{Training Set}: A set of $N$ example input-output pairs $(x_1, y_1), (x_2, y_2), \ldots, (x_N, y_N)$, where each pair is generated by an unknown function $y = f(x)$.
            \item \textbf{Hypothesis}: The function $h$ learned by the agent, which approximates the true function $f$.
            \item \textbf{Objective}: Discover a function $h$ such that $h(x) \approx f(x)$ for all $x$, minimizing the discrepancy between the predicted outputs and the true outputs.
            \item \textbf{Example}: In an image classification task, the input $x$ might be a pixel representation of an image, and the output $y$ a label such as "cat" or "dog".
        \end{itemize}
    
    \end{highlight}
    
    \subsubsection*{Hypothesis Space and Model Selection}
    
    The hypothesis space is the set of all possible functions $h$ that the learning algorithm can choose from. The choice of hypothesis space affects the learning process and the model's ability to generalize.
    
    \begin{highlight}[Hypothesis Space and Model Selection]
    
        \begin{itemize}
            \item \textbf{Hypothesis Space $H$}: The set of possible functions, e.g., linear functions, polynomials, decision trees.
            \item \textbf{Consistent Hypothesis}: A hypothesis $h$ is consistent if $h(x_i) = y_i$ for all training examples $(x_i, y_i)$.
            \item \textbf{Best-Fit Function}: When outputs are continuous, find a function that minimizes the error $|h(x_i) - y_i|$.
            \item \textbf{Generalization}: The ability of a hypothesis to perform well on unseen data, not just on the training set.
        \end{itemize}
    
    \end{highlight}
    
    \subsubsection*{Bias and Variance}
    
    Bias and variance are critical factors in supervised learning that describe the behavior of the learned function in relation to the training data and the true function.
    
    \begin{highlight}[Bias and Variance]
    
        \begin{itemize}
            \item \textbf{Bias}: The error due to overly simplistic assumptions in the learning algorithm. High bias can lead to underfitting, where the model fails to capture the underlying pattern in the data.
            \item \textbf{Variance}: The error due to the model's sensitivity to small fluctuations in the training set. High variance can lead to overfitting, where the model captures noise in the training data.
            \item \textbf{Bias-Variance Tradeoff}: The balance between bias and variance; finding the right complexity level of the model to minimize total error.
            \item \textbf{Example}: A linear model may have high bias but low variance, while a high-degree polynomial may have low bias but high variance.
        \end{itemize}
    
    \end{highlight}
    
    \subsubsection*{Overfitting and Underfitting}
    
    Overfitting and underfitting are common issues in supervised learning, affecting the model's performance on new data.
    
    \begin{highlight}[Overfitting and Underfitting]
    
        \begin{itemize}
            \item \textbf{Overfitting}: When the model learns the training data too well, including noise, leading to poor performance on unseen data. This is often characterized by a model that fits the training data almost perfectly but generalizes poorly.
            \item \textbf{Underfitting}: When the model is too simple to capture the underlying structure of the data, leading to poor performance on both training and test data.
            \item \textbf{Example}: A degree-12 polynomial may overfit a set of 13 points, capturing all fluctuations in the data, while a simple linear model may underfit the same data by failing to capture important trends.
        \end{itemize}
    
    \end{highlight}
    
    \subsubsection*{Model Selection and Evaluation}
    
    Choosing the right model involves balancing complexity and generalization. The performance of a model is often evaluated using a separate test set.
    
    \begin{highlight}[Model Selection and Evaluation]
    
        \begin{itemize}
            \item \textbf{Test Set}: A separate set of input-output pairs not seen during training, used to evaluate the model's generalization performance.
            \item \textbf{Performance Metrics}: Metrics such as accuracy, precision, recall, and F1 score are used to evaluate classification models, while mean squared error (MSE) and root mean squared error (RMSE) are common for regression models.
            \item \textbf{Cross-Validation}: A technique to assess how the model performs on different subsets of the data, providing a more robust estimate of its generalization ability.
            \item \textbf{Example}: In a classification problem, the accuracy might be measured as the proportion of correctly classified examples in the test set.
        \end{itemize}
    
    \end{highlight}
    
    \begin{highlight}[Summary of Key Concepts]
    
        \begin{itemize}
            \item \textbf{Supervised Learning}: Learning from labeled training data to map inputs to outputs.
            \item \textbf{Hypothesis Space}: The set of possible models the learning algorithm can consider.
            \item \textbf{Bias and Variance}: Key factors affecting the model's performance and generalization.
            \item \textbf{Overfitting and Underfitting}: Common issues where models either learn too much noise or fail to capture the underlying pattern.
            \item \textbf{Model Selection}: Choosing the right model based on its performance on unseen data.
        \end{itemize}
    
        Understanding these concepts is crucial for developing effective supervised learning models that generalize well to new, unseen data, balancing the trade-offs between model complexity and generalization ability.
    
    \end{highlight}
\end{notes}

The next section that is being covered from this chapter this week is \textbf{Section 19.3: Learning Decision Trees}.

\begin{notes}{Section 19.3: Learning Decision Trees}
    \subsection*{Overview}

    This section covers the concept and methodology of learning decision trees, a powerful tool for classification tasks. Decision trees represent functions that map a set of attribute values to specific 
    outputs. They are particularly useful for tasks requiring interpretable models and handle both discrete and continuous input data effectively.
    
    \subsubsection*{Decision Trees Explained}
    
    A decision tree reaches a decision by performing a sequence of tests on input attributes, each internal node representing a test, branches representing possible outcomes, and leaf nodes representing 
    the final decision or output.
    
    \begin{highlight}[Decision Trees Explained]
    
        \begin{itemize}
            \item \textbf{Structure}: Internal nodes correspond to tests on attributes, branches correspond to attribute values, and leaf nodes represent decisions.
            \item \textbf{Example}: In the restaurant domain, attributes like Patrons, Type, and WaitEstimate are used to decide whether to wait for a table.
            \item \textbf{Boolean Classification}: Decision trees can perform Boolean classification, where the output is either true (positive example) or false (negative example).
        \end{itemize}
    
    \end{highlight}
    
    \subsubsection*{Expressiveness of Decision Trees}
    
    Decision trees can represent any Boolean function, but their size and complexity depend on the nature of the function being represented.
    
    \begin{highlight}[Expressiveness of Decision Trees]
    
        \begin{itemize}
            \item \textbf{Disjunctive Normal Form (DNF)}: A decision tree can represent functions in DNF, where the output is true if at least one conjunction of conditions is met.
            \item \textbf{Example}: The decision tree for the restaurant domain represents a function where the output is "Wait" based on conditions like the number of patrons and estimated waiting time.
            \item \textbf{Limitations}: Some functions, like the majority function or parity function, require large decision trees. Real-valued functions with non-rectangular boundaries are also challenging 
            to represent compactly.
        \end{itemize}
    
    \end{highlight}
    
    \subsubsection*{Learning Decision Trees from Examples}
    
    The process involves finding a decision tree that fits a set of examples, using a greedy divide-and-conquer strategy to select the most informative attributes.
    
    \begin{highlight}[Learning Decision Trees from Examples]
    
        \begin{itemize}
            \item \textbf{Algorithm (LEARN-DECISION-TREE)}:
            \begin{enumerate}
                \item If all examples have the same classification, return the classification.
                \item If no attributes are left, return the most common classification.
                \item Otherwise, select the attribute with the highest importance and split the examples.
            \end{enumerate}
            \item \textbf{Importance Function}: Measures the attribute's effectiveness in classifying examples, often using information gain.
            \item \textbf{Example}: In the restaurant domain, the algorithm might first split on the attribute "Patrons" and then further split on "Hungry".
        \end{itemize}
    
    \end{highlight}
    
    \subsubsection*{Choosing Attribute Tests}
    
    The choice of attribute for splitting is crucial and is guided by measures like information gain, which is based on the concept of entropy from information theory.
    
    \begin{highlight}[Choosing Attribute Tests]
    
        \begin{itemize}
            \item \textbf{Entropy}: A measure of uncertainty or impurity in the data. For a Boolean variable $V$ with probabilities $P(v_k)$:
            \[
            H(V) = -\sum_k P(v_k) \log_2 P(v_k)
            \]
            \item \textbf{Information Gain}: The reduction in entropy from splitting on an attribute $A$:
            \[
            \text{Gain}(A) = H(\text{Output}) - \text{Remainder}(A)
            \]
            \item \textbf{Example}: In the restaurant domain, the attribute "Patrons" provides significant information gain, making it a good choice for the first split.
        \end{itemize}
    
    \end{highlight}
    
    \subsubsection*{Generalization and Overfitting}
    
    A key challenge in learning decision trees is ensuring the model generalizes well to new data, avoiding overfitting.
    
    \begin{highlight}[Generalization and Overfitting]
    
        \begin{itemize}
            \item \textbf{Overfitting}: Occurs when the model captures noise in the training data, leading to poor generalization.
            \item \textbf{Pruning}: A technique to remove parts of the tree that provide little predictive power, thereby reducing complexity and preventing overfitting.
            \item \textbf{Significance Tests}: Used to decide whether an attribute split is meaningful, often employing statistical methods like the chi-squared test.
            \item \textbf{Example}: In the restaurant domain, attributes like "Raining" or "Reservation" might be pruned if they do not significantly improve classification.
        \end{itemize}
    
    \end{highlight}
    
    \subsubsection*{Broadening the Applicability of Decision Trees}
    
    Decision trees can be adapted to handle missing data, continuous attributes, and even regression tasks, broadening their applicability.
    
    \begin{highlight}[Broadening the Applicability of Decision Trees]
    
        \begin{itemize}
            \item \textbf{Handling Missing Data}: Techniques like assigning a probability distribution to missing values or using surrogate splits.
            \item \textbf{Continuous and Multi-Valued Attributes}: Techniques like using split points for continuous data and information gain ratio for attributes with many values.
            \item \textbf{Regression Trees}: Extend decision trees to predict continuous outputs, using methods like linear regression at the leaves.
            \item \textbf{Example}: Predicting house prices using features like square footage and number of bedrooms, where the output is continuous.
        \end{itemize}
    
    \end{highlight}
    
    \begin{highlight}[Summary of Key Concepts]
    
        \begin{itemize}
            \item \textbf{Decision Trees}: A method for classifying inputs based on a sequence of attribute tests.
            \item \textbf{Expressiveness}: Can represent any Boolean function but may require large trees for complex functions.
            \item \textbf{Learning Algorithm}: Uses a greedy approach to build the tree based on the most informative attributes.
            \item \textbf{Information Gain}: A measure used to select the best attribute for splitting.
            \item \textbf{Generalization}: Ensuring the model performs well on unseen data, often using techniques like pruning.
            \item \textbf{Adaptability}: Can handle various data types and tasks, including regression and handling missing data.
        \end{itemize}
    
        Understanding decision trees and their properties is crucial for applying them effectively in various classification and regression tasks, ensuring that they generalize well to new data while remaining interpretable.
    
    \end{highlight}
\end{notes}

The next section that is being covered from this chapter this week is \textbf{Section 19.4: Model Selection And Optimization}.

\begin{notes}{Section 19.4: Model Selection And Optimization}
    \subsection*{Overview}

    This section discusses the critical aspects of model selection and optimization in machine learning. It covers the concepts of selecting a suitable hypothesis, measuring performance, dealing with 
    overfitting and underfitting, and the various methods and strategies for tuning models and hyperparameters.
    
    \subsubsection*{Stationarity and i.i.d. Assumptions}
    
    Model selection often relies on the assumption that future examples will resemble past ones, described by the stationarity assumption. This assumption is crucial for predicting the model's performance 
    on unseen data.
    
    \begin{highlight}[Stationarity and i.i.d. Assumptions]
    
        \begin{itemize}
            \item \textbf{Stationarity Assumption}: Assumes that the distribution of examples remains consistent over time:
            \[
            P(E_j) = P(E_{j+1}) = \cdots
            \]
            \item \textbf{Independent and Identically Distributed (i.i.d.)}: Assumes that each example $E_j$ is independent of others:
            \[
            P(E_j) = P(E_j|E_{j-1}, E_{j-2}, \ldots)
            \]
        \end{itemize}
    
    \end{highlight}
    
    \subsubsection*{Error Rate and Hypothesis Testing}
    
    The error rate measures the proportion of incorrect predictions made by a model. Model selection involves finding a hypothesis that minimizes this error rate.
    
    \begin{highlight}[Error Rate and Hypothesis Testing]
    
        \begin{itemize}
            \item \textbf{Error Rate}: The proportion of times the predicted output $h(x)$ does not match the true output $y$:
            \[
            \text{Error rate} = \frac{1}{N} \sum_{i=1}^{N} \mathbb{I}(h(x_i) \neq y_i)
            \]
            \item \textbf{Training, Validation, and Test Sets}: The data is typically split into these sets to prevent overfitting and ensure unbiased evaluation.
        \end{itemize}
    
    \end{highlight}
    
    \subsubsection*{Overfitting and Underfitting}
    
    Overfitting occurs when a model learns the noise in the training data, while underfitting happens when a model is too simple to capture the underlying pattern.
    
    \begin{highlight}[Overfitting and Underfitting]
    
        \begin{itemize}
            \item \textbf{Overfitting}: Characterized by a model that performs well on training data but poorly on unseen data.
            \item \textbf{Underfitting}: A model that fails to capture the underlying trend in the data, resulting in high bias.
            \item \textbf{Example}: A decision tree with too many nodes may overfit, while a simple linear model may underfit complex data.
        \end{itemize}
    
    \end{highlight}
    
    \subsubsection*{Cross-Validation}
    
    Cross-validation is a technique used to assess the generalization performance of a model, typically involving partitioning the data into subsets.
    
    \begin{highlight}[Cross-Validation]
    
        \begin{itemize}
            \item \textbf{K-Fold Cross-Validation}: The data is divided into $k$ subsets, with the model trained on $k-1$ subsets and validated on the remaining subset. This process is repeated $k$ 
            times, each subset serving as the validation set once.
            \item \textbf{Leave-One-Out Cross-Validation (LOOCV)}: A special case of k-fold cross-validation where $k = N$, the number of examples.
        \end{itemize}
    
    \end{highlight}
    
    \subsubsection*{Model Complexity and Regularization}
    
    Model complexity affects both the fit and the generalization ability of a model. Regularization techniques are employed to control this complexity.
    
    \begin{highlight}[Model Complexity and Regularization]
    
        \begin{itemize}
            \item \textbf{Regularization}: Introduces a penalty for complexity, often expressed as a term added to the error function:
            \[
            \text{Cost}(h) = \text{EmpLoss}(h) + \lambda \text{Complexity}(h)
            \]
            \item \textbf{Regularization Functions}: Common choices include the sum of squares of coefficients for polynomial models.
            \item \textbf{Example}: Lasso (L1 regularization) and Ridge (L2 regularization) are common techniques that penalize the absolute or squared values of model coefficients.
        \end{itemize}
    
    \end{highlight}
    
    \subsubsection*{Hyperparameter Tuning}
    
    Selecting optimal hyperparameters is crucial for improving model performance. Various techniques are used for tuning, including grid search, random search, and more sophisticated methods like Bayesian 
    optimization.
    
    \begin{highlight}[Hyperparameter Tuning]
    
        \begin{itemize}
            \item \textbf{Grid Search}: Exhaustively searches over a specified set of hyperparameter values.
            \item \textbf{Random Search}: Samples hyperparameter values randomly within a specified range, often more efficient than grid search.
            \item \textbf{Bayesian Optimization}: Treats hyperparameter tuning as an optimization problem, balancing exploration and exploitation.
        \end{itemize}
    
    \end{highlight}
    
    \begin{highlight}[Summary of Key Concepts]
    
        \begin{itemize}
            \item \textbf{Model Selection}: Involves choosing the best hypothesis based on performance metrics like error rate.
            \item \textbf{Overfitting and Underfitting}: Critical challenges in machine learning, managed through techniques like cross-validation and regularization.
            \item \textbf{Cross-Validation}: A robust method for assessing a model's generalization ability.
            \item \textbf{Regularization}: Helps control model complexity, preventing overfitting.
            \item \textbf{Hyperparameter Tuning}: Essential for optimizing model performance, with methods ranging from simple grid search to advanced Bayesian optimization.
        \end{itemize}
    
        Understanding these aspects of model selection and optimization is crucial for developing robust, generalizable machine learning models that perform well on unseen data.
    
    \end{highlight}
\end{notes}

The next section that is being covered from this chapter this week is \textbf{Section 19.5: The Theory Of Learning}.

\begin{notes}{Section 19.5: The Theory Of Learning}
    \subsection*{Overview}

    This section delves into the theoretical foundations of learning in artificial intelligence, focusing on the conditions under which learned hypotheses can be expected to generalize well to new data. 
    It discusses key concepts such as Probably Approximately Correct (PAC) learning, sample complexity, and the PAC learning model's application to decision lists.
    
    \subsubsection*{Probably Approximately Correct (PAC) Learning}
    
    PAC learning is a framework that provides a probabilistic guarantee on the accuracy of learned hypotheses, assuming the examples are drawn from a stationary distribution.
    
    \begin{highlight}[Probably Approximately Correct (PAC) Learning]
    
        \begin{itemize}
            \item \textbf{PAC Learning Model}: A hypothesis $h$ is said to be probably approximately correct if, with high probability, it has a low generalization error:
            \[
            \text{error}(h) = \sum_{x,y} L_{0/1}(y, h(x)) P(x,y)
            \]
            where $L_{0/1}(y, h(x))$ is the 0/1 loss function, indicating a misclassification.
            \item \textbf{Generalization Error}: The expected error over the distribution of all possible examples, not just the training set.
            \item \textbf{PAC Condition}: A hypothesis $h$ is approximately correct if its error is at most $\epsilon$, and the probability that all consistent hypotheses are approximately correct is at least $1 - \delta$.
        \end{itemize}
    
    \end{highlight}
    
    \subsubsection*{Sample Complexity}
    
    Sample complexity refers to the number of training examples required to ensure a hypothesis is probably approximately correct.
    
    \begin{highlight}[Sample Complexity]
    
        \begin{itemize}
            \item \textbf{Sample Complexity Bound}: The number of samples $N$ needed depends on the desired error bound $\epsilon$ and confidence level $1 - \delta$:
            \[
            N \geq \frac{1}{\epsilon} \left(\ln \frac{1}{\delta} + \ln |H|\right)
            \]
            where $|H|$ is the size of the hypothesis space.
            \item \textbf{Implications}: As the hypothesis space grows, the number of examples needed increases, making it crucial to balance complexity and generalization.
        \end{itemize}
    
    \end{highlight}
    
    \subsubsection*{Hypothesis Space and PAC Learning}
    
    The hypothesis space $H$ defines the set of all hypotheses the learning algorithm can consider. The complexity of this space impacts the PAC learnability of a problem.
    
    \begin{highlight}[Hypothesis Space and PAC Learning]
    
        \begin{itemize}
            \item \textbf{VC Dimension}: A measure of the capacity of a hypothesis space, indicating the largest number of points that can be shattered (classified in all possible ways) by the hypotheses in the space.
            \item \textbf{Relation to Sample Complexity}: The VC dimension helps determine the sample complexity; higher VC dimensions generally require more samples for reliable learning.
            \item \textbf{Example}: The class of linear functions in two dimensions has a VC dimension of 3, as three points can be arranged in such a way that they can be classified in all possible ways by linear functions.
        \end{itemize}
    
    \end{highlight}
    
    \subsubsection*{Decision Lists and PAC Learning}
    
    Decision lists are a specific type of hypothesis space that is PAC-learnable under certain conditions, offering a practical example of the PAC framework.
    
    \begin{highlight}[Decision Lists and PAC Learning]
    
        \begin{itemize}
            \item \textbf{Decision List}: A series of tests (conjunctions of literals) where each test leads to a decision if it matches the input, otherwise proceeding to the next test.
            \item \textbf{Learning Algorithm}: The DECISION-LIST-LEARNING algorithm constructs a decision list by selecting tests that match subsets of training examples, ensuring consistency with the examples.
            \item \textbf{PAC Learnability}: For k-DL, the class of decision lists with up to k literals per test, the sample complexity is polynomial in the number of attributes $n$ and k:
            \[
            N \geq \frac{1}{\epsilon} \left(\ln \frac{1}{\delta} + O(nk \log_2(nk))\right)
            \]
        \end{itemize}
    
    \end{highlight}
    
    \begin{highlight}[Summary of Key Concepts]
    
        \begin{itemize}
            \item \textbf{PAC Learning}: Provides a framework for understanding the conditions under which learning algorithms generalize well.
            \item \textbf{Sample Complexity}: The number of training examples needed to achieve a given accuracy and confidence.
            \item \textbf{VC Dimension}: A measure of the capacity of the hypothesis space, influencing sample complexity.
            \item \textbf{Decision Lists}: A practical example of a hypothesis space that can be PAC-learned with a reasonable number of examples.
        \end{itemize}
    
        Understanding these theoretical foundations is crucial for designing learning algorithms that not only fit training data but also generalize well to unseen examples, ensuring robust and reliable performance in real-world applications.
    
    \end{highlight}
\end{notes}

The next section that is being covered from this chapter this week is \textbf{Section 19.6: Linear Regression And Classification}.

\begin{notes}{Section 19.6: Linear Regression And Classification}
    \subsection*{Overview}

    This section introduces linear regression and classification, fundamental techniques in machine learning. It covers the basic concepts of univariate and multivariable linear regression, gradient descent, 
    and extends these concepts to linear classifiers, including logistic regression.
    
    \subsubsection*{Univariate Linear Regression}
    
    Univariate linear regression involves fitting a straight line to a set of data points, finding the best line that minimizes the squared errors between the predicted and actual values.
    
    \begin{highlight}[Univariate Linear Regression]
    
        \begin{itemize}
            \item \textbf{Model}: The model is represented by $y = w_1x + w_0$, where $w_0$ and $w_1$ are the coefficients to be learned.
            \item \textbf{Objective}: Minimize the empirical loss using the squared-error loss function:
            \[
            \text{Loss}(h_w) = \sum_{j=1}^{N} (y_j - h_w(x_j))^2 = \sum_{j=1}^{N} (y_j - (w_1x_j + w_0))^2
            \]
            \item \textbf{Optimal Weights}: The weights $w_0$ and $w_1$ that minimize the loss can be found using:
            \[
            w_1 = \frac{N\sum x_jy_j - (\sum x_j)(\sum y_j)}{N\sum x_j^2 - (\sum x_j)^2}
            \]
            \[
            w_0 = \frac{\sum y_j - w_1\sum x_j}{N}
            \]
        \end{itemize}
    
    \end{highlight}
    
    \subsubsection*{Gradient Descent}
    
    Gradient descent is an optimization technique used to find the minimum of a function. It is particularly useful in linear regression for finding the best-fitting line.
    
    \begin{highlight}[Gradient Descent]
    
        \begin{itemize}
            \item \textbf{Process}: Iteratively update weights to reduce the loss:
            \[
            w_i \leftarrow w_i - \alpha \frac{\partial}{\partial w_i} \text{Loss}(w)
            \]
            \item \textbf{Learning Rate ($\alpha$)}: Controls the step size during each update. It can be fixed or decay over time.
            \item \textbf{Stochastic Gradient Descent (SGD)}: Uses a small random subset of data (minibatch) to update the weights, making the process faster and more efficient.
        \end{itemize}
    
    \end{highlight}
    
    \subsubsection*{Multivariable Linear Regression}
    
    Extends univariate linear regression to multiple input variables, where the output is a linear combination of multiple input features.
    
    \begin{highlight}[Multivariable Linear Regression]
    
        \begin{itemize}
            \item \textbf{Model}: $h_w(x) = w_0 + \sum_{i} w_i x_i$ or in vector form $h_w(x) = w^T x$.
            \item \textbf{Optimal Weights}: Determined using:
            \[
            w^* = (X^TX)^{-1}X^Ty
            \]
            where $X$ is the matrix of input features and $y$ is the vector of target values.
            \item \textbf{Regularization}: Includes additional terms in the loss function to prevent overfitting, such as L1 (Lasso) or L2 (Ridge) regularization:
            \[
            \text{Cost}(h) = \text{EmpLoss}(h) + \lambda \text{Complexity}(h)
            \]
        \end{itemize}
    
    \end{highlight}
    
    \subsubsection*{Linear Classification with Hard and Soft Thresholds}
    
    Linear functions can also be used for classification tasks, distinguishing between classes using a decision boundary.
    
    \begin{highlight}[Linear Classification with Hard and Soft Thresholds]
    
        \begin{itemize}
            \item \textbf{Hard Threshold (Perceptron)}: Classifies inputs as 0 or 1 based on whether $w^T x$ is above or below a threshold.
            \item \textbf{Soft Threshold (Logistic Regression)}: Uses the logistic function to output probabilities, providing a more nuanced classification:
            \[
            \text{Logistic}(z) = \frac{1}{1 + e^{-z}}
            \]
            \[
            h_w(x) = \text{Logistic}(w^T x)
            \]
            \item \textbf{Gradient Descent in Logistic Regression}: The weights are updated based on the derivative of the logistic function:
            \[
            w_i \leftarrow w_i + \alpha (y - h_w(x))h_w(x)(1 - h_w(x))x_i
            \]
        \end{itemize}
    
    \end{highlight}
    
    \begin{highlight}[Summary of Key Concepts]
    
        \begin{itemize}
            \item \textbf{Linear Regression}: Involves fitting a linear model to data to predict continuous outcomes.
            \item \textbf{Gradient Descent}: An optimization technique for finding the best-fit parameters in regression models.
            \item \textbf{Multivariable Linear Regression}: Extends univariate regression to multiple input features.
            \item \textbf{Regularization}: Prevents overfitting by adding penalty terms to the loss function.
            \item \textbf{Linear Classification}: Uses linear models for classification tasks, with hard thresholds (Perceptron) and soft thresholds (Logistic Regression).
        \end{itemize}
    
        These concepts form the foundation of many machine learning models, providing tools for both regression and classification tasks, with applications ranging from predictive analytics to decision-making systems.
    
    \end{highlight}
\end{notes}

The next section that is being covered from this chapter this week is \textbf{Section 19.7: Nonparametric Models}.

\begin{notes}{Section 19.7: Nonparametric Models}
    \subsection*{Overview}

    This section delves into the theory of learning, focusing on the key concepts of generalization, overfitting, and the theoretical frameworks that guide learning algorithms. It explores the challenges 
    of ensuring that a learned hypothesis performs well not just on training data but also on new, unseen data. The section also introduces key concepts such as Probably Approximately Correct (PAC) 
    learning and discusses methods for balancing model complexity with the need for accurate predictions.
    
    \subsubsection*{Generalization and Overfitting}
    
    Generalization refers to a model's ability to perform well on new, unseen data. Overfitting occurs when a model is too closely tailored to the training data, capturing noise rather than the underlying 
    data distribution.
    
    \begin{highlight}[Generalization and Overfitting]
    
        \begin{itemize}
            \item \textbf{Generalization}: The capacity of a model to predict accurately on new data that was not used during training.
            \item \textbf{Overfitting}: When a model learns not only the underlying patterns in the training data but also the noise, leading to poor generalization.
            \item \textbf{Trade-off}: Balancing the complexity of the model with its ability to generalize. Simpler models may underfit, while more complex models may overfit.
            \item \textbf{Example}: A high-degree polynomial may fit the training data perfectly but fail to generalize to new data points.
        \end{itemize}
    
    \end{highlight}
    
    \subsubsection*{PAC Learning}
    
    PAC (Probably Approximately Correct) learning is a framework that formalizes the conditions under which a learning algorithm can generalize well from a limited sample of training data.
    
    \begin{highlight}[PAC Learning]
    
        \begin{itemize}
            \item \textbf{Definition}: A learning algorithm is PAC if, for any hypothesis space, it can find a hypothesis that with high probability (1 - δ) is approximately correct (error ≤ ϵ) after seeing a 
            polynomial number of examples.
            \item \textbf{Sample Complexity}: The number of training examples required to ensure that with high probability, the learned hypothesis has a small generalization error. Given by:
            \[
            N \geq \frac{1}{\epsilon} \left( \ln \frac{1}{\delta} + \ln |H| \right)
            \]
            \item \textbf{Hypothesis Space}: The set of all hypotheses that can be learned by the algorithm. The size and complexity of the hypothesis space directly impact the number of examples needed 
            for learning.
        \end{itemize}
    
    \end{highlight}
    
    \subsubsection*{VC Dimension}
    
    The Vapnik-Chervonenkis (VC) dimension is a measure of the capacity of a statistical model, defined as the maximum number of points that can be shattered (i.e., correctly classified in all possible ways) 
    by the model.
    
    \begin{highlight}[VC Dimension]
    
        \begin{itemize}
            \item \textbf{Definition}: The VC dimension of a model class is the size of the largest set of points that can be shattered by the model.
            \item \textbf{Significance}: A higher VC dimension indicates a more complex model that can capture more intricate patterns but also risks overfitting.
            \item \textbf{Bound on Generalization Error}: The VC dimension provides a bound on the generalization error, helping in the analysis of model performance.
            \item \textbf{Example}: For a linear classifier in a 2D space, the VC dimension is 3, as it can shatter any three points.
        \end{itemize}
    
    \end{highlight}
    
    \subsubsection*{Regularization and Model Selection}
    
    Regularization techniques are used to prevent overfitting by adding a penalty for more complex models. Model selection involves choosing the best model from a set of candidates based on performance 
    criteria.
    
    \begin{highlight}[Regularization and Model Selection]
    
        \begin{itemize}
            \item \textbf{Regularization}: Introduces a penalty term to the loss function to discourage complex models:
            \[
            \text{Cost}(h) = \text{EmpLoss}(h) + \lambda \text{Complexity}(h)
            \]
            where $\lambda$ controls the trade-off between fitting the data and model complexity.
            \item \textbf{Model Selection}: The process of selecting a hypothesis from a set of hypotheses that best matches the true function. This involves balancing the trade-off between bias and variance.
            \item \textbf{Cross-Validation}: A technique used to evaluate model performance by dividing the data into training and validation sets. Common methods include k-fold cross-validation.
        \end{itemize}
    
    \end{highlight}
    
    \begin{highlight}[Summary of Key Concepts]
    
        \begin{itemize}
            \item \textbf{Generalization}: The ability of a model to perform well on new data.
            \item \textbf{Overfitting}: When a model captures noise in the training data, leading to poor performance on new data.
            \item \textbf{PAC Learning}: A theoretical framework that ensures a model's accuracy with high probability after a sufficient number of examples.
            \item \textbf{VC Dimension}: A measure of a model's capacity to classify data points.
            \item \textbf{Regularization}: Techniques to prevent overfitting by adding penalties for complexity.
            \item \textbf{Model Selection}: The process of choosing the most appropriate model based on performance.
        \end{itemize}
    
        Understanding these concepts is crucial for developing robust and accurate learning algorithms that generalize well across different datasets and conditions.
    
    \end{highlight}
\end{notes}

The last section that is being covered from this chapter this week is \textbf{Section 19.8: Ensemble Learning}.

\begin{notes}{Section 19.8: Ensemble Learning}
    \subsection*{Overview}

    Ensemble learning involves combining multiple models, called base models, to create a more robust and accurate predictive model. This section explores different ensemble methods, including bagging, 
    random forests, stacking, boosting, and gradient boosting. Ensemble methods can reduce both bias and variance, improving the model's generalization performance.
    
    \subsubsection*{Bagging}
    
    Bagging (Bootstrap Aggregating) involves training multiple versions of a model on different random subsets of the training data, sampled with replacement. The final prediction is made by averaging 
    (prediction) or voting (classification) the outputs of the individual models.
    
    \begin{highlight}[Bagging]
    
        \begin{itemize}
            \item \textbf{Process}: Generate $K$ training sets by sampling with replacement from the original dataset. Train $K$ models and combine their predictions.
            \item \textbf{Application}: Particularly useful for models like decision trees that can vary significantly with small changes in the training data.
            \item \textbf{Formula}: For regression, the final output is:
            \[
            h(x) = \frac{1}{K} \sum_{i=1}^{K} h_i(x)
            \]
        \end{itemize}
    
    \end{highlight}
    
    \subsubsection*{Random Forests}
    
    Random forests are an extension of bagging applied to decision trees, where additional randomness is introduced by selecting a random subset of features for each split in the decision tree.
    
    \begin{highlight}[Random Forests]
    
        \begin{itemize}
            \item \textbf{Randomness in Feature Selection}: At each split, a random subset of features is selected, and the best feature from this subset is chosen.
            \item \textbf{Ensemble of Decision Trees}: Multiple decision trees are grown, each using a random subset of the training data and features.
            \item \textbf{Advantages}: Reduces variance and helps prevent overfitting, as different trees may capture different aspects of the data.
        \end{itemize}
    
    \end{highlight}
    
    \subsubsection*{Stacking}
    
    Stacking, or stacked generalization, involves training multiple models (potentially of different types) and then training a meta-model to combine their predictions.
    
    \begin{highlight}[Stacking]
    
        \begin{itemize}
            \item \textbf{Process}: Train base models on the original data and use their outputs as inputs for the meta-model, which learns to combine these outputs.
            \item \textbf{Meta-Model}: The meta-model can be any machine learning model, including linear regression or neural networks.
            \item \textbf{Benefits}: Can improve performance by leveraging the strengths of different types of models.
        \end{itemize}
    
    \end{highlight}
    
    \subsubsection*{Boosting}
    
    Boosting combines multiple weak learners to create a strong learner. It focuses on examples that previous models misclassified, adjusting the weights of these examples in subsequent iterations.
    
    \begin{highlight}[Boosting]
    
        \begin{itemize}
            \item \textbf{Weighted Training Sets}: Each example is assigned a weight, and weights are adjusted to focus on difficult-to-classify examples.
            \item \textbf{Algorithm (e.g., ADABOOST)}: Train weak learners sequentially, each focusing on the mistakes of the previous ones. The final model is a weighted combination of all weak learners.
            \item \textbf{Formula}: For binary classification, the final prediction is:
            \[
            h(x) = \text{sign}\left(\sum_{i=1}^{K} \alpha_i h_i(x)\right)
            \]
            where $\alpha_i$ is the weight of the $i$-th learner.
        \end{itemize}
    
    \end{highlight}
    
    \subsubsection*{Gradient Boosting}
    
    Gradient boosting is an advanced form of boosting that optimizes a loss function by adding models in a stage-wise fashion, minimizing the residual errors of the combined model.
    
    \begin{highlight}[Gradient Boosting]
    
        \begin{itemize}
            \item \textbf{Optimization}: Uses gradient descent to minimize the loss function of the ensemble.
            \item \textbf{Component Models}: Often uses decision trees as the base models. The new model added at each stage fits to the negative gradient of the loss function.
            \item \textbf{Popular Implementations}: Includes algorithms like XGBoost, which are widely used in practice for their efficiency and performance.
        \end{itemize}
    
    \end{highlight}
    
    \begin{highlight}[Summary of Key Concepts]
    
        \begin{itemize}
            \item \textbf{Bagging}: Reduces variance by averaging over multiple models trained on different data subsets.
            \item \textbf{Random Forests}: Adds feature randomness to bagging, improving robustness and reducing overfitting.
            \item \textbf{Stacking}: Combines different model types using a meta-model to enhance predictive performance.
            \item \textbf{Boosting}: Focuses on difficult examples by adjusting weights, creating a strong learner from weak learners.
            \item \textbf{Gradient Boosting}: Uses gradient descent to optimize an ensemble, particularly effective with decision trees.
        \end{itemize}
    
        Ensemble learning enhances model performance and robustness by combining multiple models, leveraging their collective strengths, and mitigating individual weaknesses.
    
    \end{highlight}
\end{notes}