\clearpage

\renewcommand{\ChapTitle}{Bayes Network}
\renewcommand{\SectionTitle}{Bayes Network}

\chapter{\ChapTitle}

\section{\SectionTitle}
\horizontalline{0}{0}

\subsection{Assigned Reading}

The reading for this week is from, \AITextbook \hspace*{1pt} and \RLTextbook.

\begin{itemize}
    \item \textbf{Artificial Intelligence - A Modern Approach - Chapter 13.1 - Representing Knowledge In An Uncertain Domain}
    \item \textbf{Artificial Intelligence - A Modern Approach - Chapter 13.2 - The Semantics Of Bayesian Networks}
    \item \textbf{Artificial Intelligence - A Modern Approach - Chapter 13.3 - Exact Inference In Bayesian Networks}
\end{itemize}

\subsection{Piazza}

Must post at least \textbf{three} times this week to Piazza.

\subsection{Lectures}

The lectures for this week are:

\begin{itemize}
    \item \lecture{https://www.youtube.com/watch?v=ayGGK5TcoqY}{Probabilistic Reasoning 1 - Probability Review}{69}
    \item \lecture{https://www.youtube.com/watch?v=G4Ly02ydaj8}{Bayes Net - Representation}{34}
    \item \lecture{https://www.youtube.com/watch?v=nEhzDUYn1FI}{Bayes Net - Independence}{32}
    \item \lecture{https://www.youtube.com/watch?v=_iYrM0Y7D2w}{Bayes Net - Inference}{50}
\end{itemize}

\noindent The lecture notes for this week are:

\begin{itemize}
    \item \pdflink{\LecNoteDir/Notes/Bayes Net - Independence Lecture Notes.pdf}{Bayes Net - Independence Lecture Notes}
    \item \pdflink{\LecNoteDir/Notes/Bayes Net - Inference Lecture Notes.pdf}{Bayes Net - Inference Lecture Notes}
    \item \pdflink{\LecNoteDir/Notes/Bayes Net - Representation Lecture Notes.pdf}{Bayes Net - Representation Lecture Notes}
\end{itemize}

\subsection{Quiz}

The quiz for this week is:

\begin{itemize}
    \item \pdflink{\QuizDir Quiz 9 - Bayes Network.pdf}{Quiz 9 - Bayes Network}
\end{itemize}

\subsection{Chapter Summary}

This weeks material is from \textbf{Chapter 13: Probabilistic Reasoning}. The first section that is covered from this chapter this week is \textbf{Section 13.1: Representing Knowledge In An Uncertain Domain}.

\begin{notes}{Section 13.1: Representing Knowledge In An Uncertain Domain}
    \subsection*{Overview}

    This section introduces Bayesian networks as a powerful tool for representing and reasoning with probabilistic knowledge in uncertain domains. It explains the limitations of full joint probability 
    distributions and the advantages of using Bayesian networks to capture conditional independence relationships efficiently.
    
    \subsubsection*{Bayesian Networks}
    
    Bayesian networks are graphical models that use directed acyclic graphs (DAGs) to represent the dependencies among variables. Each node in the network represents a random variable, and the directed 
    edges between nodes represent probabilistic dependencies.
    
    \begin{highlight}[Bayesian Networks]
    
        \begin{itemize}
            \item \textbf{Nodes and Edges}: Each node corresponds to a random variable, which can be either discrete or continuous. Directed edges indicate a dependency between the connected nodes.
            \item \textbf{Directed Acyclic Graph (DAG)}: The structure of the network is a DAG, meaning it has no directed cycles. This hierarchical structure allows for a clear representation of causal relationships.
            \item \textbf{Conditional Probability Table (CPT)}: Each node $X_i$ is associated with a CPT $P(X_i | \text{Parents}(X_i))$, which quantifies the effect of its parents on the node. The CPT 
            provides the probability distribution of the node given its parents' values.
            \item \textbf{Example}: Consider a network with nodes for Burglary ($B$), Earthquake ($E$), Alarm ($A$), JohnCalls ($J$), and MaryCalls ($M$). The edges from $B$ and $E$ to $A$, 
            and from $A$ to $J$ and $M$, capture the dependencies among these variables.
        \end{itemize}
    
    \end{highlight}
    
    \subsubsection*{Constructing Bayesian Networks}
    
    Constructing a Bayesian network involves determining the variables, their dependencies, and the corresponding conditional probabilities. The process typically follows these steps:
    
    \begin{highlight}[Constructing Bayesian Networks]
    
        \begin{itemize}
            \item \textbf{Identifying Variables}: Determine the set of relevant variables for the domain.
            \item \textbf{Ordering Variables}: Order the variables in a sequence where causes precede effects.
            \item \textbf{Adding Links}: For each variable, identify its parents from the preceding variables in the order. Add directed edges from the parents to the variable.
            \item \textbf{Specifying CPTs}: Define the conditional probability table for each variable given its parents.
            \item \textbf{Example}: In a burglary scenario, the variable "Alarm" might depend on "Burglary" and "Earthquake," with a CPT that specifies the probability of the alarm going off given the 
            occurrence of a burglary or earthquake.
        \end{itemize}
    
    \end{highlight}
    
    \subsubsection*{Mathematical Representation}
    
    The joint probability distribution of a set of variables in a Bayesian network can be factored into a product of conditional probabilities, leveraging the conditional independence relationships.
    
    \begin{highlight}[Mathematical Representation]
    
        \begin{itemize}
            \item \textbf{Chain Rule for Bayesian Networks}: The joint probability $P(X_1, X_2, \ldots, X_n)$ of the variables can be written as:
            \[
            P(X_1, X_2, \ldots, X_n) = \prod_{i=1}^n P(X_i | \text{Parents}(X_i))
            \]
            \item \textbf{Example}: For the Burglary example, the joint probability is:
            \[
            P(B, E, A, J, M) = P(B) P(E) P(A | B, E) P(J | A) P(M | A)
            \]
        \end{itemize}
    
    \end{highlight}
    
    \subsubsection*{Advantages of Bayesian Networks}
    
    Bayesian networks offer several advantages over full joint probability distributions, particularly in handling large and complex domains.
    
    \begin{highlight}[Advantages of Bayesian Networks]
    
        \begin{itemize}
            \item \textbf{Compact Representation}: By exploiting conditional independence, Bayesian networks require fewer parameters to represent the joint distribution.
            \item \textbf{Ease of Specification}: Experts can often specify the local relationships (CPTs) more easily than the full joint distribution.
            \item \textbf{Efficient Inference}: Algorithms can leverage the network structure to perform probabilistic inference more efficiently.
            \item \textbf{Scalability}: Suitable for large-scale applications due to their compactness and efficiency.
            \item \textbf{Example}: A network with 30 binary variables, each having up to 5 parents, requires significantly fewer parameters (960) compared to the full joint distribution (over a billion parameters).
        \end{itemize}
    
    \end{highlight}
    
    \subsubsection*{Applications of Bayesian Networks}
    
    Bayesian networks are widely used in various fields, including medical diagnosis, risk assessment, and decision support systems. They provide a systematic way to model uncertainty and make informed 
    decisions.
    
    \begin{highlight}[Applications of Bayesian Networks]
    
        \begin{itemize}
            \item \textbf{Medical Diagnosis}: Used to model the probabilistic relationships between diseases and symptoms, aiding in diagnostic reasoning.
            \item \textbf{Risk Assessment}: Employed to evaluate the likelihood of different risks and their impacts in fields such as finance and engineering.
            \item \textbf{Decision Support}: Assist in making decisions under uncertainty by modeling the dependencies among various factors and their probable outcomes.
            \item \textbf{Example}: A Bayesian network in medical diagnosis might include variables for different diseases, symptoms, and test results, with edges representing causal relationships and 
            CPTs providing the conditional probabilities.
        \end{itemize}
    
    \end{highlight}
    
    \begin{highlight}[Summary of Key Concepts]
    
        \begin{itemize}
            \item \textbf{Bayesian Networks}: Graphical models that represent probabilistic dependencies using directed acyclic graphs (DAGs).
            \item \textbf{Nodes and Edges}: Nodes represent random variables, and directed edges indicate dependencies.
            \item \textbf{Conditional Probability Tables (CPTs)}: Quantify the effect of parent nodes on each variable.
            \item \textbf{Constructing Bayesian Networks}: Involves identifying variables, ordering them, adding links, and specifying CPTs.
            \item \textbf{Mathematical Representation}: The joint probability distribution is factored into a product of conditional probabilities.
            \item \textbf{Advantages}: Offer compact representation, ease of specification, efficient inference, and scalability.
            \item \textbf{Applications}: Used in medical diagnosis, risk assessment, decision support, and more.
        \end{itemize}
    
        Understanding Bayesian networks is crucial for modeling and reasoning under uncertainty, enabling the development of robust and efficient probabilistic models for various applications.
    
    \end{highlight}
\end{notes}

The next section that is covered from this chapter this week is \textbf{Section 13.2: The Semantics Of Bayesian Networks}.

\begin{notes}{Section 13.2: The Semantics Of Bayesian Networks}
    \subsection*{Overview}

    This section delves into the semantics of Bayesian networks, explaining how the structure of the network encodes probabilistic relationships among variables. It covers the formal interpretation of 
    nodes and edges, and how to understand the joint probability distribution represented by a Bayesian network.
    
    \subsubsection*{Conditional Independence}
    
    The key idea behind Bayesian networks is the representation of conditional independence among variables. The network structure encodes these dependencies, simplifying the representation of the joint 
    probability distribution.
    
    \begin{highlight}[Conditional Independence]
    
        \begin{itemize}
            \item \textbf{Definition}: A variable $X_i$ is conditionally independent of its non-descendants, given its parents.
            \item \textbf{Mathematical Formulation}: For variables $X, Y, Z$, $X$ is conditionally independent of $Y$ given $Z$ if:
            \[
            P(X | Y, Z) = P(X | Z)
            \]
            \item \textbf{Example}: In the Burglary network, JohnCalls and MaryCalls are conditionally independent given Alarm.
        \end{itemize}
    
    \end{highlight}
    
    \subsubsection*{Local Semantics}
    
    The local semantics of a Bayesian network specify that each node is independent of its non-parents given its parents. This local property leads to the global property of the joint distribution.
    
    \begin{highlight}[Local Semantics]
    
        \begin{itemize}
            \item \textbf{Node Independence}: Each node is conditionally independent of its non-parents given its parents.
            \item \textbf{Mathematical Expression}: For a node $X_i$ with parents $\text{Parents}(X_i)$:
            \[
            P(X_i | \text{NonParents}(X_i), \text{Parents}(X_i)) = P(X_i | \text{Parents}(X_i))
            \]
            \item \textbf{Example}: In the Burglary network, $P(\text{JohnCalls} | \text{Burglary}, \text{Earthquake}, \text{Alarm}) = P(\text{JohnCalls} | \text{Alarm})$.
        \end{itemize}
    
    \end{highlight}
    
    \subsubsection*{Global Semantics}
    
    The global semantics of a Bayesian network describe the joint probability distribution over all variables as the product of the local conditional probabilities defined by the network structure.
    
    \begin{highlight}[Global Semantics]
    
        \begin{itemize}
            \item \textbf{Joint Probability Distribution}: The joint probability of all variables is the product of the conditional probabilities of each variable given its parents:
            \[
            P(X_1, X_2, \ldots, X_n) = \prod_{i=1}^n P(X_i | \text{Parents}(X_i))
            \]
            \item \textbf{Example}: For the Burglary network, the joint probability is:
            \[
            P(B, E, A, J, M) = P(B) P(E) P(A | B, E) P(J | A) P(M | A)
            \]
        \end{itemize}
    
    \end{highlight}
    
    \subsubsection*{d-Separation}
    
    d-Separation is a graphical criterion used to determine whether a set of variables is conditionally independent given another set of variables. It provides a method for reading conditional 
    independencies directly from the network structure.
    
    \begin{highlight}[d-Separation]
    
        \begin{itemize}
            \item \textbf{Definition}: A set of nodes $X$ is d-separated from a set of nodes $Y$ given a set of nodes $Z$ if all paths from any node in $X$ to any node in $Y$ are blocked 
            by $Z$.
            \item \textbf{Blocking Criteria}: A path is blocked if:
                \begin{itemize}
                    \item It contains a node such that one of its arcs is directed towards the node and neither the node nor any of its descendants are in $Z$.
                    \item It contains a collider (a node with two incoming arcs) that is not in $Z$ and has no descendants in $Z$.
                \end{itemize}
            \item \textbf{Example}: In the Burglary network, JohnCalls is d-separated from Earthquake given Alarm.
        \end{itemize}
    
    \end{highlight}
    
    \subsubsection*{Markov Blanket}
    
    The Markov blanket of a node in a Bayesian network is the minimal set of nodes that renders the node conditionally independent of the rest of the network. It consists of the node's parents, children, 
    and children's parents.
    
    \begin{highlight}[Markov Blanket]
    
        \begin{itemize}
            \item \textbf{Definition}: The Markov blanket of a node $X$ includes:
                \begin{itemize}
                    \item The parents of $X$.
                    \item The children of $X$.
                    \item The parents of the children of $X$.
                \end{itemize}
            \item \textbf{Mathematical Formulation}: Given the Markov blanket $\text{MB}(X)$ of $X$:
            \[
            P(X | \text{All nodes except } X) = P(X | \text{MB}(X))
            \]
            \item \textbf{Example}: In the Burglary network, the Markov blanket of Alarm includes Burglary, Earthquake, JohnCalls, and MaryCalls.
        \end{itemize}
    
    \end{highlight}
    
    \begin{highlight}[Summary of Key Concepts]
    
        \begin{itemize}
            \item \textbf{Conditional Independence}: Encodes dependencies among variables, reducing complexity.
            \item \textbf{Local Semantics}: Each node is conditionally independent of its non-parents given its parents.
            \item \textbf{Global Semantics}: The joint probability distribution is the product of local conditional probabilities.
            \item \textbf{d-Separation}: A criterion to determine conditional independence from the network structure.
            \item \textbf{Markov Blanket}: The minimal set of nodes that renders a node conditionally independent of the rest of the network.
        \end{itemize}
    
        Understanding the semantics of Bayesian networks is crucial for accurately modeling probabilistic dependencies and performing efficient inference in complex domains.
    
    \end{highlight}
\end{notes}

The last section that is covered from this chapter this week is \textbf{Section 13.3: Exact Inference In Bayesian Networks}.

\begin{notes}{Section 13.3: Exact Inference In Bayesian Networks}
    \subsection*{Overview}

    This section covers exact inference methods in Bayesian networks, which involve computing the posterior distribution of a set of query variables given some evidence. It discusses several algorithms 
    for performing exact inference, including enumeration, variable elimination, and belief propagation.
    
    \subsubsection*{Inference by Enumeration}
    
    Inference by enumeration involves summing over the joint probability distribution to compute the desired posterior probabilities. While straightforward, it can be computationally expensive for large 
    networks.
    
    \begin{highlight}[Inference by Enumeration]
    
        \begin{itemize}
            \item \textbf{Posterior Probability}: To compute $P(X | e)$ for query variable $X$ and evidence $e$, sum over all possible values of the hidden variables $Y$:
            \[
            P(X | e) = \alpha P(X, e) = \alpha \sum_{Y} P(X, Y, e)
            \]
            where $\alpha$ is a normalization constant.
            \item \textbf{Example}: In the Burglary network, to find $P(\text{Burglary} | \text{JohnCalls} = \text{true})$, sum over all possible values of Earthquake, Alarm, and MaryCalls.
        \end{itemize}
    
    \end{highlight}
    
    \subsubsection*{Variable Elimination}
    
    Variable elimination is an efficient method for exact inference that systematically eliminates variables by summing out their probabilities, reducing the complexity compared to enumeration.
    
    \begin{highlight}[Variable Elimination]
    
        \begin{itemize}
            \item \textbf{Procedure}: 
                \begin{itemize}
                    \item Express the joint probability distribution as a product of factors.
                    \item Eliminate variables one by one by summing them out, combining factors as necessary.
                \end{itemize}
            \item \textbf{Mathematical Formulation}: For a query $P(X | e)$, where $Z$ are the variables to be eliminated:
            \[
            P(X | e) = \alpha \sum_{Z} \prod_{i} f_i
            \]
            \item \textbf{Example}: In the Burglary network, to compute $P(\text{JohnCalls} = \text{true})$, eliminate variables Earthquake, Burglary, Alarm, and MaryCalls in sequence.
        \end{itemize}
    
    \end{highlight}
    
    \subsubsection*{Belief Propagation}
    
    Belief propagation (also known as the sum-product algorithm) is an exact inference method for tree-structured Bayesian networks. It passes messages between nodes to compute marginal probabilities.
    
    \begin{highlight}[Belief Propagation]
    
        \begin{itemize}
            \item \textbf{Message Passing}: Nodes send messages to their neighbors containing summarized information from the rest of the network.
            \item \textbf{Procedure}: 
                \begin{itemize}
                    \item Initialize messages at the leaf nodes.
                    \item Pass messages inward to a root node.
                    \item Pass messages outward from the root node to compute marginals.
                \end{itemize}
            \item \textbf{Mathematical Formulation}: For a node $X$, the message $\mu_{Y \to X}(X)$ from node $Y$ to node $X$ is:
            \[
            \mu_{Y \to X}(X) = \sum_{Y} P(Y | \text{Parents}(Y)) \prod_{Z \in \text{Children}(Y) \setminus X} \mu_{Z \to Y}(Y)
            \]
            \item \textbf{Example}: In a chain-structured network, belief propagation efficiently computes marginals by passing messages along the chain.
        \end{itemize}
    
    \end{highlight}
    
    \subsubsection*{Complexity of Exact Inference}
    
    The complexity of exact inference in Bayesian networks depends on the network structure and the chosen method. In general, exact inference is NP-hard, but certain structures allow for more efficient 
    computation.
    
    \begin{highlight}[Complexity of Exact Inference]
    
        \begin{itemize}
            \item \textbf{Treewidth}: The complexity of variable elimination and belief propagation is exponential in the treewidth of the network. The treewidth is a measure of how tree-like the network is.
            \item \textbf{Efficient Structures}: For tree-structured networks, belief propagation is efficient. For networks with low treewidth, variable elimination is feasible.
            \item \textbf{Intractable Cases}: For networks with high treewidth, exact inference becomes intractable, necessitating approximate methods.
        \end{itemize}
    
    \end{highlight}
    
    \begin{highlight}[Summary of Key Concepts]
    
        \begin{itemize}
            \item \textbf{Inference by Enumeration}: Summing over the joint probability distribution, straightforward but computationally expensive.
            \item \textbf{Variable Elimination}: Systematically eliminates variables by summing out their probabilities, more efficient than enumeration.
            \item \textbf{Belief Propagation}: An exact inference method for tree-structured networks, using message passing to compute marginals.
            \item \textbf{Complexity of Exact Inference}: Depends on the network's treewidth; exact inference is generally NP-hard but feasible for certain structures.
        \end{itemize}
    
        Exact inference methods in Bayesian networks allow for precise probabilistic reasoning in structured domains, but their feasibility depends on the network's complexity and structure.
    
    \end{highlight}
\end{notes}