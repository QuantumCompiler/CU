\clearpage

\chapter{\documentName}
\section{\documentName}

\testimonial

\begin{itemize}
    \item The instructions for this assignment can be found here \pdflink{\AssDir Assignment 1A - Intro To AI, Search Problems - Instruction.pdf}{here}
    \item The submission for this assignment can be found here \pdflink{\AssDir Assignment 1A - Submission.pdf}{here}
\end{itemize}

% Problem 1
\begin{problem}{Problem 1}
    \begin{statement}{Problem Statement}
        Implement a function \ttt{breadth\_first(start, goal, state\_graph, return\_cost)} to search the state space (and step costs) defined by \ttt{state\_graph} using breadth-first search:
    \end{statement}

    \begin{highlight}[Solution]
    \begin{code}[Python]
    from collections import deque
    from collections import OrderedDict
    map_distances = dict(
        chi=OrderedDict([("det",283), ("cle",345), ("ind",182)]),
        cle=OrderedDict([("chi",345), ("det",169), ("col",144), ("pit",134), ("buf",189)]),
        ind=OrderedDict([("chi",182), ("col",176)]),
        col=OrderedDict([("ind",176), ("cle",144), ("pit",185)]),
        det=OrderedDict([("chi",283), ("cle",169), ("buf",256)]),
        buf=OrderedDict([("det",256), ("cle",189), ("pit",215), ("syr",150)]),
        pit=OrderedDict([("col",185), ("cle",134), ("buf",215), ("phi",305), ("bal",247)]),
        syr=OrderedDict([("buf",150), ("phi",253), ("new",254), ("bos",312)]),
        bal=OrderedDict([("phi",101), ("pit",247)]),
        phi=OrderedDict([("pit",305), ("bal",101), ("syr",253), ("new",97)]),
        new=OrderedDict([("syr",254), ("phi",97), ("bos",215), ("pro",181)]),
        pro=OrderedDict([("bos",50), ("new",181)]),
        bos=OrderedDict([("pro",50), ("new",215), ("syr",312), ("por",107)]),
        por=OrderedDict([("bos",107)]))
    
    map_times = dict(
        chi=dict(det=280, cle=345, ind=200),
        cle=dict(chi=345, det=170, col=155, pit=145, buf=185),
        ind=dict(chi=200, col=175),
        col=dict(ind=175, cle=155, pit=185),
        det=dict(chi=280, cle=170, buf=270),
        buf=dict(det=270, cle=185, pit=215, syr=145),
        pit=dict(col=185, cle=145, buf=215, phi=305, bal=255),
        syr=dict(buf=145, phi=245, new=260, bos=290),
        bal=dict(phi=145, pit=255),
        phi=dict(pit=305, bal=145, syr=245, new=150),
        new=dict(syr=260, phi=150, bos=270, pro=260),
        pro=dict(bos=90, new=260),
        bos=dict(pro=90, new=270, syr=290, por=120),
        por=dict(bos=120))
    
    def path(previous, s): 
        '''
        `previous` is a dictionary chaining together the predecessor state that led to each state
        `s` will be None for the initial state
        otherwise, start from the last state `s` and recursively trace `previous` back to the initial state,
        constructing a list of states visited as we go
        '''
        if s is None:
            return []
        else:
            return path(previous, previous[s])+[s]
    
    def pathcost(path, step_costs):
        '''
        add up the step costs along a path, which is assumed to be a list output from the `path` function above
        '''
        cost = 0
        for s in range(len(path)-1):
            cost += step_costs[path[s]][path[s+1]]
        return cost
    
    
    # Solution:
    
    """ breadth_first - Performs a breadth first search on cities
        Input:
            start - Node that represents the start of the path
            goal - Node that represents the desired end point of the path
            state_graph - Represents the graph that is being searched in
            return_cost - Boolean value that indicates the cost of the path
        Algorithm:
            * Create a queue with the start node as the first node
            * Create a visited set where the first node is visited
            * Create a dictionary for the previous nodes that have been visited
            * While the queue is not empty:
                * Pop the current node from the queue
                * If we reach the goal
                    * Create a path with the previous nodes and the goal
                    * Return the path and the cost if return_cost is set to true, otherwise just the path
                * Iterate over the neighbors of the current node
                * Add the neighbor to the visited set if it isn't visited
                * Update the previous node with the current node
                * Add the neighbor to the queue
            * Return the cost of the traversal
        Output:
            Returns the path in the search as well as the cost in the path
    """
    def breadth_first(start, goal, state_graph, return_cost=False):
        queue = deque([start])
        visited = set([start])
        previous = {start: None}
        while (queue):
            current = queue.popleft()
            if (current == goal):
                path_to_goal = path(previous, goal)
                if (return_cost):
                    cost = pathcost(path_to_goal, state_graph)
                    return path_to_goal, cost
                else:
                    return path_to_goal
            for neighbor in state_graph[current]:
                if (neighbor not in visited):
                    visited.add(neighbor)
                    previous[neighbor] = current
                    queue.append(neighbor)
        return None if not return_cost else (None, 0)
    \end{code}
    \end{highlight}
\end{problem}

% Problem 2
\begin{problem}{Problem 2}
    \begin{statement}{Problem Statement}
        Implement a function \ttt{depth\_first(start, goal, state\_graph, return\_cost)} to search the state space (and step costs) defined by \ttt{state\_graph} using depth-first search:
    \end{statement}

    \begin{highlight}[Solution]
    \begin{code}[Python]
    from collections import deque
    from collections import OrderedDict
    
    map_distances = dict(
        chi=OrderedDict([("det",283), ("cle",345), ("ind",182)]),
        cle=OrderedDict([("chi",345), ("det",169), ("col",144), ("pit",134), ("buf",189)]),
        ind=OrderedDict([("chi",182), ("col",176)]),
        col=OrderedDict([("ind",176), ("cle",144), ("pit",185)]),
        det=OrderedDict([("chi",283), ("cle",169), ("buf",256)]),
        buf=OrderedDict([("det",256), ("cle",189), ("pit",215), ("syr",150)]),
        pit=OrderedDict([("col",185), ("cle",134), ("buf",215), ("phi",305), ("bal",247)]),
        syr=OrderedDict([("buf",150), ("phi",253), ("new",254), ("bos",312)]),
        bal=OrderedDict([("phi",101), ("pit",247)]),
        phi=OrderedDict([("pit",305), ("bal",101), ("syr",253), ("new",97)]),
        new=OrderedDict([("syr",254), ("phi",97), ("bos",215), ("pro",181)]),
        pro=OrderedDict([("bos",50), ("new",181)]),
        bos=OrderedDict([("pro",50), ("new",215), ("syr",312), ("por",107)]),
        por=OrderedDict([("bos",107)]))
    
    
    map_times = dict(
        chi=dict(det=280, cle=345, ind=200),
        cle=dict(chi=345, det=170, col=155, pit=145, buf=185),
        ind=dict(chi=200, col=175),
        col=dict(ind=175, cle=155, pit=185),
        det=dict(chi=280, cle=170, buf=270),
        buf=dict(det=270, cle=185, pit=215, syr=145),
        pit=dict(col=185, cle=145, buf=215, phi=305, bal=255),
        syr=dict(buf=145, phi=245, new=260, bos=290),
        bal=dict(phi=145, pit=255),
        phi=dict(pit=305, bal=145, syr=245, new=150),
        new=dict(syr=260, phi=150, bos=270, pro=260),
        pro=dict(bos=90, new=260),
        bos=dict(pro=90, new=270, syr=290, por=120),
        por=dict(bos=120))
    
    def path(previous, s): 
        '''
        `previous` is a dictionary chaining together the predecessor state that led to each state
        `s` will be None for the initial state
        otherwise, start from the last state `s` and recursively trace `previous` back to the initial state,
        constructing a list of states visited as we go
        '''
        if s is None:
            return []
        else:
            return path(previous, previous[s])+[s]
    
    def pathcost(path, step_costs):
        '''
        add up the step costs along a path, which is assumed to be a list output from the `path` function above
        '''
        cost = 0
        for s in range(len(path)-1):
            cost += step_costs[path[s]][path[s+1]]
        return cost
    
    """ depth_first - Performs a DFS on the state graph for the path between a start and goal
        Input:
            start - Node that represents the start of the path
            goal - Node that represents the desired end point of the path
            state_graph - Represents the graph that is being searched in
            return_cost - Boolean value that indicates the cost of the path
        Algorithm:
            * Create a stack of nodes with the start node as the original element in it
            * Create a set of visited nodes with the start node as the original element in it
            * Create a dictionary of previous nodes that is empty
            * While the stack is not empty
                * Pop the top most node from the stack
                * Check if that node is the goal
                * If it is
                    * Update the path to the goal with the previous nodes and the goal
                    * Return the path to goal and the cost if return_cost is set to true
                    * Otherwise, just return the path
                * If it is not
                    * Iterate over the neighbors of the current node
                    * Add the neighbor to the visited set if it is not visited
                    * Update the previous nodes with the current node
                    * Add the neighbor to the stack
            * Return the path and cost, or just the path if designated
        Output:
            Returns the path in the search as well as the cost in the path
    """
    def depth_first(start, goal, state_graph, return_cost=False):
        stack = [start]
        visited = set([start])
        previous = {start: None}
        while (stack):
            current = stack.pop()
            if (current == goal):
                path_to_goal = path(previous, goal)
                if (return_cost):
                    cost = pathcost(path_to_goal, state_graph)
                    return path_to_goal, cost
                else:
                    return path_to_goal
            for neighbor in state_graph[current]:
                if (neighbor not in visited):
                    visited.add(neighbor)
                    previous[neighbor] = current
                    stack.append(neighbor)
        return None if not return_cost else (None, 0)
    \end{code}
    \end{highlight}
\end{problem}

% Problem 3
\begin{problem}{Problem 3}
    \begin{statement}{Problem Statement}
        First, let's create our own \ttt{Frontier\_PQ} class to represent the frontier (priority queue) for \ttt{uniformcost} search. Note that the \ttt{heapq} package is imported in the helpers at 
        the bottom of this assignment; you may find that package useful. You could also use the Queue package. Your implementation of the uniform-cost search frontier should adhere to these specifications
    \end{statement}

    \begin{highlight}[Solution]
    \begin{code}[Python]
    from collections import deque
    import heapq
    
    map_distances = dict(
        chi=dict(det=283, cle=345, ind=182),
        cle=dict(chi=345, det=169, col=144, pit=134, buf=189),
        ind=dict(chi=182, col=176),
        col=dict(ind=176, cle=144, pit=185),
        det=dict(chi=283, cle=169, buf=256),
        buf=dict(det=256, cle=189, pit=215, syr=150),
        pit=dict(col=185, cle=134, buf=215, phi=305, bal=247),
        syr=dict(buf=150, phi=253, new=254, bos=312),
        bal=dict(phi=101, pit=247),
        phi=dict(pit=305, bal=101, syr=253, new=97),
        new=dict(syr=254, phi=97, bos=215, pro=181),
        pro=dict(bos=50, new=181),
        bos=dict(pro=50, new=215, syr=312, por=107),
        por=dict(bos=107))
    
    
    map_times = dict(
        chi=dict(det=280, cle=345, ind=200),
        cle=dict(chi=345, det=170, col=155, pit=145, buf=185),
        ind=dict(chi=200, col=175),
        col=dict(ind=175, cle=155, pit=185),
        det=dict(chi=280, cle=170, buf=270),
        buf=dict(det=270, cle=185, pit=215, syr=145),
        pit=dict(col=185, cle=145, buf=215, phi=305, bal=255),
        syr=dict(buf=145, phi=245, new=260, bos=290),
        bal=dict(phi=145, pit=255),
        phi=dict(pit=305, bal=145, syr=245, new=150),
        new=dict(syr=260, phi=150, bos=270, pro=260),
        pro=dict(bos=90, new=260),
        bos=dict(pro=90, new=270, syr=290, por=120),
        por=dict(bos=120))
    
    def path(previous, s): 
        '''
        `previous` is a dictionary chaining together the predecessor state that led to each state
        `s` will be None for the initial state
        otherwise, start from the last state `s` and recursively trace `previous` back to the initial state,
        constructing a list of states visited as we go
        '''
        if s is None:
            return []
        else:
            return path(previous, previous[s])+[s]
    
    def pathcost(path, step_costs):
        '''
        add up the step costs along a path, which is assumed to be a list output from the `path` function above
        '''
        cost = 0
        for s in range(len(path)-1):
            cost += step_costs[path[s]][path[s+1]]
        return cost
    
    # Solution:
    """ Frontier_PQ - Implements a priority queue ordered by path cost for uniform cost search
        Methods:
            __init__ - Initializes an empty priority queue
            is_empty - Checks if the priority queue is empty
            put - Adds an item with a specified priority to the priority queue
            get - Removes and returns the item with the lowest priority from the priority queue
        Algorithm:
            * __init__ initializes an empty list to represent the priority queue
            * is_empty returns True if the list is empty, otherwise False
            * put uses heapq.heappush to add an item to the priority queue with the given priority
            * get uses heapq.heappop to remove and return the item with the lowest priority
        Output:
            * is_empty returns a boolean indicating whether the priority queue is empty
            * put does not return a value
            * get returns the item with the lowest priority
    """
    class Frontier_PQ:
        ''' frontier class for uniform search, ordered by path cost '''
        # add your code here
        def __init__(self):
            self.elements = []
        def is_empty(self):
            return len(self.elements) == 0
        def put(self, item, priority):
            heapq.heappush(self.elements, (priority, item))
        def get(self):
            return heapq.heappop(self.elements)
    
    # Solution:
    """ uniform_cost - Performs a Uniform Cost Search (UCS) on the state graph for the path between a start and goal
        Input:
            start - Node that represents the start of the path
            goal - Node that represents the desired end point of the path
            state_graph - Dictionary representing the graph being searched, with costs for each edge
            return_cost - Boolean value that indicates whether to return the cost of the path
        Algorithm:
            * Initialize a Frontier_PQ instance and add the start node with a priority of 0
            * Initialize a dictionary of previous nodes with the start node set to None
            * Initialize a dictionary to keep track of the cost to reach each node with the start node set to 0
            * While the priority queue is not empty
                * Get the node with the lowest cost from the priority queue
                * Check if that node is the goal
                * If it is
                    * Update the path to the goal using the previous nodes and the goal
                    * Calculate the cost if return_cost is True
                    * Return the path to the goal and the cost if return_cost is set to True
                    * Otherwise, just return the path
                * If it is not
                    * Iterate over the neighbors of the current node
                    * Calculate the new cost to reach each neighbor
                    * If the neighbor has not been visited or the new cost is lower than the recorded cost
                        * Update the cost to reach the neighbor
                        * Add the neighbor to the priority queue with the new cost as priority
                        * Update the previous nodes with the current node
            * Return None if the goal is not reachable or return (None, 0) if return_cost is True
        Output:
            Returns the path in the search as well as the cost in the path if return_cost is True
    """
    def uniform_cost(start, goal, state_graph, return_cost=False):
        frontier = Frontier_PQ()
        frontier.put(start, 0)
        previous = {start: None}
        cost_so_far = {start: 0}
        while not frontier.is_empty():
            current_priority, current = frontier.get()
            if current == goal:
                path_to_goal = path(previous, goal)
                if return_cost:
                    cost = pathcost(path_to_goal, state_graph)
                    return path_to_goal, cost
                else:
                    return path_to_goal
            for neighbor in state_graph[current]:
                new_cost = cost_so_far[current] + state_graph[current][neighbor]
                if neighbor not in cost_so_far or new_cost < cost_so_far[neighbor]:
                    cost_so_far[neighbor] = new_cost
                    priority = new_cost
                    frontier.put(neighbor, priority)
                    previous[neighbor] = current
        return None if not return_cost else (None, 0)
    \end{code}
    \end{highlight}
\end{problem}

% Problem 4
\begin{problem}{Problem 4}
    \begin{statement}{Problem Statement}
        Use each of your search functions to find routes for Neal to travel from New York to Chicago, with path costs defined by the distance between cities.
    \end{statement}
    
    \begin{highlight}[Solution]
    \begin{code}[Python]
    from collections import deque
    from collections import OrderedDict
    import heapq
    map_distances = dict(
        chi=OrderedDict([("det",283), ("cle",345), ("ind",182)]),
        cle=OrderedDict([("chi",345), ("det",169), ("col",144), ("pit",134), ("buf",189)]),
        ind=OrderedDict([("chi",182), ("col",176)]),
        col=OrderedDict([("ind",176), ("cle",144), ("pit",185)]),
        det=OrderedDict([("chi",283), ("cle",169), ("buf",256)]),
        buf=OrderedDict([("det",256), ("cle",189), ("pit",215), ("syr",150)]),
        pit=OrderedDict([("col",185), ("cle",134), ("buf",215), ("phi",305), ("bal",247)]),
        syr=OrderedDict([("buf",150), ("phi",253), ("new",254), ("bos",312)]),
        bal=OrderedDict([("phi",101), ("pit",247)]),
        phi=OrderedDict([("pit",305), ("bal",101), ("syr",253), ("new",97)]),
        new=OrderedDict([("syr",254), ("phi",97), ("bos",215), ("pro",181)]),
        pro=OrderedDict([("bos",50), ("new",181)]),
        bos=OrderedDict([("pro",50), ("new",215), ("syr",312), ("por",107)]),
        por=OrderedDict([("bos",107)]))
        
    map_times = dict(
        chi=dict(det=280, cle=345, ind=200),
        cle=dict(chi=345, det=170, col=155, pit=145, buf=185),
        ind=dict(chi=200, col=175),
        col=dict(ind=175, cle=155, pit=185),
        det=dict(chi=280, cle=170, buf=270),
        buf=dict(det=270, cle=185, pit=215, syr=145),
        pit=dict(col=185, cle=145, buf=215, phi=305, bal=255),
        syr=dict(buf=145, phi=245, new=260, bos=290),
        bal=dict(phi=145, pit=255),
        phi=dict(pit=305, bal=145, syr=245, new=150),
        new=dict(syr=260, phi=150, bos=270, pro=260),
        pro=dict(bos=90, new=260),
        bos=dict(pro=90, new=270, syr=290, por=120),
        por=dict(bos=120))
    
    def path(previous, s): 
        '''
        `previous` is a dictionary chaining together the predecessor state that led to each state
        `s` will be None for the initial state
        otherwise, start from the last state `s` and recursively trace `previous` back to the initial state,
        constructing a list of states visited as we go
        '''
        if s is None:
            return []
        else:
            return path(previous, previous[s])+[s]
    
    def pathcost(path, step_costs):
        '''
        add up the step costs along a path, which is assumed to be a list output from the `path` function above
        '''
        cost = 0
        for s in range(len(path)-1):
            cost += step_costs[path[s]][path[s+1]]
        return cost
    
    
    # Solution:
    """ Frontier_PQ - Implements a priority queue ordered by path cost for uniform cost search
        Methods:
            __init__ - Initializes an empty priority queue
            is_empty - Checks if the priority queue is empty
            put - Adds an item with a specified priority to the priority queue
            get - Removes and returns the item with the lowest priority from the priority queue
        Algorithm:
            * __init__ initializes an empty list to represent the priority queue
            * is_empty returns True if the list is empty, otherwise False
            * put uses heapq.heappush to add an item to the priority queue with the given priority
            * get uses heapq.heappop to remove and return the item with the lowest priority
        Output:
            * is_empty returns a boolean indicating whether the priority queue is empty
            * put does not return a value
            * get returns the item with the lowest priority
    """
    class Frontier_PQ:
        ''' frontier class for uniform search, ordered by path cost '''
        # add your code here
        def __init__(self):
            self.elements = []
        def is_empty(self):
            return len(self.elements) == 0
        def put(self, item, priority):
            heapq.heappush(self.elements, (priority, item))
        def get(self):
            return heapq.heappop(self.elements)
        
    
    """ breadth_first - Performs a breadth first search on cities
        Input:
            start - Node that represents the start of the path
            goal - Node that represents the desired end point of the path
            state_graph - Represents the graph that is being searched in
            return_cost - Boolean value that indicates the cost of the path
        Algorithm:
            * Create a queue with the start node as the first node
            * Create a visited set where the first node is visited
            * Create a dictionary for the previous nodes that have been visited
            * While the queue is not empty:
                * Pop the current node from the queue
                * If we reach the goal
                    * Create a path with the previous nodes and the goal
                    * Return the path and the cost if return_cost is set to true, otherwise just the path
                * Iterate over the neighbors of the current node
                * Add the neighbor to the visited set if it isn't visited
                * Update the previous node with the current node
                * Add the neighbor to the queue
            * Return the cost of the traversal
        Output:
            Returns the path in the search as well as the cost in the path
    """
    def breadth_first(start, goal, state_graph, return_cost=False):
        queue = deque([start])
        visited = set([start])
        previous = {start: None}
        while (queue):
            current = queue.popleft()
            if (current == goal):
                path_to_goal = path(previous, goal)
                if (return_cost):
                    cost = pathcost(path_to_goal, state_graph)
                    return path_to_goal, cost
                else:
                    return path_to_goal
            for neighbor in state_graph[current]:
                if (neighbor not in visited):
                    visited.add(neighbor)
                    previous[neighbor] = current
                    queue.append(neighbor)
        return None if not return_cost else (None, 0)
    
    """ depth_first - Performs a DFS on the state graph for the path between a start and goal
        Input:
            start - Node that represents the start of the path
            goal - Node that represents the desired end point of the path
            state_graph - Represents the graph that is being searched in
            return_cost - Boolean value that indicates the cost of the path
        Algorithm:
            * Create a stack of nodes with the start node as the original element in it
            * Create a set of visited nodes with the start node as the original element in it
            * Create a dictionary of previous nodes that is empty
            * While the stack is not empty
                * Pop the top most node from the stack
                * Check if that node is the goal
                * If it is
                    * Update the path to the goal with the previous nodes and the goal
                    * Return the path to goal and the cost if return_cost is set to true
                    * Otherwise, just return the path
                * If it is not
                    * Iterate over the neighbors of the current node
                    * Add the neighbor to the visited set if it is not visited
                    * Update the previous nodes with the current node
                    * Add the neighbor to the stack
            * Return the path and cost, or just the path if designated
        Output:
            Returns the path in the search as well as the cost in the path
    """
    def depth_first(start, goal, state_graph, return_cost=False):
        stack = [start]
        visited = set([start])
        previous = {start: None}
        while (stack):
            current = stack.pop()
            if (current == goal):
                path_to_goal = path(previous, goal)
                if (return_cost):
                    cost = pathcost(path_to_goal, state_graph)
                    return path_to_goal, cost
                else:
                    return path_to_goal
            for neighbor in state_graph[current]:
                if (neighbor not in visited):
                    visited.add(neighbor)
                    previous[neighbor] = current
                    stack.append(neighbor)
        return None if not return_cost else (None, 0)
    
    # Solution:
    """ uniform_cost - Performs a Uniform Cost Search (UCS) on the state graph for the path between a start and goal
        Input:
            start - Node that represents the start of the path
            goal - Node that represents the desired end point of the path
            state_graph - Dictionary representing the graph being searched, with costs for each edge
            return_cost - Boolean value that indicates whether to return the cost of the path
        Algorithm:
            * Initialize a Frontier_PQ instance and add the start node with a priority of 0
            * Initialize a dictionary of previous nodes with the start node set to None
            * Initialize a dictionary to keep track of the cost to reach each node with the start node set to 0
            * While the priority queue is not empty
                * Get the node with the lowest cost from the priority queue
                * Check if that node is the goal
                * If it is
                    * Update the path to the goal using the previous nodes and the goal
                    * Calculate the cost if return_cost is True
                    * Return the path to the goal and the cost if return_cost is set to True
                    * Otherwise, just return the path
                * If it is not
                    * Iterate over the neighbors of the current node
                    * Calculate the new cost to reach each neighbor
                    * If the neighbor has not been visited or the new cost is lower than the recorded cost
                        * Update the cost to reach the neighbor
                        * Add the neighbor to the priority queue with the new cost as priority
                        * Update the previous nodes with the current node
            * Return None if the goal is not reachable or return (None, 0) if return_cost is True
        Output:
            Returns the path in the search as well as the cost in the path if return_cost is True
    """
    def uniform_cost(start, goal, state_graph, return_cost=False):
        frontier = Frontier_PQ()
        frontier.put(start, 0)
        previous = {start: None}
        cost_so_far = {start: 0}
        while not frontier.is_empty():
            current_priority, current = frontier.get()
            if current == goal:
                path_to_goal = path(previous, goal)
                if return_cost:
                    cost = pathcost(path_to_goal, state_graph)
                    return path_to_goal, cost
                else:
                    return path_to_goal
            for neighbor in state_graph[current]:
                new_cost = cost_so_far[current] + state_graph[current][neighbor]
                if neighbor not in cost_so_far or new_cost < cost_so_far[neighbor]:
                    cost_so_far[neighbor] = new_cost
                    priority = new_cost
                    frontier.put(neighbor, priority)
                    previous[neighbor] = current
        return None if not return_cost else (None, 0)
    \end{code}
    \end{highlight}
\end{problem}

% Problem 5
\begin{problem}{Problem 5}
    \begin{statement}{Problem Statement}
        Which algorithm yields the shortest path?
    \end{statement}

    \begin{highlight}[Solution]
        \textbf{Uniform Cost Search}
    \end{highlight}
\end{problem}

% Problem 6
\begin{problem}{Problem 6}
    \begin{statement}{Problem Statement}
        Use your choice of search function to show the list of cities that Neal will traverse to get to Chicago on time, should such a path exist.
    \end{statement}

    \begin{highlight}[Solution]
    \begin{code}[Python]
    from collections import deque
    import heapq
    
    map_distances = dict(
        chi=dict(det=283, cle=345, ind=182),
        cle=dict(chi=345, det=169, col=144, pit=134, buf=189),
        ind=dict(chi=182, col=176),
        col=dict(ind=176, cle=144, pit=185),
        det=dict(chi=283, cle=169, buf=256),
        buf=dict(det=256, cle=189, pit=215, syr=150),
        pit=dict(col=185, cle=134, buf=215, phi=305, bal=247),
        syr=dict(buf=150, phi=253, new=254, bos=312),
        bal=dict(phi=101, pit=247),
        phi=dict(pit=305, bal=101, syr=253, new=97),
        new=dict(syr=254, phi=97, bos=215, pro=181),
        pro=dict(bos=50, new=181),
        bos=dict(pro=50, new=215, syr=312, por=107),
        por=dict(bos=107))
    
    
    map_times = dict(
        chi=dict(det=280, cle=345, ind=200),
        cle=dict(chi=345, det=170, col=155, pit=145, buf=185),
        ind=dict(chi=200, col=175),
        col=dict(ind=175, cle=155, pit=185),
        det=dict(chi=280, cle=170, buf=270),
        buf=dict(det=270, cle=185, pit=215, syr=145),
        pit=dict(col=185, cle=145, buf=215, phi=305, bal=255),
        syr=dict(buf=145, phi=245, new=260, bos=290),
        bal=dict(phi=145, pit=255),
        phi=dict(pit=305, bal=145, syr=245, new=150),
        new=dict(syr=260, phi=150, bos=270, pro=260),
        pro=dict(bos=90, new=260),
        bos=dict(pro=90, new=270, syr=290, por=120),
        por=dict(bos=120))
    
    def path(previous, s): 
        '''
        `previous` is a dictionary chaining together the predecessor state that led to each state
        `s` will be None for the initial state
        otherwise, start from the last state `s` and recursively trace `previous` back to the initial state,
        constructing a list of states visited as we go
        '''
        if s is None:
            return []
        else:
            return path(previous, previous[s])+[s]
    
    def pathcost(path, step_costs):
        '''
        add up the step costs along a path, which is assumed to be a list output from the `path` function above
        '''
        cost = 0
        for s in range(len(path)-1):
            cost += step_costs[path[s]][path[s+1]]
        return cost
    
    # Solution:
    """ Frontier_PQ - Implements a priority queue ordered by path cost for uniform cost search
        Methods:
            __init__ - Initializes an empty priority queue
            is_empty - Checks if the priority queue is empty
            put - Adds an item with a specified priority to the priority queue
            get - Removes and returns the item with the lowest priority from the priority queue
        Algorithm:
            * __init__ initializes an empty list to represent the priority queue
            * is_empty returns True if the list is empty, otherwise False
            * put uses heapq.heappush to add an item to the priority queue with the given priority
            * get uses heapq.heappop to remove and return the item with the lowest priority
        Output:
            * is_empty returns a boolean indicating whether the priority queue is empty
            * put does not return a value
            * get returns the item with the lowest priority
    """
    class Frontier_PQ:
        ''' frontier class for uniform search, ordered by path cost '''
        # add your code here
        def __init__(self):
            self.elements = []
        def is_empty(self):
            return len(self.elements) == 0
        def put(self, item, priority):
            heapq.heappush(self.elements, (priority, item))
        def get(self):
            return heapq.heappop(self.elements)                  
    
    
    # Solution:
    """ uniform_cost - Performs a Uniform Cost Search (UCS) on the state graph for the path between a start and goal
        Input:
            start - Node that represents the start of the path
            goal - Node that represents the desired end point of the path
            state_graph - Dictionary representing the graph being searched, with costs for each edge
            return_cost - Boolean value that indicates whether to return the cost of the path
        Algorithm:
            * Initialize a Frontier_PQ instance and add the start node with a priority of 0
            * Initialize a dictionary of previous nodes with the start node set to None
            * Initialize a dictionary to keep track of the cost to reach each node with the start node set to 0
            * While the priority queue is not empty
                * Get the node with the lowest cost from the priority queue
                * Check if that node is the goal
                * If it is
                    * Update the path to the goal using the previous nodes and the goal
                    * Calculate the cost if return_cost is True
                    * Return the path to the goal and the cost if return_cost is set to True
                    * Otherwise, just return the path
                * If it is not
                    * Iterate over the neighbors of the current node
                    * Calculate the new cost to reach each neighbor
                    * If the neighbor has not been visited or the new cost is lower than the recorded cost
                        * Update the cost to reach the neighbor
                        * Add the neighbor to the priority queue with the new cost as priority
                        * Update the previous nodes with the current node
            * Return None if the goal is not reachable or return (None, 0) if return_cost is True
        Output:
            Returns the path in the search as well as the cost in the path if return_cost is True
    """
    def uniform_cost(start, goal, state_graph, return_cost=False):
        frontier = Frontier_PQ()
        frontier.put(start, 0)
        previous = {start: None}
        cost_so_far = {start: 0}
        while not frontier.is_empty():
            current_priority, current = frontier.get()
            if current == goal:
                path_to_goal = path(previous, goal)
                if return_cost:
                    cost = pathcost(path_to_goal, state_graph)
                    return path_to_goal, cost
                else:
                    return path_to_goal
            for neighbor in state_graph[current]:
                new_cost = cost_so_far[current] + state_graph[current][neighbor]
                if neighbor not in cost_so_far or new_cost < cost_so_far[neighbor]:
                    cost_so_far[neighbor] = new_cost
                    priority = new_cost
                    frontier.put(neighbor, priority)
                    previous[neighbor] = current
        return None if not return_cost else (None, 0)
    \end{code}
    \end{highlight}
\end{problem}

% Problem 7
\begin{problem}{Problem 7}
    \begin{statement}{Problem Statement}
        Use your choice of search function to show the list of cities that Neal would traverse to get to Chicago as quickly as possible.
    \end{statement}

    \begin{highlight}[Solution]
    \begin{code}[Python]
    from collections import deque
    import heapq
    
    map_distances = dict(
        chi=dict(det=283, cle=345, ind=182),
        cle=dict(chi=345, det=169, col=144, pit=134, buf=189),
        ind=dict(chi=182, col=176),
        col=dict(ind=176, cle=144, pit=185),
        det=dict(chi=283, cle=169, buf=256),
        buf=dict(det=256, cle=189, pit=215, syr=150),
        pit=dict(col=185, cle=134, buf=215, phi=305, bal=247),
        syr=dict(buf=150, phi=253, new=254, bos=312),
        bal=dict(phi=101, pit=247),
        phi=dict(pit=305, bal=101, syr=253, new=97),
        new=dict(syr=254, phi=97, bos=215, pro=181),
        pro=dict(bos=50, new=181),
        bos=dict(pro=50, new=215, syr=312, por=107),
        por=dict(bos=107))
    
    
    map_times = dict(
        chi=dict(det=280, cle=345, ind=200),
        cle=dict(chi=345, det=170, col=155, pit=145, buf=185),
        ind=dict(chi=200, col=175),
        col=dict(ind=175, cle=155, pit=185),
        det=dict(chi=280, cle=170, buf=270),
        buf=dict(det=270, cle=185, pit=215, syr=145),
        pit=dict(col=185, cle=145, buf=215, phi=305, bal=255),
        syr=dict(buf=145, phi=245, new=260, bos=290),
        bal=dict(phi=145, pit=255),
        phi=dict(pit=305, bal=145, syr=245, new=150),
        new=dict(syr=260, phi=150, bos=270, pro=260),
        pro=dict(bos=90, new=260),
        bos=dict(pro=90, new=270, syr=290, por=120),
        por=dict(bos=120))
    
    def path(previous, s): 
        '''
        `previous` is a dictionary chaining together the predecessor state that led to each state
        `s` will be None for the initial state
        otherwise, start from the last state `s` and recursively trace `previous` back to the initial state,
        constructing a list of states visited as we go
        '''
        if s is None:
            return []
        else:
            return path(previous, previous[s])+[s]
    
    def pathcost(path, step_costs):
        '''
        add up the step costs along a path, which is assumed to be a list output from the `path` function above
        '''
        cost = 0
        for s in range(len(path)-1):
            cost += step_costs[path[s]][path[s+1]]
        return cost
    
    # Solution:
    """ Frontier_PQ - Implements a priority queue ordered by path cost for uniform cost search
        Methods:
            __init__ - Initializes an empty priority queue
            is_empty - Checks if the priority queue is empty
            put - Adds an item with a specified priority to the priority queue
            get - Removes and returns the item with the lowest priority from the priority queue
        Algorithm:
            * __init__ initializes an empty list to represent the priority queue
            * is_empty returns True if the list is empty, otherwise False
            * put uses heapq.heappush to add an item to the priority queue with the given priority
            * get uses heapq.heappop to remove and return the item with the lowest priority
        Output:
            * is_empty returns a boolean indicating whether the priority queue is empty
            * put does not return a value
            * get returns the item with the lowest priority
    """
    class Frontier_PQ:
        ''' frontier class for uniform search, ordered by path cost '''
        # add your code here
        def __init__(self):
            self.elements = []
        def is_empty(self):
            return len(self.elements) == 0
        def put(self, item, priority):
            heapq.heappush(self.elements, (priority, item))
        def get(self):
            return heapq.heappop(self.elements)
        
    # Solution:
    """ uniform_cost - Performs a Uniform Cost Search (UCS) on the state graph for the path between a start and goal
        Input:
            start - Node that represents the start of the path
            goal - Node that represents the desired end point of the path
            state_graph - Dictionary representing the graph being searched, with costs for each edge
            return_cost - Boolean value that indicates whether to return the cost of the path
        Algorithm:
            * Initialize a Frontier_PQ instance and add the start node with a priority of 0
            * Initialize a dictionary of previous nodes with the start node set to None
            * Initialize a dictionary to keep track of the cost to reach each node with the start node set to 0
            * While the priority queue is not empty
                * Get the node with the lowest cost from the priority queue
                * Check if that node is the goal
                * If it is
                    * Update the path to the goal using the previous nodes and the goal
                    * Calculate the cost if return_cost is True
                    * Return the path to the goal and the cost if return_cost is set to True
                    * Otherwise, just return the path
                * If it is not
                    * Iterate over the neighbors of the current node
                    * Calculate the new cost to reach each neighbor
                    * If the neighbor has not been visited or the new cost is lower than the recorded cost
                        * Update the cost to reach the neighbor
                        * Add the neighbor to the priority queue with the new cost as priority
                        * Update the previous nodes with the current node
            * Return None if the goal is not reachable or return (None, 0) if return_cost is True
        Output:
            Returns the path in the search as well as the cost in the path if return_cost is True
    """
    def uniform_cost(start, goal, state_graph, return_cost=False):
        frontier = Frontier_PQ()
        frontier.put(start, 0)
        previous = {start: None}
        cost_so_far = {start: 0}
        while not frontier.is_empty():
            current_priority, current = frontier.get()
            if current == goal:
                path_to_goal = path(previous, goal)
                if return_cost:
                    cost = pathcost(path_to_goal, state_graph)
                    return path_to_goal, cost
                else:
                    return path_to_goal
            for neighbor in state_graph[current]:
                new_cost = cost_so_far[current] + state_graph[current][neighbor]
                if neighbor not in cost_so_far or new_cost < cost_so_far[neighbor]:
                    cost_so_far[neighbor] = new_cost
                    priority = new_cost
                    frontier.put(neighbor, priority)
                    previous[neighbor] = current
        return None if not return_cost else (None, 0)
    \end{code}
    \end{highlight}
\end{problem}

% Problem 8
\begin{problem}{Problem 8}
    \begin{statement}{Problem Statement}
        Pass the maze-to-graph unit test.
    \end{statement}

    \begin{highlight}[Solution]
    \begin{code}[Python]
    import numpy as np

    """ maze_to_graph - Converts a maze represented as a numpy array into a graph
        Input:
            maze - 2D numpy array where 0 represents walkable cells and 1 represents walls
        Algorithm:
            * Initialize an empty dictionary to represent the graph
            * Define the directions for North, South, East, and West movements
            * Iterate over each cell in the maze
                * Initialize an empty dictionary for each cell in the graph
                * For each direction, calculate the neighboring cell's coordinates
                * Check if the neighboring cell is within the maze bounds and is walkable (contains 0)
                    * If it is, add the neighbor to the current cell's dictionary in the graph with the direction as the value
        Output:
            Returns a dictionary representing the graph where keys are coordinates of cells and values are dictionaries
            of neighboring cells with directions
    """
    def maze_to_graph(maze):
        ''' takes in a maze as a numpy array, converts to a graph '''
        graph = {}
        rows, cols = maze.shape
        directions = {
            'N': (1, 0),  # North
            'S': (-1, 0),   # South
            'E': (0, 1),   # East
            'W': (0, -1)   # West
        }
        for r in range(rows):
            for c in range(cols):
                graph[(c, r)] = {}
                for direction, (dr, dc) in directions.items():
                    nr, nc = r + dr, c + dc
                    if 0 <= nr < rows and 0 <= nc < cols and maze[nr, nc] == 0:
                        graph[(c, r)][(nc, nr)] = direction
        return graph
    \end{code}
    \end{highlight}
\end{problem}

% Problem 9
\begin{problem}{Problem 9}
    \begin{statement}{Problem Statement}
        Use your depth-first search function to solve the maze and provide the solution path.
    \end{statement}

    \begin{highlight}[Solution]
    \begin{code}[Python]
    import numpy as np
    from collections import OrderedDict
    from collections import deque
    maze = np.array([[1, 1, 1, 1, 1, 1, 1, 1, 1, 1, 1, 1],
                    [1, 0, 0, 0, 0, 0, 0, 0, 0, 0, 0, 1],
                    [1, 0, 1, 1, 1, 1, 1, 1, 0, 1, 1, 1],
                    [1, 0, 1, 0, 0, 0, 0, 0, 0, 0, 0, 1],
                    [1, 0, 1, 0, 1, 1, 1, 1, 1, 1, 0, 1],
                    [1, 0, 1, 0, 1, 0, 0, 0, 0, 0, 0, 1],
                    [1, 0, 0, 0, 1, 1, 0, 1, 1, 1, 0, 1],
                    [1, 0, 1, 0, 0, 0, 0, 1, 0, 1, 1, 1],
                    [1, 0, 1, 1, 0, 1, 0, 0, 0, 0, 0, 1],
                    [1, 0, 1, 0, 0, 1, 1, 1, 1, 1, 0, 1],
                    [1, 0, 0, 0, 1, 1, 0, 0, 0, 0, 0, 1],
                    [1, 1, 1, 1, 1, 1, 1, 1, 1, 1, 1, 1]])
    
    map_distances = dict(
        chi=OrderedDict([("det",283), ("cle",345), ("ind",182)]),
        cle=OrderedDict([("chi",345), ("det",169), ("col",144), ("pit",134), ("buf",189)]),
        ind=OrderedDict([("chi",182), ("col",176)]),
        col=OrderedDict([("ind",176), ("cle",144), ("pit",185)]),
        det=OrderedDict([("chi",283), ("cle",169), ("buf",256)]),
        buf=OrderedDict([("det",256), ("cle",189), ("pit",215), ("syr",150)]),
        pit=OrderedDict([("col",185), ("cle",134), ("buf",215), ("phi",305), ("bal",247)]),
        syr=OrderedDict([("buf",150), ("phi",253), ("new",254), ("bos",312)]),
        bal=OrderedDict([("phi",101), ("pit",247)]),
        phi=OrderedDict([("pit",305), ("bal",101), ("syr",253), ("new",97)]),
        new=OrderedDict([("syr",254), ("phi",97), ("bos",215), ("pro",181)]),
        pro=OrderedDict([("bos",50), ("new",181)]),
        bos=OrderedDict([("pro",50), ("new",215), ("syr",312), ("por",107)]),
        por=OrderedDict([("bos",107)]))
    
    map_times = dict(
        chi=dict(det=280, cle=345, ind=200),
        cle=dict(chi=345, det=170, col=155, pit=145, buf=185),
        ind=dict(chi=200, col=175),
        col=dict(ind=175, cle=155, pit=185),
        det=dict(chi=280, cle=170, buf=270),
        buf=dict(det=270, cle=185, pit=215, syr=145),
        pit=dict(col=185, cle=145, buf=215, phi=305, bal=255),
        syr=dict(buf=145, phi=245, new=260, bos=290),
        bal=dict(phi=145, pit=255),
        phi=dict(pit=305, bal=145, syr=245, new=150),
        new=dict(syr=260, phi=150, bos=270, pro=260),
        pro=dict(bos=90, new=260),
        bos=dict(pro=90, new=270, syr=290, por=120),
        por=dict(bos=120))
    
    def path(previous, s): 
        '''
        `previous` is a dictionary chaining together the predecessor state that led to each state
        `s` will be None for the initial state
        otherwise, start from the last state `s` and recursively trace `previous` back to the initial state,
        constructing a list of states visited as we go
        '''
        if s is None:
            return []
        else:
            return path(previous, previous[s])+[s]
    
    def pathcost(path, step_costs):
        '''
        add up the step costs along a path, which is assumed to be a list output from the `path` function above
        '''
        cost = 0
        for s in range(len(path)-1):
            cost += step_costs[path[s]][path[s+1]]
        return cost
    
    
    """ depth_first - Performs a DFS on the state graph for the path between a start and goal
        Input:
            start - Node that represents the start of the path
            goal - Node that represents the desired end point of the path
            state_graph - Represents the graph that is being searched in
            return_cost - Boolean value that indicates the cost of the path
        Algorithm:
            * Create a stack of nodes with the start node as the original element in it
            * Create a set of visited nodes with the start node as the original element in it
            * Create a dictionary of previous nodes that is empty
            * While the stack is not empty
                * Pop the top most node from the stack
                * Check if that node is the goal
                * If it is
                    * Update the path to the goal with the previous nodes and the goal
                    * Return the path to goal and the cost if return_cost is set to true
                    * Otherwise, just return the path
                * If it is not
                    * Iterate over the neighbors of the current node
                    * Add the neighbor to the visited set if it is not visited
                    * Update the previous nodes with the current node
                    * Add the neighbor to the stack
            * Return the path and cost, or just the path if designated
        Output:
            Returns the path in the search as well as the cost in the path
    """
    def depth_first(start, goal, state_graph, return_cost=False):
        stack = [start]
        visited = set([start])
        previous = {start: None}
        while (stack):
            current = stack.pop()
            if (current == goal):
                path_to_goal = path(previous, goal)
                if (return_cost):
                    cost = pathcost(path_to_goal, state_graph)
                    return path_to_goal, cost
                else:
                    return path_to_goal
            for neighbor in state_graph[current]:
                if (neighbor not in visited):
                    visited.add(neighbor)
                    previous[neighbor] = current
                    stack.append(neighbor)
        return None if not return_cost else (None, 0)
        
    
    # Solution:
    
    """ maze_to_graph - Converts a maze represented as a numpy array into a graph
        Input:
            maze - 2D numpy array where 0 represents walkable cells and 1 represents walls
        Algorithm:
            * Initialize an empty dictionary to represent the graph
            * Define the directions for North, South, East, and West movements
            * Iterate over each cell in the maze
                * Initialize an empty dictionary for each cell in the graph
                * For each direction, calculate the neighboring cell's coordinates
                * Check if the neighboring cell is within the maze bounds and is walkable (contains 0)
                    * If it is, add the neighbor to the current cell's dictionary in the graph with the direction as the value
        Output:
            Returns a dictionary representing the graph where keys are coordinates of cells and values are dictionaries
            of neighboring cells with directions
    """
    def maze_to_graph(maze):
        ''' takes in a maze as a numpy array, converts to a graph '''
        graph = {}
        rows, cols = maze.shape
        directions = {
            'N': (1, 0),  # North
            'S': (-1, 0),   # South
            'E': (0, 1),   # East
            'W': (0, -1)   # West
        }
        for r in range(rows):
            for c in range(cols):
                graph[(c, r)] = {}
                for direction, (dr, dc) in directions.items():
                    nr, nc = r + dr, c + dc
                    if 0 <= nr < rows and 0 <= nc < cols and maze[nr, nc] == 0:
                        graph[(c, r)][(nc, nr)] = direction
        return graph
    \end{code}
    \end{highlight}
\end{problem}

% Problem 10
\begin{problem}{Problem 10}
    \begin{statement}{Problem Statement}
        Use your breadth-first search function to solve the maze and provide the solution path and its length.
    \end{statement}

    \begin{highlight}[Solution]
    \begin{code}[Python]
    import numpy as np
    maze = np.array([[1, 1, 1, 1, 1, 1, 1, 1, 1, 1, 1, 1],
                    [1, 0, 0, 0, 0, 0, 0, 0, 0, 0, 0, 1],
                    [1, 0, 1, 1, 1, 1, 1, 1, 0, 1, 1, 1],
                    [1, 0, 1, 0, 0, 0, 0, 0, 0, 0, 0, 1],
                    [1, 0, 1, 0, 1, 1, 1, 1, 1, 1, 0, 1],
                    [1, 0, 1, 0, 1, 0, 0, 0, 0, 0, 0, 1],
                    [1, 0, 0, 0, 1, 1, 0, 1, 1, 1, 0, 1],
                    [1, 0, 1, 0, 0, 0, 0, 1, 0, 1, 1, 1],
                    [1, 0, 1, 1, 0, 1, 0, 0, 0, 0, 0, 1],
                    [1, 0, 1, 0, 0, 1, 1, 1, 1, 1, 0, 1],
                    [1, 0, 0, 0, 1, 1, 0, 0, 0, 0, 0, 1],
                    [1, 1, 1, 1, 1, 1, 1, 1, 1, 1, 1, 1]])
    
    ##################
    from collections import deque
    
    map_distances = dict(
        chi=dict(det=283, cle=345, ind=182),
        cle=dict(chi=345, det=169, col=144, pit=134, buf=189),
        ind=dict(chi=182, col=176),
        col=dict(ind=176, cle=144, pit=185),
        det=dict(chi=283, cle=169, buf=256),
        buf=dict(det=256, cle=189, pit=215, syr=150),
        pit=dict(col=185, cle=134, buf=215, phi=305, bal=247),
        syr=dict(buf=150, phi=253, new=254, bos=312),
        bal=dict(phi=101, pit=247),
        phi=dict(pit=305, bal=101, syr=253, new=97),
        new=dict(syr=254, phi=97, bos=215, pro=181),
        pro=dict(bos=50, new=181),
        bos=dict(pro=50, new=215, syr=312, por=107),
        por=dict(bos=107))
    
    
    map_times = dict(
        chi=dict(det=280, cle=345, ind=200),
        cle=dict(chi=345, det=170, col=155, pit=145, buf=185),
        ind=dict(chi=200, col=175),
        col=dict(ind=175, cle=155, pit=185),
        det=dict(chi=280, cle=170, buf=270),
        buf=dict(det=270, cle=185, pit=215, syr=145),
        pit=dict(col=185, cle=145, buf=215, phi=305, bal=255),
        syr=dict(buf=145, phi=245, new=260, bos=290),
        bal=dict(phi=145, pit=255),
        phi=dict(pit=305, bal=145, syr=245, new=150),
        new=dict(syr=260, phi=150, bos=270, pro=260),
        pro=dict(bos=90, new=260),
        bos=dict(pro=90, new=270, syr=290, por=120),
        por=dict(bos=120))
    
    def path(previous, s): 
        '''
        `previous` is a dictionary chaining together the predecessor state that led to each state
        `s` will be None for the initial state
        otherwise, start from the last state `s` and recursively trace `previous` back to the initial state,
        constructing a list of states visited as we go
        '''
        if s is None:
            return []
        else:
            return path(previous, previous[s])+[s]
    
    def pathcost(path, step_costs):
        '''
        add up the step costs along a path, which is assumed to be a list output from the `path` function above
        '''
        cost = 0
        for s in range(len(path)-1):
            cost += step_costs[path[s]][path[s+1]]
        return cost
    
    
    """ breadth_first - Performs a breadth first search on cities
        Input:
            start - Node that represents the start of the path
            goal - Node that represents the desired end point of the path
            state_graph - Represents the graph that is being searched in
            return_cost - Boolean value that indicates the cost of the path
        Algorithm:
            * Create a queue with the start node as the first node
            * Create a visited set where the first node is visited
            * Create a dictionary for the previous nodes that have been visited
            * While the queue is not empty:
                * Pop the current node from the queue
                * If we reach the goal
                    * Create a path with the previous nodes and the goal
                    * Return the path and the cost if return_cost is set to true, otherwise just the path
                * Iterate over the neighbors of the current node
                * Add the neighbor to the visited set if it isn't visited
                * Update the previous node with the current node
                * Add the neighbor to the queue
            * Return the cost of the traversal
        Output:
            Returns the path in the search as well as the cost in the path
    """
    def breadth_first(start, goal, state_graph, return_cost=False):
        queue = deque([start])
        visited = set([start])
        previous = {start: None}
        while (queue):
            current = queue.popleft()
            if (current == goal):
                path_to_goal = path(previous, goal)
                if (return_cost):
                    cost = pathcost(path_to_goal, state_graph)
                    return path_to_goal, cost
                else:
                    return path_to_goal
            for neighbor in state_graph[current]:
                if (neighbor not in visited):
                    visited.add(neighbor)
                    previous[neighbor] = current
                    queue.append(neighbor)
        return None if not return_cost else (None, 0)
    
    # Solution:
    
    """ maze_to_graph - Converts a maze represented as a numpy array into a graph
        Input:
            maze - 2D numpy array where 0 represents walkable cells and 1 represents walls
        Algorithm:
            * Initialize an empty dictionary to represent the graph
            * Define the directions for North, South, East, and West movements
            * Iterate over each cell in the maze
                * Initialize an empty dictionary for each cell in the graph
                * For each direction, calculate the neighboring cell's coordinates
                * Check if the neighboring cell is within the maze bounds and is walkable (contains 0)
                    * If it is, add the neighbor to the current cell's dictionary in the graph with the direction as the value
        Output:
            Returns a dictionary representing the graph where keys are coordinates of cells and values are dictionaries
            of neighboring cells with directions
    """
    def maze_to_graph(maze):
        ''' takes in a maze as a numpy array, converts to a graph '''
        graph = {}
        rows, cols = maze.shape
        directions = {
            'N': (1, 0),  # North
            'S': (-1, 0),   # South
            'E': (0, 1),   # East
            'W': (0, -1)   # West
        }
        for r in range(rows):
            for c in range(cols):
                graph[(c, r)] = {}
                for direction, (dr, dc) in directions.items():
                    nr, nc = r + dr, c + dc
                    if 0 <= nr < rows and 0 <= nc < cols and maze[nr, nc] == 0:
                        graph[(c, r)][(nc, nr)] = direction
        return graph
    \end{code}
    \end{highlight}
\end{problem}