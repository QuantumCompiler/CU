\clearpage
\chapter{Week 8: (10/16 - 10/22)}

\section{Vision 3C}
\horizontalline{0}{0}

\subsection{Assigned Reading}

The reading assignments for this week can be found below:

\begin{itemize}
    \item \pdflink{./Reading Material/Mental Imagery And The Visual System.pdf}{Mental Imagery And The Visual System}
\end{itemize}

\subsection{Piazza}

Must post / respond to at least \textbf{five} Piazza posts this week. \due{(10/20/23)} \cbox{piazza-week8}

\subsection{Lectures}

The lectures for this week and their links can be found below:

\begin{itemize}
    \item \href{https://applied.cs.colorado.edu/mod/hvp/view.php?id=49385}{Mental Imagery And The Brain} $\approx$ 63 min.
    \item \href{https://applied.cs.colorado.edu/mod/hvp/view.php?id=49386}{Mental Imagery And The "Turn Towards Neuroscience"} $\approx$ 61 min.
\end{itemize}

\noindent Below is a list of lecture notes for this week:

\begin{itemize}
    \item \pdflink{./Lecture Notes/Mental Imagery Lecture Notes.pdf}{Mental Imagery}
\end{itemize}

\subsection{Assignments}

The assignment for this week is \textbf{Mini-Project Three}. \due{(11/3/23)} \cbox{minproj3}

\subsection{Quiz}

The quiz's for this week can be found below:

\begin{itemize}
    \item \href{https://applied.cs.colorado.edu/mod/quiz/view.php?id=49389}{Mental Imagery} \due{(10/23/23)} \cbox{quiz-week8}
\end{itemize}

\subsection{Chapter Summary}

Below is a summary of the reading assignment `Mental Imagery And The Visual System'.

\begin{notes}{Mental Imagery And The Visual System}
    Mental imagery and the visual system are intricately connected aspects of human cognition and perception. Mental imagery refers to the ability of the mind to generate and manipulate visual 
    representations of objects, scenes, or concepts without actual sensory input. This phenomenon plays a crucial role in various cognitive processes and is closely intertwined with the functioning 
    of the visual system.

    The visual system, encompassing the eyes and the brain's neural networks, is responsible for processing and interpreting visual information from the external world. However, mental imagery 
    operates internally, allowing individuals to "see" images in their mind's eye without the need for external stimuli. Understanding the relationship between mental imagery and the visual system 
    offers valuable insights into human cognition.

    Research in this field explores how mental imagery is generated, its neural underpinnings, and the functional roles it serves. Studies using neuroimaging techniques like fMRI have identified 
    brain regions associated with mental imagery, indicating that the same neural pathways involved in processing external visual stimuli are activated during mental imagery. Moreover, mental 
    imagery contributes to memory, problem-solving, and creative thinking, showcasing its significance in various cognitive domains.

    The study of mental imagery and the visual system has practical applications in areas like psychology, education, sports, and rehabilitation. It can enhance learning and memory strategies, 
    aid in therapeutic interventions, and even help athletes visualize and improve their performance. Overall, this topic bridges the realms of psychology, neuroscience, and cognitive science, 
    shedding light on how the mind's eye interacts with the external world through the visual system.
\end{notes}