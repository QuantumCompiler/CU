\clearpage

\renewcommand{\ChapTitle}{Vision 3A \& 3B}
\renewcommand{\SectionTitle}{Vision 3A \& 3B}

\chapter{\ChapTitle}
\section{\SectionTitle}
\horizontalline{0}{0}

\subsection{Assigned Reading}

The reading assignments for this week can be found below:

\begin{itemize}
    \item \pdflink{\ReadingMatDir Vision.pdf}{Vision}
\end{itemize}

\subsection{Piazza}

Must post / respond to at least \textbf{five} Piazza posts this week. \due{(10/13/23)} \cbox{piazza-week7}

\subsection{Lectures}

The lectures for this week and their links can be found below:

\begin{itemize}
    \item \href{https://www.youtube.com/watch?v=c8yacaB0TEs}{3.1 Vision as a Computational Problem} $\approx$ 19 min.
    \item \href{https://www.youtube.com/watch?v=4-LCyFr6hRQ}{3.2 Finding Edges} $\approx$ 31 min.
    \item \href{https://www.youtube.com/watch?v=_psVot3Y8Lo}{3.3 Depth Perception} $\approx$ 36 min.
    \item \href{https://www.youtube.com/watch?v=OBm_lyWIm9Q}{3.4 Object Recognition} $\approx$ 37 min.
\end{itemize}

\noindent Below is a list of lecture notes for this week:

\begin{itemize}
    \item \pdflink{\LectNotesDir Vision - Depth Perception And Object Recognition Lecture Notes.pdf}{Heuristics Lecture Notes}
\end{itemize}

\subsection{Assignments}

The assignment for this week is \pdflink{\AssDir Assignment 3 - Pixel Spreadsheet.pdf}{Assignment 3 - Pixel Spreadsheet}. \due{(11/3/23)} \cbox{minproj3}

\subsection{Quiz}

The quiz's for this week can be found below:

\begin{itemize}
    \item \href{https://applied.cs.colorado.edu/mod/quiz/view.php?id=49380}{Vision} \due{(10/16/23)} \cbox{quiz-week7}
    \item \pdflink{\QuizDir Quiz 5 - Vision 3A & 3B.pdf}{Quiz 5 - Vision 3A \& 3B}
\end{itemize}

\subsection{Chapter Summary}

Below is a summary of the reading assignment `Vision'.

\begin{notes}{Vision}
    Computer vision is a multidisciplinary field of artificial intelligence and computer science focused on enabling machines, particularly computers and robots, to interpret and understand visual 
    information from the world, much like the way humans do with their vision. It aims to replicate and simulate human vision processes to enable machines to perceive and interact with the visual
    world effectively.

    At its core, computer vision involves the development of algorithms and systems that can analyze and extract information from images or video data. This process includes tasks like object 
    recognition, scene understanding, motion tracking, facial recognition, and more. The field has wide-ranging applications, including autonomous vehicles, surveillance systems, medical image 
    analysis, industrial automation, augmented and virtual reality, and even creative arts.

    The relationship between computer vision and human vision is profound. Computer vision researchers draw inspiration from the human visual system, seeking to understand and replicate its mechanisms 
    for processing and interpreting visual information. Key aspects of human vision, such as feature detection, pattern recognition, and depth perception, serve as foundational principles for 
    computer vision algorithms. Additionally, computer vision systems are designed to mimic how humans perceive and understand the visual world, albeit in a more computational and systematic manner.

    While computer vision has made significant strides, it still faces challenges in achieving the robustness and versatility of human vision. Human vision is highly adaptable, capable of 
    recognizing objects in various lighting conditions, orientations, and cluttered environments. Achieving this level of flexibility and generalization in computer vision remains an ongoing 
    research goal. Nonetheless, computer vision continues to evolve and contribute to a wide range of industries, promising innovative solutions to complex real-world problems.
\end{notes}