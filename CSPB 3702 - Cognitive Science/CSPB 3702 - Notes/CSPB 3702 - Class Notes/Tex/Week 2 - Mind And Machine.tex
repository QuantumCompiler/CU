\clearpage
\chapter{Week 2: (9/4 - 9/10)}

\section{Mind And Machine 1B}
\horizontalline{0}{0}

\subsection{Assigned Reading}

The reading assignments for this week can be found below:

\begin{itemize}
    \item \pdflink{./Reading Material/The Nature of Mental States.pdf}{The Nature of Mental States}
\end{itemize}

\subsection{Piazza}

Must post / respond to at least \textbf{five} Piazza posts this week. \due{(9/8/23)} \cbox{piazza-week2}

\subsection{Lectures}

The lectures for this week and their links can be found below:

\begin{itemize}
    \item \href{https://applied.cs.colorado.edu/mod/hvp/view.php?id=49327}{What Makes a Problem Hard? Part 1} $\approx$ 23 min.
    \item \href{https://applied.cs.colorado.edu/mod/hvp/view.php?id=49328}{What Makes a Problem Hard? Part 2} $\approx$ 35 min.
    \item \href{https://applied.cs.colorado.edu/mod/hvp/view.php?id=49329}{What Makes a Problem Hard? Part 3} $\approx$ 23 min.
\end{itemize}

\noindent Below is a list of lecture notes for this week:

\begin{itemize}
    \item \pdflink{./Lecture Notes/What Makes a Problem Hard Lecture Notes.pdf}{What Makes A Problem Hard}
    \item \href{https://www.youtube.com/watch?v=6Lm9EHhbJAY}{Why Can You Not Trisect An Angle?}
    \item \href{https://www.youtube.com/watch?v=SL2lYcggGpc}{Trisect An Angle With Origami}
\end{itemize}

\subsection{Assignments}

The assignment for this week is \textbf{Mini-Project One}. \due{(9/15/23)} \cbox{minproj2}

\subsection{Quiz}

The quiz's for this week can be found below:

\begin{itemize}
    \item \href{https://applied.cs.colorado.edu/mod/quiz/view.php?id=49332}{Mind And Machine 1B} \due{(9/11/23)} \cbox{quiz-week2}
\end{itemize}

\subsection{Chapter Summary}

Below is a summary of the reading assignment `The Nature of Mental States'.

\begin{notes}{The Nature of Mental States}
    `The Nature of Mental States' by Hilary Putnam is a seminal essay that delves into the intricate relationship between mental states and their physical counterparts, touching on crucial themes in
    philosophy of mind, cognitive science, and the exploration of artificial intelligence. Published in 1967, this essay continues to resonate with contemporary discussions and remains a cornerstone 
    in the study of the mind and its connection to the physical world.

    Putnam's central argument revolves around the concept of functionalism, which stands in contrast to the traditional `type identity theory.' Type identity theory posits that mental states are 
    directly identical to specific brain states—a viewpoint that Putnam challenges by presenting the thought experiment of the "brain in a vat." This provocative scenario prompts us to contemplate the 
    possibility that our mental experiences might be disconnected from the external world, thereby raising questions about the nature of reality, knowledge, and consciousness.
    
    Functionalism, as presented by Putnam, posits that mental states are better understood through their functional roles within cognitive systems and their relationships with other mental states. This 
    approach allows for the possibility that different physical systems could give rise to identical mental states if their functional organizations match. This notion has far-reaching implications for 
    the philosophy of mind, as it redefines the criteria for identifying and explaining mental states.
    
    In the context of cognitive science, `The Nature of Mental States' holds immense relevance. Cognitive science seeks to understand the mind's processes, its relationship to the brain, and the nature 
    of human consciousness. Putnam's functionalist perspective aligns with cognitive science's focus on studying mental processes in terms of information processing, representation, and computation. The 
    essay also touches upon the nature of artificial intelligence and how we might conceptualize the mental states of AI systems.
    
    In the context of the `Mind and Machine' module within cognitive science, this essay can provide a foundational understanding of the relationship between mental states and physical systems. The 
    exploration of functionalism can shed light on how cognitive processes might be modeled and understood in both human minds and artificial systems. Additionally, Putnam's thought experiment can stimulate 
    discussions on the boundaries of human cognition, the challenges of studying consciousness, and the implications of artificial intelligence for our understanding of the mind.
    
    By engaging with `The Nature of Mental States,' one will delve into the heart of questions that underpin cognitive science, including the nature of consciousness, the relationship between mind and body, 
    and the potential for replicating cognitive processes in machines. This essay offers a thought-provoking journey into the intersection of philosophy, cognitive science, and the study of artificial 
    intelligence, enriching one's understanding of the complexities of the mind and its connection to machines.
\end{notes}