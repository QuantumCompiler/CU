\clearpage
\chapter{Week 12: (11/13 - 11/19)}

\section{Games 4B}
\horizontalline{0}{0}

\subsection{Assigned Reading}

The reading assignments for this week can be found below:

\begin{itemize}
    \item \pdflink{./Reading Material/The Evolution Of Institutions For Collective Action.pdf}{The Evolution Of Institutions For Collective Action}
\end{itemize}

\subsection{Piazza}

Must post / respond to at least \textbf{five} Piazza posts this week. \due{(11/17/23)} \cbox{piazza-week12}

\subsection{Lectures}

The lectures for this week and their links can be found below:

\begin{itemize}
    \item \href{https://applied.cs.colorado.edu/mod/hvp/view.php?id=49420}{Games "In the Large"} $\approx$ 49 min.
\end{itemize}

\noindent Below is a list of lecture notes for this week:

\begin{itemize}
    \item \pdflink{./Lecture Notes/Games In The Large Lecture Notes.pdf}{Games In The Large Lecture Notes}
\end{itemize}

\subsection{Assignments}

The assignment for this week is \href{https://applied.cs.colorado.edu/mod/assign/view.php?id=49425}{Mini Project 4} \due{(12/1/23)} \cbox{minproj4}

\subsection{Quiz}

The quiz's for this week can be found below:

\begin{itemize}
    \item \href{https://applied.cs.colorado.edu/mod/quiz/view.php?id=49423}{Games} \due{(11/20/23)} \cbox{quiz-week12}
\end{itemize}

\subsection{Chapter Summary}

Below is a summary of the reading assignment `The Evolution Of Institutions For Collective Action'.

\begin{notes}{The Evolution Of Institutions For Collective Action}
    \subsubsection*{Overview}

    "The Evolution of Institutions for Collective Action" is a seminal work by Elinor Ostrom that explores the development and functioning of institutions designed to facilitate collective action and 
    manage common-pool resources. The summary of the key concepts and ideas in this work includes:

    \begin{enumerate}
        \item \textbf{Common-Pool Resources (CPRs):} Ostrom's research focuses on common-pool resources, such as fisheries, forests, and irrigation systems, where multiple individuals or groups have 
        access and use rights. The challenge is to prevent overuse and depletion.
        \item \textbf{Institutional Analysis and Design (IAD) Framework:} Ostrom presents the IAD framework, which provides a systematic way to analyze and design institutions for managing CPRs. It 
        considers various factors, including the characteristics of the resource, the users, and the context.
        \item \textbf{Polycentric Governance:} Ostrom argues for polycentric governance, where multiple levels of authority and institutions coexist. This allows for flexibility and adaptation to 
        local conditions, fostering sustainable resource management.
        \item \textbf{Eight Design Principles:} Ostrom identifies eight design principles that successful CPR institutions tend to follow. These principles include clear boundaries, proportional costs 
        and benefits, and collective-choice arrangements.
        \item \textbf{Empirical Studies:} Ostrom's work is grounded in extensive empirical studies of real-world CPR management, challenging the conventional wisdom that CPRs are inevitably subject 
        to the "tragedy of the commons."
        \item \textbf{Policy Implications:} Ostrom's research has significant policy implications, suggesting that communities can effectively manage CPRs without centralized government control. Her 
        work has influenced discussions on environmental and resource management.
    \end{enumerate}
    
    "The Evolution of Institutions for Collective Action" contributes to our understanding of how communities can self-organize and govern common-pool resources effectively. It emphasizes the importance 
    of context-specific solutions, collective decision-making, and the role of institutions in sustainable resource management.
\end{notes}