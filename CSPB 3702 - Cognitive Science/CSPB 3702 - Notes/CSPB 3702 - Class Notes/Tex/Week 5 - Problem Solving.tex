\clearpage
\chapter{Week 5: (9/25 - 10/1)}

\section{Problem Solving 2B}
\horizontalline{0}{0}

\subsection{Assigned Reading}

The reading assignments for this week can be found below:

\begin{itemize}
    \item \pdflink{./Reading Material/Judgement Under Uncertainty - Heuristics And Biases.pdf}{Judgement Under Uncertainty - Heuristics And Biases}
    \item \pdflink{./Reading Material/Prospect Theory.pdf}{Prospect Theory}
\end{itemize}

\subsection{Piazza}

Must post / respond to at least \textbf{five} Piazza posts this week. \due{(9/29/23)} \cbox{piazza-week5}

\subsection{Lectures}

The lectures for this week and their links can be found below:

\begin{itemize}
    \item \href{https://applied.cs.colorado.edu/mod/hvp/view.php?id=49359}{2.4 Machines And Logic} $\approx$ 25 min.
    \item \href{https://applied.cs.colorado.edu/mod/hvp/view.php?id=49360}{2.5 Judgement And Decision Makeing} $\approx$ 28 min.
    \item \href{https://applied.cs.colorado.edu/mod/hvp/view.php?id=49361}{2.6 Heuristics And Biases In Judgement} $\approx$ 22 min.
\end{itemize}

\noindent Below is a list of lecture notes for this week:

\begin{itemize}
    \item \pdflink{./Lecture Notes/Heuristics Lecture Notes.pdf}{Heuristics}
\end{itemize}

\subsection{Assignments}

The assignment for this week is \textbf{Mini-Project Two}. \due{(10/6/23)} \cbox{minproj2}

\subsection{Quiz}

The quiz's for this week can be found below:

\begin{itemize}
    \item \href{https://applied.cs.colorado.edu/mod/quiz/view.php?id=49364}{Problem Solving} \due{(10/2/23)} \cbox{quiz-week5}
\end{itemize}

\subsection{Chapter Summary}

Below is a summary of the reading assignment `Judgement Under Uncertainty - Heuristics And Biases'.

\begin{notes}{Judgement Under Uncertainty - Heuristics And Biases}
    "Judgment under Uncertainty: Heuristics and Biases" by Amos Tversky and Daniel Kahneman represents a landmark in the field of behavioral economics and cognitive psychology. This influential paper 
    dives deep into the intricacies of human decision-making when confronted with situations characterized by uncertainty and limited information. 

    At the heart of the paper are the concepts of heuristics and biases. Heuristics are the mental shortcuts or rules of thumb that individuals employ when faced with complex choices or judgments. These 
    cognitive shortcuts help simplify decision-making processes, allowing people to make relatively quick assessments in situations where exhaustive analysis would be impractical. While heuristics serve 
    as valuable cognitive tools, Tversky and Kahneman's research reveals that they often lead to systematic cognitive biases—predictable patterns of irrational judgment.

    The paper identifies and elucidates various cognitive biases that arise from heuristics. For example, the availability heuristic is a tendency to rely on readily available information when making 
    judgments, often leading individuals to overestimate the likelihood of events based on their ease of recall. The representativeness heuristic involves making judgments based on how closely an object 
    or event matches a prototype or stereotype, which can lead to errors in probabilistic reasoning. Additionally, the anchoring bias describes the phenomenon where individuals rely heavily on the first 
    piece of information encountered (the anchor) when making subsequent judgments, even when the anchor is irrelevant or misleading.

    These cognitive biases are not mere curiosities; they have profound implications for decision-making in various contexts. From finance to healthcare, from politics to personal choices, individuals 
    regularly encounter situations where these biases influence their judgments and decisions. Recognizing these biases and understanding how they operate is crucial for improving decision-making 
    processes, designing effective policies, and developing strategies to mitigate their impact.

    "Judgment under Uncertainty" has played a pivotal role in reshaping how we view human decision-making. It challenges the traditional economic model of humans as perfectly rational agents and instead 
    paints a more nuanced picture of cognitive limitations and the systematic errors we are prone to make. This paper's enduring legacy has led to a flourishing field of research in behavioral economics 
    and cognitive psychology, inspiring numerous studies and practical applications aimed at helping individuals make better, more informed decisions in an uncertain world.
\end{notes}

Below is a summary of the reading assignment `Prospect Theory'.

\begin{notes}{Prospect Theory}
    Prospect Theory, developed by Daniel Kahneman and Amos Tversky, represents a significant departure from traditional economic models, offering a more comprehensive and psychologically grounded framework 
    for understanding human decision-making in situations involving risk and uncertainty. This theory challenges the conventional economic concept of "expected utility theory," which assumes that individuals 
    make rational choices by maximizing expected outcomes. Instead, Prospect Theory introduces two fundamental behavioral elements: the reference point and loss aversion.

    The concept of the reference point plays a central role in Prospect Theory, as individuals assess potential outcomes in relation to this point, often rooted in their current circumstances or the status 
    quo. This reference point acts as a psychological anchor against which gains and losses are measured. For example, receiving a sudden windfall of \$1,000 may be perceived differently depending on whether 
    it elevates someone from financial hardship to security or simply adds to their existing wealth.
    
    Another key tenet of Prospect Theory is loss aversion, which posits that individuals tend to be more averse to losses than they are inclined to value equivalent gains. In practical terms, this implies 
    that the emotional impact of losing \$100 is typically more significant than the pleasure derived from gaining an identical amount, leading to risk-averse behavior in scenarios involving potential losses.
    
    Additionally, Prospect Theory introduces the notion of "diminishing sensitivity to gains and losses," suggesting that as individuals move farther from their reference point, their emotional responses to 
    changes in wealth become less pronounced. For instance, the emotional contrast between gaining \$1,000 and gaining an additional \$1,000 is usually less substantial than the difference between gaining 
    and losing \$1,000.
    
    The real-world implications of Prospect Theory are substantial, particularly in fields like finance, investment, and behavioral economics. It provides insights into various phenomena, including how loss 
    aversion induces risk aversion, the disposition effect (the tendency to prematurely sell winning investments and hold onto losing ones), and the equity premium puzzle (the question of why people demand 
    a premium for holding risky assets). By offering a more psychologically realistic lens through which to view human decision-making, Prospect Theory has deepened our understanding of how individuals 
    navigate the intricate landscape of uncertainty and risk in practical scenarios.
\end{notes}