\clearpage

\renewcommand{\ChapTitle}{Vision 3D}
\renewcommand{\SectionTitle}{Vision 3D}

\chapter{\ChapTitle}
\section{\SectionTitle}
\horizontalline{0}{0}

\subsection{Assigned Reading}

The reading assignments for this week can be found below:

\begin{itemize}
    \item \textbf{Mind, Body, World: Foundations of Cognitive Science}
\end{itemize}

\subsection{Piazza}

Must post / respond to at least \textbf{five} Piazza posts this week.

\subsection{Lectures}

The lectures for this week and their links can be found below:

\begin{itemize}
    \item \href{https://www.youtube.com/watch?v=XAN8u7voWw0}{Neural Networks: Perceptrons} $\approx$ 47 min.
    \item \href{https://www.youtube.com/watch?v=zAXDn4zGXxs}{Neural Networks 2: Multi-Layer Networks} $\approx$ 25 min.
    \item \href{https://www.youtube.com/watch?v=zttSZyM8HPI}{Deep Learning - Object Recognition} $\approx$ 27 min.
\end{itemize}

\subsection{Assignments}

The assignment for this week is:

\begin{itemize}
    \item \pdflink{\AssDir Assignment 3 - Pixel Spreadsheet.pdf}{Assignment 3 - Pixel Spreadsheet}
\end{itemize}

\subsection{Quiz}

The quiz's for this week can be found below:

\begin{itemize}
    \item \pdflink{\QuizDir Quiz 7 - Vision 3D.pdf}{Quiz 7 - Vision 3D}
\end{itemize}

\subsection{Chapter Summary}

Below is a summary of the reading assignment `Mind, Body, World: Foundations of Cognitive Science'.

\begin{notes}{Mind, Body, World: Foundations of Cognitive Science}
    "Elements of Connectionist Cognitive Science" by Michael R. W. Dawson is a foundational book within the realm of cognitive science that delves deep into the principles and applications of 
    connectionist modeling. Connectionism, a computational framework inspired by the neural networks in the human brain, is at the core of this exploration. Dawson introduces readers to the 
    fundamental concepts of connectionism, emphasizing that cognitive processes can be seen as interactions among simple processing units connected through weighted links. These networks have the 
    capacity to adapt and learn from experience, making them potent tools for modeling various aspects of human cognition.

    The book covers different types of neural network architectures, including feedforward and recurrent networks, and elucidates how these structures relate to specific cognitive processes. Learning 
    algorithms, particularly backpropagation, are discussed in detail. Backpropagation is a pivotal learning algorithm that enables neural networks to adjust their connection weights based on prediction 
    errors, mirroring the process of learning and adaptation in the human brain.
    
    Dawson demonstrates how connectionist models can be applied to simulate and understand diverse cognitive functions, ranging from memory retrieval and language processing to pattern recognition. 
    Additionally, the book navigates the connections and distinctions between connectionism and other cognitive theories, such as symbolic and information-processing models, shedding light on the 
    strengths and weaknesses of connectionist approaches and their potential to complement or challenge existing theories.
    
    Moreover, "Elements of Connectionist Cognitive Science" explores the practical implications of connectionist cognitive science, notably its relevance to artificial intelligence and machine learning. 
    Connectionist models have significantly influenced the development of neural networks in AI, contributing to their successes in various applications. In summary, this book provides an extensive and 
    insightful exploration of connectionism as a framework for comprehending and modeling human cognition, making it a valuable resource for researchers, students, and anyone intrigued by the computational 
    principles underpinning the intricate processes of the human mind.
\end{notes}