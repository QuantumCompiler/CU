\clearpage

\renewcommand{\ChapTitle}{Problem Solving 2A}
\renewcommand{\SectionTitle}{Problem Solving 2A}

\chapter{\ChapTitle}
\section{\SectionTitle}
\horizontalline{0}{0}

\subsection{Assigned Reading}

The reading assignments for this week can be found below:

\begin{itemize}
    \item \pdflink{\ReadingMatDir Recognizing, Defining, and Representing Problems.pdf}{Recognizing, Defining, and Representing Problems}
\end{itemize}

\subsection{Piazza}

Must post / respond to at least \textbf{five} Piazza posts this week.

\subsection{Lectures}

The lectures for this week and their links can be found below:

\begin{itemize}
    \item \href{https://www.youtube.com/watch?v=YvkDeQP-pQM}{Solving Problems} $\approx$ 25 min.
    \item \href{https://www.youtube.com/watch?v=8h0okzEM13M}{Problems For Minds And Machines} $\approx$ 32 min.
    \item \href{https://www.youtube.com/watch?v=6qwk8NKR0UM}{Hard Problems For Computers} $\approx$ 23 min.
\end{itemize}

\noindent Below is a list of lecture notes for this week:

\begin{itemize}
    \item \pdflink{\LectNotesDir Problem Solving Lecture Notes.pdf}{Problem Solving Lecture Notes}
\end{itemize}

\subsection{Assignments}

The assignment for this week is:

\begin{itemize}
    \item \pdflink{\AssDir Assignment 2 - Problem Solving.pdf}{Assignment 2 - Problem Solving}
\end{itemize}

\subsection{Quiz}

The quiz's for this week can be found below:

\begin{itemize}
    \item \pdflink{\QuizDir Quiz 3 - Problem Solving 1A.pdf}{Quiz 3 - Problem Solving 1A}
\end{itemize}

\subsection{Chapter Summary}

Below is a summary of the reading assignment `Recognizing, Defining, and Representing Problems'.

\begin{notes}{Recognizing, Defining, And Representing Problems}
    The exploration of recognizing, defining, and representing problems stands at the forefront of cognitive psychology and problem-solving research. It delves into the intricate cognitive processes and 
    strategies individuals employ when confronted with a wide array of challenges and predicaments. This field encompasses a multifaceted journey of understanding, beginning with problem recognition. In 
    this initial phase, individuals discern the presence of a problem, often detecting disparities between their present state and an aspired one. Such recognition relies upon the capacity to spot 
    discrepancies, irregularities, or hurdles that necessitate attention and resolution.

    Once a problem is identified, the subsequent step involves problem definition, a pivotal stage where the problem must be clearly and explicitly articulated. This encompasses the delineation of the 
    problem's nature, boundaries, as well as the establishment of its objectives, constraints, and pertinent factors. Effective problem definition is of paramount importance, for it serves as the foundation 
    upon which problem-solving strategies are constructed.

    Problem representation follows, encompassing the mental structuring and organization of problem-related information. This cognitive process often entails the construction of mental models, diagrams, 
    or frameworks, facilitating a comprehensive grasp of the problem's intricacies, components, and relationships. A well-crafted problem representation significantly streamlines the ensuing problem-solving 
    endeavor.

    With a meticulously defined and represented problem in place, individuals proceed to employ a spectrum of problem-solving strategies. These strategies span from systematic, algorithmic approaches, 
    characterized by methodical, step-by-step procedures, to heuristic methods grounded in intuitive, rule-of-thumb decision-making. The selection of an appropriate strategy hinges upon the nature of the 
    problem at hand and the individual's level of expertise.

    Furthermore, the expertise possessed within a specific domain significantly influences problem recognition, definition, and representation. Experts tend to harbor more refined mental problem 
    representations, often capable of recognizing elusive patterns or solutions that might elude novices. Creativity emerges as another vital component in this narrative, stimulating the generation of 
    innovative problem definitions and representations.

    Lastly, metacognitive skills occupy a central role. These skills involve the active monitoring and regulation of one's problem-solving processes. Through metacognition, individuals continuously evaluate 
    the efficiency of their problem representations and strategies, making adjustments and refinements as deemed necessary.

    In summation, the effective ability to recognize, define, and represent problems constitutes a fundamental facet of human cognition and the broader realm of problem-solving. It forms the bedrock upon 
    which subsequent problem-solving efforts are constructed, profoundly influencing the quality and efficacy of the solutions derived. This intricate cognitive landscape continues to captivate researchers 
    in cognitive psychology, propelling the exploration of the underlying cognitive mechanisms and strategies that govern human problem-solving capacities.
\end{notes}