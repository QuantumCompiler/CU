\clearpage

\renewcommand{\ChapTitle}{Games 4A}
\renewcommand{\SectionTitle}{Games 4A}

\chapter{\ChapTitle}
\section{\SectionTitle}
\horizontalline{0}{0}

\subsection{Assigned Reading}

The reading assignments for this week can be found below:

\begin{itemize}
    \item \textbf{Is Tit-for-Tat The Answer?}
\end{itemize}

\subsection{Piazza}

Must post / respond to at least \textbf{five} Piazza posts this week.

\subsection{Lectures}

The lectures for this week and their links can be found below:

\begin{itemize}
    \item \href{https://www.youtube.com/watch?v=JDg9Z2W9Mfk}{Multiple Deciding Agents - A Simple Model} $\approx$ 23 min.
    \item \href{https://www.youtube.com/watch?v=6rD16HNCs50}{Game Theory: Prisoner's Dilemma} $\approx$ 36 min.
    \item \href{https://www.youtube.com/watch?v=5D5huKHZgu0}{Game Theory: Axelrod's Tournament} $\approx$ 26 min.
\end{itemize}

\noindent Below is a list of lecture notes for this week:

\begin{itemize}
    \item \pdflink{\LectNotesDir Multiple Deciding Agents Lecture Notes.pdf}{Multiple Deciding Agents Lecture Notes}
\end{itemize}

\subsection{Assignments}

The assignment for this week is:

\begin{itemize}
    \item \pdflink{\AssDir Assignment 4 - Making A Teachable Machine.pdf}{Assignment 4 - Making A Teachable Machine}
\end{itemize}

\subsection{Quiz}

The quiz's for this week can be found below:

\begin{itemize}
    \item \pdflink{\QuizDir Quiz 8 - Games 4A.pdf}{Quiz 8 - Games 4A}
\end{itemize}

\subsection{Chapter Summary}

Below is a summary of the reading assignment `Is Tit-For-Tat The Answer?'.

\begin{notes}{Is Tit-For-Tat The Answer?}
    \subsubsection*{Summary of Tit-for-Tat Strategy:}

    Tit-for-Tat is a cooperative strategy in which an individual initially cooperates with another individual or entity and subsequently mirrors their opponent's previous move. In other words, it 
    responds to cooperation with cooperation and defection with defection. The strategy's key features include:

    \begin{enumerate}
        \item \textbf{Start with Cooperation:} Tit-for-Tat begins with a cooperative move, assuming goodwill and cooperation from the opponent at the outset of an interaction.
        \item \textbf{Reciprocity:} It responds to the opponent's last move by reciprocating in kind. If the opponent cooperated in the previous round, Tit-for-Tat cooperates in the next round. If 
        the opponent defected, Tit-for-Tat defects in response.
        \item \textbf{No Retaliation:} Tit-for-Tat does not seek revenge or escalate conflicts. It merely reflects the opponent's behavior, maintaining a fair and balanced approach.
        \item \textbf{Adaptive:} Tit-for-Tat adapts to the opponent's actions over time. If the opponent consistently cooperates, Tit-for-Tat continues to cooperate. If the opponent defects, 
        Tit-for-Tat adjusts accordingly.
    \end{enumerate}
    
    The Tit-for-Tat strategy has been studied extensively in the context of the Prisoner's Dilemma and other game scenarios. What makes it remarkable is its simplicity, effectiveness in promoting 
    cooperation, and robustness in various competitive situations. It can lead to mutually beneficial outcomes in repeated interactions by fostering trust and encouraging reciprocation of positive 
    behavior. However, it is not immune to exploitation by strategies that continuously defect, making its success contingent on the behavior of the opponent. Despite this, Tit-for-Tat serves as a 
    valuable model for understanding cooperation, trust, and reciprocity in both human interactions and computer simulations.
\end{notes}