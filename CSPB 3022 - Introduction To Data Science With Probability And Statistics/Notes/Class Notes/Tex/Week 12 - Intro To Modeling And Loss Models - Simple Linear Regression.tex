\clearpage

\renewcommand{\ChapTitle}{Intro To Modeling And Loss Models: Simple Linear Regression}
\renewcommand{\SectionTitle}{Intro To Modeling And Loss Models: Simple Linear Regression}

\chapter{\ChapTitle}
\section{\SectionTitle}
\horizontalline{0}{0}

\subsection{Optional Reading}

The optional reading for this week is from \LearnDSBook \hspace*{1pt} and \PyDSBook.

\subsection{Piazza}

Must post / respond to two Piazza posts.

\subsection{Lectures}

The lecture videos for this week are:

\begin{itemize}
    \item \lecture{https://applied.cs.colorado.edu/pluginfile.php/76485/mod_resource/content/1/CSCI3022_002_CASEE240_11_8_2023\%20\%281\%29.mp4}{Intro To Modeling And Loss Models}{54}
    \item \lecture{https://applied.cs.colorado.edu/pluginfile.php/76486/mod_resource/content/1/Lecture27.mp4}{Simple Linear Regression}{23}
    \item \lecture{https://applied.cs.colorado.edu/pluginfile.php/76487/mod_resource/content/1/CSCI3022_002_CASEE240_11_13_2023.mp4}{SLR Inference}{54}
    \item \lecture{https://applied.cs.colorado.edu/pluginfile.php/76488/mod_resource/content/1/lecture29.mp4}{sklearn And Linear Regression}{21}
    \item \lecture{https://applied.cs.colorado.edu/pluginfile.php/76489/mod_resource/content/1/notebook11_video.mp4}{Simple And Multiple Linear Regression Walkthrough}{40}
\end{itemize}

\noindent The lecture notes for this week are:

\begin{itemize}
    \item \pdflink{\LecNoteDir Introduction To Modeling Lecture Notes.pdf}{Introduction To Modeling Lecture Notes}
    \item \pdflink{\LecNoteDir Simple Linear Regression Lecture Notes.pdf}{Simple Linear Regression Lecture Notes}
    \item \pdflink{\LecNoteDir Inference With Simple Linear Regression Lecture Notes.pdf}{Inference With Simple Linear Regression Lecture Notes}
    \item \pdflink{\LecNoteDir Sklearn And Multiple Linear Regression Lecture Notes.pdf}{Sklearn And Multiple Linear Regression Lecture Notes}
\end{itemize}

\subsection{Assignments}

The assignment for this week is:

\begin{itemize}
    \item \href{https://github.com/QuantumCompiler/CU/tree/main/CSPB%203022%20-%20Introduction%20To%20Data%20Science%20With%20Probability%20And%20Statistics/Assignments/Assignment%2010%20-%20Intro%20To%20Modeling%20And%20Loss%20Models%20-%20Simple%20Linear%20Regression}{Assignment 10 - Intro To Modeling And Loss Models: Simple Linear Regression}
\end{itemize}

\newpage

\subsection{Concept Summary}

The concept that is being covered this week is \textbf{Intro To Modeling And Loss Models: Simple Linear Regression}.

\begin{notes}{Intro To Modeling And Loss Models: Simple Linear Regression}
    \subsection*{Overview}

    Modeling is a fundamental aspect of statistical analysis and predictive analytics, allowing researchers and analysts to understand and predict behaviors based on observed data. Simple Linear 
    Regression is one of the most basic forms of statistical modeling techniques used to predict an outcome variable (dependent variable) based on one predictor variable (independent variable). It 
    provides a clear and straightforward way to quantify the relationship between two variables and is extensively used in economics, finance, natural sciences, and social sciences. \vspace*{1em}
    
    \subsubsection*{Principles of Simple Linear Regression}
    
    Simple Linear Regression aims to model the relationship between two variables by fitting a linear equation to observed data. The steps in building a regression model include:
    \begin{itemize}
        \item \textbf{Model Formulation}: Determine the dependent variable and the independent variable. The model takes the form $Y = \beta_0 + \beta_1X + \epsilon$, where $Y$ is the dependent 
        variable, $X$ is the independent variable, $\beta_0$ is the y-intercept, $\beta_1$ is the slope of the line, and $\epsilon$ is the error term.
        \item \textbf{Parameter Estimation}: Use statistical methods, typically the method of least squares, to estimate the parameters $\beta_0$ and $\beta_1$ that minimize the sum of the squared 
        difference between the observed values and the values predicted by the model.
    \end{itemize}
    
    \subsubsection*{Assumptions of Simple Linear Regression}
    
    For the model to provide reliable predictions, several key assumptions must be satisfied:
    \begin{itemize}
        \item \textbf{Linearity}: The relationship between the independent and dependent variable must be linear.
        \item \textbf{Independence}: Observations must be independent of each other.
        \item \textbf{Homoscedasticity}: The variance of residual terms (differences between observed and predicted values) should be constant.
        \item \textbf{Normality}: The residuals of the model should be normally distributed.
    \end{itemize}
    
    \subsubsection*{Loss Models in Simple Linear Regression}
    
    In the context of regression, the concept of 'loss' refers to the penalty for a bad prediction, which in this case is quantified as the error between the observed values and the values predicted by the model:
    \begin{itemize}
        \item \textbf{Squared Loss Function}: Simple Linear Regression typically uses the squared loss function (also known as least squares), where the loss is calculated as the square of the difference 
        between the predicted and actual values. This method emphasizes larger errors more significantly than smaller ones, which can be particularly useful in finding a line of best fit that minimizes these errors.
    \end{itemize}
    
    \subsubsection*{Evaluating Model Performance}
    
    The performance of a simple linear regression model can be evaluated using several metrics:
    \begin{itemize}
        \item \textbf{Coefficient of Determination ($R^2$)}: Measures the proportion of the variance in the dependent variable that is predictable from the independent variable.
        \item \textbf{Residual Standard Error}: Measures the average amount that the response will deviate from the true regression line.
    \end{itemize}
    
    \subsubsection*{Summary}
    
    Simple Linear Regression is a powerful tool for predictive modeling, providing valuable insights into linear relationships between variables. Understanding its principles, assumptions, and methods 
    for evaluating its performance is crucial for effectively applying this technique in practical scenarios, ensuring that predictions are both accurate and reliable.    
\end{notes}