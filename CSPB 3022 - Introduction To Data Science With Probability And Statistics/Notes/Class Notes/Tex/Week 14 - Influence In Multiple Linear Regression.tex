\clearpage

\renewcommand{\ChapTitle}{Influence In Multiple Linear Regression}
\renewcommand{\SectionTitle}{Influence In Multiple Linear Regression}

\chapter{\ChapTitle}
\section{\SectionTitle}
\horizontalline{0}{0}

\subsection{Optional Reading}

The optional reading for this week is from \LearnDSBook \hspace*{1pt} and \PyDSBook.

\subsection{Piazza}

Must post / respond to two Piazza posts.

\subsection{Lectures}

The lecture videos for this week are:

\begin{itemize}
    \item \lecture{https://applied.cs.colorado.edu/pluginfile.php/76521/mod_resource/content/1/CSCI3022_002_CASEE240_12_6_2023.mp4}{Influence In Multiple Linear Regression Models}{54}
    \item \lecture{https://applied.cs.colorado.edu/pluginfile.php/76522/mod_resource/content/1/quizwalkthrough.mp4}{Quiz Walkthroug, Project, Fairness In Housing Appraisal}{25}
    \item \lecture{https://applied.cs.colorado.edu/pluginfile.php/76523/mod_resource/content/2/nb12_video.mp4}{Modeling And Analyzing COVID-19 Cases Walkthrough}{55}
    \item \lecture{https://applied.cs.colorado.edu/pluginfile.php/76524/mod_resource/content/1/CSCI3022_001_ECCR200_12_11_2023.mp4}{Intro To Classification - Logistic Regression}{54}
\end{itemize}

\noindent The lecture notes for this week are:

\begin{itemize}
    \item \pdflink{\LecNoteDir Inference In Multiple Linear Regression Lecture Notes.pdf}{Inference In Multiple Linear Regression Lecture Notes}
    \item \pdflink{\LecNoteDir Logistic Regression Lecture Notes.pdf}{Logistic Regression Lecture Notes}
\end{itemize}

\subsection{Assignments}

The assignment for this week is:

\begin{itemize}
    \item \href{https://github.com/QuantumCompiler/CU/tree/main/CSPB%203022%20-%20Introduction%20To%20Data%20Science%20With%20Probability%20And%20Statistics/CSPB%203022%20-%20Final%20Project/CSPB%203022%20-%20Final%20Project%20Part%202}{Final Project Part 2}
\end{itemize}

\subsection{Concept Summary}

The concept that is being covered this week is \textbf{Influence In Multiple Linear Regression}.

\begin{notes}{Influence In Multiple Linear Regression}
    \subsection*{Overview}

    Influence in multiple linear regression refers to the effect that specific observations or groups of observations have on the fitted regression model. Highly influential points can disproportionately 
    affect the slope of the regression line and other inferential statistics, potentially leading to misleading results. Understanding and identifying influential observations are crucial for ensuring 
    the reliability and accuracy of regression analysis. \vspace*{1em}
    
    \subsubsection*{Assessing Influence in Multiple Linear Regression}
    
    The assessment of influence in multiple linear regression involves several statistical measures and diagnostics that help in identifying observations that have a significant impact on the coefficients 
    and predictions of the model:
    
    \begin{itemize}
        \item \textbf{Leverage}: Leverage measures the influence an observation has on its own fitted value, relative to its position with respect to the mean of the independent variables. High leverage 
        points are those that have extreme values on one or more predictors.
        \item \textbf{Residuals}: While residuals (the differences between observed and predicted values) themselves do not measure influence, large residuals, when combined with high leverage, can 
        indicate potential influence.
        \item \textbf{Cook's Distance}: A commonly used measure that combines the leverage of an observation with the size of its residual to determine its influence on the fitted values across all 
        observations in the dataset. Observations with large Cook's Distance values may be unduly influencing the regression model.
    \end{itemize}
    
    \subsubsection*{Mitigating the Effects of Influential Observations}
    
    Once influential observations are identified, there are several strategies for dealing with them to ensure the robustness of the model:
    
    \begin{itemize}
        \item \textbf{Robust Regression}: Using regression techniques that are less sensitive to outliers and influential points, such as weighted least squares or ridge regression, can help mitigate 
        the effects of these observations.
        \item \textbf{Removing Influential Points}: In cases where the influence is due to data errors or anomalies that do not represent the population, removing these points might be justified. 
        However, care must be taken to ensure that the removal of data does not bias the results.
        \item \textbf{Transforming Variables}: Applying transformations to the variables (e.g., logarithmic, square root) can reduce the influence of outliers by making the data more homoscedastic 
        (having uniform variance) and normally distributed.
    \end{itemize}
    
    \subsubsection*{Importance of Understanding Influence}
    
    Understanding the influence of observations in multiple linear regression is vital for:
    \begin{itemize}
        \item \textbf{Model Accuracy}: Ensuring that the model accurately reflects the underlying data without being overly affected by anomalies.
        \item \textbf{Decision Making}: Providing reliable information for decision-making processes in business, economics, health sciences, and other fields.
        \item \textbf{Statistical Inference}: Maintaining the integrity of statistical inferences made from the model.
    \end{itemize}
    
    \subsubsection*{Summary}
    
    Influence in multiple linear regression is a critical aspect of model diagnostics. It involves identifying and addressing observations that unduly affect the model's results, ensuring that the 
    conclusions drawn from the model are valid and representative of the broader data set. By effectively managing influential data points, analysts can enhance the robustness and reliability of 
    their regression models, leading to more accurate predictions and insights.    
\end{notes}