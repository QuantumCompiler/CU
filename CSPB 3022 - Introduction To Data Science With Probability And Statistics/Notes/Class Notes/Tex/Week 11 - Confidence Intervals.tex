\clearpage

\renewcommand{\ChapTitle}{Confidence Intervals}
\renewcommand{\SectionTitle}{Confidence Intervals}

\chapter{\ChapTitle}
\section{\SectionTitle}
\horizontalline{0}{0}

\subsection{Optional Reading}

The optional reading for this week is from \LearnDSBook \hspace*{1pt} and \PyDSBook.

\subsection{Piazza}

Must post / respond to two Piazza posts.

\subsection{Lectures}

The lecture videos for this week are:

\begin{itemize}
    \item \lecture{https://applied.cs.colorado.edu/pluginfile.php/76463/mod_resource/content/1/CSCI3022_002_CASEE240_11_1_2023.mp4}{Interpretting Confidence Intervals}{54}
    \item \lecture{https://applied.cs.colorado.edu/pluginfile.php/76464/mod_resource/content/1/CSCI3022_002_CASEE240_11_3_2023.mp4}{Confidence Intervals And Designing Experiments}{54}
    \item \lecture{https://applied.cs.colorado.edu/pluginfile.php/76465/mod_resource/content/2/lecture_26.mp4}{Modeling}{54}
    \item \lecture{https://applied.cs.colorado.edu/pluginfile.php/76466/mod_resource/content/1/nb10_video.mp4}{Climate Change Notebook}{44}
\end{itemize}

\noindent The lecture notes for this week are:

\begin{itemize}
    \item \pdflink{\LecNoteDir Confidence Interval Guide Lecture Notes.pdf}{Confidence Interval Guide Lecture Notes}
    \item \pdflink{\LecNoteDir Interpreting Confidence Intervals Lecture Notes.pdf}{Interpreting Confidence Intervals Lecture Notes}
    \item \pdflink{\LecNoteDir Confidence Intervals And Designing Experiments Lecture Notes.pdf}{Confidence Intervals And Designing Experiments Lecture Notes}
    \item \pdflink{\LecNoteDir Introduction To Modeling Lecture Notes.pdf}{Introduction To Modeling Lecture Notes}
\end{itemize}

\subsection{Assignments}

The assignment for this week is:

\begin{itemize}
    \item \href{https://github.com/QuantumCompiler/CU/tree/main/CSPB%203022%20-%20Introduction%20To%20Data%20Science%20With%20Probability%20And%20Statistics/Assignments/Assignment%209%20-%20Confidence%20Intervals}{Assignment 9 - Confidence Intervals}
\end{itemize}

\newpage

\subsection{Confidence Intervals}

The concept that is being covered this week is \textbf{Confidence Intervals}.

\begin{notes}{Confidence Intervals}
    \subsection*{Overview}

    Confidence intervals (CIs) are a fundamental statistical tool used to estimate the range within which a population parameter is likely to lie based on sample data. These intervals provide a 
    measure of the uncertainty associated with sampling variability and offer a range of values that are believed to cover the true parameter with a certain level of confidence, typically expressed 
    as 90\%, 95\%, or 99\%. Understanding and using confidence intervals correctly is essential for making informed decisions in scientific research, business analytics, and policy making. \vspace*{1em}
    
    \subsubsection*{Construction of Confidence Intervals}
    
    The process of constructing a confidence interval typically involves several key steps:
    \begin{itemize}
        \item \textbf{Determine the Sample Statistic}: Calculate a point estimate of the parameter (e.g., the sample mean or proportion) from the observed data.
        \item \textbf{Select the Confidence Level}: Choose the confidence level (e.g., 95\%) which reflects the degree of certainty desired in the estimate.
        \item \textbf{Calculate the Margin of Error}: Determine the margin of error for the interval using the appropriate standard error and the critical value from the corresponding distribution (usually z or t-distribution).
    \end{itemize}
    
    \subsubsection*{Significance and Interpretation}
    
    \textbf{Interpretation of Confidence Intervals}:
    \begin{itemize}
        \item A 95\% confidence interval means that if we were to take 100 different samples and compute a confidence interval for each, approximately 95 of these intervals would be expected to contain the true population parameter.
        \item Confidence intervals are particularly valuable in research for assessing the reliability of estimates and for comparing different studies or datasets.
    \end{itemize}
    
    \textbf{Importance of Confidence Intervals in Statistical Analysis}:
    \begin{itemize}
        \item \textbf{Assessing Estimate Precision}: Confidence intervals provide more information than a simple point estimate, indicating the range within which the true value lies and the precision of the estimate.
        \item \textbf{Testing Hypotheses}: Confidence intervals can be used for hypothesis testing. If a confidence interval does not contain a value of interest (e.g., zero in the case of estimating 
        differences), this can be seen as evidence against the null hypothesis that this value is true.
    \end{itemize}
    
    \subsubsection*{Limitations}
    
    While confidence intervals are a powerful statistical tool, they do have limitations:
    \begin{itemize}
        \item \textbf{Dependence on Correct Model Assumptions}: The accuracy of confidence intervals depends heavily on the correctness of the model and assumptions used in their construction, such 
        as the assumption of normality or independence.
        \item \textbf{Misinterpretation}: Confidence intervals can be misunderstood. They do not say that the true parameter has a 95\% chance of being within the interval—rather, they provide a 
        range that is likely to capture the true parameter in 95\% of samples.
    \end{itemize}
    
    \subsubsection*{Summary}
    
    Confidence intervals are an essential part of inferential statistics, aiding in decision making by providing a range of plausible values for an unknown parameter. They enhance the understanding 
    of data variability and help quantify the certainty of statistical estimates, thereby playing a crucial role in scientific research and quantitative analysis.    
\end{notes}