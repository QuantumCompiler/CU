\clearpage

\renewcommand{\ChapTitle}{Pandas, Exploring And Cleaning Tabular Data}
\renewcommand{\SectionTitle}{Pandas, Exploring And Cleaning Tabular Data}
\chapter{\ChapTitle}

\section{\SectionTitle}
\horizontalline{0}{0}

\subsection{Optional Reading}

The optional reading for this week is from \LearnDSBook \hspace*{1pt} and \PyDSBook.

\subsection{Piazza}

Must post / respond to two Piazza posts.

\subsection{Lectures}

The lecture videos for this week are:

\begin{itemize}
    \item \lecture{https://applied.cs.colorado.edu/pluginfile.php/76329/mod_resource/content/5/CSCI3022_002_CASEE240_9_6_2023.mp4}{Pandas Part 3}{54}
    \item \lecture{https://applied.cs.colorado.edu/pluginfile.php/76330/mod_resource/content/2/CSCI3022_002_CASEE240_9_8_2023.mp4}{Pandas Part 4}{54}
    \item \lecture{https://applied.cs.colorado.edu/pluginfile.php/76328/mod_resource/content/2/lab2_zoom\%20\%281\%29.mp4}{Discussion 2 Video Walkthrough: Basics with Pandas and In-Depth with NumPy}{73}
\end{itemize}

\noindent The lecture notes for this week are:

\begin{itemize}
    \item \pdflink{\LecNoteDir Pandas Part III Lecture Notes.pdf}{Pandas Part 3 Lecture Notes}
    \item \pdflink{\LecNoteDir Pandas Part IV Lecture Notes.pdf}{Pandas Part 4 Lecture Notes}
    \item \pdflink{\LecNoteDir Permutations And Combinations Lecture Notes.pdf}{Permutations And Combinations Lecture Notes}
    \item \pdflink{\LecNoteDir Section 7.1 Intro To Discrete Probability Lecture Notes.pdf}{Section 7.1 Intro To Discrete Probability Lecture Notes}
    \item \pdflink{\LecNoteDir Section 7.2 Probability Theory And Applications Lecture Notes.pdf}{Section 7.2 Probability Theory And Applications Lecture Notes}
\end{itemize}

\subsection{Assignments}

The assignment for this week is:

\begin{itemize}
    \item \href{https://github.com/QuantumCompiler/CU/tree/main/CSPB%203022%20-%20Introduction%20To%20Data%20Science%20With%20Probability%20And%20Statistics/Assignments/Assignment%202%20-%20Pandas%2C%20Exploring%20And%20Cleaning%20Tabular%20Data}{Assignment 2 - Pandas, Exploring And Cleaning Tabular Data}
\end{itemize}

\subsection{Quiz}

The quizzes for this week are:

\begin{itemize}
    \item \pdflink{\QuizDir Quiz 2 - Python And Pandas.pdf}{Quiz 2 - Python And Pandas}
\end{itemize}

\subsection{Concept Summary}

The concept that is being summarized this week is \textbf{Pandas, Exploring And Cleaning Tabular Data}.

\begin{notes}{Pandas, Exploring And Cleaning Tabular Data}
    \subsection*{Overview}

    Pandas is a powerful Python library used for data manipulation and analysis, particularly suited for working with tabular data. It provides robust tools for cleaning, transforming, and analyzing 
    data efficiently, making it indispensable in the data science toolkit. The ability to handle large datasets with ease and integrate smoothly with other data analysis libraries makes pandas a 
    cornerstone for data exploration and preprocessing tasks. \vspace*{1em}
    
    \subsubsection*{Detailed Phases of Exploring and Cleaning Data with Pandas}
    
    Data Exploration is an essential first step in understanding the structure and quality of the dataset. Pandas provides numerous functions to facilitate this process:
    \begin{itemize}
        \item \textbf{Reading Data}: Pandas can read data from various formats like CSV, Excel, JSON, SQL databases, and more, allowing for easy ingestion of data into Python environments.
        \item \textbf{Viewing Data}: Functions like \texttt{head()}, \texttt{tail()}, and \texttt{describe()} give a quick overview of the data, displaying the first and last rows and summarizing the 
        statistical attributes of numerical columns.
        \item \textbf{Data Profiling}: This involves more comprehensive analysis using methods such as \texttt{info()} and \texttt{describe()} to assess data types, non-null values, and other attributes 
        that indicate data quality and completeness.
    \end{itemize}
    
    Data Cleaning with pandas is critical for preparing data for analysis:
    \begin{itemize}
        \item \textbf{Handling Missing Values}: Pandas provides several methods to detect, remove, or fill missing values, such as \texttt{isnull()}, \texttt{dropna()}, and \texttt{fillna()}.
        \item \textbf{Data Transformation}: This includes operations like merging, joining, or reshaping data. Functions like \texttt{merge()}, \texttt{concat()}, and \texttt{pivot()} are extensively 
        used to modify data structures and prepare datasets for analysis.
        \item \textbf{Filtering and Sorting Data}: Pandas enables filtering and sorting of data based on conditions, which is crucial for narrowing down data to relevant subsets for specific analyses, 
        using \texttt{loc()}, \texttt{iloc()}, and \texttt{sort\_values()}.
    \end{itemize}
    
    \subsubsection*{Advanced Features}
    
    Beyond basic cleaning and exploration, pandas also offers advanced features that enhance its functionality:
    \begin{itemize}
        \item \textbf{Time Series Analysis}: Pandas has built-in support for date and time data types and time series functionalities, which are essential for analyzing chronological data.
        \item \textbf{High-Performance Operations}: With tools like \texttt{groupby()} for grouping large data sets and \texttt{apply()} for applying functions to data frames, pandas supports complex operations 
        that are highly performant on large datasets.
    \end{itemize}
    
    \subsubsection*{Summary}
    
    The use of pandas for exploring and cleaning tabular data is pivotal in data science. It streamlines the process of data preparation, allowing for more efficient and accurate analysis. By providing 
    a rich set of tools for dealing with diverse data types and structures, pandas not only simplifies data manipulation but also enhances the overall data analysis workflow, promoting a culture of 
    data-driven decision-making.    
\end{notes}