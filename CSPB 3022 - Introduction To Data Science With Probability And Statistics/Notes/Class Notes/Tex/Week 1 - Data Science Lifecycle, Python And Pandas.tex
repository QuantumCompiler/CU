\clearpage

\newcommand{\ChapTitle}{Data Science Lifecycle, Python And Pandas}
\newcommand{\SectionTitle}{Data Science Lifecycle, Python And Pandas}
\chapter{\ChapTitle}

\section{\SectionTitle}
\horizontalline{0}{0}

\subsection{Optional Reading}

The optional reading for this week is from \LearnDSBook \hspace*{1pt} and \PyDSBook.

\subsection{Piazza}

Must post / respond to two Piazza posts.

\subsection{Lectures}

The lecture videos for this week are:

\begin{itemize}
    \item \lecture{https://applied.cs.colorado.edu/pluginfile.php/76307/mod_resource/content/3/Lecture1CourseOverview.mp4}{Course Overview}{54}
    \item \lecture{https://applied.cs.colorado.edu/pluginfile.php/76309/mod_resource/content/1/CSCI3022_001_ECCR200_8_30_2023.mp4}{Pandas Part 1}{54}
    \item \lecture{https://applied.cs.colorado.edu/pluginfile.php/76310/mod_resource/content/1/CSCI3022_001_ECCR200_9_1_2023.mp4}{Pandas Part 2}{54}
    \item \lecture{https://applied.cs.colorado.edu/pluginfile.php/76308/mod_resource/content/2/lab1_zoom.mp4}{Python Walkthrough}{47}
    \item \lecture{https://applied.cs.colorado.edu/mod/hvp/view.php?id=56692}{Learning to Use Jupyter Notebooks}{9}
    \item \lecture{https://applied.cs.colorado.edu/mod/hvp/view.php?id=56694}{Using Markdown In Jupyter Notebooks}{11}
    \item \lecture{https://applied.cs.colorado.edu/mod/hvp/view.php?id=56695}{Writing Mathematics In Jupyter Notebooks}{16}
    \item \lecture{https://applied.cs.colorado.edu/mod/hvp/view.php?id=56696}{Beginning Python}{44}
    \item \lecture{https://applied.cs.colorado.edu/mod/hvp/view.php?id=56697}{Python - Basic NumPy}{18}
    \item \lecture{https://applied.cs.colorado.edu/mod/hvp/view.php?id=56700}{Quick Calculus Refresher}{21}
\end{itemize}

\noindent The lecture notes for this week are:

\begin{itemize}
    \item \pdflink{\LecNoteDir Course Overview Data Science Lifecycle Lecture Notes.pdf}{Course Overview Data Science Lifecycle Lecture Notes}
    \item \pdflink{\LecNoteDir Pandas Part I Lecture Notes.pdf}{Pandas Part 1 Lecture Notes}
    \item \pdflink{\LecNoteDir Pandas Part II Lecture Notes.pdf}{Pandas Part 2 Lecture Notes}
    \item \pdflink{\LecNoteDir LaTeX Lecture Notes.pdf}{\LaTeX \hspace*{1pt} Lecture Notes}
\end{itemize}

\subsection{Assignments}

The assignment for this week is:

\begin{itemize}
    \item \href{https://github.com/QuantumCompiler/CU/tree/main/CSPB%203022%20-%20Introduction%20To%20Data%20Science%20With%20Probability%20And%20Statistics/Assignments/Assignment%201%20-%20Data%20Science%20Lifecycle%2C%20Python%2C%20And%20Pandas}{Assignment 1 - Data Science Lifecycle, Python And Pandas}
\end{itemize}

\subsection{Quiz}

The quizzes for this week are:

\begin{itemize}
    \item \pdflink{\QuizDir Quiz 1 - Math Concept.pdf}{Quiz 1 - Math Concept}
\end{itemize}

\newpage

\subsection{Concept Summary}

The concept that is being explored this week is \textbf{The Data Science Lifecycle}.

\begin{notes}{The Data Science Lifecycle}
    \subsection*{Overview}

    The Data Science Lifecycle provides a systematic approach to managing data science projects. This structured methodology ensures that projects progress efficiently from conception through to 
    deployment, facilitating the transformation of raw data into actionable insights. It is designed to align data science projects with business objectives, ensuring relevance and value in the outcomes. \vspace*{1em}
    
    \subsubsection*{Detailed Phases of the Data Science Lifecycle}
    
    Business Understanding is the cornerstone of any data science project. It involves defining the objectives and scope of the project in alignment with business goals. This phase requires intensive 
    discussions with stakeholders to pinpoint the critical questions that the project should address and to establish clear success metrics. \vspace*{1em}
    
    \subsubsection*{Data Acquisition and Understanding}
    
    \begin{itemize}
        \item \textbf{Data Collection}: Gathering data from a variety of sources including databases, files, and external APIs, ensuring that the data aligns with the needs identified in the business understanding phase.
        \item \textbf{Data Exploration}: Applying statistical methods to gain insights into the data, understand its structure, and uncover any underlying patterns or anomalies.
        \item \textbf{Data Quality Assessment}: Identifying and addressing issues related to data quality such as missing values or inconsistent formats, which are critical for the next stages of the lifecycle.
    \end{itemize}
    
    Data Preparation involves cleaning and transforming the data. This stage is pivotal as it prepares the data for effective analysis by addressing any quality issues identified previously, normalizing 
    data formats, and engineering features that will be used in modeling.
    
    Modeling is at the heart of the data science lifecycle where theoretical models are turned into practical tools:
    \begin{itemize}
        \item \textbf{Model Selection}: Choosing appropriate algorithms that best fit the business problem, data characteristics, and expected outcomes.
        \item \textbf{Model Training}: Fitting models to the data, adjusting parameters to improve their accuracy and effectiveness.
        \item \textbf{Model Validation}: Rigorously testing models to ensure they perform well against predefined metrics and real-world scenarios.
    \end{itemize}

    \subsubsection*{Evaluation}
    
    Evaluation involves assessing whether the models meet the business objectives set in the initial phase. This includes reviewing the model outputs in detail and ensuring that they provide the insights 
    needed to make informed business decisions.

    \subsubsection*{Deployment}
    
    Deployment marks the integration of the model into the business environment where it can start providing value by generating actionable insights. This stage also includes setting up mechanisms for 
    monitoring the model's performance and establishing processes for ongoing maintenance.
    
    The lifecycle is characterized by a Feedback Loop, which involves continual monitoring and refining of the models based on feedback and new data. This ensures that the models remain relevant and perform optimally over time.
    
    \subsubsection*{Summary}
    
    The Data Science Lifecycle is critical for ensuring that data science projects deliver maximum value by aligning closely with business objectives and adapting to new data and insights. This structured 
    approach aids organizations in navigating the complexities of data-driven decision-making, fostering a culture of continuous improvement and innovation.
\end{notes}