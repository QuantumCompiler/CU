\clearpage

\renewcommand{\ChapTitle}{CLT And Hypothesis Testing}
\renewcommand{\SectionTitle}{CLT And Hypothesis Testing}

\chapter{\ChapTitle}
\section{\SectionTitle}
\horizontalline{0}{0}

\subsection{Piazza}

Must post / respond to two Piazza posts. \assignment{3/19/24}{Piazza9DueDate}

\subsection{Lectures}

The lecture videos for this week are:

\begin{itemize}
    \item \lecture{https://applied.cs.colorado.edu/pluginfile.php/76436/mod_resource/content/2/CSCI3022_002_CASEE240_10_23_2023.mp4}{Hypothesis Testing}{54}
    \item \lecture{https://applied.cs.colorado.edu/pluginfile.php/76437/mod_resource/content/1/lab08_zoom.mp4}{Testing Hypothesis And The Central Limit Theorem Walkthrough}{63}
    \item \lecture{https://applied.cs.colorado.edu/pluginfile.php/76438/mod_resource/content/2/CSCI3022_002_CASEE240_10_25_2023.mp4}{A/B Testing}{54}
    \item \lecture{https://applied.cs.colorado.edu/pluginfile.php/76439/mod_resource/content/1/lab9_zoom.mp4}{A/B Testing Walkthrough}{33}
    \item \lecture{https://applied.cs.colorado.edu/course/view.php?id=261&section=10}{Errors In Hypothesis Testing}{54}
\end{itemize}

\noindent The lecture notes for this week are:

\begin{itemize}
    \item \pdflink{\LecNoteDir Hypothesis Testing Lecture Notes.pdf}{Hypothesis Testing Lecture Notes}
    \item \pdflink{\LecNoteDir AB Testing, Randomized Control Tests And Causality Lecture Notes.pdf}{AB Testing, Randomized Control Tests And Causality Lecture Notes}
    \item \pdflink{\LecNoteDir Errors in Hypothesis Testing, PHacking Lecture Notes.pdf}{Errors in Hypothesis Testing, PHacking Lecture Notes}
\end{itemize}

\subsection{Assignments}

The assignment for this week is:

\begin{itemize}
    \item The assignment for this week is \href{https://github.com/QuantumCompiler/CU/tree/main/CSPB%203022%20-%20Introduction%20To%20Data%20Science%20With%20Probability%20And%20Statistics/CSPB%203022%20-%20Assignments/CSPB%203022%20-%20Assignment%208%20-%20CLT%20And%20Hypothesis%20Testing}{Assignment 8 - CLT And Hypothesis Testing}. \assignment{3/19/24}{Ass8DueDate}
\end{itemize}

\subsection{Concept Summary}

The concept that is being covered this week is \textbf{CLT And Hypothesis Testing}.

\begin{notes}{CLT And Hypothesis Testing}
    \subsection*{Overview}

    The Central Limit Theorem (CLT) and Hypothesis Testing are foundational concepts in statistics, crucial for making inferences about populations from sample data. The CLT provides the theoretical 
    backbone for understanding how sample means behave, especially with large sample sizes, while hypothesis testing is a methodological framework used to make decisions and test claims about 
    population parameters based on sample statistics. Understanding both concepts allows statisticians to perform reliable and robust statistical analyses. \vspace*{1em}
    
    \subsubsection*{Central Limit Theorem}
    
    The Central Limit Theorem is a fundamental principle in probability theory that explains why the normal distribution arises so commonly and why it is generally applicable in statistical methodologies:
    \begin{itemize}
        \item \textbf{Statement of CLT}: It states that the distribution of the sample means will approximate a normal distribution, regardless of the shape of the population distribution, as long as 
        the sample size is sufficiently large (usually n > 30).
        \item \textbf{Implications}: This theorem allows for the simplification of analysis in many statistical applications because it justifies the use of the normal probability model in the sampling 
        distribution of the mean. This is particularly useful for inferential statistics where exact distributions are unknown.
    \end{itemize}
    
    \subsubsection*{Hypothesis Testing}
    
    Hypothesis Testing is a systematic method used to evaluate assumptions (hypotheses) about a parameter in a given population:
    \begin{itemize}
        \item \textbf{Steps in Hypothesis Testing}: 
            \begin{enumerate}
                \item Formulation of Null (H0) and Alternative (H1) Hypotheses: The null hypothesis typically represents a theory that there is no effect or no difference, and the alternative is what 
                the researcher aims to prove.
                \item Selection of a Significance Level $(\alpha)$: Commonly set at 0.05, this is the threshold for deciding whether to reject the null hypothesis.
                \item Calculation of Test Statistic: Based on the sample data, calculate a statistic that is appropriate for the test (e.g., t-statistic for a t-test).
                \item Determination of the P-value or Critical Value: Compare the calculated statistic to a critical value derived from an appropriate distribution.
                \item Decision: Reject or fail to reject the null hypothesis based on the comparison between the P-value and the significance level.
            \end{enumerate}
        \item \textbf{Types of Errors}: Understanding Type I (false positive) and Type II (false negative) errors is critical in hypothesis testing. These errors reflect the probabilities of incorrectly 
        rejecting a true null hypothesis or failing to reject a false null hypothesis, respectively.
    \end{itemize}
    
    \subsubsection*{Combining CLT and Hypothesis Testing}
    
    Integrating the Central Limit Theorem with Hypothesis Testing provides a powerful approach to statistical inference:
    \begin{itemize}
        \item By assuming the normal distribution of the sample mean, as justified by the CLT, statisticians can apply hypothesis testing even when the population standard deviation is unknown, using 
        the sample standard deviation as an estimate.
        \item This integration simplifies the calculation of test statistics under the normality assumption, enhancing the accuracy and reliability of hypothesis tests.
    \end{itemize}
    
    \subsubsection*{Summary}
    
    The Central Limit Theorem and Hypothesis Testing are indispensable tools in the field of statistics, enabling practitioners to make precise inferences about population parameters from sample data. 
    These concepts are not only theoretical but also immensely practical, forming the basis for data analysis across scientific research, business analytics, and many other domains.    
\end{notes}