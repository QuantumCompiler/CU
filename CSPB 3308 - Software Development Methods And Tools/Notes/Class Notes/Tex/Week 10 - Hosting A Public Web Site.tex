\clearpage

\renewcommand{\ChapTitle}{Hosting A Public Web Site}
\renewcommand{\SectionTitle}{Hosting A Public Web Site}

\chapter{\ChapTitle}

\section{\SectionTitle}
\horizontalline{0}{0}

\subsection{Assigned Reading}

The reading for this week is from, \AgileBook, \EngSoftBook, \ProGitBook, and \LinuxBook.

\begin{itemize}
    \item \textbf{N/A}
\end{itemize}

\subsection{Lectures}

The lectures for this week are:

\begin{itemize}
    \item \lecture{https://www.youtube.com/watch?v=Pp9l_fLBAJ4}{Introduction To Cloud Computing}{25}
    \item \lecture{https://www.youtube.com/watch?v=gQUeWiDBe4c}{How To Deploy A Flask App And Postgres Database To Render}{5}
    \item \lecture{https://www.youtube.com/watch?v=IBfj_0Zf2Mo}{How To Deploy A Flask App And Postgres Database To Render (Full Video)}{17}
\end{itemize}

\subsection{Assignments}

The assignment(s) for this week are:

\begin{itemize}
    \item \href{https://github.com/cu-cspb-3308-software-dev-summer-2024/lab-10-QuantumCompiler}{Assignment 10 - Website Hosting Tutorial}
\end{itemize}

\subsection{Chapter Summary}

The topic that is being covered this week is \textbf{Cloud Computing}.

\begin{notes}{Cloud Computing}
    \subsection*{Overview}

    Cloud Computing refers to the delivery of computing services over the internet, enabling users to access and manage data and applications remotely. This technology provides on-demand availability 
    of computing resources like servers, storage, databases, and networking, without the need for direct management by the user. Cloud computing is essential for businesses of all sizes, offering 
    scalability, cost-efficiency, and flexibility.
    
    \subsubsection*{Types of Cloud Computing Services}
    
    Cloud computing services are generally categorized into three main types: Infrastructure as a Service (IaaS), Platform as a Service (PaaS), and Software as a Service (SaaS).
    
    \begin{highlight}[Types of Cloud Computing Services]
    
        Different types of cloud services cater to varying levels of control and management, suitable for diverse business needs.
        
        \begin{itemize}
            \item \textbf{Infrastructure as a Service (IaaS)}: Provides virtualized computing resources over the internet. IaaS includes essential services such as virtual machines, storage, and networking. 
            It offers high flexibility and control over IT resources, allowing businesses to scale as needed.
            \begin{code}[Example]
            Use case: Hosting a website or a large-scale application that requires flexible resource allocation.
            \end{code}
            \item \textbf{Platform as a Service (PaaS)}: Offers a platform allowing customers to develop, run, and manage applications without the complexity of building and maintaining the underlying 
            infrastructure. PaaS includes services like development tools, database management, and business analytics.
            \begin{code}[Example]
            Use case: Developing a web application with pre-configured environments for testing and deployment.
            \end{code}
            \item \textbf{Software as a Service (SaaS)}: Delivers software applications over the internet, on a subscription basis. SaaS providers manage the infrastructure and platforms that run the 
            applications, which are accessible via a web browser.
            \begin{code}[Example]
            Use case: Using CRM software or email services.
            \end{code}
        \end{itemize}
    
    \end{highlight}
    
    \subsubsection*{Benefits of Cloud Computing}
    
    Cloud computing offers numerous benefits that have revolutionized how businesses and individuals interact with technology.
    
    \begin{highlight}[Benefits of Cloud Computing]
    
        The advantages of cloud computing include:
        
        \begin{itemize}
            \item \textbf{Cost Efficiency}: Reduces the need for physical infrastructure and maintenance, converting capital expenses to operational expenses. Users pay only for the resources they use, 
            which can lead to significant cost savings.
            \item \textbf{Scalability and Flexibility}: Allows businesses to scale resources up or down based on demand, ensuring optimal resource utilization and avoiding over-provisioning.
            \item \textbf{Global Reach}: Deploy applications globally within minutes, allowing companies to reach customers around the world and improve latency and user experience.
            \item \textbf{Security and Compliance}: Leading cloud providers offer robust security measures and compliance certifications, helping businesses meet regulatory requirements and protect data integrity.
            \item \textbf{Innovation and Agility}: Cloud services enable rapid development and deployment of new applications and services, fostering innovation and reducing time-to-market.
        \end{itemize}
    
    \end{highlight}
    
    \subsubsection*{Key Cloud Computing Models}
    
    Cloud computing models vary based on deployment and service delivery, offering different levels of management and control.
    
    \begin{highlight}[Key Cloud Computing Models]
    
        The primary models include:
        
        \begin{itemize}
            \item \textbf{Public Cloud}: Resources are owned and operated by a third-party cloud service provider and delivered over the internet. Public clouds are ideal for workloads with varying or unpredictable demand.
            \item \textbf{Private Cloud}: Dedicated infrastructure for a single organization, offering more control and security. It can be hosted on-premises or by a third-party provider.
            \item \textbf{Hybrid Cloud}: Combines public and private clouds, allowing data and applications to move between them. This model offers greater flexibility and optimization of existing 
            infrastructure, security, and compliance requirements.
        \end{itemize}
    
    \end{highlight}
    
    \subsubsection*{Applications of Cloud Computing}
    
    Cloud computing supports a wide range of applications across various industries, enhancing efficiency and innovation.
    
    \begin{highlight}[Applications of Cloud Computing]
    
        Key applications include:
        
        \begin{itemize}
            \item \textbf{Big Data Analytics}: Processing large datasets to derive business insights and improve decision-making.
            \item \textbf{Web Hosting}: Hosting websites and applications with scalable and reliable infrastructure.
            \item \textbf{Disaster Recovery and Backup}: Protecting data and applications from disruptions and ensuring business continuity.
            \item \textbf{AI and Machine Learning}: Running complex AI models and algorithms with scalable compute resources.
            \item \textbf{IoT (Internet of Things)}: Connecting and managing devices and sensors, processing data in real-time.
        \end{itemize}
    
    \end{highlight}
    
    \begin{highlight}[Summary of Key Concepts]
    
        Key concepts from Cloud Computing:
        
        \begin{itemize}
            \item \textbf{Types of Services}: IaaS, PaaS, and SaaS each offer different levels of control and management.
            \item \textbf{Benefits}: Includes cost efficiency, scalability, global reach, security, and innovation.
            \item \textbf{Cloud Models}: Public, private, and hybrid clouds cater to different organizational needs.
            \item \textbf{Applications}: Encompasses big data analytics, web hosting, disaster recovery, AI, and IoT.
        \end{itemize}
        
    \end{highlight}
\end{notes}