\clearpage

\renewcommand{\ChapTitle}{Flask Web Framework And More Python}
\renewcommand{\SectionTitle}{Flask Web Framework And More Python}

\chapter{\ChapTitle}

\section{\SectionTitle}
\horizontalline{0}{0}

\subsection{Assigned Reading}

The reading for this week is from, \AgileBook, \EngSoftBook, \ProGitBook, and \LinuxBook.

\begin{itemize}
    \item \textbf{Engineering Software As A Service - Chapter 2 - The Architecture Of SaaS Applications}
\end{itemize}

\subsection{Lectures}

The lectures for this week are:

\begin{itemize}
    \item \lecture{https://www.youtube.com/watch?v=KxT5j0RPtnc}{Python Decorators}{8}
    \item \lecture{https://www.youtube.com/watch?v=LryIlBhZxtU}{Python Virtual Environments}{10}
    \item \lecture{https://www.youtube.com/watch?v=lZhxrjDp1Ek}{Flask Quickstart}{12}
    \item \lecture{https://www.youtube.com/watch?v=MNQ6Rah-EZI}{Creating Tables In HTML}{10}
\end{itemize}

\noindent The lecture notes for this week are:

\begin{itemize}
    \item \pdflink{\LecNoteDir Routes Lecture Notes.pdf}{Routes Lecture Notes}
\end{itemize}

\subsection{Assignments}

The assignment(s) for this week are:

\begin{itemize}
    \item \href{https://github.com/cu-cspb-3308-software-dev-summer-2024/lab-6-QuantumCompiler}{Assignment 6 - Flask Tutorial}
\end{itemize}

\subsection{Project}

The assignment(s) for the project this week is:

\begin{itemize}
    \item \href{https://applied.cs.colorado.edu/mod/assign/view.php?id=61352}{Project Milestone 3: Weekly Status}
\end{itemize}

\subsection{Chapter Summary}

The reading for this week is from \textbf{Engineering Software As A Service}. The chapter that is being covered this week is \textbf{Chapter 2: The Architecture Of SaaS Applications}.

\begin{notes}{Chapter 2: The Architecture Of SaaS Applications}
    \subsection*{Overview}

    This chapter discusses the architecture of SaaS (Software as a Service) applications, outlining various levels of detail from high-level client-server architecture to specific implementation patterns. 
    Understanding these architectures helps in designing scalable and efficient SaaS applications.
    
    \subsubsection*{100,000 Feet: Client-Server Architecture}
    
    The client-server architecture is a fundamental pattern in SaaS applications, where clients request services and servers respond to these requests.
    
    \begin{highlight}[Client-Server Architecture]
    
        Clients and servers communicate over the network, with clients initiating requests and servers responding.
        
        \begin{itemize}
            \item Clients, such as web browsers, request services.
            \item Servers, such as WEBrick, respond to client requests.
            \item Example: Firefox requests a page from WEBrick, which serves content from the RottenPotatoes app.
        \end{itemize}
    
    \end{highlight}
    
    \subsubsection*{50,000 Feet: Communication—HTTP and URIs}
    
    HTTP and URIs are the foundation of web communication, enabling resource identification and data exchange.
    
    \begin{highlight}[Communication—HTTP and URIs]
    
        HTTP (HyperText Transfer Protocol) and URIs (Uniform Resource Identifiers) facilitate client-server communication.
        
        \begin{itemize}
            \item HTTP is a stateless protocol used for web communication.
            \item URIs identify resources on the web.
            \item Example: A browser uses an HTTP GET request to retrieve a web page via its URI.
        \end{itemize}
    
    \end{highlight}
    
    \subsubsection*{10,000 Feet: Representation—HTML and CSS}
    
    HTML and CSS define the structure and style of web content, respectively, providing a clear separation between content and presentation.
    
    \begin{highlight}[Representation—HTML and CSS]
    
        HTML (HyperText Markup Language) and CSS (Cascading Style Sheets) structure and style web content.
        
        \begin{itemize}
            \item HTML elements structure web pages.
            \item CSS styles HTML elements.
            \item Example: An HTML document with embedded CSS defines the layout and appearance of a web page.
        \end{itemize}
    
    \end{highlight}
    
    \subsubsection*{5,000 Feet: 3-Tier Architecture & Horizontal Scaling}
    
    The 3-tier architecture separates concerns into presentation, logic, and persistence tiers, enabling scalability.
    
    \begin{highlight}[3-Tier Architecture & Horizontal Scaling]
    
        The 3-tier architecture divides applications into presentation, logic, and persistence tiers.
        
        \begin{itemize}
            \item Presentation tier handles user interfaces.
            \item Logic tier processes application functionality.
            \item Persistence tier manages data storage.
            \item Horizontal scaling allows adding more servers to each tier as needed.
        \end{itemize}
    
    \end{highlight}
    
    \subsubsection*{1,000 Feet: Model-View-Controller Architecture}
    
    The MVC pattern separates applications into models, views, and controllers, each with distinct responsibilities.
    
    \begin{highlight}[Model-View-Controller Architecture]
    
        MVC (Model-View-Controller) pattern divides application logic, user interface, and user input handling.
        
        \begin{itemize}
            \item Models manage data and business logic.
            \item Views display data and user interfaces.
            \item Controllers handle user input and interactions.
        \end{itemize}
    
    \end{highlight}
    
    \subsubsection*{500 Feet: Active Record for Models}
    
    Active Record is a pattern used to map objects to database records, facilitating data persistence.
    
    \begin{highlight}[Active Record for Models]
    
        Active Record maps objects to database tables, enabling CRUD operations.
        
        \begin{itemize}
            \item Models correspond to database tables.
            \item CRUD operations: Create, Read, Update, Delete.
            \item Example: A Movie model maps to a movies table in the database.
        \end{itemize}
    
    \end{highlight}
    
    \subsubsection*{500 Feet: Routes, Controllers, and REST}
    
    Routes map HTTP requests to controller actions, and RESTful principles ensure stateless and self-descriptive interactions.
    
    \begin{highlight}[Routes, Controllers, and REST]
    
        Routes map URIs and HTTP methods to controller actions; REST (Representational State Transfer) organizes resource interactions.
        
        \begin{itemize}
            \item Routes define URL patterns and corresponding actions.
            \item RESTful design ensures each request contains all necessary information.
            \item Example: GET /movies maps to the index action in the Movies controller.
        \end{itemize}
    
    \end{highlight}
    
    \subsubsection*{500 Feet: Template Views}
    
    Template Views render dynamic content by combining static templates with dynamic data.
    
    \begin{highlight}[Template Views]
    
        Template Views generate dynamic content by integrating data into HTML templates.
        
        \begin{itemize}
            \item Templates contain static HTML with placeholders for dynamic data.
            \item Example: Haml template renders a list of movies dynamically.
        \end{itemize}
    
    \end{highlight}
    
    \begin{highlight}[Summary of Key Concepts]
    
        Key concepts from Chapter 2 on the architecture of SaaS applications:
        
        \begin{itemize}
            \item \textbf{Client-Server Architecture}: Separates clients and servers for specialized roles.
            \item \textbf{Communication—HTTP and URIs}: Facilitates web communication and resource identification.
            \item \textbf{Representation—HTML and CSS}: Structures and styles web content.
            \item \textbf{3-Tier Architecture & Horizontal Scaling}: Divides applications into scalable tiers.
            \item \textbf{Model-View-Controller Architecture}: Separates data, presentation, and input handling.
            \item \textbf{Active Record for Models}: Maps objects to database records for data persistence.
            \item \textbf{Routes, Controllers, and REST}: Maps requests to actions and ensures stateless interactions.
            \item \textbf{Template Views}: Generates dynamic content from static templates.
        \end{itemize}
        
    \end{highlight}
\end{notes}