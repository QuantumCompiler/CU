\clearpage

\renewcommand{\ChapTitle}{Testing}
\renewcommand{\SectionTitle}{Testing}

\chapter{\ChapTitle}

\section{\SectionTitle}
\horizontalline{0}{0}

\subsection{Assigned Reading}

The reading for this week is from, \AgileBook, \EngSoftBook, \ProGitBook, and \LinuxBook.

\begin{itemize}
    \item \textbf{Engineering Software As A Service - Chapter 8 - Software Testing - Test-Driven Development}
\end{itemize}

\subsection{Lectures}

The lectures for this week are:

\begin{itemize}
    \item \lecture{https://www.youtube.com/watch?v=tIrcxwLqzjQ}{Introduction To Unit Testing In Python}{51}
    \item \lecture{https://www.youtube.com/watch?v=h_TkFMiuSOE}{Introduction To Lab 5}{28}
\end{itemize}

\subsection{Assignments}

The assignment(s) for this week are:

\begin{itemize}
    \item \href{https://github.com/cu-cspb-3308-software-dev-summer-2024/lab-5-QuantumCompiler}{Lab 5 - Automated Unit Test}
\end{itemize}

\subsection{Chapter Summary}

The reading for this week is from \textbf{Engineering Software As A Service}. The chapter that is being covered this week is \textbf{Chapter 8: Software Testing - Test-Driven Development}.

\begin{notes}{Chapter 8: Software Testing - Test-Driven Development}
    \subsection*{Overview}

    This chapter explores Test-Driven Development (TDD) in the context of software testing. It introduces the principles of TDD, the process of writing tests before code, and the benefits of this 
    methodology. By the end of this chapter, you will understand how to apply TDD to create robust, maintainable, and well-documented software.
    
    \subsubsection*{Background: A RESTful API and a Ruby Gem}
    
    A RESTful API, such as TMDb, allows for self-contained HTTP requests to interact with external services. The TMDb gem simplifies the use of this API by constructing the necessary URIs and handling responses.
    
    \begin{highlight}[Background: A RESTful API and a Ruby Gem]
    
        Using the TMDb gem to interact with the TMDb API for movie information.
        
        \begin{itemize}
            \item Construct RESTful URIs with the API key.
            \item Use the TMDb gem to handle API requests and responses.
        \end{itemize}
    
    \end{highlight}
    
    \subsubsection*{FIRST, TDD, and Red-Green-Refactor}
    
    The FIRST principles (Fast, Independent, Repeatable, Self-checking, Timely) guide good test creation. TDD involves writing a failing test first, then writing code to pass the test, and finally refactoring the code.
    
    \begin{highlight}[FIRST, TDD, and Red-Green-Refactor]
    
        TDD ensures code quality and maintainability through iterative testing and refactoring.
        
        \begin{itemize}
            \item \textbf{Fast}: Quick tests avoid disrupting workflow.
            \item \textbf{Independent}: Tests should not depend on each other.
            \item \textbf{Repeatable}: Consistent results regardless of external factors.
            \item \textbf{Self-checking}: Tests automatically verify results.
            \item \textbf{Timely}: Write tests concurrently with code development.
        \end{itemize}
    
    \end{highlight}
    
    \subsubsection*{Seams and Doubles}
    
    Seams allow altering program behavior without changing source code. Doubles, such as mocks and stubs, help isolate tests by mimicking the behavior of real objects.
    
    \begin{highlight}[Seams and Doubles]
    
        Use seams and doubles to isolate the code under test.
        
        \begin{itemize}
            \item Seams change program behavior during tests without altering the source code.
            \item Doubles (mocks and stubs) simulate interactions with other objects.
        \end{itemize}
    
    \end{highlight}
    
    \subsubsection*{Expectations, Mocks, Stubs, Setup}
    
    RSpec allows setting expectations for method calls and return values, using mocks and stubs to isolate test behavior.
    
    \begin{highlight}[Expectations, Mocks, Stubs, Setup]
    
        Set expectations for method interactions and use mocks/stubs to control test behavior.
        
        \begin{itemize}
            \item \textbf{Mocks}: Verify method calls.
            \item \textbf{Stubs}: Control method return values.
            \item Use \texttt{before} blocks for common test setup.
        \end{itemize}
    
    \end{highlight}
    
    \subsubsection*{Fixtures and Factories}
    
    Fixtures provide a fixed state for tests, while factories create objects dynamically, reducing interdependencies between tests.
    
    \begin{highlight}[Fixtures and Factories]
    
        Use fixtures for predefined states and factories for dynamic object creation.
        
        \begin{itemize}
            \item \textbf{Fixtures}: Load predefined objects for testing.
            \item \textbf{Factories}: Create objects with specific attributes as needed.
        \end{itemize}
    
    \end{highlight}
    
    \subsubsection*{Implicit Requirements and Stubbing the Internet}
    
    Testing external services requires handling implicit requirements and stubbing out service calls to ensure tests are fast and reliable.
    
    \begin{highlight}[Implicit Requirements and Stubbing the Internet]
    
        Handle implicit requirements and stub external service calls for reliable testing.
        
        \begin{itemize}
            \item Identify and test implicit requirements.
            \item Use stubs to simulate external service responses.
        \end{itemize}
    
    \end{highlight}
    
    \begin{highlight}[Summary of Key Concepts]
    
        Key concepts from Chapter 8 on Test-Driven Development:
        
        \begin{itemize}
            \item \textbf{RESTful API and Ruby Gem}: Simplify API interactions.
            \item \textbf{FIRST Principles}: Guide good test creation.
            \item \textbf{TDD and Red-Green-Refactor}: Iterative testing and refactoring.
            \item \textbf{Seams and Doubles}: Isolate code for testing.
            \item \textbf{Expectations, Mocks, Stubs, Setup}: Control and verify test behavior.
            \item \textbf{Fixtures and Factories}: Manage test objects and dependencies.
            \item \textbf{Implicit Requirements and Stubbing}: Ensure fast, reliable tests for external services.
        \end{itemize}
        
    \end{highlight}
\end{notes}