\clearpage

\renewcommand{\ChapTitle}{JavaScript}
\renewcommand{\SectionTitle}{JavaScript}

\chapter{\ChapTitle}

\section{\SectionTitle}
\horizontalline{0}{0}

\subsection{Assigned Reading}

The reading for this week is from, \AgileBook, \EngSoftBook, \ProGitBook, and \LinuxBook.

\begin{itemize}
    \item \pdflink{\LecNoteDir JavaScript Tutorial Point.pdf}{JavaScript Tutorial Point}
\end{itemize}

\subsection{Lectures}

The lectures for this week are:

\begin{itemize}
    \item \lecture{https://www.youtube.com/watch?v=jANBreqEvXU}{Introduction To JavaScript}{16}
    \item \lecture{https://www.youtube.com/watch?v=4-A_Y3rLLzA}{JavaScript Syntax And Practice}{14}
\end{itemize}

\subsection{Assignments}

The assignment(s) for this week are:

\begin{itemize}
    \item \href{https://github.com/cu-cspb-3308-software-dev-summer-2024/lab-9-QuantumCompiler}{Assignment 9 - JavaScript}
\end{itemize}

\subsection{Quiz}

The quiz for this week is:

\begin{itemize}
    \item \pdflink{\QuizDir Quiz 6 - JavaScript.pdf}{Quiz 6 - JavaScript}
\end{itemize}

\subsection{Project}

The assignment(s) for the project this week is:

\begin{itemize}
    \item \href{https://applied.cs.colorado.edu/mod/assign/view.php?id=61422}{Project Milestone 5: SQL Design}
\end{itemize}

\subsection{Chapter Summary}

The topic that is being covered this week is \textbf{JavaScript}.

\begin{notes}{JavaScript}
    \subsection*{Overview}

    JavaScript is a versatile and powerful programming language commonly used in web development to create dynamic and interactive user experiences. It allows developers to implement complex features 
    on web pages, such as real-time updates, interactive maps, animations, and much more. JavaScript is a core technology of the web, alongside HTML and CSS.
    
    \subsubsection*{Basic JavaScript Syntax and Functions}
    
    JavaScript syntax is influenced by Java and C, making it familiar to developers with experience in those languages. It supports various programming paradigms, including procedural, object-oriented, 
    and functional programming.
    
    \begin{highlight}[Basic JavaScript Syntax and Functions]
    
        JavaScript syntax forms the foundation for writing scripts to control web page behavior.
        
        \begin{itemize}
            \item \textbf{Variables}: Used to store data values.
    \begin{code}[JavaScript]
    let name = "Alice";
    const pi = 3.14159;
    \end{code}
            \item \textbf{Functions}: Blocks of code designed to perform a particular task.
    \begin{code}[JavaScript]
    function greet() {
        console.log("Hello, world!");
    }
    greet();
    \end{code}
        \end{itemize}
    
    \end{highlight}
    
    \subsubsection*{DOM Manipulation}
    
    The Document Object Model (DOM) is an API for HTML and XML documents. It represents the page so that programs can change the document structure, style, and content.
    
    \begin{highlight}[DOM Manipulation]
    
        JavaScript interacts with the DOM to dynamically update web content.
        
        \begin{itemize}
            \item \textbf{Selecting Elements}: Access elements in the DOM.
    \begin{code}[JavaScript]
    const element = document.getElementById("myElement");
    \end{code}
            \item \textbf{Changing Content}: Modify the content of elements.
    \begin{code}[JavaScript]
    element.textContent = "New content";
    \end{code}
            \item \textbf{Event Listeners}: Respond to user actions.
    \begin{code}[JavaScript]
    element.addEventListener("click", function() {
        alert("Element clicked!");
    });
    \end{code}
        \end{itemize}
    
    \end{highlight}
    
    \subsubsection*{JavaScript Objects and Prototypes}
    
    JavaScript is an object-oriented language. Objects are collections of related data and functionality, and prototypes allow for the inheritance of properties and methods.
    
    \begin{highlight}[JavaScript Objects and Prototypes]
    
        Objects and prototypes are key concepts for organizing and reusing code.
        
        \begin{itemize}
            \item \textbf{Creating Objects}: Define objects using literal notation.
    \begin{code}[JavaScript]
    let person = {
        name: "Alice",
        age: 25,
        greet: function() {
            console.log("Hello, " + this.name);
        }
    };
    person.greet();
    \end{code}
            \item \textbf{Prototypes}: Share properties and methods across instances.
    \begin{code}[JavaScript]
    function Person(name, age) {
        this.name = name;
        this.age = age;
    }
    Person.prototype.greet = function() {
        console.log("Hello, " + this.name);
    };
    let bob = new Person("Bob", 30);
    bob.greet();
    \end{code}
        \end{itemize}
    
    \end{highlight}
    
    \subsubsection*{Asynchronous JavaScript}
    
    Asynchronous programming is essential in JavaScript for handling operations that take time to complete, such as data fetching or file reading, without blocking the execution of other code.
    
    \begin{highlight}[Asynchronous JavaScript]
    
        Asynchronous techniques improve the responsiveness of web applications.
        
        \begin{itemize}
            \item \textbf{Callbacks}: Functions passed as arguments to other functions.
    \begin{code}[JavaScript]
    function fetchData(callback) {
        setTimeout(() => {
            callback("Data received");
        }, 1000);
    }
    fetchData(data => {
        console.log(data);
    });
    \end{code}
            \item \textbf{Promises}: Objects representing the eventual completion or failure of an asynchronous operation.
    \begin{code}[JavaScript]
    let promise = new Promise((resolve, reject) => {
        setTimeout(() => {
            resolve("Data received");
        }, 1000);
    });
    promise.then(data => {
        console.log(data);
    });
    \end{code}
            \item \textbf{Async/Await}: Syntactic sugar over promises for writing asynchronous code.
    \begin{code}[JavaScript]
    async function fetchData() {
        let data = await new Promise((resolve) => {
            setTimeout(() => {
                resolve("Data received");
            }, 1000);
        });
        console.log(data);
    }
    fetchData();
    \end{code}
        \end{itemize}
    
    \end{highlight}
    
    \subsubsection*{JavaScript Modules}
    
    Modules allow you to break up your code into reusable pieces. They are essential for maintaining clean and manageable codebases.
    
    \begin{highlight}[JavaScript Modules]
    
        Use modules to organize and encapsulate code.
        
        \begin{itemize}
            \item \textbf{Exporting and Importing}: Share functionality between files.
    \begin{code}[JavaScript]
    // In math.js
    export function add(a, b) {
        return a + b;
    }
    // In main.js
    import { add } from './math.js';
    console.log(add(2, 3));
    \end{code}
        \end{itemize}
    
    \end{highlight}
    
    \begin{highlight}[Summary of Key Concepts]
    
        Key concepts from JavaScript overview:
        
        \begin{itemize}
            \item \textbf{Syntax and Functions}: Basic building blocks of JavaScript.
            \item \textbf{DOM Manipulation}: Interacting with and modifying the document structure.
            \item \textbf{Objects and Prototypes}: Organizing code and enabling inheritance.
            \item \textbf{Asynchronous Programming}: Handling time-consuming tasks without blocking the main thread.
            \item \textbf{Modules}: Structuring code into reusable components.
        \end{itemize}
        
    \end{highlight}
\end{notes}