\clearpage

\renewcommand{\ChapTitle}{Web Services}
\renewcommand{\SectionTitle}{Web Services}

\chapter{\ChapTitle}

\section{\SectionTitle}
\horizontalline{0}{0}

\subsection{Assigned Reading}

The reading for this week is from, \AgileBook, \EngSoftBook, \ProGitBook, and \LinuxBook.

\begin{itemize}
    \item \pdflink{\LecNoteDir Cloud Computing Introduction.pdf}{Cloud Computing Introduction}
\end{itemize}

\subsection{Lectures}

The lectures for this week are:

\begin{itemize}
    \item \lecture{https://www.youtube.com/watch?v=_GmW3MLf7O0}{Web API And Protocols}{24}
    \item \lecture{https://www.youtube.com/watch?v=U7Asq7U24mU}{JSON And XML}{18}
\end{itemize}

\subsection{Assignments}

The assignment(s) for this week are:

\begin{itemize}
    \item \href{https://github.com/QuantumCompiler/CU/tree/main/CSPB%203308%20-%20Software%20Development%20Methods%20And%20Tools/Assignments/Assignment%2011%20-%20REST%20Weather%20App}{Assignment 11 - REST Weather Map}
\end{itemize}

\newpage

\subsection{Chapter Summary}

The first topic that is being covered this week is \textbf{Web API And Protocols}.

\begin{notes}{Web API And Protocols}
    \subsection*{Overview}

    Web APIs (Application Programming Interfaces) are a set of defined protocols and tools that allow different software applications to communicate over the internet. They provide a way for developers 
    to interact with web services and integrate various functionalities into their applications. Web APIs are integral to modern web development, enabling the connection and interaction of disparate 
    systems and services.
    
    \subsubsection*{Types of APIs and Protocols}
    
    APIs can be categorized based on their architecture and the protocols they use. The most common types include REST, SOAP, GraphQL, and WebSocket.
    
    \begin{highlight}[Types of APIs and Protocols]
    
        Different API protocols offer various advantages and are suitable for different use cases.
        
        \begin{itemize}
            \item \textbf{REST (Representational State Transfer)}: REST APIs use standard HTTP methods (GET, POST, PUT, DELETE) and are widely used for their simplicity and scalability. They allow 
            clients to interact with resources via URIs and standard HTTP status codes.
    \begin{code}[Example]
    GET /users/123
    \end{code}
            \item \textbf{SOAP (Simple Object Access Protocol)}: SOAP is a protocol that uses XML to encode its messages and is often used in enterprise environments. It supports a wide range of 
            transport protocols, including HTTP and SMTP.
    \begin{code}[SOAP Example]
    <SOAP-ENV:Envelope xmlns:SOAP-ENV="http://schemas.xmlsoap.org/soap/envelope/">
        <SOAP-ENV:Header/>
        <SOAP-ENV:Body>
            <GetUser>
                <UserID>123</UserID>
            </GetUser>
        </SOAP-ENV:Body>
    </SOAP-ENV:Envelope>
    \end{code}
            \item \textbf{GraphQL}: A query language for APIs, GraphQL allows clients to request only the data they need, reducing the number of requests required. It uses a single endpoint for all operations.
    \begin{code}[Example]
    {
        user(id: "123") {
        name
        email
        }
    }
    \end{code}
    \item \textbf{WebSocket}: This protocol provides full-duplex communication channels over a single TCP connection, enabling real-time interaction between a client and a server.
    \begin{code}[JavaScript]
    const socket = new WebSocket("wss://example.com/socket");
    socket.onmessage = function(event) {
        console.log("Received data: " + event.data);
    };
    \end{code}
        \end{itemize}
    
    \end{highlight}
    
    \subsubsection*{Benefits of Web APIs}
    
    Web APIs offer several key benefits, including ease of integration, enhanced functionality, and support for a wide range of devices and platforms.
    
    \begin{highlight}[Benefits of Web APIs]
    
        Web APIs provide numerous advantages that enhance application development and user experience.
        
        \begin{itemize}
            \item \textbf{Integration}: APIs enable the integration of new applications with existing systems, facilitating faster development and deployment of new features.
            \item \textbf{Innovation}: By leveraging APIs, businesses can quickly adapt to new technologies and trends, fostering innovation without needing to overhaul entire systems.
            \item \textbf{Scalability}: APIs can support a wide range of devices and platforms, making it easier to scale applications across different environments.
            \item \textbf{Ease of Maintenance}: Using APIs allows for modular code, where individual components can be updated or replaced without affecting the entire system.
        \end{itemize}
    
    \end{highlight}
    
    \subsubsection*{Common Use Cases of Web APIs}
    
    APIs are used in a variety of applications, from integrating third-party services to enhancing user experiences with interactive features.
    
    \begin{highlight}[Common Use Cases of Web APIs]
    
        Web APIs support a broad range of applications across different domains.
        
        \begin{itemize}
            \item \textbf{Integrating with Third-Party Services}: Examples include payment processing, social media integration, and map services.
            \item \textbf{Enhancing User Interaction}: APIs enable features like real-time notifications, live data feeds, and user authentication.
            \item \textbf{Data Exchange and Automation}: APIs facilitate the automated transfer of data between systems, reducing manual work and improving efficiency.
        \end{itemize}
    
    \end{highlight}
    
    \subsubsection*{Security and API Management}
    
    Security is a critical aspect of API management, ensuring that data is protected and access is controlled.
    
    \begin{highlight}[Security and API Management]
    
        Proper API security and management are essential for protecting data and ensuring reliable service.
        
        \begin{itemize}
            \item \textbf{Authentication and Authorization}: Use of tokens and API keys to verify the identity of users and restrict access to certain resources.
            \item \textbf{Monitoring and Logging}: Tracking API usage and performance to detect and respond to potential issues.
            \item \textbf{Rate Limiting}: Implementing limits on the number of API requests to prevent abuse and ensure fair usage.
        \end{itemize}
    
    \end{highlight}
    
    \begin{highlight}[Summary of Key Concepts]
    
        Key concepts from Web APIs and Protocols:
        
        \begin{itemize}
            \item \textbf{Types of APIs and Protocols}: REST, SOAP, GraphQL, and WebSocket, each with unique characteristics and use cases.
            \item \textbf{Benefits of APIs}: Including integration, innovation, scalability, and ease of maintenance.
            \item \textbf{Common Use Cases}: Integration with third-party services, enhancing user interaction, and data automation.
            \item \textbf{Security}: Critical measures like authentication, monitoring, and rate limiting.
        \end{itemize}
        
    \end{highlight}
\end{notes}

The last topic that is being covered this week is \textbf{JSON And XML}.

\begin{notes}{JSON And XML}
    \subsection*{Overview}

    JSON (JavaScript Object Notation) and XML (eXtensible Markup Language) are two popular data interchange formats used in web development and other applications. Both are text-based and designed to 
    facilitate data exchange between systems. However, they have different structures and use cases, making them suitable for various scenarios depending on the requirements.
    
    \subsubsection*{JSON}
    
    JSON is a lightweight data-interchange format that is easy for humans to read and write, and easy for machines to parse and generate. It is based on a subset of the JavaScript programming language 
    and is language-independent.
    
    \begin{highlight}[JSON]
    
        JSON structures data in a collection of name/value pairs and arrays, making it an ideal format for transmitting data between a server and a web application.
        
        \begin{itemize}
            \item \textbf{Data Structures}: JSON supports simple data structures such as objects (name/value pairs) and arrays.
    \begin{code}[JSON]
    {
        "name": "John Doe",
        "age": 30,
        "isStudent": false,
        "courses": ["Math", "Science", "History"]
    }
    \end{code}
            \item \textbf{Human Readability}: JSON's structure is easy to understand and write, which simplifies debugging and development.
            \item \textbf{Compatibility}: JSON is compatible with most programming languages, making it versatile for various applications.
        \end{itemize}
    
    \end{highlight}
    
    \subsubsection*{XML}
    
    XML is a markup language that defines rules for encoding documents in a format that is both human-readable and machine-readable. It is more verbose than JSON but provides greater flexibility through 
    its use of custom tags.
    
    \begin{highlight}[XML]
    
        XML is designed to store and transport data, emphasizing the self-descriptive nature of the data and the flexibility of defining custom tags.
        
        \begin{itemize}
            \item \textbf{Structure}: XML documents are structured with a root element, child elements, attributes, and text content.
    \begin{code}[XML]
    <person>
        <name>John Doe</name>
        <age>30</age>
        <isStudent>false</isStudent>
        <courses>
            <course>Math</course>
            <course>Science</course>
            <course>History</course>
        </courses>
    </person>
    \end{code}
            \item \textbf{Extensibility}: XML allows the definition of custom tags, making it highly extensible and suitable for complex data structures.
            \item \textbf{Use Cases}: XML is widely used in systems where data needs to be validated against a schema (XSD) or when the data interchange requires rich data descriptions.
        \end{itemize}
    
    \end{highlight}
    
    \subsubsection*{Comparison and Use Cases}
    
    Both JSON and XML are widely used for data interchange, but they have distinct advantages and are suited for different scenarios.
    
    \begin{highlight}[Comparison and Use Cases]
    
        When choosing between JSON and XML, consider the specific requirements of your project, including data complexity, readability, and interoperability needs.
        
        \begin{itemize}
            \item \textbf{JSON Advantages}: Simplicity, ease of use, faster parsing, and native support in JavaScript.
            \item \textbf{XML Advantages}: Extensive support for complex data structures, schema validation, and data transformation capabilities (XSLT).
            \item \textbf{Typical Use Cases}: JSON is commonly used in web APIs and data transmission in web applications, while XML is often used in document storage, configuration files, and data 
            interchange in enterprise applications.
        \end{itemize}
    
    \end{highlight}
    
    \begin{highlight}[Summary of Key Concepts]
    
        Key concepts from JSON and XML overview:
        
        \begin{itemize}
            \item \textbf{JSON}: Lightweight, easy to read/write, ideal for data interchange in web applications.
            \item \textbf{XML}: Flexible, supports complex data structures, used in applications requiring rich data descriptions and validation.
            \item \textbf{Comparison}: JSON is simpler and more efficient for most web applications, while XML offers more extensive capabilities for document-based data.
        \end{itemize}
        
    \end{highlight}
\end{notes}