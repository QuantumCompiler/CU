\clearpage

\renewcommand{\ChapTitle}{\CSPBIntro}
\renewcommand{\SectionTitle}{CSPB 1300}

\chapter{\ChapTitle}
\section{\SectionTitle}
\horizontalline{0}{0}

\subsection{Course Overview}

The overview of this course can be seen below.

\course{\hypertarget{CSPB:1300}{\CSPBIntro} \hspace*{1pt} - Prerequisites: N/A - Credits: 4}{
    \vspace*{-1.5em}
    \begin{highlight}[\CSPBIntro]
        \subsubsection*{Brief Description of Course Content}

        The course covers techniques for writing computer programs in high level programming languages to solve problems of interest in a range of application domains. This class is intended for students 
        with little to no experience with programming. \vspace*{1em}
    
        \subsubsection*{Specific Goals}
    
        By the end of this course, students should be well positioned to learn any mainstream programming language, and have a foundation for learning more advanced concepts for software engineering and 
        computer science. \vspace*{1em}
        
        \subsubsection*{Specific Outcomes of Instruction}
        
        \begin{itemize}
            \item Understand how to break down hard problems into a series of sub-problems.
            \item Be able to use fundamental programming constructs (such as variables, conditional and iterative control structures) in Python and C++.
            \item Understand and be able to implement simple input and output (I/O) (e.g. interactive input from the user, or using disk storage).
            \item Design functions and reason about their role in programs, including an understanding of passing arguments and returning values.
            \item Learn the properties of data types, including primitive types like numbers and booleans, as well as complex data types like lists and dictionaries.
            \item Use an Integrated Development Environment (IDE) to write code.  Begin to understand the art of debugging as part of software development.
            \item Design and create code using the fundamentals of object-oriented design methods.
            \item Develop an understanding of software development as a dynamic, social process, and that learning how to seek out information is a necessary skill for success.
            \item Leverage two different programming languages to understand programming concepts in general rather than just in the particular.
            \item Understand type systems (dynamic vs static).
            \item Know the differences between interpreted and compiled languages.
        \end{itemize}
        
        \subsubsection*{Brief List of Topics to be Covered}
    
        \begin{itemize}
            \item Python Basics
            \item Debugging
            \item Modules and Functions
            \item Selection
            \item Iterable Data Structures
            \item Classes and Objects
            \item Intro to C++ \& C++ program composition
        \end{itemize}
        
        \subsubsection*{Mathematic Concepts Used}
    
        \begin{itemize}
            \item Basic Algebra
            \item Modulo
        \end{itemize}
    \end{highlight}
}

\subsection{Course Notes}

The transfer equivalent of this course's work can be found at the following repo: \coursedoc{\href{https://github.com/QuantumCompiler/Introduction-To-Computer-Science}{CSCI 111 - Introduction To Computer Science}}