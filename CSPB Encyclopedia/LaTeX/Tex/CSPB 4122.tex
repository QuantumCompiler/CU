\clearpage

\renewcommand{\ChapTitle}{\CSPBIV}
\renewcommand{\SectionTitle}{CSPB 4122}

\chapter{\ChapTitle}
\section{\SectionTitle}
\horizontalline{0}{0}

\subsection{Course Overview}

The overview of this course can be seen below.

\course{\hypertarget{CSPB:4122}{\CSPBIV} \hspace*{1pt} - Prerequisites: \hyperlink{CSPB:1300}{CSPB 1300} - Credits: \textbf{3}}{
    \vspace*{-1.5em}
    \begin{highlight}[\CSPBIV]
        \subsubsection*{Brief Description of Course Content}
    
        Studies interactive visualization techniques that help people analyze data. This course introduces design, development, and validation approaches for interactive visualizations with applications 
        in various domains, including the analysis of text collections, software visualization, network analytics, and the biomedical sciences. It covers underlying principles, provides an overview of 
        existing techniques, and teaches the background necessary to design innovative visualizations. \vspace*{1em}
        
        \subsubsection*{Specific Outcomes of Instruction}
    
        \begin{itemize}
            \item Define information visualization and describe its role in data analysis
            \item Illustrate underlying principles in information visualization
            \item Explain the four nested levels of visualization design
            \item Design an interactive data visualization
            \item Build a prototype implementation of an interactive visualization using existing tools/frameworks
            \item Critique visualization techniques and tools
            \item Evaluate the utility of visualization system
        \end{itemize}
        
        \subsubsection*{Brief List of Topics to be Covered}
    
        \begin{itemize}
            \item Review existing design approaches
            \item Learn a framework for analyzing and critiquing interactive visualizations
            \item Develop your own visualizations using existing tools and frameworks
        \end{itemize}
        
        \subsubsection*{Mathematical Concepts Used}
    
        \begin{itemize}
            \item No significant mathematical concepts needed for this course.
        \end{itemize}
    \end{highlight}
}