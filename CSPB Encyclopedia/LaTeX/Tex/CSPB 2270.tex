\clearpage

\renewcommand{\ChapTitle}{\CSPBDataStruct}
\renewcommand{\SectionTitle}{CSPB 2270}

\chapter{\ChapTitle}
\section{\SectionTitle}
\horizontalline{0}{0}

\subsection{Course Overview}

The overview of this course can be seen below.

\course{\hypertarget{CSPB:2270}{\CSPBDataStruct} \hspace*{1pt} - Prerequisites: \hyperlink{CSPB:1300}{CSPB 1300} - Credits: 4}{
    \vspace*{-1.5em}
    \begin{highlight}[\CSPBDataStruct]
        \subsubsection*{Brief Description of Course Content}
    
        Studies data abstractions (e.g., stacks, queues, lists, trees) and their representation techniques (e.g., linking, arrays). Introduces concepts used in algorithm design and analysis including criteria 
        for selecting data structures to fit their applications. Topics include data and program representations, computer organization effect on performance and mechanisms used for program isolation and 
        memory management. \vspace*{1em}
        
        \subsubsection*{Specific Outcomes of Instruction}
    
        \begin{itemize}
            \item Document code including precondition/postcondition contracts for functions and invariants for classes.
            \item Determine quadratic, linear and logarithmic running time behavior in simple algorithms, write big-O expressions to describe this behavior, and state the running time behaviors for all basic operations on the data structures presented in the course.
            \item Create and recognize appropriate test data for simple problems, including testing boundary conditions and creating/running test cases, and writing simple interactive test programs to test any newly implemented class.
            \item Define basic data types (vector, stack, queue, priority queue, map, list).
            \item Specify, design and test new classes using the principle of information hiding for the following data structures: array-based collections (including dynamic arrays), list-based collections (singly-linked lists, doubly-linked lists, circular-linked lists), stacks, queues, priority queues, binary search trees, heaps, hash tables, graphs (e.g. for depth-first and breadth-first search), and at least one balanced search tree.
            \item Be able to describe how basic data types are stored in memory (sequential or distributed), predict what may happen when they exceed those bounds.
            \item Correctly use and manipulate pointer variables to change variables and build dynamic data structures.
            \item Determine an appropriate data structure for given problems.
            \item Follow, explain, trace, and be able to implement standard computer science algorithms using standard data types, such as a stack-based evaluation of arithmetic expressions or a traversal of a graph.
            \item Recognize situations in which a subtask is nothing more than a simpler version of the larger problem and design recursive solutions for these problems.
            \item Follow, explain, trace, and be able to implement binary search and a variety of quadratic sorting algorithms including mergesort, quicksort and heapsort.
        \end{itemize}
        
        \subsubsection*{Brief List of Topics to be Covered}
    
        \begin{itemize}
            \item Cost of algorithms and Big O notation.
            \item Memory and pointers, structs, and dynamic memory allocation.
            \item Linked lists, stacks and queues.
            \item Trees: Binary trees, binary search trees, tree traversal, recursion.
            \item Tree balancing: red-black trees.
            \item Graphs: graph traversal algorithms, depth-first and breadth-first search.
            \item Hash tables, hash functions, collision resolution algorithms.
            \item Algorithms for sorting, such as insertion sort, bubble sort, quick sort, and merge sort.
        \end{itemize}
        
        \subsubsection*{Mathematic Concepts Used}
    
        \begin{itemize}
            \item Logarithms
            \item Big O
            \item Recursion
            \item Trees
            \item Graphs
        \end{itemize}
    \end{highlight}
}

\subsection{Course Notes}

The notes for this course can be found below: \coursedoc{\pdflink{\CSPBDataStructDir CSPB 2270 - Data Structures Course Notes.pdf}{CSPB 2270 - Data Structures Course Notes}}