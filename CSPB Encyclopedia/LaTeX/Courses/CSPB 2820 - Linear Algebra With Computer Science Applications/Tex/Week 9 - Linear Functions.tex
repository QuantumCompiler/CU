\clearpage

\renewcommand{\ChapTitle}{Linear Functions}
\renewcommand{\SectionTitle}{Linear Functions}

\chapter{\ChapTitle}
\section{\SectionTitle}
\horizontalline{0}{0}

\subsection{Assigned Reading}

The reading assignments for this week are from, \VMLS \hspace*{1pt} and \PyCap:

\begin{itemize}
    \item \pdflink{\ReadMatDir VMLS Chapter 8.1 - Linear And Affine Functions.pdf}{VMLS Chapter 8.1 - Linear And Affine Functions}
    \item \pdflink{\ReadMatDir VMLS Chapter 8.2 - Linear Function Models.pdf}{VMLS Chapter 8.2 - Linear Function Models}
    \item \pdflink{\ReadMatDir VMLS Chapter 8.3 - Systems Of Linear Equations.pdf}{VMLS Chapter 8.3 - Systems Of Linear Equations}
    \item \pdflink{\ReadMatDir Python Companion Chapter 8.1 - Linear And Affine Functions.pdf}{Python Companion Chapter 8.1 - Linear And Affine Functions}
    \item \pdflink{\ReadMatDir Python Companion Chapter 8.2 - Linear Function Models.pdf}{Python Companion Chapter 8.2 - Linear Function Models}
    \item \pdflink{\ReadMatDir Python Companion Chapter 8.3 - Systems Of Linear Equations.pdf}{Python Companion Chapter 8.3 - Systems Of Linear Equations}
\end{itemize}

\subsection{Piazza}

Must post / respond to at least \textbf{four} Piazza posts this week.

\subsection{Lectures}

The lectures for this week and their links can be found below:

\begin{itemize}
    \item \href{https://applied.cs.colorado.edu/mod/hvp/view.php?id=50757}{8.1 Linear Functions} $\approx$ 22 min.
    \item \href{https://applied.cs.colorado.edu/mod/hvp/view.php?id=50758}{8.2 Regression Models} $\approx$ 9 min.
    \item \href{https://applied.cs.colorado.edu/mod/hvp/view.php?id=50759}{8.3 Linear Equations} $\approx$ 15 min.
    \item \href{https://applied.cs.colorado.edu/mod/hvp/view.php?id=50760}{8.3 Over And Underdetermined Systems} $\approx$ 11 min.
    \item \href{https://applied.cs.colorado.edu/mod/hvp/view.php?id=50761}{Random Exercise: Chapter 8} $\approx$ 8 min.
\end{itemize}

\subsection{Assignments}

The assignment for this week is:

\begin{itemize}
    \item \pdflink{\AssDir Assignment 8 - Linear Equations.pdf}{Assignment 8 - Linear Equations}
\end{itemize}

\subsection{Quiz}

The quiz for this week is:

\begin{itemize}
    \item \pdflink{\QuizDir Quiz 8 - Linear Equations.pdf}{Quiz 8 - Linear Equations}
\end{itemize}

\subsection{Chapter Summary}

The chapter that we will review this week is \textbf{VMLS Chapter 8 - Linear Equations}. The first section that we will cover this week is \textbf{VMLS Section 8.1 - Linear And Affine Functions}.

\begin{notes}{VMLS Section 8.1 - Linear And Affine Functions}
    \subsubsection*{Linear Functions:}

    Linear functions are perhaps the most straightforward mathematical relationships between variables. They adhere to two key principles:

    \begin{enumerate}
        \item \textbf{Additivity:} This means that when you add two inputs to the function, the corresponding outputs will add as well. In other words, $f(x + y) = f(x) + f(y)$. For example, if you're 
        measuring the cost of producing a certain number of items, a linear cost function implies that the cost of producing 10 items is the sum of the costs of producing 5 items and then producing 
        another 5 items.
        \item \textbf{Homogeneity:} Homogeneity implies that when you scale the input by a factor (multiply it by a constant), the output scales accordingly. In mathematical terms, $f(ax) = af(x)$. 
        In our cost example, if you double the number of items to produce, the cost should double as well if it follows a linear relationship.
    \end{enumerate}
    
    \subsubsection*{Affine Functions:}
    
    Affine functions build upon the principles of linearity but introduce an extra element: a constant offset. This constant, often denoted as "$b$," adds versatility to the function:

    \begin{enumerate}
        \item \textbf{Additivity and Homogeneity:} Like linear functions, affine functions maintain additivity and homogeneity. However, the presence of the constant term "$b$" means that they may not 
        pass through the origin (0,0) on a graph, unlike linear functions.
        \item \textbf{Translation and Shifting:} The constant term "$b$" allows affine functions to model more complex relationships where there is an initial value or a shift. For instance, in finance, 
        an affine function could represent a time series of stock prices, accounting for both the underlying trend and an initial value (offset) due to factors like an opening price.
    \end{enumerate}

    \subsubsection*{Relationship:}

    The relationship between linear and affine functions is straightforward: every linear function is also an affine function, but not vice versa. This is because an affine function can have a constant 
    term "$b$" that makes it non-linear when "$b$" is not zero.
    
    In essence, linear and affine functions are essential tools for modeling and understanding relationships between variables in various scientific and practical contexts. They provide a mathematical 
    foundation for describing how different quantities interact, evolve, and impact one another, making them indispensable in the toolkit of mathematicians, scientists, engineers, and economists.
\end{notes}

The second section that we will cover this week is \textbf{VMLS Section 8.2 - Linear Function Models}.

\begin{notes}{VMLS Section 8.2 - Linear Function Models}
    \subsection*{Overview}

    Linear function models are a fundamental concept in the field of linear equations and mathematics. These models represent relationships between variables in a straightforward and easily interpretable 
    way. A linear function is defined by two key principles:

    \begin{enumerate}
        \item \textbf{Linearity:} Linearity means that the relationship between variables is proportional and additive. In a linear function, if you double one variable, the output also doubles, and 
        combining multiple variables is as simple as adding or subtracting their contributions. Mathematically, this property is expressed as $f(ax + by) = af(x) + bf(y)$, where $a$ and $b$ are constants.
        \item \textbf{Constant Rate of Change:} Linear functions exhibit a constant rate of change. This means that as one variable increases or decreases by a fixed amount, the output changes by a 
        consistent amount. The rate of change is represented by the slope of the line in a linear function.
    \end{enumerate}
    
    Linear function models are widely used in various fields for their simplicity and applicability. Some key points to consider:

    \begin{itemize}
        \item \textbf{Graphical Representation:} Linear functions are represented by straight lines on a graph, where the slope of the line indicates the rate of change, and the y-intercept represents 
        the initial value when the input is zero.
        \item \textbf{Real-Life Applications:} Linear function models are extensively applied in fields such as physics, economics, engineering, and data analysis. They are used to describe relationships 
        like distance vs. time, cost vs. quantity, and temperature vs. time.
        \item \textbf{Regression Analysis:} Linear regression is a statistical technique that employs linear function models to analyze and predict relationships between variables. It is widely used in 
        data science for predictive modeling.
        \item \textbf{Slope-Intercept Form:} Linear functions are often expressed in the slope-intercept form, $y = mx + b$, where $m$ is the slope (rate of change) and $b$ is the y-intercept (initial 
        value).
        \item \textbf{Limitations:} While linear function models are powerful, they have limitations. They may not accurately represent complex, nonlinear relationships. In such cases, more sophisticated 
        models, like quadratic or exponential functions, are needed.
    \end{itemize}
    
    Linear function models are a foundational concept in linear equations, providing a simple yet effective way to describe relationships between variables. Their versatility makes them a valuable tool 
    for modeling, analysis, and prediction in various scientific, engineering, and real-world applications.

    \subsubsection*{Taylor Approximations}

    Taylor Approximations are a powerful tool used in linear function models to approximate complex nonlinear functions with simpler linear representations. These approximations are based on Taylor 
    series expansions and provide a means to locally linearize a function around a specific point.

    Key points to understand about Taylor Approximations in linear function models:

    \begin{itemize}
        \item \textbf{Local Linearity:} Taylor Approximations are used when dealing with functions that exhibit nonlinear behavior. By selecting a reference point (usually denoted as 'a') within the 
        function's domain, a Taylor series expansion is performed to approximate the function as a linear equation around that point. This results in a linear model that closely approximates the 
        function's behavior within a small neighborhood of 'a.'
        \item \textbf{Taylor Series Expansion:} The Taylor series expansion of a function f(x) around a point 'a' involves finding the function's derivatives at 'a.' The general form of a first-order 
        Taylor Approximation (linearization) is given by: $f(x) \approx f(a) + f'(a)(x - a)$, where f'(a) represents the first derivative of the function at 'a.' Higher-order approximations consider 
        additional derivatives for increased accuracy.
        \item \textbf{Applications:} Taylor Approximations are extensively applied in various fields, including physics, engineering, and numerical analysis. They are particularly useful for approximating 
        nonlinear physical phenomena and simplifying complex mathematical models. For instance, in physics, they are used to linearize equations of motion, making them easier to solve.
        \item \textbf{Error Analysis:} The accuracy of Taylor Approximations depends on the proximity of the selected point 'a' to the point of interest. The smaller the interval around 'a,' the more 
        accurate the linear approximation. Error analysis is essential to assess the quality of the approximation and determine its suitability for a given application.
        \item \textbf{Higher-Order Approximations:} While first-order Taylor Approximations provide local linearizations, higher-order approximations incorporate additional derivatives and result in 
        polynomial models of higher degrees. These can offer even more accurate representations of nonlinear functions but are more computationally intensive.
        \item \textbf{Limitations:} Taylor Approximations are valid only in a local neighborhood of the chosen point 'a.' They may not accurately represent the entire function if it exhibits significant 
        nonlinear behavior over a wide range.
    \end{itemize}

    Taylor Approximations are valuable tools in linear function modeling when dealing with nonlinear functions. They allow for the creation of local linear models that simplify complex relationships and 
    facilitate analysis and prediction. The choice of 'a' and the order of the approximation depend on the specific problem and the desired level of accuracy.

    \subsubsection*{Regression Model}

    Regression models are fundamental tools within the realm of linear function models, aimed at understanding and predicting relationships between variables. These models assume that the relationships 
    between the response variable and predictor variables can be represented linearly. Here are key insights into regression models in this context:

    \begin{itemize}
        \item \textbf{Linear Assumption:} In linear function models, regression models assume that the relationship between the response variable (usually denoted as 'Y') and predictor variables 
        ($'X_{1},' 'X_{2},'$ etc.) can be expressed as a linear combination of these variables. The relationship is represented as $Y = \beta_{0} + \beta_{1}X_{1} + \beta_{2}X_{2} + \dots + \epsilon$, 
        where $\beta_{0}, \beta_{1}, \beta_{2}$, etc., are coefficients, and $\epsilon$ represents the error term.
        \item \textbf{Simple Linear Regression:} Simple linear regression models the relationship between a single predictor variable (X) and the response variable (Y). It aims to find the best-fitting 
        linear equation $Y = \beta_{0} + \beta_{1}X$ that minimizes the sum of squared differences between observed and predicted values.
        \item \textbf{Multiple Linear Regression:} Multiple linear regression extends the simple linear model to incorporate multiple predictor variables ($X_{1}, X_{2},$ etc.). It seeks to find coefficients 
        $\beta_{0}, \beta_{1}, \beta_{2}$, etc., that optimize the linear equation $Y = \beta_{0} + \beta_{1}X_{1} + \beta_{2}X_{2} + \dots + \epsilon$.
        \item \textbf{Model Interpretation:} In linear function models, the coefficients ($\beta$ values) provide insights into the strength and direction of the relationships between predictor variables 
        and the response variable. Positive coefficients indicate a positive relationship, while negative coefficients signify a negative relationship.
        \item \textbf{Assumptions:} Regression models rely on assumptions such as linearity, independence of errors, constant variance of errors (homoscedasticity), and normality of error terms. Violations 
        of the1se assumptions may impact model accuracy.
        \item \textbf{Model Evaluation:} Regression models are evaluated based on metrics like the coefficient of determination (R-squared), mean squared error (MSE), and others. These metrics gauge the 
        model's fit and predictive performance.
        \item \textbf{Applications:} Regression models are widely used in various fields, including economics, social sciences, finance, and machine learning. They help researchers and analysts uncover 
        relationships, make predictions, and inform decision-making.
        \item \textbf{Extensions:} Linear function models can be extended to include interactions, polynomial terms, and other transformations to capture complex relationships. These extensions allow for 
        modeling nonlinear patterns when necessary.
    \end{itemize}
    
    Regression models are essential components of linear function models, providing a framework to understand and predict relationships between variables. They are versatile tools with applications across 
    diverse domains, and their interpretability makes them valuable for making informed decisions based on data analysis.
\end{notes}

The last section that we will cover this week is \textbf{VMLS Section 8.3 - Systems Of Linear Equations}.

\begin{notes}{VMLS Section 8.3 - Systems Of Linear Equations}
    \subsection*{Overview}

    A system of linear equations is a collection of two or more linear equations involving the same set of variables. These systems are fundamental in various areas of mathematics, science, and engineering. 
    Here are key insights into systems of linear equations:

    \begin{itemize}
        \item \textbf{Linear Equations:} Each equation in the system is a linear equation, meaning it can be expressed in the form of $ax + by + cz + \dots = d$, where 'a,' 'b,' 'c,' etc., are constants, and 
        $'x,' 'y,' 'z,'$ etc., are variables. The goal is to find values for the variables that satisfy all equations simultaneously.        
        \item \textbf{Variables and Coefficients:} Variables represent unknown quantities, while coefficients represent known constants. The system may involve more variables than equations, resulting in an 
        overdetermined or underdetermined system.        
        \item \textbf{Matrix Representation:} Systems of linear equations can be represented using matrices and vectors. The coefficient matrix $'A'$ contains the coefficients of the variables, the vector 
        $'X'$ represents the variables, and the constant vector $'B'$ holds the constants from the right-hand side of the equations. The system is written as $'AX = B.'$        
        \item \textbf{Solution Space:} The set of all possible solutions to the system forms a solution space. The system can have one unique solution, infinitely many solutions, or no solution at all, 
        depending on the relationships between equations.        
        \item \textbf{Methods of Solution:} Various methods are employed to solve systems of linear equations, including Gaussian elimination, matrix inversion, Cramer's rule (for square systems), and 
        numerical techniques like the Gauss-Jordan method and iterative methods.        
        \item \textbf{Applications:} Systems of linear equations have widespread applications in diverse fields. In physics, they describe equilibrium conditions and electrical circuits. In economics, they 
        model supply and demand relationships. In engineering, they solve problems related to structural analysis and control systems.        
        \item \textbf{Linear Independence:} The concept of linear independence is crucial in determining whether a system has a unique solution. Linearly independent equations provide the best chance of a 
        unique solution.        
        \item \textbf{Homogeneous Systems:} When all equations in a system have a right-hand side of zero, it's referred to as a homogeneous system. Such systems always have a trivial solution (all 
        variables equal to zero) and may have non-trivial solutions.        
        \item \textbf{Nonlinear Systems:} While linear equations deal with linear relationships, nonlinear systems involve equations with nonlinear relationships between variables. Solving nonlinear 
        systems is generally more complex and may require numerical methods.
    \end{itemize}

    Systems of linear equations are essential tools for modeling real-world problems and finding solutions to unknown variables. Their mathematical representation, methods of solution, and applications 
    make them a fundamental concept in mathematics and its various applications.
\end{notes}