\clearpage

\renewcommand{\ChapTitle}{Matrix Operations}
\renewcommand{\SectionTitle}{Matrix Operations}

\chapter{\ChapTitle}
\section{\SectionTitle}
\horizontalline{0}{0}

\subsection{Assigned Reading}

The reading assignments for this week are from, \VMLS \hspace*{1pt} and \PyCap:

\begin{itemize}
    \item \textbf{Sections 7.1, 7.2, 7.3, 7.4} from VMLS.
    \item \pdflink{\ReadMatDir VMLS Chapter 7.1 - Geometric Transformations.pdf}{VMLS Chapter 7.1 - Geometric Transformations}
    \item \pdflink{\ReadMatDir VMLS Chapter 7.2 - Selectors.pdf}{VMLS Chapter 7.2 - Selectors}
    \item \pdflink{\ReadMatDir VMLS Chapter 7.3 - Incidence Matrix.pdf}{VMLS Chapter 7.3 - Incidence Matrix}
    \item \pdflink{\ReadMatDir VMLS Chapter 7.4 - Convolution.pdf}{VMLS Chapter 7.4 - Convolution}
    \item \pdflink{\ReadMatDir Python Companion Chapter 7.1 - Geometric Transformations.pdf}{Python Companion Chapter 7.1 - Geometric Transformations}
    \item \pdflink{\ReadMatDir Python Companion Chapter 7.2 - Selectors.pdf}{Python Companion Chapter 7.2 - Selectors}
    \item \pdflink{\ReadMatDir Python Companion Chapter 7.3 - Incidence Matrix.pdf}{Python Companion Chapter 7.3 - Incidence Matrix}
    \item \pdflink{\ReadMatDir Python Companion Chapter 7.4 - Convolution.pdf}{Python Companion Chapter 7.4 - Convolution}
\end{itemize}

\subsection{Piazza}

Must post / respond to at least \textbf{four} Piazza posts this week.

\subsection{Lectures}

The lectures for this week and their links can be found below:

\begin{itemize}
    \item \href{https://www.youtube.com/watch?v=jXN50fuRSqE}{VMLS Chapter 7 Part 1} $\approx$ 6 min.
    \item \href{https://www.youtube.com/watch?v=v7CH2ry-vrQ}{VMLS Chapter 7 Part 2} $\approx$ 16 min.
    \item \href{https://www.youtube.com/watch?v=kTg0vKppOl8}{VMLS Chapter 7 Part 3} $\approx$ 16 min.
\end{itemize}

\subsection{Assignments}

The assignment for this week is:

\begin{itemize}
    \item \pdflink{\AssDir Assignment 7 - Matrix Operations.pdf}{Assignment 7 - Matrix Operations}
\end{itemize}

\subsection{Quiz}

The quiz for this week is:

\begin{itemize}
    \item \pdflink{\QuizDir Quiz 7 - Matrix Operations.pdf}{Quiz 7 - Matrix Operations}
\end{itemize}

\subsection{Chapter Summary}

The chapter that we will review this week is \textbf{VMLS Chapter 7 - Matrix Examples}. The first section that we will cover this week is \textbf{VMLS Section 7.1 - Geometric Transformations}.

\begin{notes}{VMLS Section 7.1 - Geometric Transformations}
    \subsubsection*{Overview}

    Geometric transformations refer to a fundamental concept in mathematics and computer graphics where the shape, position, or orientation of objects in a geometric space are altered. These transformations 
    are commonly used to manipulate and manipulate shapes, figures, and images. There are several types of geometric transformations, including translation, rotation, scaling, shearing, and reflection, each 
    serving a unique purpose in transforming objects in a 2D or 3D space.

    \begin{enumerate}
        \item \textbf{Translation:} Translation involves shifting an object's position without altering its shape or orientation. In 2D, this is done by adding or subtracting values to the x and y coordinates 
        of each point, effectively moving the object in a specific direction.
        
        \item \textbf{Rotation:} Rotation transforms an object by changing its orientation around a fixed point or axis. In 2D, this can be achieved through trigonometric functions to determine new coordinates 
        based on a specified angle of rotation.
        
        \item \textbf{Scaling:} Scaling changes the size of an object, making it either larger or smaller while maintaining its shape. Scaling factors are applied to the object's dimensions, modifying its 
        proportions accordingly.
        
        \item \textbf{Shearing:} Shearing involves skewing or distorting an object along one axis while keeping the other axis unchanged. This transformation is often used for creating perspective effects.
        
        \item \textbf{Reflection:} Reflection flips an object over a line or plane, creating its mirror image. In 2D, this is achieved by reversing the sign of one coordinate while leaving the other unchanged.
    \end{enumerate}

    Geometric transformations are widely used in computer graphics, computer-aided design (CAD), image processing, and various mathematical applications. They play a crucial role in manipulating and animating 
    objects, simulating physical phenomena, and solving geometric problems. Understanding these transformations is essential for working with graphical data and mathematical modeling in various fields.

    \subsubsection*{Rotations}

    Rotations are fundamental geometric transformations that involve altering the orientation of objects or vectors within a given space. In these transformations, objects are rotated about a specified point, 
    axis, or origin by a certain angle, resulting in a new arrangement while preserving the objects' size and shape. Rotations are essential in various fields, including mathematics, physics, computer graphics, 
    and engineering, where they are used to manipulate and analyze spatial data, model physical phenomena, and create realistic visual representations. These transformations are mathematically represented using 
    rotation matrices, and they play a crucial role in understanding symmetry, transforming coordinate systems, and solving complex spatial problems.

    \begin{highlight}[Rotations Formula]
        The provided equation represents a 2D rotation transformation. In this transformation, a 2D vector \(x\) is rotated by an angle \(\theta\), resulting in a new vector \(y\). The rotation is performed 
        using a 2x2 matrix:

        \begin{equation*}
            y =
            \begin{bmatrix}
                \cos{(\theta)} & -\sin{(\theta)} \\
                \sin{(\theta)} & \cos{(\theta)} \\
            \end{bmatrix}
            x.
        \end{equation*}

        Here's what each component of this equation means:

        \begin{itemize}
            \item \(\cos{(\theta)}\) and \(\sin{(\theta)}\) are trigonometric functions representing the cosine and sine of the rotation angle \(\theta\).
            \item The matrix structure represents the transformation. The top row corresponds to the new x-axis, and the bottom row corresponds to the new y-axis after the rotation.
            \item \(x\) is the input 2D vector that you want to rotate.
            \item \(y\) is the resulting vector after applying the rotation.
        \end{itemize}

        This equation essentially describes how to perform a 2D rotation of a vector \(x\) by an angle \(\theta\) using a matrix. The matrix elements are determined by trigonometric functions of the rotation 
        angle, ensuring that the rotated vector \(y\) retains its length and orientation in the new coordinate system. This concept is fundamental in geometry, graphics, and various engineering applications.
    \end{highlight}

    \textbf{Principles of Rotation:}

    \begin{enumerate}
        \item \textbf{Rotation Center:} A rotation is performed around a fixed point called the "center of rotation" or "pivot point." All points in the object move in a circular path around this center.
        
        \item \textbf{Angle of Rotation:} The angle through which an object is rotated is a crucial parameter. It is measured in degrees or radians and determines the extent of the rotation.
        
        \item \textbf{Direction:} Rotations can be clockwise or counterclockwise, depending on the choice of orientation. Conventionally, counterclockwise rotations are considered positive.
    \end{enumerate}

    \textbf{Properties of Rotation:}

    \begin{enumerate}
        \item \textbf{Preservation of Length:} Rotations preserve the length of line segments. If two points are a certain distance apart before a rotation, they will remain at the same distance after 
        the rotation.
        
        \item \textbf{Preservation of Angles:} Rotations also maintain the angles between lines. If two lines intersect at a specific angle before a rotation, they will intersect at the same angle after 
        the rotation.
        
        \item \textbf{Center of Rotation:} The center of rotation is an invariant point. It remains fixed, and all other points move in circular paths around it.
    \end{enumerate}

    \textbf{Types of Rotations:}

    \begin{enumerate}
        \item \textbf{2D Rotations:} In a two-dimensional space (2D), rotations involve changing the angle of an object around a fixed point. These rotations are often represented using trigonometric 
        functions like sine and cosine.
        
        \item \textbf{3D Rotations:} In a three-dimensional space (3D), rotations are more complex, involving changes in orientation about multiple axes. Representations like Euler angles, rotation 
        matrices, and quaternions are used to describe 3D rotations.
    \end{enumerate}

    \textbf{Challenges in Rotations:}

    \begin{enumerate}
        \item \textbf{Angle Representation:} Choosing the right angle representation (degrees or radians) and dealing with angle wrap-around can be challenging.
        
        \item \textbf{Singularities:} Certain rotation representations may encounter singularities or gimbal lock, limiting their practicality.
    \end{enumerate}

    Rotation is a versatile and powerful geometric transformation with wide-ranging applications in mathematics, science, engineering, art, and technology. Understanding its principles and properties 
    is fundamental for anyone working with spatial transformations or manipulating objects in a 2D or 3D space.

    \subsubsection*{Reflection}

    Reflection is a fundamental geometric transformation that involves flipping or mirroring an object or a point across a specified axis or plane. It plays a crucial role in various fields, including 
    geometry, physics, computer graphics, and engineering. This transformation essentially changes the orientation of an object or point while preserving its size and shape. \vspace*{1em}

    \begin{highlight}[Reflections Formula]
        The provided equation represents a 2D reflection transformation matrix. In this transformation, a 2D vector \(x\) is reflected across a line (or mirror line) at an angle \(2\theta\) with respect 
        to the x-axis, resulting in a new vector \(y\). The reflection matrix is defined as:
        
        \begin{equation*}
            y = 
            \begin{bmatrix}
                \cos{(2\theta)} & \sin{(2\theta)} \\
                \sin{(2\theta)} & -\cos{(2\theta)} \\
            \end{bmatrix}
            x.
        \end{equation*}
        
        Here's what each component of this equation means:

        \begin{itemize}
            \item \(\cos{(2\theta)}\) and \(\sin{(2\theta)}\) are trigonometric functions representing the cosine and sine of twice the reflection angle \(2\theta\).
            \item The matrix structure represents the transformation. It describes how the vector \(x\) is reflected across the mirror line.
            \item \(x\) is the input 2D vector that you want to reflect.
            \item \(y\) is the resulting vector after applying the reflection.
        \end{itemize}
        
        This equation essentially describes how to perform a 2D reflection of a vector \(x\) across a line at a specified angle \(2\theta\) using a matrix. The matrix elements are determined by 
        trigonometric functions of the reflection angle, ensuring that the reflected vector \(y\) retains its orientation with respect to the mirror line. Reflection matrices are fundamental tools in 
        geometry, graphics, and various engineering applications.
        
    \end{highlight}
    
    \textbf{Key points about reflection:}
    
    \begin{enumerate}
        \item \textbf{Reflection Across a Line:} In 2D, reflection typically occurs across a line, often referred to as the "mirror line" or "axis of reflection." Anything on one side of this line is 
        mirrored to the other side. The line of reflection is perpendicular to the mirror plane.
    
        \item \textbf{Mirror Plane:} In 3D, reflection occurs across a plane called the "mirror plane." Objects or points are reflected from one side of the plane to the other. The normal vector of 
        the mirror plane determines the direction of reflection.
    
        \item \textbf{Mathematical Representation:} Reflection can be represented mathematically using matrices. In 2D, a reflection matrix is defined based on the angle of reflection. In 3D, reflection 
        matrices depend on the orientation of the mirror plane.
    
        \item \textbf{Preservation of Distance:} Reflection preserves the distance between points. If a point is a certain distance from the mirror line or mirror plane, its reflected counterpart will 
        be the same distance on the other side.
    
        \item \textbf{Symmetry:} Reflection is a symmetry operation. Objects that possess reflective symmetry are identical on both sides of the mirror line or plane.
    
        \item \textbf{Multiple Reflections:} Multiple reflections can occur when an object is reflected multiple times across different mirror lines or planes. These transformations are cumulative.
    
        \item \textbf{Reflection in Mathematics:} In mathematics, reflection is an essential concept in linear algebra, where reflection matrices are used for various transformations. It's also crucial 
        in geometry when studying symmetry and transformations.
    
    \end{enumerate}
    
    Reflection is a powerful tool for understanding the behavior of light, sound, and various physical phenomena. It also plays a significant role in creating realistic graphics and simulations, making 
    it a fundamental concept in several scientific and technical disciplines.
\end{notes}

The next section that we will cover this week is \textbf{VMLS Section 7.2 - Selectors}.

\begin{notes}{VMLS Section 7.2 - Selectors}
    \subsubsection*{Overview}

    \textbf{Selectors in Linear Algebra}

    Selectors in the context of matrix operations in linear algebra refer to the process of choosing specific elements or subsets of a matrix to perform operations on or extract information from. These 
    selectors play a crucial role in various matrix-related tasks, such as matrix addition, multiplication, transposition, and decomposition.

    \begin{enumerate}
        \item \textbf{Element Selection:} Selectors allow targeting individual elements within a matrix, denoted as A[i][j], where i and j represent the row and column indices.
        
        \item \textbf{Row and Column Selection:} Entire rows or columns of a matrix can be selected, enabling operations involving entire vectors within the matrix.
        
        \item \textbf{Submatrix Selection:} Selectors facilitate the extraction of submatrices from a larger matrix, useful for focusing on specific regions of interest.
        
        \item \textbf{Block Matrix Operations:} Targeting blocks or submatrices of a larger matrix is possible using selectors, which is valuable for tasks like matrix multiplication.
        
        \item \textbf{Transpose Selection:} Selectors are used in transpose operations ($A^T$), rearranging rows and columns efficiently.
        
        \item \textbf{Diagonal Selection:} Diagonal selectors are employed to extract or manipulate diagonal elements, essential in various linear algebra operations.
        
        \item \textbf{Matrix Concatenation:} Selectors enable the combination or concatenation of matrices horizontally or vertically, particularly useful in block matrix operations.
        
        \item \textbf{Element-wise Operations:} For element-wise operations, where corresponding elements of two matrices are combined using arithmetic operations, selectors are fundamental.
    \end{enumerate}

    Selectors in linear algebra provide a flexible and precise way to access and manipulate elements within matrices. They are fundamental for performing a wide range of matrix operations and 
    transformations, making them a crucial concept in linear algebra and its various applications.

    \subsubsection*{Down Sampling}

    Down-sampling is a data reduction technique commonly used in signal processing, image processing, and data analysis. Its primary purpose is to decrease the amount of data by selecting a 
    subset of data points while preserving essential information. In the context of digital signals or images, down-sampling involves reducing the sample rate or resolution, effectively reducing 
    the size of the data. \vspace*{1em}

    \textbf{Key points about down-sampling:}

    \begin{enumerate}
        \item \textbf{Data Reduction:} Down-sampling is employed when working with large datasets to make data more manageable and computationally efficient.
        
        \item \textbf{Anti-Aliasing:} In image processing, down-sampling is often accompanied by anti-aliasing filters to prevent aliasing artifacts, ensuring that the reduced data still 
        represents the original information accurately.
        
        \item \textbf{Applications:} Down-sampling is used in various fields, such as audio processing, where high-frequency components are removed to reduce file size without significant 
        loss in audio quality.
        
        \item \textbf{Sub-sampling:} Sub-sampling is a common down-sampling technique where every nth data point is selected, effectively reducing the data's resolution. For example, reducing 
        the frame rate of a video.
        
        \item \textbf{Down-sampling Factor:} The degree of down-sampling is determined by the down-sampling factor, which indicates how many data points are retained for every set of 
        contiguous data points.
        
        \item \textbf{Decimation:} Decimation is a more specific form of down-sampling where data points are removed, often used in digital signal processing.
        
        \item \textbf{Loss of Information:} Down-sampling typically results in a loss of fine details or high-frequency components. The choice of down-sampling method and factor depends on 
        the specific application and the acceptable level of information loss.
        
        \item \textbf{Compression:} Down-sampling is a fundamental step in data compression techniques, helping reduce storage requirements while maintaining essential information.
        
        \item \textbf{Preprocessing:} In machine learning and data analysis, down-sampling can be used to balance imbalanced datasets by reducing the size of the majority class.
        
        \item \textbf{Trade-Off:} Down-sampling involves a trade-off between reducing data size and preserving critical information. The optimal down-sampling strategy depends on the specific 
        task and data characteristics.
    \end{enumerate}

    Down-sampling is a valuable technique for managing large datasets, reducing storage requirements, and simplifying data analysis. However, it requires careful consideration to balance the 
    reduction in data size with the retention of essential information, making it a common preprocessing step in various fields.
\end{notes}

The next section that we will cover this week is \textbf{VMLS Section 7.3 - Incidence Matrix}.

\begin{notes}{VMLS Section 7.3 - Incidence Matrix}
    \subsubsection*{Overview}

    An incidence matrix is a fundamental concept in graph theory and linear algebra used to represent relationships between elements of two sets, particularly nodes (vertices) and edges in a graph. 
    Denoted as $A$, the incidence matrix has rows corresponding to nodes and columns corresponding to edges. Entries in the matrix indicate whether a node is incident to an edge. Key points about 
    incidence matrices include:

    \begin{enumerate}
        \item \textbf{Binary Representation:} Incidence matrices are often binary, with entries of 0 or 1. A 1 in row $i$ and column $j$ indicates node $i$ is incident to edge $j$, while 0 indicates 
        no incidence.
        
        \item \textbf{Directed and Undirected Graphs:} In directed graphs, matrices may contain 1 to indicate edge direction. For undirected graphs, entries are typically 0 and 1.
        
        \item \textbf{Size:} The matrix size depends on the number of nodes and edges in the graph. For $n$ nodes and $m$ edges, the matrix size is $n \times m$.
        
        \item \textbf{Sparse Matrices:} They are often sparse, particularly in large graphs where most nodes are not directly connected to all edges.
        
        \item \textbf{Node Degrees:} Row sums of an incidence matrix reveal node degrees in the graph. In undirected graphs, the row sum represents the node's degree.
        
        \item \textbf{Graph Connectivity:} Incidence matrices aid in studying graph connectivity and determining if a graph is connected.
        
        \item \textbf{Linear Dependence:} In linear algebra, columns of an incidence matrix help analyze linear dependence relationships among edges.
        
        \item \textbf{Spanning Trees:} They are essential for finding minimum spanning trees and solving network optimization problems.
    \end{enumerate}

    Incidence matrices are a powerful tool for efficiently representing and analyzing graph structures, making them a cornerstone of graph theory and related fields.

    \begin{highlight}[Incidence Matrix Example With Graph]
        The provided matrix and graph together represent an incidence matrix and its corresponding directed graph. In the context of graph theory, the incidence matrix illustrates the relationships 
        between nodes and edges in a directed graph. Here's how to interpret this:
        \begin{equation*}
            \begin{bmatrix}
                -1 & 1 & -1 & 0 \\
                1 & -1 & 0 & -1 \\
                0 & 0 & 1 & 1 \\
            \end{bmatrix}
        \end{equation*}
        The matrix shown is an incidence matrix where each row represents a node (labeled 1, 2, 3, and 4), and each column represents an edge. The entries in the matrix indicate the incidence of each 
        edge on each node. Specifically, a '-1' in the matrix at row 'i' and column 'j' signifies that node 'i' is the tail of edge 'j,' while a '1' indicates that node 'i' is the head of edge 'j.' 
        Zeros represent no connection between the node and edge.
        \begin{center}
            \begin{tikzpicture}[->, >=stealth', auto, node distance=2cm, main node/.style={circle, draw}]
                \node[main node] (1) at (0, 0) {1};
                \node[main node] (2) at (2, 0) {2};
                \node[main node] (3) at (0, -2) {3};
                \node[main node] (4) at (2, -2) {4};
            
                \path
                (1) edge[bend left] node {} (2)
                (2) edge[bend left] node {} (1)
                (1) edge[bend left] node {} (3)
                (3) edge[bend left] node {} (1)
                (2) edge[bend left] node {} (3)
                (3) edge[bend left] node {} (2)
                (3) edge[bend left] node {} (4)
                (4) edge[bend left] node {} (3);
                
                \node[right=1cm of 4] (matrix) {
                };
            \end{tikzpicture}
        \end{center}
        The directed graph depicted below the matrix shows the relationships between nodes (labeled 1, 2, 3, and 4) and directed edges (indicated by arrows). Each directed edge connects a source node 
        (tail) to a target node (head), consistent with the entries in the incidence matrix. The graph visually represents the connections and flow of information or influence within the system.

        This combination of an incidence matrix and directed graph provides a concise representation of the structure and relationships within a directed graph, making it a valuable tool in various 
        fields such as network analysis, optimization, and transportation planning.
    \end{highlight}
\end{notes}

The last section that we will cover this week is \textbf{VMLS Section 7.4 - Convolution}.

\begin{notes}{VMLS Section 7.4 - Convolution}
    \subsubsection*{Overview}

    Convolution is a fundamental mathematical operation widely used in signal processing, image analysis, and deep learning. It involves the blending of two functions to produce a third function, 
    which represents how one function modifies the other. In the context of discrete signals or images, convolution computes the output signal by sliding one function (often referred to as the 
    kernel or filter) over the other while calculating the integral of their pointwise product. This operation has several key characteristics:

    \begin{enumerate}
        \item \textbf{Linearity:} Convolution is a linear operation, making it a fundamental building block in linear systems theory. This linearity property simplifies the analysis of complex 
        systems composed of interconnected linear components.

        \item \textbf{Translation-Invariance:} Convolution is translation-invariant, meaning that the same filter applied at different positions produces consistent results. This property is 
        crucial in image processing and feature detection.

        \item \textbf{Commutativity:} Convolution is commutative, meaning that the order in which two functions are convolved does not affect the result. This property is useful in various 
        mathematical and computational contexts.

        \item \textbf{Associativity:} Convolution is associative, allowing multiple convolutions to be combined into a single convolution operation. This property simplifies the implementation 
        of deep neural networks.
    \end{enumerate}

    Convolution finds applications in diverse fields, including image and audio processing for tasks like edge detection and noise reduction, as well as in deep learning for feature extraction 
    and convolutional neural networks (CNNs) used in image recognition and natural language processing. Understanding convolution is essential for anyone working with digital signals, images, 
    or neural networks.
\end{notes}