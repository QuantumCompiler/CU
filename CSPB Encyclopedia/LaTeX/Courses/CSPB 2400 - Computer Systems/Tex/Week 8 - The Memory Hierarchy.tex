\clearpage

\renewcommand{\ChapTitle}{The Memory Hierarchy}
\renewcommand{\SectionTitle}{The Memory Hierarchy}

\chapter{\ChapTitle}
\section{\SectionTitle}
\horizontalline{0}{0}

\subsection{Assigned Reading}

The reading assignment for this week is from, \Textbook:

\begin{itemize}
    \item \pdflink{\ReadingMatDir Chapter 6.1 - Storage Technologies.pdf}{Chapter 6.1 - Storage Technologies}
    \item \pdflink{\ReadingMatDir Chapter 6.2 - Locality.pdf}{Chapter 6.2 - Locality}
    \item \pdflink{\ReadingMatDir Chapter 6.3 - The Memory Hierarchy.pdf}{Chapter 6.3 - The Memory Hierarchy}
    \item \pdflink{\ReadingMatDir Chapter 6.4 - Cache Memories.pdf}{Chapter 6.4 - Cache Memories}
    \item \pdflink{\ReadingMatDir Chapter 6.5 - Writing Cache-Friendly Code.pdf}{Chapter 6.5 - Writing Cache-Friendly Code}
    \item \pdflink{\ReadingMatDir Chapter 6.6 - Putting It Together - The Impact Of Caches On Program Performance.pdf}{Chapter 6.6 - Putting It Together - The Impact Of Caches On Program Performance}
\end{itemize}

\subsection{Lectures}

The lecture videos for this week are:

\begin{itemize}
    \item \lecture{https://www.youtube.com/watch?v=meO9vKW-ZIM}{Memory - Memory Storage Technology}{37}
    \item \lecture{https://www.youtube.com/watch?v=nuR2UV31Nvo}{Memory - Memory Hierarchy}{15}
    \item \lecture{https://www.youtube.com/watch?v=a-s1tdvFnC8}{Memory - Introduction To Caches \& Memory Trends}{28}
    \item \lecture{https://www.youtube.com/watch?v=S0LrHMUXpbk}{The Memory Hierarchy - Cache Memory Organization}{34}
    \item \lecture{https://www.youtube.com/watch?v=cpqHghg4gcI}{The Memory Hierarchy - Memory Mountain \& Prefetching}{10}
    \item \lecture{https://www.youtube.com/watch?v=E63pvAn3OhE}{The Memory Hierarchy - Writing Cache Friendly Code}{22}
\end{itemize}

\noindent The lecture notes for this week are:

\begin{itemize}
    \item \pdflink{\LecNoteDir Memory Hierarchy III Lecture Notes.pdf}{Memory Hierarchy III Lecture Notes}
    \item \pdflink{\LecNoteDir Cache Memories I Lecture Notes.pdf}{Cache Memories I Lecture Notes}
    \item \pdflink{\LecNoteDir Cache Memories II Lecture Notes.pdf}{Cache Memories II Lecture Notes}
    \item \pdflink{\LecNoteDir Cache Friendly Code Lecture Notes.pdf}{Cache Friendly Code Lecture Notes}
\end{itemize}

\subsection{Assignments}

The assignment for this week is:

\begin{itemize}
    \item \href{https://github.com/QuantumCompiler/CU/tree/main/CSPB%202400%20-%20Computer%20Systems/CSPB%202400%20-%20Assignments/CSPB%202400%20-%20Assignment%203%20-%20Attack%20Lab}{Attack Lab}
    \item \href{https://github.com/QuantumCompiler/CU/tree/main/CSPB%202400%20-%20Computer%20Systems/CSPB%202400%20-%20Assignments/CSPB%202400%20-%20Assignment%203%20-%20Attack%20Lab}{Attack Lab Interview}
\end{itemize}

\subsection{Quiz}

The quizzes for this week are:

\begin{itemize}
    \item \pdflink{\QuizDir Quiz 8 - Chapter 6.pdf}{Quiz 8 - Chapter 6}
\end{itemize}

\subsection{Chapter Summary}

The chapter that is being covered this week is \textbf{Chapter 6: The Memory Hierarchy}. The first section that is being covered from this chapter this week is \textbf{Section 6.1: Storage Technologies}.

\begin{notes}{Section 6.1: Storage Technologies}
    \subsubsection*{Storage Technologies}

    Storage technologies are essential components of the memory hierarchy, each offering a unique balance of capacity, speed, cost, and volatility. They range from fast, expensive, and volatile memory 
    near the top of the hierarchy, such as registers and cache, to slower, cheaper, and non-volatile storage at the bottom, like hard disk drives (HDDs) and solid-state drives (SSDs).
    
    \vspace*{1em}
    
    \subsubsection*{Key Points of Storage Technologies}
    
    \begin{itemize}
        \item \textbf{Registers}: The fastest type of memory, used directly by the CPU to store the data it is currently processing. However, they have very limited capacity.
        \item \textbf{Cache Memory}: Slightly slower than registers but still very fast. Cache memory stores copies of frequently accessed data to speed up access times. It comes in multiple levels 
        (L1, L2, L3) with increasing size and decreasing speed.
        \item \textbf{Main Memory (RAM)}: Volatile memory used for storing data and programs that are in active use. It offers a good balance between speed and capacity but loses data when power is off.
        \item \textbf{Solid-State Drives (SSD)}: Fast, non-volatile storage that uses flash memory. SSDs have no moving parts, which makes them faster and more reliable than HDDs but more expensive 
        per gigabyte.
        \item \textbf{Hard Disk Drives (HDD)}: Traditional storage devices with mechanical parts. They offer large storage capacities at a lower cost than SSDs but are slower and more prone to physical 
        damage.
        \item \textbf{Optical Drives and Media}: Including CDs, DVDs, and Blu-ray discs, these are used for distributing and storing large amounts of data. They are relatively slow and have been largely 
        superseded by flash storage for many applications.
    \end{itemize}
    
    \begin{highlight}[Example of Storage Technologies Usage]
        \subsubsection*{Comparing SSD and HDD in Personal Computers}
    
        In a personal computing context, choosing between an SSD and an HDD can significantly affect the system's performance:
    
    \begin{code}[Comparison]
    - Boot Time: 
        SSD: Seconds
        HDD: Minutes
    
    - Data Transfer Speed:
        SSD: Up to 550 MB/s
        HDD: 80-160 MB/s
    
    - Noise and Vibration:
        SSD: None
        HDD: Noticeable
    \end{code}
        SSDs provide faster boot times, quicker data access, and are silent in operation, making them preferable for performance-sensitive applications. However, HDDs may still be used for bulk storage 
        due to their cost-effectiveness.
    \end{highlight}
    
    \subsubsection*{Considerations}
    
    \begin{itemize}
        \item The choice of storage technology greatly influences the overall system performance and cost. High-performance applications benefit from faster storage technologies like SSDs and cache memory.
        \item The evolution of storage technologies continues to address the trade-offs between speed, size, cost, and durability. Future advancements may further blur the lines between these categories.
        \item Environmental conditions, such as temperature and humidity, can affect the longevity and reliability of storage devices, particularly mechanical HDDs.
    \end{itemize}
\end{notes}

The next section that is being covered from this chapter this week is \textbf{Section 6.2: Locality}.

\begin{notes}{Section 6.2: Locality}
    \subsubsection*{Locality}

    Locality in computer systems refers to the pattern of memory access by a processor, where certain locations or groups of locations are accessed more frequently over a period of time. This concept 
    is pivotal in optimizing computer architecture and software design, as it leverages the predictable patterns of memory access to enhance overall system performance. Locality is primarily categorized 
    into two types: temporal locality and spatial locality. \vspace*{1em}
    
    \subsubsection*{Temporal Locality}
    
    Temporal locality occurs when a program accesses the same memory location multiple times within a relatively short timeframe. This pattern suggests that once a piece of data is accessed, it is 
    likely to be accessed again soon. Temporal locality is exploited by caching mechanisms, where recently accessed data is stored in a cache close to the processor. Since accessing data from the 
    cache is significantly faster than accessing it from main memory, programs that exhibit strong temporal locality can achieve substantial performance improvements.
    
    \subsubsection*{Spatial Locality}
    
    Spatial locality refers to the pattern where if a memory location is accessed, locations physically near it are likely to be accessed soon after. This tendency is based on the way programs are 
    typically structured and how data is often grouped and accessed in blocks or sequences. Exploiting spatial locality involves prefetching blocks of data into the cache before they are actually 
    needed, thus reducing the number of slow main memory accesses. This is effective because modern processors can fetch and cache multiple adjacent memory locations with a single memory access 
    operation, improving the efficiency of data retrieval. \vspace*{1em}
    
    \subsubsection*{Key Points of Locality}
    
    \begin{itemize}
        \item \textbf{Temporal Locality}: Important for the design of cache policies that determine which data to store and replace. It ensures that data which is likely to be reused in the near 
        future remains accessible at high speeds.
        \item \textbf{Spatial Locality}: Influences the organization of memory and the prefetching strategies of processors. By anticipating the need for nearby data, systems can preload this 
        information to avoid delays associated with accessing main memory.
        \item \textbf{Impact on System Design}: Both types of locality have profound implications for the architecture of computer systems, influencing the design of memory hierarchies, caching 
        algorithms, and even the physical layout of memory.
        \item \textbf{Programming for Locality}: Programmers can significantly impact the performance of their applications by structuring data and code in ways that enhance locality. This includes 
        practices like looping over data in contiguous blocks and organizing data structures to be locality-friendly.
    \end{itemize}
    
    \begin{highlight}[Example of Enhancing Locality]
        \subsubsection*{Improving Spatial Locality in Matrix Multiplication}
    
        Consider the algorithmic task of multiplying two matrices. The traditional approach may not optimally exploit spatial locality, especially if one matrix is accessed row-wise and the other 
        column-wise. This mismatch can lead to inefficient use of the cache, as the data needed for subsequent operations may not be physically adjacent in memory.
    
    \begin{code}[Optimization]
    - To enhance spatial locality, one can adopt a blocking technique, where the matrices are divided into smaller submatrices or blocks. This reorganization allows for the submatrices to be loaded into cache, ensuring that the subsequent accesses to these blocks are cache hits, significantly improving performance.
    \end{code}
        This optimized approach not only reduces the number of cache misses but also aligns with the principles of spatial locality by ensuring that data accessed in succession is located close 
        together in memory.
    \end{highlight}
    
    \subsubsection*{Considerations}
    
    \begin{itemize}
        \item The exploitation of locality is a key strategy in the design of efficient computing systems, from low-level hardware architecture to high-level software development.
        \item The balance between temporal and spatial locality optimizations must be considered, as focusing too much on one can sometimes diminish the other.
        \item Advanced compiler optimizations and programming techniques can further exploit locality, but understanding the underlying hardware's memory hierarchy is essential for maximizing these benefits.
    \end{itemize}
\end{notes}

The next section that is being covered from this chapter this week is \textbf{Section 6.3: The Memory Hierarchy}.

\begin{notes}{Section 6.3: The Memory Hierarchy}
    \subsubsection*{The Memory Hierarchy}

    The Memory Hierarchy is a structured framework of computer memory that consists of multiple layers, each with varying speeds, sizes, and costs. This hierarchy is designed to bridge the gap between 
    the fastest and most costly types of memory (like CPU registers) and the slowest but most affordable ones (like hard disk drives). The core idea behind the memory hierarchy is to provide the illusion 
    of a large, fast, and cheap memory system by strategically using different types of memory storage at different levels. \vspace*{1em}
    
    \subsubsection*{Levels of the Memory Hierarchy}
    
    From fastest and most expensive to slowest and least expensive, the typical levels of the memory hierarchy include:
    
    \begin{itemize}
        \item \textbf{Registers}: Located inside the CPU, registers are the fastest form of memory for data storage and retrieval but have very limited storage capacity.
        \item \textbf{Cache Memory}: A small-sized but fast memory located close to the CPU cores. It is used to store copies of frequently accessed data from main memory, significantly reducing data 
        access times. Cache memory is usually divided into three levels (L1, L2, and L3), with L1 being the fastest.
        \item \textbf{Main Memory (RAM)}: A larger pool of memory that stores data and programs that are currently in use. It is slower and cheaper per bit than cache memory and registers but offers 
        much greater capacity.
        \item \textbf{Secondary Storage}: Non-volatile storage devices like hard disk drives (HDDs), solid-state drives (SSDs), and optical media. These storage solutions provide substantial storage 
        capacity at a lower cost but have much slower access times compared to RAM and cache.
        \item \textbf{Tertiary Storage and Off-line Storage}: Includes storage solutions like tape drives, used for backup and archival purposes. These storage media offer the lowest cost per bit and 
        are used for data that is rarely accessed.
    \end{itemize}
    
    \vspace*{1em}
    
    \subsubsection*{Key Principles of the Memory Hierarchy}
    
    \begin{itemize}
        \item \textbf{Locality of Reference}: Programs tend to access a relatively small portion of their address space at any given time. The memory hierarchy exploits this by keeping active data 
        and instructions close to the processor.
        \item \textbf{Cost-Performance Trade-off}: By combining various types of storage media, the memory hierarchy aims to achieve a balance between cost, size, and speed, optimizing overall system performance.
        \item \textbf{Automatic Management}: Modern computer systems automatically manage data movement between the levels of the memory hierarchy. For example, data is promoted to higher levels 
        (closer to the CPU) as it is accessed more frequently.
    \end{itemize}
    
    \begin{highlight}[Example of Memory Hierarchy Utilization]
        \subsubsection*{Accessing Data in a Computing Task}
    
        When a processor needs to access data, it first checks the fastest level of memory, the registers. If the data is not found there (a condition known as a cache miss), it proceeds to check the 
        L1 cache, then L2, and so on down the hierarchy until the data is found.
    
    \begin{code}[Memory Access Pattern]
    - If data is found in an upper level, it is used immediately (cache hit).
    - If data must be retrieved from a lower level (e.g., main memory or secondary storage), it incurs a significant time penalty but is then stored in upper levels to speed up future accesses.
    \end{code}
        This process exemplifies how the memory hierarchy efficiently manages data access, reducing average access time by leveraging faster memory levels for more frequently accessed data.
    \end{highlight}
    
    \subsubsection*{Considerations}
    
    \begin{itemize}
        \item The efficiency of the memory hierarchy is crucial for overall system performance. Optimizations at each level can lead to significant performance gains.
        \item The specific design and implementation of the memory hierarchy can vary significantly between different computing systems, influenced by their particular performance requirements and 
        application domains.
        \item Software developers can influence the efficiency of memory usage through programming techniques that take advantage of the memory hierarchy, such as optimizing algorithms for cache efficiency.
    \end{itemize}
\end{notes}

The next section that is being covered from this chapter this week is \textbf{Section 6.4: Cache Memories}.

\begin{notes}{Section 6.4: Cache Memories}
    \subsubsection*{Cache Memories}

    Cache memories are a critical component of the memory hierarchy, designed to bridge the significant speed gap between the central processing unit (CPU) and main memory (RAM). By storing frequently 
    accessed data and instructions close to the CPU, caches reduce the average time to access memory, thereby enhancing overall system performance. Cache memories are smaller and faster than RAM but 
    offer limited storage capacity due to their higher cost per bit. \vspace*{1em}
    
    \subsubsection*{Structure and Types of Cache}
    
    Cache memories are typically structured in multiple levels (L1, L2, and L3):
    
    \begin{itemize}
        \item \textbf{L1 Cache}: The fastest and smallest cache directly integrated into the processor's core. It is split into separate sections for data and instructions (d-cache and i-cache) to 
        optimize access.
        \item \textbf{L2 Cache}: Larger and slightly slower than L1, L2 cache serves as a bridge between the ultra-fast L1 cache and the larger but slower L3 cache. It can be exclusive or shared among 
        cores, depending on the processor design.
        \item \textbf{L3 Cache}: The largest but slowest level of cache, often shared across all cores in a processor. L3 cache acts as a last resort before the CPU has to access the significantly 
        slower main memory.
    \end{itemize}
    
    \vspace*{1em}
    
    \subsubsection*{Cache Operation}
    
    Caches operate on the principles of locality of reference, utilizing both spatial and temporal locality to predict and pre-load data that the CPU is likely to need soon. The basic operations of 
    cache memory include:
    
    \begin{itemize}
        \item \textbf{Reading Data}: When the CPU requests data, the cache is the first place checked. If the data is present (a cache hit), it is delivered quickly to the CPU. If not (a cache miss), 
        the data is fetched from a lower level of the memory hierarchy.
        \item \textbf{Writing Data}: Writing policies, such as write-through (data is written to both the cache and the main memory) and write-back (data is only written to the cache and later synchronized 
        with the main memory), affect performance and data coherence.
        \item \textbf{Cache Coherence}: In multi-core systems, ensuring that all caches have the most recent data is crucial. Protocols like MESI (Modified, Exclusive, Shared, Invalid) help maintain 
        consistency across caches.
    \end{itemize}
    
    \vspace*{1em}
    
    \subsubsection*{Cache Performance Optimization}
    
    Optimizing cache performance involves several strategies, including:
    
    \begin{itemize}
        \item \textbf{Cache Size}: Larger caches can store more data but have diminishing returns due to increased cost and complexity.
        \item \textbf{Associativity}: The method by which cache locations are selected for storing data. Higher associativity reduces cache misses but increases complexity and access time.
        \item \textbf{Block Size}: The unit of data exchange between cache and main memory. Optimal block size depends on the access pattern of applications.
        \item \textbf{Replacement Policies}: Algorithms to decide which cache entries to replace, like Least Recently Used (LRU), aim to minimize cache misses by keeping the most relevant data in cache.
    \end{itemize}
    
    \begin{highlight}[Example of Cache Utilization]
        \subsubsection*{Web Browsing Cache Example}
    
        In web browsing, the browser cache stores copies of recently accessed web pages. When a user revisits a page, the browser first checks the cache:
    
    \begin{code}[Web Browsing]
    - If the page is in the cache (hit), it loads almost instantly.
    - If the page is not in the cache (miss), it is fetched from the internet, which is slower.
    \end{code}
        This caching mechanism significantly speeds up web browsing by reducing load times for frequently visited pages.
    \end{highlight}
    
    \subsubsection*{Considerations}
    
    \begin{itemize}
        \item The effectiveness of a cache memory system is highly dependent on the specific workloads and access patterns of the applications running on the system.
        \item Cache memories are a critical part of system design for both hardware architects and software developers, who must understand and optimize for cache behavior to achieve high performance.
        \item Emerging technologies and approaches to cache design continue to evolve, aiming to address the challenges of increasing processor speeds and the need for more efficient memory access patterns.
    \end{itemize}    
\end{notes}

The next section that is being covered from this chapter this week is \textbf{Section 6.5: Writing Cache-Friendly Code}.

\begin{notes}{Section 6.5: Writing Cache-Friendly Code}
    \subsubsection*{Writing Cache-Friendly Code}

    Writing cache-friendly code involves understanding and leveraging the memory hierarchy, specifically the cache, to improve software performance. By aligning code and data structures with the principles 
    of cache operation, programmers can significantly reduce cache misses, leading to faster execution times. Key strategies include optimizing for spatial and temporal locality, minimizing cache line 
    evictions, and organizing data access patterns to align with cache design. \vspace*{1em}
    
    \subsubsection*{Key Strategies for Cache Optimization}
    
    \begin{itemize}
        \item \textbf{Maximize Temporal Locality}: Access data and instructions as close together in time as possible if they will be reused, to keep them in the cache.
        \item \textbf{Enhance Spatial Locality}: Organize data so that elements accessed closely together in time are also stored closely together in memory.
        \item \textbf{Loop Interchange}: Adjust the order of nested loops to access data in memory sequentially, reducing cache line misses.
        \item \textbf{Block or Tile Loops}: Break down large data sets into smaller chunks that fit into the cache, processing each chunk at a time to prevent cache evictions.
        \item \textbf{Avoid Unnecessary Data Structure Padding}: Align data structures to cache line boundaries when possible but avoid excessive padding that can waste cache space and lead to evictions.
    \end{itemize}
    
    \begin{highlight}[Example of Cache-Friendly Code]
        \subsubsection*{Matrix Multiplication Optimization}
    
        A common example of optimizing for cache usage is the modification of a simple matrix multiplication algorithm. Consider a naive implementation:
    
    \begin{code}[C]
    void matrix_multiply(int N, double A[N][N], double B[N][N], double C[N][N]) {
        for (int i = 0; i < N; i++) {
            for (int j = 0; j < N; j++) {
                C[i][j] = 0.0;
                for (int k = 0; k < N; k++) {
                    C[i][j] += A[i][k] * B[k][j];
                }
            }
        }
    }
    \end{code}
    
        This naive approach can lead to poor cache utilization due to the access patterns of B[k][j]. By transposing matrix B beforehand and adjusting the loop to multiply A[i][k] with B[j][k], spatial 
        locality is improved:
    
    \begin{code}[C]
    void transpose_matrix(int N, double B[N][N], double BT[N][N]) {
        for (int i = 0; i < N; i++) {
            for (int j = 0; j < N; j++) {
                BT[j][i] = B[i][j];
            }
        }
    }
    
    void optimized_matrix_multiply(int N, double A[N][N], double BT[N][N], double C[N][N]) {
        for (int i = 0; i < N; i++) {
            for (int j = 0; j < N; j++) {
                C[i][j] = 0.0;
                for (int k = 0; k < N; k++) {
                    C[i][j] += A[i][k] * BT[j][k]; // Notice B is transposed
                }
            }
        }
    }
    \end{code}
        Transposing matrix B improves spatial locality because consecutive iterations of the innermost loop access consecutive elements of BT, making better use of cache lines loaded into the cache.
    \end{highlight}
    
    \subsubsection*{Considerations}
    
    \begin{itemize}
        \item Writing cache-friendly code often requires profiling and benchmarking to identify bottlenecks and understand the cache behavior of specific hardware.
        \item The benefits of optimization can vary depending on the architecture, including the size of caches and their associativity.
        \item Balancing readability and maintainability of code with optimization efforts is crucial. Highly optimized code can sometimes be more difficult to understand and maintain.
    \end{itemize}    
\end{notes}

The last section that is being covered from this chapter this week is \textbf{Section 6.5: Putting It Together: The Impact of Caches on Program Performance}.

\begin{notes}{Section 6.5: Putting It Together: The Impact of Caches on Program Performance}
    \subsubsection*{Putting It Together: The Impact of Caches on Program Performance}

    The impact of caches on program performance is profound and multifaceted, encompassing aspects of computer architecture, software design, and programming practices. Caches serve as a critical 
    intermediary between the fast CPU and the slower main memory, significantly reducing the average memory access time by exploiting the principles of locality. Understanding and optimizing for cache 
    behavior can lead to substantial performance gains in software applications. \vspace*{1em}
    
    \subsubsection*{Understanding Cache Impact}
    
    The effectiveness of cache memory in enhancing program performance is primarily determined by several key factors:
    
    \begin{itemize}
        \item \textbf{Cache Hit Rate}: The percentage of memory accesses that are successfully served by the cache. A higher cache hit rate means fewer accesses to slower main memory, directly 
        improving program performance.
        \item \textbf{Cache Miss Penalty}: The additional time required to fetch data from main memory or lower levels of the cache hierarchy on a cache miss. Reducing the miss penalty or avoiding 
        misses altogether is crucial for maintaining high performance.
        \item \textbf{Data and Instruction Locality}: Programs exhibiting high spatial and temporal locality can make more efficient use of cache, as they access memory in patterns well-suited to 
        caching mechanisms.
        \item \textbf{Cache-Friendly Algorithms and Data Structures}: Choosing or designing algorithms and data structures that align with cache operation principles can significantly impact performance.
    \end{itemize}
    
    \vspace*{1em}
    
    \subsubsection*{Case Study: Optimizing Cache Usage}
    
    To illustrate the impact of caches on program performance, consider a software application that processes large data sets:
    
    \begin{highlight}[Performance Improvement Through Cache Optimization]
        Initial observations show the application spends a significant portion of its execution time on memory accesses. Profiling indicates a low cache hit rate, suggesting poor cache utilization.
    
        \textbf{Optimization Steps:}
        \begin{enumerate}
            \item Analyze the application's memory access patterns to identify bottlenecks.
            \item Reorganize data structures to improve spatial locality, ensuring related data is stored contiguously in memory.
            \item Refactor critical loops and algorithms to enhance temporal locality, making frequent accesses to the same data.
            \item Implement loop tiling or blocking to process data in chunks that fit within the cache, minimizing cache line evictions.
        \end{enumerate}
    
        After optimization, the application demonstrates a significantly higher cache hit rate and reduced memory access times, leading to an overall performance boost.
    \end{highlight}
    
    \subsubsection*{Considerations}
    
    \begin{itemize}
        \item The impact of cache optimization varies across different hardware architectures, requiring targeted profiling and testing.
        \item While optimizing for cache performance, it's important to maintain code clarity and maintainability, balancing optimization efforts with software engineering best practices.
        \item Continuous performance monitoring and analysis are essential, as changes in software behavior or data usage patterns can alter cache efficiency.
    \end{itemize}    
\end{notes}