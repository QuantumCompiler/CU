\clearpage

\renewcommand{\ChapTitle}{Logistic Regression}
\renewcommand{\SectionTitle}{Logistic Regression}

\chapter{\ChapTitle}
\section{\SectionTitle}
\horizontalline{0}{0}

\subsection{Assigned Reading}

The reading for this week comes from \ISLRPython, \ISLRR, and \ESLII \hspace*{1pt} and is:

\begin{itemize}
    \item \textbf{ISLR Chapter 4.1: An Overview Of Classification}
    \item \textbf{ISLR Chapter 4.2: Why Not Linear Regression?}
    \item \textbf{ISLR Chapter 4.3: Logistic Regression}
    \item \href{https://en.wikipedia.org/wiki/Confusion_matrix}{Confusion Matrix}
    \item \href{https://scikit-learn.org/stable/modules/generated/sklearn.linear_model.LogisticRegression.html}{Logistic Regression}
\end{itemize}

\subsection{Piazza}

Must post \textbf{one} dataset that aligns with weekly material.

\subsection{Lectures}

The lectures for this week are:

\begin{itemize}
    \item \lecture{https://www.youtube.com/watch?v=Gr_y-C3U_O8}{Logistic Regression Intro - Examples, Logistic Function And Softmax Function}{15}
    \item \lecture{https://www.youtube.com/watch?v=_qQsvyDZh3E}{Logistic Regression Optimization}{20}
    \item \lecture{https://www.youtube.com/watch?v=fdJ6DcrZYu4}{Performance Metrics In Classification}{14}
    \item \lecture{https://www.youtube.com/watch?v=Uj3V1G2YYfA}{SKlearn Library Usage And Examples}{15}
\end{itemize}

\noindent The lecture notes for this week are:

\begin{itemize}
    \item \pdflink{\LecNoteDir Logistic Regression Introduction Lecture Notes.pdf}{Logistic Regression Introduction Lecture Notes}
\end{itemize}

\subsection{Assignments}

The assignment(s) for the week is:

\begin{itemize}
    \item \textbf{Assignment 2 - Logistic Regression}
\end{itemize}

\subsection{Quiz}

The quiz for this week is:

\begin{itemize}
    \item \href{https://applied.cs.colorado.edu/mod/quiz/view.php?id=67076}{Quiz 2 - Logistic Regression}
\end{itemize}

\newpage

\subsection{Chapter Summary}

The first section that is being covered from the chapter this week is \textbf{Section 4.1: An Overview Of Classification}. 

\begin{notes}{Section 4.1: An Overview Of Classification}
    \subsection*{Overview}

    This section introduces classification, a predictive modeling technique used when the response variable is qualitative (categorical). Unlike linear regression, which predicts a continuous outcome, 
    classification assigns an observation to a discrete category or class. Classification methods predict the probability of an observation belonging to each class, and the observation is then assigned 
    to the class with the highest probability.
    
    \subsubsection*{Examples of Classification Problems}
    
    Classification problems arise in many real-world situations, such as:
    \begin{itemize}
        \item Diagnosing a medical condition based on a patient’s symptoms.
        \item Detecting fraudulent transactions in online banking.
        \item Identifying deleterious DNA mutations based on genetic data.
    \end{itemize}
    
    \begin{highlight}[Key Concepts in Classification]
        \begin{itemize}
            \item \textbf{Qualitative Response}: The response variable represents categories or classes (e.g., default or no default).
            \item \textbf{Classifier}: A method or algorithm used to assign a class label to an observation.
            \item \textbf{Training Data}: Used to build a classification model, which generalizes to unseen test data.
        \end{itemize}
    \end{highlight}
    
    \subsubsection*{Common Classification Methods}
    
    Several widely used classifiers include:
    \begin{itemize}
        \item \textbf{Logistic Regression}: Estimates the probability of a binary outcome.
        \item \textbf{Linear Discriminant Analysis (LDA)}: A method that finds a linear combination of predictors that best separates two or more classes.
        \item \textbf{Quadratic Discriminant Analysis (QDA)}: Similar to LDA but allows for non-linear boundaries.
        \item \textbf{Naive Bayes}: Assumes independence between predictors to compute class probabilities.
        \item \textbf{K-Nearest Neighbors (KNN)}: Classifies an observation based on the majority class of its nearest neighbors.
    \end{itemize}
    
    \subsubsection*{Example: Default Data Set}
    
    The Default data set demonstrates how classification can be applied to predict whether a person will default on a credit card payment. The predictors are annual income and monthly credit card balance. 
    The response is binary (default or no default).
    
    \begin{highlight}[Key Takeaways from Default Data Set]
        \begin{itemize}
            \item \textbf{Predictors}: Income and balance are used to predict default.
            \item \textbf{Visualizations}: Boxplots show the distribution of balance and income for defaulters and non-defaulters, indicating a relationship between balance and default status.
        \end{itemize}
    \end{highlight}
    
    \subsubsection*{Why Not Use Linear Regression for Classification?}
    
    Linear regression is unsuitable for qualitative responses for several reasons:
    \begin{itemize}
        \item \textbf{Unbounded Predictions}: Linear regression can predict values outside the [0, 1] range, which are not valid probabilities.
        \item \textbf{Non-linear Boundaries}: Linear regression assumes a linear relationship, which may not effectively separate classes.
        \item \textbf{Interpretation Issues}: Assigning probabilities or classifications based on continuous outputs from a regression model is conceptually flawed for qualitative data.
    \end{itemize}
    
    \begin{highlight}[Reasons to Avoid Linear Regression for Classification]
        \begin{itemize}
            \item Linear models are not designed to handle categorical responses and can lead to invalid predictions.
            \item Classification methods provide better accuracy and interpretability for qualitative outcomes.
        \end{itemize}
    \end{highlight}
\end{notes}

The next section that is being covered from the chapter this week is \textbf{Section 4.2: Why Not Linear Regression?}.

\begin{notes}{Section 4.2: Why Not Linear Regression?}
    \subsection*{Overview}

    This section discusses why linear regression is not suitable for classification problems, particularly when the response variable is qualitative (categorical). While linear regression can be applied 
    to binary classification, it has limitations that make it inappropriate for classification tasks with more than two classes or where probabilities are required.
    
    \subsubsection*{Why Linear Regression Fails for Classification}
    
    Linear regression assumes that the response variable is quantitative and continuous. For classification tasks, where the response variable is categorical, linear regression presents several problems:
    
    \begin{highlight}[Key Limitations of Linear Regression for Classification]
        \begin{itemize}
            \item \textbf{Arbitrary Ordering for Multi-Class Problems}: Encoding categorical classes with numbers (e.g., 1 for stroke, 2 for drug overdose, 3 for seizure) imposes an artificial order, 
            which may not exist in reality.
            \item \textbf{No Natural Way to Handle Multiple Classes}: There is no general method to map qualitative responses with more than two levels to numerical values suitable for linear regression.
            \item \textbf{Non-Probabilistic Predictions}: Linear regression can produce predicted values outside the range of valid probabilities (0 and 1), making it hard to interpret these predictions as probabilities.
        \end{itemize}
    \end{highlight}
    
    \subsubsection*{Binary Classification and Linear Regression}
    
    For binary classification, linear regression can still be applied by coding the response variable as 0 or 1 (e.g., 0 for stroke, 1 for drug overdose). The least squares solution estimates the 
    probability of a given class as $P(Y = 1 | X)$. However, even in this case, linear regression has notable limitations:
    \begin{itemize}
        \item Predicted probabilities can exceed the valid range of [0, 1].
        \item It assumes a linear relationship between the predictors and the response, which may not hold in practice.
    \end{itemize}
    
    \begin{highlight}[Issues with Binary Classification Using Linear Regression]
        \begin{itemize}
            \item \textbf{Probability Estimates}: While the method produces estimates of class probabilities, the estimates can be unreliable, with values outside the [0, 1] range.
            \item \textbf{Non-Linear Boundaries}: Linear regression does not model non-linear decision boundaries, which are often needed in classification tasks.
        \end{itemize}
    \end{highlight}
    
    \subsubsection*{Conclusion}
    
    There are two main reasons not to use linear regression for classification:
    \begin{itemize}
        \item Linear regression cannot handle multi-class qualitative responses effectively.
        \item It does not provide meaningful or valid probability estimates, even for binary responses.
    \end{itemize}
    
    \begin{highlight}[Summary of Key Points]
        \begin{itemize}
            \item Linear regression is not well-suited for categorical response variables.
            \item For classification, it is better to use methods specifically designed for qualitative responses, such as logistic regression or linear discriminant analysis.
        \end{itemize}
    \end{highlight}
\end{notes}

The last section that is being covered from the chapter this week is \textbf{Section 4.3: Logistic Regression}.

\begin{notes}{Section 4.3: Logistic Regression}
    \subsection*{Overview}

    This section introduces logistic regression, a classification method used for predicting a binary outcome. Unlike linear regression, which predicts a continuous response, logistic regression models 
    the probability that a given observation belongs to a particular class. The method is widely used when the response variable is qualitative, and it is particularly suited for binary classification problems.
    
    \subsubsection*{The Logistic Model}
    
    Logistic regression models the probability that the response variable $Y$ belongs to a class (e.g., default = Yes) based on the predictor variables $X$. The probability is modeled using the logistic function:
    \[
    p(X) = \frac{e^{\beta_0 + \beta_1 X}}{1 + e^{\beta_0 + \beta_1 X}},
    \]
    where $p(X)$ is the probability that $Y = 1$. The logistic function ensures that the predicted probabilities fall between 0 and 1.
    
    \begin{highlight}[Key Features of the Logistic Model]
        \begin{itemize}
            \item \textbf{Log Odds}: The log odds of the probability are modeled as a linear function of the predictors:
            \[
            \log\left(\frac{p(X)}{1 - p(X)}\right) = \beta_0 + \beta_1 X.
            \]
            \item \textbf{Non-Linearity}: Unlike linear regression, logistic regression captures non-linear relationships between the predictors and the response.
            \item \textbf{Odds and Probability}: The odds $p(X)/(1 - p(X))$ can range from 0 to infinity, while the probability remains bounded between 0 and 1.
        \end{itemize}
    \end{highlight}
    
    \subsubsection*{Coefficient Estimation}
    
    The logistic regression coefficients $\beta_0$ and $\beta_1$ are estimated using the method of maximum likelihood. This approach finds the parameter values that maximize the likelihood of the 
    observed data. The likelihood function for logistic regression is given by:
    \[
    \ell(\beta_0, \beta_1) = \prod_{i:y_i=1} p(x_i) \prod_{i':y_{i'}=0} (1 - p(x_{i'})).
    \]
    The coefficients are chosen to maximize this likelihood.
    
    \begin{highlight}[Coefficient Estimation]
        \begin{itemize}
            \item \textbf{Maximum Likelihood Estimation (MLE)}: Used to estimate the parameters that make the observed outcomes most probable.
            \item \textbf{Standard Errors and $z$-statistics}: These are computed to assess the significance of the coefficients, similar to $t$-statistics in linear regression.
        \end{itemize}
    \end{highlight}
    
    \subsubsection*{Making Predictions}
    
    Once the coefficients are estimated, logistic regression can be used to compute the probability of an outcome for any new observation. For example, given a balance of \$1,000 in the Default data 
    set, the predicted probability of default is:
    \[
    p(\text{balance} = 1000) = \frac{e^{\beta_0 + \beta_1 \times 1000}}{1 + e^{\beta_0 + \beta_1 \times 1000}}.
    \]
    Logistic regression is particularly effective for binary outcomes, but it can also handle qualitative predictors by coding them as dummy variables.
    
    \subsubsection*{Multiple Logistic Regression}
    
    Multiple logistic regression extends the basic model to include multiple predictors. The generalized model is:
    \[
    \log\left(\frac{p(X)}{1 - p(X)}\right) = \beta_0 + \beta_1 X_1 + \cdots + \beta_p X_p,
    \]
    where $X_1, \dots, X_p$ are the predictor variables. The coefficients can be interpreted similarly to those in simple logistic regression, where each $\beta_j$ represents the change in the log 
    odds of the outcome for a one-unit change in $X_j$, holding all other variables constant.
    
    \begin{highlight}[Multiple Logistic Regression]
        \begin{itemize}
            \item Allows for the inclusion of multiple predictors to improve the accuracy of the model.
            \item The interpretation of coefficients remains focused on the log odds, with each $\beta_j$ indicating the effect of the corresponding predictor on the likelihood of the outcome.
        \end{itemize}
    \end{highlight}
\end{notes}