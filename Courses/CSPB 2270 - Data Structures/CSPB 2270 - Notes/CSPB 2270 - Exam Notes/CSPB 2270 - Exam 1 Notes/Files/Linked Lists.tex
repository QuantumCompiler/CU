\section*{Linked Lists}

\subsection*{Overview}

In C++, a linked list is a data structure consisting of a sequence of nodes, where each node contains a value and a pointer/reference to the next node. Unlike arrays and vectors, linked lists do not 
store elements in contiguous memory locations. Instead, each node in a linked list is dynamically allocated and connected through pointers/references. This allows for efficient insertion and removal 
of elements at any position in the list, but it requires traversal from the beginning to access specific elements. Linked lists have a flexible size and can expand or shrink as needed by allocating 
or deallocating nodes, unlike arrays whose size is fixed. However, accessing elements in a linked list requires iterating through the nodes, while arrays and vectors provide direct access using indices.

