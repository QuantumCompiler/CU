\clearpage

\renewcommand{\ChapTitle}{Types And Type Checking}
\renewcommand{\SectionTitle}{Types And Type Checking}

\chapter{\ChapTitle}

\section{\SectionTitle}
\horizontalline{0}{0}

\subsection{Assigned Reading}

The reading for this week is from, \AtomicScalaBook, \EssOfPPLBook, \FuncOfPPLBook, and \ProgInScalaBook:

\begin{itemize}
    \item \textbf{N/A}
\end{itemize}

\subsection{Lectures}

The lectures for this week are:

\begin{itemize}
    \item \lecture{https://www.youtube.com/watch?v=Z6VjnruK9fI}{Type Checking: Introduction}{18}
    \item \lecture{https://www.youtube.com/watch?v=rbtVO70Qjeo}{Type Checking In Lettuce}{33}
    \item \lecture{https://www.youtube.com/watch?v=g21xY-lC-gE}{Type Inference Recitation}{52}
\end{itemize}

\noindent The lecture notes for this week are:

\begin{itemize}
    \item Jupyter Notebooks:
    \begin{itemize}
        \item \textbf{Types And Type Checking - Type Inference Lettuce Lecture Notes}
        \item \textbf{Types And Type Checking - Types And Type Checking In Lettuce Lecture Notes}
        \item \textbf{Types And Type Checking Recitation}
        \item \textbf{Types And Type Checking Recitation Solutions}
    \end{itemize}
\end{itemize}

\subsection{Assignments}

The assignment(s) for this week are:

\begin{itemize}
    \item \textbf{Problem Set 8 - Types And Type Checking}
\end{itemize}

\subsection{Quiz}

The quiz for this week is:

\begin{itemize}
    \item \pdflink{\QuizDir Quiz 8 - Types And Type Checking.pdf}{Quiz 8 - Types And Type Checking}
\end{itemize}

\subsection{Chapter Summary}

The topic that is being covered this week is \textbf{Types And Type Checking}. The first topic that is being covered this week is \textbf{Type Checking: Introduction}.

\begin{notes}{Type Checking: Introduction}
    \subsection*{Overview}

    Type checking is a crucial process in programming languages that ensures code correctness by verifying that operations are performed on compatible types. It prevents type errors during execution, 
    enhancing program reliability and safety.
    
    \subsubsection*{What is Type Checking?}
    
    Type checking involves examining the types of expressions and statements in a program to ensure they adhere to the language's type rules. It can be performed statically at compile-time or dynamically at runtime.
    
    \begin{highlight}[What is Type Checking?]
    
        This example provides a basic understanding of type checking.
    
    \begin{code}[Scala]
    // Statically typed example
    val x: Int = 10
    val y: String = "Hello"
    // val z: Int = "World"  // This would cause a compile-time type error
    
    // Dynamically typed example
    val a = 10
    val b = "Hello"
    // val c = a + b  // This would cause a runtime type error
    \end{code}
    
        In this example, the static type checker ensures that "x" is an "Int" and "y" is a "String". A compile-time error occurs if an "Int" is assigned a "String". In dynamically typed languages, type 
        errors are detected at runtime.
    
        \begin{itemize}
            \item Type checking verifies that operations are performed on compatible types.
            \item Prevents type errors during execution.
            \item Can be performed statically at compile-time or dynamically at runtime.
        \end{itemize}
    
    \end{highlight}
    
    \subsubsection*{Static vs. Dynamic Type Checking}
    
    Static type checking occurs at compile-time, ensuring type safety before program execution. Dynamic type checking occurs at runtime, verifying types as the program executes.
    
    \begin{highlight}[Static vs. Dynamic Type Checking]
    
        This example compares static and dynamic type checking.
    
    \begin{code}[Scala]
    // Static type checking
    def add(x: Int, y: Int): Int = x + y
    
    // Dynamic type checking (in a dynamically typed language like Python)
    // def add(x, y):
    //     return x + y
    
    // Static type checking error
    // add(1, "two")  // Compile-time error
    
    // Dynamic type checking error
    // add(1, "two")  // Runtime error
    \end{code}
    
        In this example, the static type checker ensures that "add" only accepts "Int" parameters, catching errors at compile-time. In dynamically typed languages, errors are caught at runtime.
    
        \begin{itemize}
            \item \textbf{Static Type Checking}: Ensures type safety at compile-time.
            \item \textbf{Dynamic Type Checking}: Verifies types at runtime.
            \item Static checking catches errors earlier, while dynamic checking provides flexibility.
        \end{itemize}
    
    \end{highlight}
    
    \subsubsection*{Type Systems}
    
    A type system defines the rules for assigning types to expressions and ensuring that operations are performed on compatible types. Type systems can be strong or weak, and static or dynamic.
    
    \begin{highlight}[Type Systems]
    
        This example illustrates a type system in Scala.
    
    \begin{code}[Scala]
    // Strongly typed system
    val x: Int = 5
    // val y: String = x  // Compile-time error
    
    // Weakly typed system (hypothetical)
    // var a: Any = 5
    // a = "Hello"  // No error in a weakly typed system
    \end{code}
    
        In this example, Scala's type system enforces strong typing, preventing an "Int" from being assigned to a "String". In weakly typed systems, type conversions may occur implicitly.
    
        \begin{itemize}
            \item A type system defines rules for assigning types and ensuring compatibility.
            \item \textbf{Strongly Typed Systems}: Enforce strict type rules, preventing implicit type conversions.
            \item \textbf{Weakly Typed Systems}: Allow more flexibility with implicit type conversions.
        \end{itemize}
    
    \end{highlight}
    
    \subsubsection*{Benefits of Type Checking}
    
    Type checking provides several benefits, including early error detection, improved code reliability, better documentation, and enhanced performance through optimizations.
    
    \begin{highlight}[Benefits of Type Checking]
    
        The benefits of type checking include:
    
        \begin{itemize}
            \item \textbf{Early Error Detection}: Catches type errors at compile-time or early in execution.
            \item \textbf{Improved Code Reliability}: Ensures operations are performed on compatible types.
            \item \textbf{Better Documentation}: Types serve as documentation, clarifying the expected data.
            \item \textbf{Enhanced Performance}: Enables compiler optimizations based on type information.
        \end{itemize}
    
    \end{highlight}
    
    \subsubsection*{Type Inference}
    
    Type inference allows the compiler to deduce types automatically, reducing the need for explicit type annotations while maintaining type safety.
    
    \begin{highlight}[Type Inference]
    
        This example demonstrates type inference in Scala.
    
    \begin{code}[Scala]
    // Explicit type annotation
    val x: Int = 10
    
    // Type inference
    val y = 20  // Compiler infers that y is of type Int
    \end{code}
    
        In this example, the Scala compiler infers that "y" is of type "Int" based on the assigned value, reducing the need for explicit type annotations.
    
        \begin{itemize}
            \item Type inference deduces types automatically.
            \item Reduces the need for explicit type annotations.
            \item Maintains type safety while simplifying code.
        \end{itemize}
    
    \end{highlight}
    
    \begin{highlight}[Summary of Key Concepts]
    
        Type checking is a fundamental process in programming languages that ensures code correctness by verifying type compatibility. Here are the key concepts covered in this section:
    
        \begin{itemize}
            \item \textbf{What is Type Checking?}: Verifies that operations are performed on compatible types, preventing type errors.
            \item \textbf{Static vs. Dynamic Type Checking}: Static type checking occurs at compile-time, while dynamic type checking occurs at runtime.
            \item \textbf{Type Systems}: Define rules for assigning types and ensuring compatibility, with strong or weak, static or dynamic systems.
            \item \textbf{Benefits of Type Checking}: Early error detection, improved code reliability, better documentation, and enhanced performance.
            \item \textbf{Type Inference}: Allows the compiler to deduce types automatically, reducing the need for explicit type annotations.
        \end{itemize}
    
        These concepts highlight the importance of type checking in ensuring code correctness, safety, and performance.
    
    \end{highlight}
\end{notes}

The last topic that is being covered this week is \textbf{Type Checking In Lettuce}.

\begin{notes}{Type Checking In Lettuce}
    \subsection*{Overview}

    Type checking in Lettuce ensures that operations are performed on compatible types, enhancing the safety and correctness of programs. Lettuce, as a functional programming language, utilizes a type 
    system to enforce type rules and catch errors at compile time.
    
    \subsubsection*{Type System in Lettuce}
    
    Lettuce employs a static type system, meaning types are checked at compile time. This prevents type errors from occurring during program execution.
    
    \begin{highlight}[Type System in Lettuce]
    
        This example illustrates the static type system in Lettuce.
    
    \begin{code}[Lettuce]
    // Valid type assignments
    let x: Int = 10
    let y: String = "Hello"
    
    // Type error
    // let z: Int = "World"  // Compile-time error
    \end{code}
    
        In this example, Lettuce's type system ensures that "x" is an "Int" and "y" is a "String". Assigning a "String" to an "Int" variable "z" results in a compile-time error.
    
        \begin{itemize}
            \item Lettuce uses a static type system.
            \item Types are checked at compile time.
            \item Prevents type errors during program execution.
        \end{itemize}
    
    \end{highlight}
    
    \subsubsection*{Type Annotations}
    
    Type annotations in Lettuce specify the types of variables and function parameters explicitly, aiding in type checking and improving code readability.
    
    \begin{highlight}[Type Annotations]
    
        This example demonstrates type annotations in Lettuce.
    
    \begin{code}[Lettuce]
    // Function with type annotations
    let add: (Int, Int) => Int = fun(x: Int, y: Int) {
        x + y
    }
    
    // Variable with type annotation
    let message: String = "Hello, Lettuce!"
    \end{code}
    
        In this example, the function "add" is annotated with types for its parameters and return type. The variable "message" is explicitly annotated as a "String".
    
        \begin{itemize}
            \item Type annotations specify variable and function parameter types.
            \item Improves type checking and code readability.
            \item Helps the type checker enforce type rules.
        \end{itemize}
    
    \end{highlight}
    
    \subsubsection*{Type Inference in Lettuce}
    
    Lettuce supports type inference, allowing the compiler to deduce types automatically, reducing the need for explicit type annotations while maintaining type safety.
    
    \begin{highlight}[Type Inference in Lettuce]
    
        This example demonstrates type inference in Lettuce.
    
    \begin{code}[Lettuce]
    // Variable with type inference
    let x = 10  // Compiler infers x is of type Int
    
    // Function with type inference
    let add = fun(x, y) {
        x + y
    }  // Compiler infers add: (Int, Int) => Int
    \end{code}
    
        In this example, the compiler infers the type of "x" as "Int" and the function "add" as "(Int, Int) => Int" based on the assigned values.
    
        \begin{itemize}
            \item Type inference deduces types automatically.
            \item Reduces the need for explicit type annotations.
            \item Maintains type safety while simplifying code.
        \end{itemize}
    
    \end{highlight}
    
    \subsubsection*{Type Checking Functions}
    
    Type checking functions in Lettuce involves verifying that function parameters and return types are consistent with their type annotations or inferred types.
    
    \begin{highlight}[Type Checking Functions]
    
        This example demonstrates type checking for functions in Lettuce.
    
    \begin{code}[Lettuce]
    // Function with correct type annotations
    let multiply: (Int, Int) => Int = fun(a: Int, b: Int) {
        a * b
    }
    
    // Function with type error
    // let concatenate: (Int, String) => String = fun(a: Int, b: Int) {
    //     a + b  // Compile-time error: type mismatch
    // }
    \end{code}
    
        In this example, "multiply" is correctly annotated and type-checked. The "concatenate" function has a type error because "a + b" is not a valid operation for "Int" and "String".
    
        \begin{itemize}
            \item Type checking verifies function parameter and return types.
            \item Ensures consistency with type annotations or inferred types.
            \item Catches type errors during function definition.
        \end{itemize}
    
    \end{highlight}
    
    \subsubsection*{Polymorphic Functions}
    
    Lettuce supports polymorphic functions, allowing functions to operate on different types. Type parameters are used to define polymorphic functions.
    
    \begin{highlight}[Polymorphic Functions]
    
        This example demonstrates defining polymorphic functions in Lettuce.
    
    \begin{code}[Lettuce]
    // Polymorphic identity function
    let identity: [T] => (T) => T = fun[T](x: T) {
        x
    }
    
    // Using the polymorphic function
    let intIdentity = identity 
    let stringIdentity = identity[String]("Hello")
    \end{code}
    
        In this example, "identity" is a polymorphic function that works with any type "T". It can be used with different types, such as "Int" and "String".
    
        \begin{itemize}
            \item Polymorphic functions operate on different types.
            \item Type parameters define the polymorphic behavior.
            \item Provides flexibility and reusability in function definitions.
        \end{itemize}
    
    \end{highlight}
    
    \subsubsection*{Type Safety and Error Prevention}
    
    Type checking in Lettuce ensures type safety and prevents errors by verifying that operations are performed on compatible types. This enhances the reliability and correctness of programs.
    
    \begin{highlight}[Type Safety and Error Prevention]
    
        The benefits of type checking in Lettuce include:
    
        \begin{itemize}
            \item \textbf{Type Safety}: Ensures operations are performed on compatible types, preventing type errors.
            \item \textbf{Error Prevention}: Catches type errors at compile-time, improving program reliability.
            \item \textbf{Improved Documentation}: Types serve as documentation, clarifying the expected data.
            \item \textbf{Enhanced Code Quality}: Maintains code consistency and correctness through rigorous type checks.
        \end{itemize}
    
    \end{highlight}
    
    \begin{highlight}[Summary of Key Concepts]
    
        Type checking in Lettuce is crucial for ensuring the correctness and safety of programs. Here are the key concepts covered in this section:
    
        \begin{itemize}
            \item \textbf{Type System in Lettuce}: Utilizes a static type system to check types at compile time.
            \item \textbf{Type Annotations}: Explicitly specify variable and function parameter types for better type checking.
            \item \textbf{Type Inference in Lettuce}: Allows the compiler to deduce types automatically, reducing the need for explicit annotations.
            \item \textbf{Type Checking Functions}: Verifies that function parameters and return types are consistent with type annotations or inferred types.
            \item \textbf{Polymorphic Functions}: Defines functions that can operate on different types using type parameters.
            \item \textbf{Type Safety and Error Prevention}: Ensures operations are performed on compatible types, preventing errors and improving program reliability.
        \end{itemize}
    
        These concepts highlight the importance of type checking in maintaining the safety, correctness, and quality of programs in Lettuce.
    
    \end{highlight}
\end{notes}