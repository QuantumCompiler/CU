\clearpage

\renewcommand{\ChapTitle}{Variance And Subtyping}
\renewcommand{\SectionTitle}{Variance And Subtyping}

\chapter{\ChapTitle}

\section{\SectionTitle}
\horizontalline{0}{0}

\subsection{Assigned Reading}

The reading for this week is from, \AtomicScalaBook, \EssOfPPLBook, \FuncOfPPLBook, and \ProgInScalaBook:

\begin{itemize}
    \item \textbf{N/A}
\end{itemize}

\subsection{Lectures}

The lectures for this week are:

\begin{itemize}
    \item \lecture{https://www.youtube.com/watch?v=_CLcfvteDUY}{Type Hierarchy, Covariance, And Contravariance}{17}
\end{itemize}

\noindent The lecture notes for this week are:

\begin{itemize}
    \item Jupyter Notebooks
    \begin{itemize}
        \item \textbf{Variance And Subtyping - Subtyping And Variance Annotations Lecture Notes}
    \end{itemize}
\end{itemize}

\subsection{Chapter Summary}

The topic that is being covered this week is \textbf{Type Hierarchy, Covariance, And Contravariance}.

\begin{notes}{Type Hierarchy, Covariance, And Contravariance}
    \subsection*{Overview}

    Type hierarchy, covariance, and contravariance are key concepts in Scala that govern the relationships between types and how they interact with each other. Understanding these concepts is essential 
    for designing flexible and type-safe APIs, especially when dealing with collections and generics.
    
    \subsubsection*{Type Hierarchy}
    
    In Scala, the type hierarchy defines the relationships between different types, where more specific types inherit from more general types. This hierarchy starts with the "Any" type at the top, which 
    is the supertype of all types, and progresses down to more specific types like "Int", "String", and custom classes.
    
    \begin{highlight}[Type Hierarchy]
    
        This example illustrates Scala's type hierarchy.
    
    \begin{code}[Scala]
    // Type hierarchy example
    val anyValue: Any = "Hello"
    val anyRefValue: AnyRef = "World"
    val stringValue: String = "Scala"
    
    // Assigning to a more general type
    val anyRef: AnyRef = stringValue  // String is a subtype of AnyRef
    val any: Any = anyRefValue        // AnyRef is a subtype of Any
    \end{code}
    
        In this example, "String" is a subtype of "AnyRef", and "AnyRef" is a subtype of "Any". The type hierarchy allows assigning more specific types to variables of more general types.
    
        \begin{itemize}
            \item The type hierarchy in Scala starts with "Any" at the top, encompassing all types.
            \item "AnyRef" is the base class of all reference types (similar to "Object" in Java).
            \item Specific types like "String" are subtypes of "AnyRef" and "Any".
        \end{itemize}
    
    \end{highlight}
    
    \subsubsection*{Covariance}
    
    Covariance allows a type parameter to be substituted with a subtype. In Scala, this is expressed using the "+" variance annotation. Covariant types are typically used in situations where a generic 
    type can accept a more specific type.
    
    \begin{highlight}[Covariance]
    
        This example demonstrates covariance in Scala.
    
    \begin{code}[Scala]
    // Covariant class definition
    class Box[+A]
    
    // Covariance example
    val intBox: Box[Int] = new Box[Int]
    val anyValBox: Box[AnyVal] = intBox  // Box[Int] is a subtype of Box[AnyVal]
    \end{code}
    
        In this example, "Box[+A]" is a covariant class, allowing "Box[Int]" to be assigned to a variable of type "Box[AnyVal]", since "Int" is a subtype of "AnyVal".
    
        \begin{itemize}
            \item Covariance allows a type parameter to be substituted with a subtype.
            \item Expressed using the "+" annotation, e.g., "Box[+A]".
            \item Enables generic types to accept more specific types, maintaining type safety.
        \end{itemize}
    
    \end{highlight}
    
    \subsubsection*{Contravariance}
    
    Contravariance allows a type parameter to be substituted with a supertype. In Scala, this is expressed using the "-" variance annotation. Contravariant types are useful in scenarios where a more 
    general type can be substituted for a more specific one, such as in function arguments.
    
    \begin{highlight}[Contravariance]
    
        This example demonstrates contravariance in Scala.
    
    \begin{code}[Scala]
    // Contravariant class definition
    class Printer[-A] {
        def print(value: A): Unit = println(value)
    }
    
    // Contravariance example
    val anyPrinter: Printer[Any] = new Printer[Any]
    val stringPrinter: Printer[String] = anyPrinter  // Printer[Any] is a supertype of Printer[String]
    stringPrinter.print("Hello, Scala!")  // Output: Hello, Scala!
    \end{code}
    
        In this example, "Printer[-A]" is a contravariant class, allowing "Printer[Any]" to be assigned to a variable of type "Printer[String]", since "Any" is a supertype of "String".
    
        \begin{itemize}
            \item Contravariance allows a type parameter to be substituted with a supertype.
            \item Expressed using the "-" annotation, e.g., "Printer[-A]".
            \item Useful for defining types that are more general in function arguments or similar contexts.
        \end{itemize}
    
    \end{highlight}
    
    \subsubsection*{Invariance}
    
    Invariance means that a type parameter cannot be substituted with either a subtype or a supertype. In Scala, this is the default behavior when no variance annotation is provided. Invariant types 
    require exact type matching.
    
    \begin{highlight}[Invariance]
    
        This example demonstrates invariance in Scala.
    
    \begin{code}[Scala]
    // Invariant class definition
    class Container[A]
    
    // Invariance example
    val stringContainer: Container[String] = new Container[String]
    // val anyContainer: Container[Any] = stringContainer  // Compile-time error: type mismatch
    \end{code}
    
        In this example, "Container[A]" is invariant, meaning "Container[String]" cannot be assigned to a variable of type "Container[Any]".
    
        \begin{itemize}
            \item Invariance means that a type parameter must match exactly; it cannot be substituted with a subtype or supertype.
            \item No variance annotation is used, making it the default behavior in Scala.
            \item Ensures strict type matching, which can be necessary in certain contexts.
        \end{itemize}
    
    \end{highlight}
    
    \subsubsection*{Applications of Covariance and Contravariance}
    
    Covariance and contravariance are particularly useful in designing type-safe APIs, especially when dealing with collections, function types, and class hierarchies. They allow developers to define 
    how types relate to one another, ensuring that subtyping relationships are preserved where appropriate.
    
    \begin{highlight}[Applications of Covariance and Contravariance]
    
        The applications and benefits of covariance and contravariance include:
    
        \begin{itemize}
            \item \textbf{Covariance}: Commonly used in collections like "List[+A]", where it makes sense to allow a "List[Dog]" to be treated as a "List[Animal]".
            \item \textbf{Contravariance}: Useful in function arguments or event handlers, where a handler for a general type can handle specific subtypes.
            \item \textbf{Invariance}: Ensures strict type matching, important in scenarios where exact type preservation is necessary, such as in mutable collections.
        \end{itemize}
    
    \end{highlight}
    
    \begin{highlight}[Summary of Key Concepts]
    
        Understanding type hierarchy, covariance, and contravariance is essential for designing flexible and type-safe Scala programs. Here are the key concepts covered in this section:
    
        \begin{itemize}
            \item \textbf{Type Hierarchy}: Defines the relationships between types, with "Any" at the top, followed by "AnyRef" and more specific types like "String".
            \item \textbf{Covariance}: Allows a type parameter to be substituted with a subtype, indicated by "+", enabling flexible and type-safe collection handling.
            \item \textbf{Contravariance}: Allows a type parameter to be substituted with a supertype, indicated by "-", useful in function arguments and other contexts.
            \item \textbf{Invariance}: Requires exact type matching, ensuring that a type cannot be substituted with a subtype or supertype.
            \item \textbf{Applications}: Covariance and contravariance are crucial in designing type-safe APIs, collections, and class hierarchies, while invariance is important for maintaining strict 
            type constraints.
        \end{itemize}
    
        These concepts are fundamental to understanding and leveraging Scala's powerful type system, enabling developers to create robust and flexible software.
    
    \end{highlight}
\end{notes}