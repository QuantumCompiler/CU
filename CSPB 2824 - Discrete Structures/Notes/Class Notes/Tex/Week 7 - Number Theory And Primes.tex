\clearpage

\renewcommand{\ChapTitle}{Number Theory And Primes}
\renewcommand{\SectionTitle}{Number Theory And Primes}

\chapter{\ChapTitle}
\section{\SectionTitle}
\horizontalline{0}{0}

\subsection{Assigned Reading}

The reading assignments for this week is from, \Textbook:

\begin{itemize}
    \item \textbf{Chapter 4.1 - Divisibility And Modular Arithmetic}
    \item \textbf{Chapter 4.2 - Integer Representation And Algorithms}
    \item \textbf{Chapter 4.3 - Primes And Greatest Common Divisors}
\end{itemize}

\subsection{Piazza}

Must post / respond to at least \textbf{two} Piazza posts this week.

\subsection{Lectures}

The lectures for this week and their links can be found below:

\begin{itemize}
    \item \href{https://applied.cs.colorado.edu/mod/hvp/view.php?id=51691}{Modular Arithmetic} $\approx$ 5 min.
    \item \href{https://applied.cs.colorado.edu/mod/hvp/view.php?id=51692}{Congruences} $\approx$ 5 min.
    \item \href{https://applied.cs.colorado.edu/mod/hvp/view.php?id=51693}{Modular Operator vs Congruence} $\approx$ 4 min.
    \item \href{https://applied.cs.colorado.edu/mod/hvp/view.php?id=51694}{Arithmetic and Modular Operator} $\approx$ 10 min.
    \item \href{https://applied.cs.colorado.edu/mod/hvp/view.php?id=51695}{Congruences Vs. Equality} $\approx$ 7 min.
    \item \href{https://applied.cs.colorado.edu/mod/hvp/view.php?id=51696}{Congruences Vs. Equality (Differences)} $\approx$ 4 min.
    \item \href{https://applied.cs.colorado.edu/mod/hvp/view.php?id=51697}{Prime Number And Fundamental Theorem Of Arithmetic (Recap)} $\approx$ 13 min.
    \item \href{https://applied.cs.colorado.edu/mod/hvp/view.php?id=51698}{Facts About Primes} $\approx$ 14 min.
    \item \href{https://applied.cs.colorado.edu/mod/hvp/view.php?id=51699}{Tests For Primality} $\approx$ 9 min.
    \item \href{https://applied.cs.colorado.edu/mod/hvp/view.php?id=51700}{GCDs Basic Definition} $\approx$ 8 min.
    \item \href{https://applied.cs.colorado.edu/mod/hvp/view.php?id=51701}{GCDs, LCMs and Prime Factorization} $\approx$ 14 min.
    \item \href{https://applied.cs.colorado.edu/mod/hvp/view.php?id=51702}{Euclid's Algorithm for GCD} $\approx$ 13 min.
    \item \href{https://applied.cs.colorado.edu/mod/hvp/view.php?id=51703}{Euclid's Algorithm [Continued]} $\approx$ 7 min.
    \item \href{https://applied.cs.colorado.edu/mod/hvp/view.php?id=51704}{Euclid's Algorithm With Jugs} $\approx$ 7 min.
    \item \href{https://o365coloradoedu-my.sharepoint.com/personal/stadee_colorado_edu/_layouts/15/stream.aspx?id=%2Fpersonal%2Fstadee%5Fcolorado%5Fedu%2FDocuments%2FG%2FAAAPostbac%2FDiscrete%20Structures%2Fweek6%5F7%20%2D%20number%20theory%5Fmod%2FFMEorSquareandMod%2Emp4&ga=1}{Square And Mod - A Simple Introduction To Fast ModularExponentiation} $\approx$ 10 min.
    \item \href{https://applied.cs.colorado.edu/mod/hvp/view.php?id=51706}{Fast Modular Exponentiation} $\approx$ 15 min.
\end{itemize}

\subsection{Assignments}

The assignment for this week is:

\begin{itemize}
    \item \pdflink{\AssDir Assignment 6 - Number Theory And Primes.pdf}{Assignment 6 - Number Theory And Primes}
\end{itemize}

\subsection{Quiz}

The quiz's for this week can be found at:

\begin{itemize}
    \item \pdflink{\QuizDir Quiz 6 - Number Theory And Primes.pdf}{Quiz 6 - Number Theory And Primes}
\end{itemize}

\newpage

\subsection{Chapter Summary}

The first section that we are covering this week is \textbf{Section 4.1 - Divisibility And Modular Arithmetic}. 

\begin{notes}{Section 4.1 - Divisibility And Modular Arithmetic}
    \subsection*{Overview}

    Divisibility and modular arithmetic are fundamental concepts in number theory, a branch of mathematics that explores the properties and relationships of integers. These concepts lay the foundation 
    for understanding how numbers can be divided and organized into congruence classes, with wide-ranging applications in mathematics and various other fields. Here, we delve into the core principles 
    of divisibility and modular arithmetic and explore their significance in number theory and practical applications.

    \subsection*{Key Concepts}

    \begin{enumerate}
        \item \textbf{Definition of Divisibility:} Divisibility is a fundamental concept in number theory that explores the relationships between integers concerning their ability to be divided by one 
        another without leaving a remainder. In mathematical terms, we say that an integer $a$ is divisible by another integer $b$, denoted as $b | a$, when there exists an integer $q$ such that $a = bq$. 
        This concept is fundamental because it lays the groundwork for understanding how integers are interconnected in terms of multiplication and division. If no such integer $q$ exists, then $b$ is 
        not a divisor of $a$, and we denote this as $b \nmid a$.
    
        \item \textbf{Properties of Divisibility:} Divisibility exhibits several important properties. For instance, if an integer $a$ is divisible by both $b$ and $c$, it is also divisible by their 
        greatest common divisor (GCD), represented as $\gcd(b, c)$. Furthermore, if $a$ is divisible by $b$ and $b$ is divisible by $c$, then $a$ is also divisible by $c$. Additionally, the concept 
        of divisibility plays a key role in understanding the least common multiple (LCM) of integers, denoted as $\mathrm{lcm}(b, c)$.
    
        \item \textbf{Prime Factorization:} Every positive integer can be expressed uniquely as a product of prime numbers, known as its prime factorization. This concept is at the heart of divisibility 
        since it allows us to determine the divisors of a number efficiently. By decomposing a number into its prime factors, we gain valuable insights into its divisibility properties, making prime 
        factorization a crucial tool in number theory.
    
        \item \textbf{Modular Arithmetic:} Modular arithmetic is a specialized field of number theory that deals with the arithmetic of remainders. It introduces the concept of congruence, where two 
        numbers share the same remainder when divided by a fixed integer $m$. Modular arithmetic, denoted as $a \equiv b \pmod{m}$, is an essential aspect of number theory and has widespread applications 
        in various domains.
    
        \item \textbf{Congruence Classes:} In modular arithmetic, numbers are organized into congruence classes, often represented by $\pmod{m}$, where $m$ represents the modulus. Each congruence 
        class contains integers that share the same remainder when divided by $m$. For example, the congruence class $\pmod{5}$ includes numbers like $0, 5, 10, -5, -10,$ and so on. These classes are 
        fundamental in modular arithmetic and help simplify computations.
    
        \item \textbf{Modular Operations:} Modular arithmetic defines operations like addition, subtraction, multiplication, and exponentiation modulo $m$. These operations follow specific rules and 
        properties, making it possible to perform calculations within congruence classes. Modular arithmetic's systematic approach to these operations is essential for various applications, especially 
        in computer science, cryptography, and coding theory.
    
        \item \textbf{Applications:} Modular arithmetic finds applications in a wide range of fields, including computer science, cryptography, coding theory, and number theory. It is used in algorithms 
        and protocols for data encryption, error detection and correction, and generating pseudorandom numbers. Its practical utility extends to computer systems, where it helps manage memory allocation 
        and addressing.
    
        \item \textbf{Fermat's Little Theorem and Euler's Totient Function:} These two key results rely on modular arithmetic. Fermat's Little Theorem states that if $p$ is a prime number and $a$ is an 
        integer not divisible by $p$, then $a^{p-1} \equiv 1 \pmod{p}$. This theorem has significant implications in cryptography and number theory. Euler's Totient function, denoted as $\phi(n)$, is 
        employed to calculate the number of positive integers less than $n$ that are coprime to $n$. It is fundamental in solving problems related to modular arithmetic and number theory.
    \end{enumerate}

    Divisibility and modular arithmetic are foundational concepts in number theory, intertwined in their importance and application. Divisibility forms the basis for understanding the relationships 
    between integers, while modular arithmetic offers a versatile framework for working with remainders and congruences. Together, they provide the foundation for tackling complex mathematical problems 
    and finding solutions in various fields, from cryptography to computer science. These concepts are not only essential for theoretical mathematics but also have practical implications in real-world 
    problem-solving.
\end{notes}

The next section that we are covering this week is \textbf{Section 4.2 - Integer Representations And Algorithms}.

\begin{notes}{Section 4.2 - Integer Representations And Algorithms}
    \subsection*{Overview}

    Integer representations and algorithms are fundamental topics in computer science and mathematics, serving as the backbone for various computational tasks. Understanding how integers are represented 
    in binary form and knowing how to perform arithmetic operations on them efficiently is crucial in computer systems, cryptography, and many other areas. This summary explores the key concepts related 
    to integer representations and algorithms, shedding light on their significance and real-world applications.

    \subsection*{Key Concepts}

    \begin{enumerate}[label=\arabic*.]
        \item \textbf{Binary Representation:} Integers are often represented in binary form, which uses only two digits: 0 and 1. In this representation, each digit's position has a weight corresponding 
        to a power of 2, allowing for efficient conversions and bitwise operations.
        
        \item \textbf{Two's Complement:} Two's complement is a widely-used representation for signed integers. It allows both positive and negative numbers to be represented using a fixed number of bits. 
        Negative numbers are obtained by taking the complement (flipping the bits) of the corresponding positive number and adding 1.
        
        \item \textbf{Overflow:} Integer overflow occurs when the result of an arithmetic operation exceeds the maximum representable value for a given number of bits. Handling overflow is critical to 
        ensure accurate computations in computer programs.
        
        \item \textbf{Bitwise Operations:} Bitwise operations (AND, OR, XOR, NOT, shifts) provide a way to manipulate individual bits in integers. They are used in various algorithms, including those 
        for data compression, encryption, and optimization.
        
        \item \textbf{Addition and Subtraction:} Algorithms for addition and subtraction involve working with binary representations, carry bits, and borrowing. Efficient methods, such as binary addition 
        trees and two's complement subtraction, are used in hardware and software implementations.
        
        \item \textbf{Multiplication and Division:} Multiplication and division of integers require more complex algorithms, such as long multiplication and long division. Optimized techniques like 
        Karatsuba multiplication and Newton-Raphson division enhance efficiency.
        
        \item \textbf{Modular Arithmetic:} Modular arithmetic is used to perform arithmetic operations within a specified modulus. It finds applications in cryptography (RSA encryption), hashing, and 
        pseudorandom number generation.
        
        \item \textbf{Euclidean Algorithm:} The Euclidean algorithm computes the greatest common divisor (GCD) of two integers efficiently. It plays a crucial role in simplifying fractions, solving linear 
        Diophantine equations, and ensuring data integrity.
        
        \item \textbf{Number Bases:} Understanding different number bases, such as hexadecimal and octal, is important for programmers and engineers. These bases are used in low-level programming, memory 
        addressing, and data representation.
        
        \item \textbf{Floating-Point Representation:} Floating-point representation is used to represent real numbers with a fractional part. It involves a sign bit, exponent, and mantissa and is 
        fundamental in scientific and engineering computing.
        
        \item \textbf{Applications:} Integer representations and algorithms find applications in various fields, including computer graphics, cryptography, database systems, network protocols, and 
        scientific simulations. Efficient algorithms are essential for optimizing software and hardware performance.
    \end{enumerate}

    In conclusion, integer representations and algorithms are the building blocks of computer science and mathematics. They underlie the fundamental operations performed by computers and are essential for 
    designing efficient algorithms and data structures. Mastery of these concepts empowers professionals in fields ranging from software engineering to cybersecurity, enabling them to solve complex problems 
    and develop robust systems. An understanding of integer representations and algorithms is invaluable in the digital age, where computational efficiency and accuracy are paramount.
\end{notes}

The last section that we will be covering this week is \textbf{Section 4.3 - Primes and Greatest Common Divisors}.

\begin{notes}{Section 4.3 - Primes and Greatest Common Divisors}
    \subsection*{Overview}

    Primes and greatest common divisors (GCD) are fundamental concepts in number theory with wide-ranging applications in various mathematical and computational domains. Prime numbers, as the building 
    blocks of integers, play a crucial role in encryption, data compression, and random number generation. GCD, on the other hand, is a foundational tool for solving Diophantine equations, simplifying 
    fractions, and ensuring data integrity. This summary explores the key concepts related to primes and GCD, highlighting their importance and real-world implications.

    \subsection*{Key Concepts}

    \begin{enumerate}[label=\arabic*.]
        \item \textbf{Prime Numbers:} Prime numbers are natural numbers greater than 1 that have no divisors other than 1 and themselves. They play a fundamental role in number theory and cryptography.
        
        \item \textbf{Composite Numbers:} Composite numbers are integers greater than 1 that are not prime, meaning they have divisors other than 1 and themselves.
        
        \item \textbf{Prime Factorization:} Prime factorization is the process of expressing a composite number as a product of prime numbers. It is the basis for understanding the internal structure 
        of integers.
        
        \item \textbf{Sieve of Eratosthenes:} The Sieve of Eratosthenes is an algorithm for efficiently finding all prime numbers up to a given limit. It eliminates multiples of each prime as it 
        iterates through the integers.
        
        \item \textbf{Greatest Common Divisor (GCD):} The GCD of two integers is the largest positive integer that divides both of them without a remainder. It is denoted as $\text{GCD}(a, b)$ or 
        $\text{gcd}(a, b)$.
        
        \item \textbf{Euclidean Algorithm:} The Euclidean algorithm is an efficient method for finding the GCD of two integers. It involves repeated division with remainder and is a fundamental tool 
        in number theory and cryptography.
        
        \item \textbf{Diophantine Equations:} Diophantine equations are equations where solutions are sought in integers. The GCD and the Euclidean algorithm are used to solve linear Diophantine equations.
        
        \item \textbf{Relatively Prime Numbers:} Two integers are relatively prime if their GCD is 1. Relatively prime numbers have no common divisors other than 1.
        
        \item \textbf{Applications:} Prime numbers and GCD have practical applications in cryptography (RSA encryption), random number generation, error detection and correction, and hashing algorithms. 
        They are essential tools in ensuring data security and integrity.
    \end{enumerate}

    Primes and greatest common divisors are foundational concepts in number theory with a significant impact on mathematics, computer science, and cryptography. Prime numbers are the basis for secure 
    encryption, while the GCD is a versatile tool for solving equations and ensuring data accuracy. Understanding these concepts is essential for professionals in fields such as cybersecurity, algorithm 
    design, and numerical analysis. The study of primes and GCD continues to be a fertile ground for mathematical research and innovation, with implications for both theoretical and practical aspects 
    of computation and data security.
\end{notes}