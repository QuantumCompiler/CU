\clearpage

\renewcommand{\ChapTitle}{Graph Theory}
\renewcommand{\SectionTitle}{Graph Theory}

\chapter{\ChapTitle}
\section{\SectionTitle}
\horizontalline{0}{0}

\subsection{Assigned Reading}

The reading assignments for this week is from, \Textbook:

\begin{itemize}
    \item \textbf{Chapter 10.1 - Graphs And Graph Models}
    \item \textbf{Chapter 10.2 - Graph Terminology And Special Types Of Graphs}
\end{itemize}

\subsection{Piazza}

Must post / respond to at least \textbf{two} Piazza posts this week.

\subsection{Lectures}

The lectures for this week and their links can be found below:

\begin{itemize}
    \item \href{https://applied.cs.colorado.edu/mod/hvp/view.php?id=51873}{Essentials: Intro} $\approx$ 2 min.
    \item \href{https://applied.cs.colorado.edu/mod/hvp/view.php?id=51874}{Essentials: Terminology} $\approx$ 15 min.
    \item \href{https://applied.cs.colorado.edu/mod/hvp/view.php?id=51875}{Essentials: Degree Sequences} $\approx$ 17 min.
    \item \href{https://applied.cs.colorado.edu/mod/hvp/view.php?id=51876}{Essentials: Wrap-Up} $\approx$ 1 min.
    \item \href{https://applied.cs.colorado.edu/mod/hvp/view.php?id=51877}{Connectedness And Eulerian Tours} $\approx$ 30 min.
    \item \href{https://applied.cs.colorado.edu/mod/hvp/view.php?id=51878}{Graph Coloring} $\approx$ 40 min.
\end{itemize}

\noindent Below is a list of lecture notes for this week:

\begin{itemize}
    \item \pdflink{\LectureNotesDir Essential Graph Theory Lecture Notes.pdf}{Essential Graph Theory Lecture Notes}
    \item \pdflink{\LectureNotesDir Connectedness And Eulerian Tours Lecture Notes.pdf}{Connectedness And Eulerian Tours Lecture Notes}
    \item \pdflink{\LectureNotesDir The Coloring Problem Lecture Notes.pdf}{The Coloring Problem Lecture Notes}
\end{itemize}

\newpage

\subsection{Chapter Summary}

The first section that we are covering this week is \textbf{Section 10.1 - Graphs And Graph Models}.

\begin{notes}{Section 10.1 - Graphs And Graph Models}
    \subsubsection*{Overview:}

    This section covers the foundational concepts of graph theory. It defines a graph as a collection of vertices and edges, and explores various types of graphs including simple graphs, multigraphs, 
    pseudographs, directed, and mixed graphs. The section delves into graph terminology such as paths, cycles, and vertex degrees. It also highlights the practical applications of graphs in areas like 
    social networks and transportation, and introduces basic graph algorithms. This concise overview encapsulates the key ideas and applications of graph theory presented in the section. \vspace*{1em}

    \subsubsection*{Graph Definition:}

    \begin{itemize}
        \item A graph is formally defined as $G=(V,E)$, where 4 is a nonempty set of vertices and $E$ is a set of edges.
        \item Simple Graphs: Consist of a set of vertices and edges, where each edge connects a pair of distinct vertices without any loops or multiple edges between the same vertices.
        \item Multigraphs: Permit multiple edges between the same pair of vertices.
        \item Pseudographs: Allow loops and multiple edges.
        \item Directed Graphs (Digraphs): Edges have a direction, represented as ordered pairs of vertices.
        \item Mixed Graphs: Contain both directed and undirected edges.
    \end{itemize}

    \subsubsection*{Graph Terminology:}

    \begin{itemize}
        \item The degree of a vertex, paths, cycles, connected graphs, and subgraphs are explained with examples.
        \item In directed graphs, concepts of in-degree and out-degree are introduced.
    \end{itemize}

    \subsubsection*{Graph Models:}

    \begin{itemize}
        \item Practical applications in various fields like social networks, communication networks, and transportation are highlighted.
        \item Models for specific applications, like representing a computer network or transportation system, are discussed with examples.
    \end{itemize}

    \subsubsection*{Graph Representations:}

    \begin{itemize}
        \item Different methods of representing graphs, such as adjacency lists and adjacency matrices, are detailed.
    \end{itemize}

    \subsubsection*{Graph Algorithms:}

    \begin{itemize}
        \item Basic algorithms for graph traversal, such as depth-first search (DFS) and breadth-first search (BFS), are introduced.
    \end{itemize}
\end{notes}

The last section that we will cover this week is \textbf{Section 10.2 - Graph Terminology And Special Types Of Graphs}.

\begin{notes}{Section 10.2 - Graph Terminology And Special Types Of Graphs}
    \subsubsection*{Overview:}

    This section elaborates on key terms in graph theory and explores various special graph types. It introduces basic vocabulary essential for solving problems in graph theory, like graph drawing and 
    vertex correspondence. The section covers terminology for both undirected and directed graphs, such as adjacency, degrees, and neighborhood concepts. It also delves into special graph types like 
    complete graphs, cycles, wheels, bipartite graphs, and n-cubes, explaining their characteristics and significance. The section ties these concepts to practical applications, illustrating how graphs 
    model complex systems in various fields. \vspace*{1em}

    \subsubsection*{Graph Foundations:}

    Clarifies the basic elements of graphs: vertices (nodes) and edges (lines). It distinguishes between simple graphs (no loops or multiple edges), multigraphs (multiple edges allowed), and pseudographs 
    (loops and multiple edges allowed). \vspace*{1em}

    \subsubsection*{Vertex Characteristics:}

    Focuses on vertex degrees - the number of edges connected to a vertex. It introduces adjacency (when two vertices are connected by an edge) and incidence (relationship between vertices and edges). \vspace*{1em}

    \subsubsection*{Directed Graphs:}

    In these, edges have directions. The section explains in-degree (number of incoming edges) and out-degree (number of outgoing edges) for vertices in directed graphs. \vspace*{1em}

    \subsubsection*{Specific Graph Types:}

    Describes complete graphs (every pair of distinct vertices is connected by a unique edge), cycles (a path that starts and ends at the same vertex), and bipartite graphs (vertices can be divided 
    into two disjoint sets where each edge connects a vertex from one set to a vertex from the other). \vspace*{1em}

    \subsubsection*{Graph Representation Methods:}

    Introduces adjacency lists and matrices as ways to represent graph structures in a compact form.
\end{notes}