\clearpage

\renewcommand{\ChapTitle}{Sets And Functions}
\renewcommand{\SectionTitle}{Sets And Functions}

\chapter{\ChapTitle}
\section{\SectionTitle}
\horizontalline{0}{0}

\subsection{Assigned Reading}

The reading assignments for this week is from, \Textbook:

\begin{itemize}
    \item \textbf{Chapter 2.1 - Sets}
    \item \textbf{Chapter 2.2 - Set Operations}
    \item \textbf{Chapter 2.3 - Functions}
    \item \textbf{Chapter 4.1 - Divisibility And Modular Arithmetic}
\end{itemize}

\subsection{Piazza}

Must post / respond to at least \textbf{two} Piazza posts this week.

\subsection{Lectures}

The lectures for this week and their links can be found below:

\begin{itemize}
    \item \href{https://applied.cs.colorado.edu/mod/hvp/view.php?id=51609}{Sets And Set Operations} $\approx$ 1 hour, 2 min.
    \item \href{https://applied.cs.colorado.edu/mod/hvp/view.php?id=51610}{Functions} $\approx$ 40 min.
    \item \href{https://applied.cs.colorado.edu/mod/hvp/view.php?id=51611}{Modular Arithmetic Review} $\approx$ 5 min.
    \item \href{https://drive.explaineverything.com/thecode/HXKUAZT}{Modular Arithmetic With Scopes And Cups} $\approx$ 7 min.
    \item \href{https://o365coloradoedu-my.sharepoint.com/personal/stadee_colorado_edu/_layouts/15/stream.aspx?id=%2Fpersonal%2Fstadee%5Fcolorado%5Fedu%2FDocuments%2FG%2FAAAPostbac%2FBest%20function%20part1%2Emp4&ga=1}{Best Function You've Never Heard Of Part 1} $\approx$ 6 min.
    \item \href{https://o365coloradoedu-my.sharepoint.com/personal/stadee_colorado_edu/_layouts/15/stream.aspx?id=%2Fpersonal%2Fstadee%5Fcolorado%5Fedu%2FDocuments%2FG%2FAAAPostbac%2FBestfunctionyouneverheardofpart2%2Emp4&ga=1}{Best Function You've Never Heard Of Part 2} $\approx$ 10 min.
    \item \href{https://o365coloradoedu-my.sharepoint.com/personal/stadee_colorado_edu/_layouts/15/stream.aspx?id=%2Fpersonal%2Fstadee%5Fcolorado%5Fedu%2FDocuments%2FG%2FAAAPostbac%2FMakeasliderule%2Emp4&ga=1}{Make A Slide Rule} $\approx$ 8 min.
    \item \href{https://applied.cs.colorado.edu/mod/hvp/view.php?id=51616}{Algorithms} $\approx$ 36 min.
    \item \href{https://applied.cs.colorado.edu/mod/hvp/view.php?id=51617}{Countable And Uncountable Sets} $\approx$ 18 min.
\end{itemize}

\noindent Below is a list of lecture notes for this week:

\begin{itemize}
    \item \pdflink{\LectureNotesDir Algorithms Lecture Notes.pdf}{Algorithms Lecture Notes}
    \item \pdflink{\LectureNotesDir Countable And Uncountable Sets Lecture Notes.pdf}{Countable And Uncountable Sets Lecture Notes}
    \item \pdflink{\LectureNotesDir Functions Lecture Notes.pdf}{Functions Lecture Notes}
    \item \pdflink{\LectureNotesDir Modular Scoops Lecture Notes.pdf}{Modular Scoops Lecture Notes}
    \item \pdflink{\LectureNotesDir Sets And Set Operations Lecture Notes.pdf}{Sets And Set Operations Lecture Notes}
\end{itemize}

\subsection{Assignments}

The assignment for this week is:

\begin{itemize}
    \item \pdflink{\AssDir Assignment 4 - Sets And Functions.pdf}{Assignment 4 - Sets And Functions}
\end{itemize}

\subsection{Quiz}

The quiz's for this week can be found at:

\begin{itemize}
    \item \pdflink{\QuizDir Quiz 4 - Sets And Modulo.pdf}{Quiz 4 - Sets And Modulo}
\end{itemize}

\newpage

\subsection{Chapter Summary}

The first section that we are covering this week is \textbf{Section 2.1 - Sets}.

\begin{notes}{Section 2.1 - Sets}
    \subsection*{Overview}

    In the realm of discrete mathematics, sets play a foundational role, serving as one of the fundamental mathematical concepts. A set is essentially a collection of distinct elements, denoted within 
    curly braces $\{\}$. These elements can encompass a wide range of entities, such as numbers, letters, symbols, or even other sets. For instance, a set of natural numbers could be represented as 
    $\{1, 2, 3, 4, \ldots\}$, while a set of uppercase letters might appear as $\{A, B, C, \ldots\}$.

    One of the key principles that govern sets is the concept of equality. Two sets are considered equal if and only if they contain the exact same elements. For example, if we have a set $A = \{1, 2, 3\}$ 
    and another set $B = \{3, 1, 2\}$, they are still deemed equal since their elements are identical. Sets also have a unique feature known as cardinality, which measures the number of elements 
    within a set. In the case of sets $A$ and $B$ mentioned earlier, their cardinality is 3 since they both contain three elements.

    Among the essential ideas in set theory are the notions of subsets, union, and intersection. A set $A$ is considered a subset of another set $B$ if every element in $A$ is also present in $B$. 
    The union of two sets, denoted by $A \cup B$, combines all distinct elements from both sets into a new set. Conversely, the intersection of two sets, denoted by $A \cap B$, contains only the 
    elements that are common to both $A$ and $B$. These concepts provide the foundation for various mathematical operations and problem-solving techniques in discrete mathematics.

    Sets also have a counterpart known as the empty set $\emptyset$, which contains no elements. The existence of the empty set is crucial for mathematical reasoning and allows us to represent 
    situations where there are no items in a particular category. For instance, the set of natural numbers less than zero would be represented as an empty set: $\emptyset$.

    In summary, sets are fundamental entities in discrete mathematics, serving as a means to organize and manipulate collections of elements. Their properties, including equality, cardinality, 
    subsets, union, intersection, and the existence of the empty set, play a crucial role in various branches of discrete mathematics, including combinatorics, set theory, probability, and graph 
    theory. Understanding sets is essential for effective problem-solving and mathematical reasoning in these fields.

    \subsection*{Subsets}

    Subsets are a foundational concept in set theory and discrete mathematics. In essence, a subset is a set that comprises only elements that are also part of another set, known as the "superset." 
    Symbolically, if every element within set $A$ is also found in set $B$, then $A$ is considered a subset of $B$, denoted as $A \subseteq B$. Subsets find extensive use in various mathematical and 
    logical contexts, helping categorize elements based on specific properties or criteria. Whether analyzing integers, real numbers, or more complex structures, subsets enable mathematicians to 
    break down intricate problems into manageable parts. Additionally, subsets can include the "empty set" ($\emptyset$ or $\{\}$), which contains no elements and is considered a subset of every 
    set, making it a fundamental concept in mathematical proofs and reasoning. Moreover, the "power set" of a set $A$, denoted as $P(A)$, encompasses all possible subsets of $A$, including $A$ itself 
    and the empty set, often applied in combinatorics and discrete mathematics to explore various combinations and permutations.

    \subsection*{Size Of A Set}

    In the realm of set theory, determining the size or cardinality of a set is a fundamental concept. The size of a set refers to the number of elements it contains. It is denoted as $\lvert A \rvert$, 
    where $A$ represents the set. The cardinality of a set can be finite or infinite. For finite sets, counting the elements directly provides the cardinality. However, for infinite sets, the cardinality 
    is assessed differently, often by establishing a correspondence with a known infinite set, such as the natural numbers.

    The concept of cardinality is essential for comparing sets and establishing relationships between them. Two sets are considered equal in size if their cardinalities are the same, regardless of the 
    elements within them. Set operations, such as union and intersection, can also be analyzed in terms of cardinality, allowing mathematicians to make precise statements about the sizes of resulting 
    sets. Additionally, cardinality plays a pivotal role in combinatorics, probability theory, and various branches of mathematics where counting and measuring the size of sets are integral to solving 
    problems and proving theorems.

    \subsection*{Cartesian Products}

    In discrete mathematics, the Cartesian product is a fundamental concept that allows us to construct new sets from existing ones. Given two sets $A$ and $B$, the Cartesian product $A \times B$ is defined 
    as the set of all ordered pairs where the first element comes from set $A$ and the second element comes from set $B$. Symbolically, $A \times B = \{(a, b) \mid a \in A, b \in B\}$.

    The Cartesian product is a versatile tool used in various mathematical and computer science applications. It is commonly employed in geometry to represent points in the Cartesian coordinate system. 
    In set theory, it plays a crucial role in defining relations between sets and functions. In computer science, Cartesian products are used to model the state spaces of systems, such as finite automata.
    
    Cartesian products can be extended to more than two sets. For example, the Cartesian product of three sets $A \times B \times C$ consists of ordered triples, and so on for higher dimensions. Understanding 
    Cartesian products and their properties is essential for solving problems involving ordered pairs, relations, and multi-dimensional data structures.
    
    \subsection*{Set Notation With Quantifiers}

    Set notation and quantifiers are essential tools in discrete mathematics for describing and making statements about collections of elements. The universal quantifier ($\forall$) is employed to express 
    statements that hold true for every element within a given set. For instance, $\forall x \in A, P(x)$ asserts that the predicate $P(x)$ is valid for all elements $x$ in set $A$. Conversely, the existential 
    quantifier ($\exists$) is used to indicate that at least one element in a set satisfies a particular predicate. For example, $\exists x \in A, P(x)$ implies that there exists an element $x$ in set $A$ 
    for which $P(x)$ is true.

    Set notation, in conjunction with quantifiers, provides a concise and precise way to represent complex statements about sets and their elements. The set-builder notation, $\{x \in A \mid P(x)\}$, signifies 
    the set of all elements in $A$ for which the predicate $P(x)$ holds. This combination of tools is indispensable in various mathematical fields, such as set theory, logic, and proof theory. It enables 
    mathematicians to systematically and rigorously reason about collections of objects, making it a fundamental aspect of discrete mathematics.
\end{notes}

The next section that we are covering this week is \textbf{Section 2.2 - Set Operations}.

\begin{notes}{Section 2.2 - Set Operations}
    \subsection*{Overview}

    In discrete mathematics, sets are fundamental objects used to represent collections of elements. Set operations allow us to manipulate and analyze sets in various ways. The primary set operations include 
    union, intersection, complement, and difference.

    The union of two sets, denoted by $A \cup B$, contains all the unique elements that belong to either set $A$ or set $B$, or both. It combines elements from both sets without duplication. The intersection of 
    two sets, denoted by $A \cap B$, consists of all the elements that are common to both set $A$ and set $B$. In other words, it contains only the elements that appear in both sets.
    
    The complement of a set, denoted by $\neg A$ or $A'$, contains all the elements from the universal set that do not belong to set $A$. It represents everything outside of set $A$ within the universal set. The 
    difference between two sets, denoted by $A - B$, contains all the elements that belong to set $A$ but not to set $B$. It essentially removes the elements of set $B$ from set $A$.
    
    Set operations are crucial tools in various mathematical disciplines, including set theory, probability, and discrete mathematics. They help us perform tasks like combining sets, finding common elements, 
    and defining subsets. These operations follow specific rules and properties that enable precise and systematic reasoning about sets and their relationships, making them essential for solving a wide range 
    of mathematical problems.
    
    \subsection*{Set Identities}

    In the realm of set theory, set identities are fundamental principles and relationships that govern the behavior of sets and their operations. These identities help mathematicians manipulate sets and 
    express complex relationships with clarity and precision. Some of the key set identities include the identity laws, domination laws, complement laws, double complement law, and De Morgan's laws.

    The identity laws state that the union of a set with the universal set is the universal set itself, and the intersection of a set with the universal set is the set itself. These laws highlight the role 
    of the universal set as an identity element for set operations.The domination laws state that the union of a set with the empty set is the set itself, and the intersection of a set with the universal set 
    is the empty set. These laws underscore the influence of the empty set as a dominating element in set operations.
    
    The complement laws describe the relationship between a set and its complement. They state that the union of a set with its complement is the universal set, and the intersection of a set with its complement 
    is the empty set. These laws highlight the complementary nature of sets. The double complement law states that the complement of the complement of a set is the set itself. It emphasizes that taking the 
    complement of a set twice returns the original set.De Morgan's laws provide a powerful tool for expressing the complement of union and intersection. They state that the complement of the union of two sets 
    is equal to the intersection of their complements, and the complement of the intersection of two sets is equal to the union of their complements. These laws enable the simplification of complex set expressions.
    
    Set identities serve as a foundation for proving theorems and making deductions in set theory and related mathematical disciplines. They provide a systematic way to reason about sets and their properties, 
    making them indispensable tools for solving problems in various areas of mathematics.

    \subsection*{Set Identities}

    Here is a table of set identities that are similar to propositional logic.

    \begin{center}
        \begin{tabular}{|c|c|}
            \hline \textbf{Identity Name} & \textbf{Set Identity} \\ \hline
            Identity laws & $A \cap U = A$ \\
                        & $A \cup \emptyset = A$ \\ \hline
            Domination laws & $A \cup U = U$ \\
                            & $A \cap \emptyset = \emptyset$ \\ \hline
            Idempotent laws & $A \cup A = A$ \\
                            & $A \cap A = A$ \\ \hline
            Complementation law & $\overline{A} = A$ \\ \hline
            Commutative laws & $A \cup B = B \cup A$ \\
                            & $A \cap B = B \cap A$ \\ \hline
            Associative laws & $A \cup (B \cup C) = (A \cup B) \cup C$ \\
                            & $A \cap (B \cap C) = (A \cap B) \cap C$ \\ \hline
            Distributive laws & $A \cup (B \cap C) = (A \cup B) \cap (A \cup C)$ \\
                            & $A \cap (B \cup C) = (A \cap B) \cup (A \cap C)$ \\ \hline
            De Morgan’s laws & $A \cup B = A \cap B$ \\
                            & $A \cap B = A \cup B$ \\ \hline
            Absorption laws & $A \cup (A \cap B) = A$ \\
                        & $A \cap (A \cup B) = A$ \\ \hline
            Complement laws & $A \cup A = U$ \\
                            & $A \cap A = \emptyset$ \\ \hline
        \end{tabular}
    \end{center}
        
    \subsection*{Generalized Unions And Intersections}

    In the realm of set theory and discrete mathematics, Generalized Unions and Intersections offer a versatile approach to dealing with collections of sets. These operations extend the principles of 
    regular unions and intersections to handle entire families of sets, making them invaluable tools in various mathematical contexts.

    The Generalized Union, often represented as $\bigcup$, allows us to combine the elements of multiple sets within a given family of sets. This operation is particularly useful when dealing with 
    infinite sets or an unspecified number of sets. The result of $\bigcup$ is a new set containing all unique elements found in any set within the family, without repetition.

    On the other hand, the Generalized Intersection, denoted as $\bigcap$, enables us to find elements that are common to all sets in a given family. Like its union counterpart, the Generalized Intersection 
    accommodates infinite or unspecified sets within the family. The outcome of $\bigcap$ is a new set that includes only the elements present in every set of the family.

    These Generalized Unions and Intersections adhere to the same fundamental principles as regular unions and intersections, including the commutative and associative properties. They play a pivotal role 
    in various mathematical disciplines, such as topology, set theory, and real analysis, where dealing with diverse and unbounded sets is commonplace. Whether handling countable or uncountable collections, 
    these operations provide a robust framework for reasoning about sets and their relationships, allowing mathematicians to explore intricate concepts and solve complex problems.
\end{notes}

The next section that we are covering this week is \textbf{Section 2.3 - Functions}.

\begin{notes}{Section 2.3 - Functions}
    \subsection*{Overview}

    Functions in discrete mathematics are a cornerstone concept with broad applicability in diverse mathematical and computational domains. These mathematical constructs are essentially relationships that 
    link two sets: a domain (comprising all possible input values) and a codomain (encompassing all potential output values). A crucial characteristic of functions is their unique mapping property, where 
    each element from the domain corresponds to precisely one element in the codomain. This mapping, typically represented as $f: A \rightarrow B$ for a function $f$ with domain $A$ and codomain $B$, allows 
    for the systematic analysis, modeling, and problem-solving across numerous mathematical and computer science disciplines.

    Functions come with several key attributes. They begin with the definition of their domain and codomain, with the codomain often serving as a broader set that includes all possible output values. The 
    range of a function denotes the subset of the codomain containing the actual output values produced by the function. Additionally, functions can be categorized based on their behavior. A function is 
    one-to-one (or injective) if it assigns distinct values to different domain elements, ensuring that no two domain elements map to the same codomain element. Conversely, a function is onto (or surjective) 
    when every element in the codomain has at least one corresponding element in the domain.
    
    Function notation, such as $f(x)$, is used to denote the relationship between elements in the domain and their corresponding elements in the codomain. This mathematical concept serves as a foundational 
    tool across various mathematical domains, including algebra, calculus, number theory, and combinatorics. Moreover, in computer science, functions are instrumental for modeling algorithms, data transformations, 
    and program behaviors. By studying functions and their properties, mathematicians and computer scientists gain the fundamental tools needed to analyze, model, and solve an extensive range of mathematical 
    and computational problems.

    \subsection*{One-to-One And Onto Functions}

    One-to-One and Onto functions are fundamental concepts in discrete mathematics and play a crucial role in understanding the relationships between sets and their mappings. A function is considered one-to-one, 
    or injective, if each element in its domain maps to a unique element in its codomain. In simpler terms, no two distinct elements in the domain can map to the same element in the codomain. This property 
    ensures that there is a one-to-one correspondence between elements in the domain and their images in the codomain, making it possible to invert the function when restricted to its range.

    Conversely, a function is termed onto, or surjective, when its range covers the entire codomain. In other words, every element in the codomain has a pre-image in the domain under the function. Onto functions 
    are exhaustive in their mappings, leaving no "gaps" or uncovered elements in the codomain. When a function is both one-to-one and onto, it is referred to as a bijection. Bijections establish a bijective 
    correspondence between the elements of the domain and codomain, providing a complete and reversible mapping.

    One-to-One and Onto functions are crucial in various mathematical fields, including algebra, calculus, and combinatorics. They allow mathematicians to analyze the behavior of functions, determine their 
    invertibility, and explore the properties of sets and their relationships. In computer science and data analysis, these concepts are essential for understanding data transformations, encryption algorithms, 
    and database operations. One-to-One and Onto functions provide powerful tools for modeling real-world situations and solving mathematical and computational problems effectively.

    \subsection*{Inverse Functions And Compositions Of Functions}

    In discrete mathematics, inverse functions and compositions of functions are fundamental concepts. An inverse function, denoted as $f^{-1}$, is a function that "undoes" the actions of another function. For 
    any element $x$ in the domain, $f$ maps it to $y$, while $f^{-1}$ maps $y$ back to $x$. The properties $f(f^{-1}(x)) = x$ for all $x$ in the domain and $f^{-1}(f(x)) = x$ for all $x$ in the codomain of $f$ 
    must hold for them to be considered inverses.

    Compositions of functions involve applying one function to the output of another. If we have two functions $f$ and $g$, the composition $g \circ f$ means applying $f$ first and then $g$. Compositions help 
    us understand how multiple functions combine and their combined effects. It's expressed as $(g \circ f)(x) = g(f(x))$. The order of composition matters, as $g \circ f$ may not be the same as $f \circ g$.
    
    These concepts are vital in various branches of mathematics, such as calculus, linear algebra, and abstract algebra, enabling us to analyze transformations, solve equations, and explore relationships between 
    functions. In computer science and engineering, they are essential for modeling systems, designing algorithms, and optimizing data processing. Inverse functions and compositions provide a powerful framework 
    for understanding complex mappings and transformations in both mathematical and real-world contexts.
\end{notes}

The last section that we are covering this week is \textbf{Section 4.1 - Divisibility And Modular Arithmetic}.

\begin{notes}{Section 4.1 - Divisibility And Modular Arithmetic}
    \subsection*{Overview}

    In discrete mathematics, divisibility and modular arithmetic are fundamental concepts with widespread applications. Divisibility refers to the property of one integer being divisible by another without leaving 
    a remainder. For integers $a$ and $b$, if there exists an integer $k$ such that $a = bk$, then $b$ is said to divide $a$, denoted as $b | a$. Divisibility plays a crucial role in number theory, prime 
    factorization, and understanding the properties of integers.

    Modular arithmetic, on the other hand, deals with remainders when integers are divided. It operates within a fixed modulus $m$ and considers integers to be congruent if they have the same remainder when 
    divided by $m$. This concept is denoted as $a \equiv b \mod m$. Modular arithmetic has various applications, including cryptography, computer science (in algorithms and data structures), and number theory 
    (in solving Diophantine equations).
    
    These concepts provide tools for solving equations, understanding patterns in number sequences, and working with cyclic data in real-world applications. Divisibility and modular arithmetic are essential in many 
    mathematical disciplines and have practical implications in various fields, making them valuable tools for mathematicians, scientists, and engineers.
\end{notes}