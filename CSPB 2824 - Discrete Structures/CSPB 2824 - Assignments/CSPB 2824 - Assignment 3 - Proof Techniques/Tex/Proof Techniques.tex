\clearpage
\chapter{Mastery Workbook 3}

% Chapter page
\section{Proof Techniques Workbook}

\horizontalline{0}{0}

\begin{center}
    \Large{\textbf{I have neither given nor received unauthorized assistance.}}
    \horizontalline{0}{0}
    \large{\textbf{Taylor James Larrechea}}
    \horizontalline{0}{0}
\end{center}

% Problem 0
\begin{problem}{Problem 0 - Warm Up THMs}
    \begin{statement}{Problem Statement}
        \begin{enumerate}[label=(\alph*)]
            \item Prove: For all positive integers, $A$, $B$, if $A$ is even and $B$ is odd, $A + B = C$ is odd.
            \item Prove: For all positive integers, $A$, $B$, if $A$ is odd and $B$ is odd, $A + B = C$ is even.
            \item Prove: For all positive integers, $A$, $B$, if $A$ and $B$ are even, $A \cdot B = C$ is even.
            \item Prove: For all positive integers, $A$, $B$, if $A$ is even $B$ is odd, $A \cdot B = C$ is even.
        \end{enumerate}
    \end{statement}

    \begin{Highlight}[Solution - Part (a)]
        \textbf{Theorem:} For all positive integers, $A$, $B$, if $A$ is even and $B$ is odd, $A + B = C$ is odd. \vspace*{1em}

        \textbf{Direct Proof:} \newline
        \horizontalline{-1.5}{-1.5}
        \begin{align*}
            \text{There is $k$ such that } A & = 2k & \text{(By definition of even)} \\
            \text{There is $l$ such that } B & = 2l + 1 & \text{(By definition of odd)} \\
            C & = A + B & \text{(Premise)} \\
            & = 2k + 2l + 1 & \text{(Substitution)} \\
            & = 2(k + l) + 1 & \text{(Factoring)} \\
            & \textcolor{blue}{\text{Let } \alpha \text{ integer} = k + l} & \text{(Closure of integers renaming $k + l$ )} \\
            & = 2\alpha + 1 & \text{(Substitution)} \\
        \end{align*}
        \horizontalline{-2.5}{0.5}
        Consequently, $C = 2\alpha + 1$ with $\alpha$ is an integer, and thus by the \textbf{definition of odd} $C$ is odd. \qed
    \end{Highlight}

    \begin{Highlight}[Solution - Part (b)]
        \textbf{Theorem:} For all positive integers, $A$, $B$, if $A$ is odd and $B$ is odd, $A + B = C$ is even. \vspace*{1em}

        \textbf{Direct Proof:} \newline
        \horizontalline{-1.5}{-1.5}
        \begin{align*}
            \text{There is $k$ such that } A & = 2k + 1 & \text{(By definition of odd)} \\
            \text{There is $l$ such that } B & = 2l + 1 & \text{(By definition of odd)} \\
            C & = A + B & \text{(Premise)} \\
            & = 2k + 1 + 2l + 1 & \text{(Substitution)} \\
            & = 2k + 2l + 1 + 1 & \text{(Commutativity)} \\
            & = 2k + 2l + 2 & \text{(Simplification by addition)} \\
            & = 2(k + l) + 2 & \text{(Factoring)} \\
            & \textcolor{blue}{\text{Let } \alpha \text{ integer} = k + l} & \text{(Closure of integers renaming $k + l$)} \\
            & = 2\alpha + 2 & \text{(Substitution)} \\
            & = 2(\alpha + 1) & \text{(Factoring)} \\
            & \textcolor{blue}{\text{Let } \beta \text{ integer} = \alpha + 1} & \text{(Closure of integers renaming $\alpha + 1$)} \\
            & = 2\beta & \text{(Substitution)} \\
        \end{align*}
        \horizontalline{-2.5}{0.5}
        Consequently, $C = 2\beta$ with $\beta$ is an integer, and thus by the \textbf{definition of even} $C$ is even. \qed
    \end{Highlight}

    \begin{Highlight}[Solution - Part (c)]
        \textbf{Theorem:} For all positive integers, $A$, $B$, if $A$ and $B$ are even, $A \cdot B = C$ is even. \vspace*{1em}

        \textbf{Direct Proof:} \newline
        \horizontalline{-1.5}{-1.5}
        \begin{align*}
            \text{There is $k$ such that } A & = 2k & \text{(By definition of even)} \\
            \text{There is $l$ such that } B & = 2l & \text{(By definition of even)} \\
            C & = A \cdot B & \text{(Premise)} \\
            & = (2k)(2l) & \text{(Substitution)} \\
            & = (2)(2)(k)(l) & \text{(Commutativity)} \\
            & = 4(k)(l) & \text{(Simplification by multiplication)} \\
            & = 4kl & \text{(Associativity)} \\
            & = 2(2kl) & \text{(Factoring)} \\
            & \textcolor{blue}{\text{Let } \alpha \text{ integer } = 2kl} & \text{(Closure of integers renaming $2kl$)} \\
            & = 2\alpha & \text{(Substitution)} \\
        \end{align*}
        \horizontalline{-2.5}{0.5}
        Consequently, $C = 2\alpha$ with $\alpha$ is an integer, and thus by the \textbf{definition of even} $C$ is even. \qed
    \end{Highlight}

    \begin{Highlight}[Solution - Part (d)]
        \textbf{Theorem:} For all positive integers, $A$, $B$, if $A$ is even $B$ is odd, $A \cdot B = C$ is even. \vspace*{1em}

        \textbf{Direct Proof:} \newline
        \horizontalline{-1.5}{-1.5}
        \begin{align*}
            \text{There is $k$ such that } A & = 2k & \text{(By definition of even)} \\
            \text{There is $l$ such that } B & = 2l + 1 & \text{(By definition of odd)} \\
            C & = A \cdot B & \text{(Premise)} \\
            & = (2k)(2l + 1) & \text{(Substitution)} \\
            & = (2k)(2l) + (2k)(1) & \text{(Distributivity)} \\
            & = (2)(2)(k)(l) + (2k)(1) & \text{(Commutativity)} \\
            & = 4(k)(l) + 2k & \text{(Simplification by multiplication)} \\
            & = 4kl + 2k & \text{(Associativity)} \\
            & = 2(kl + k) & \text{(Factoring)} \\
            & \textcolor{blue}{\text{Let } \alpha \text{ integer} = kl + k} & \text{(Closure of integers renaming $kl + k$)} \\
            & = 2\alpha & \text{(Substitution)} \\
        \end{align*}
        \horizontalline{-2.5}{0.5}
        Consequently, $C = 2\alpha$ with $\alpha$ is an integer, and thus by the \textbf{definition of even} $C$ is even. \qed
    \end{Highlight}
\end{problem}

% Problem 0 Summary
\begin{summary}{Problem 0 Summary}
    \begin{statement}{Procedure}
        \begin{enumerate}[label = (\alph*)]
            \item Part (a)
            \begin{itemize}
                \item Use the definitions of odd and even and make a definition with the use of closure
                \item Use the respective properties, make substitutions, and arrive at the correct conclusion
            \end{itemize}
            \item Part (b)
            \begin{itemize}
                \item Same as part (a)
            \end{itemize}
            \item Part (c)
            \begin{itemize}
                \item Same as parts (a) and (b)
            \end{itemize}
            \item Part (d)
            \begin{itemize}
                \item Same as parts (a)-(c)
            \end{itemize}
        \end{enumerate}
    \end{statement}
    \begin{statement}{Key Concepts}
        \begin{itemize}
            \item These proofs are done with the method of the direct proof format
            \item These proofs use the definition of even ($n = 2k$) and the definition of odd ($n = 2k + 1$)
            \item These proofs use simplification by addition, subtraction, and multiplication
            \item These proofs use properties such as associativity, commutativity, and distributivity, and closure properties
        \end{itemize}
    \end{statement}
    \begin{statement}{Variations}
        \begin{itemize}
            \item We could be asked to prove a different theorem with different initial values
            \begin{itemize}
                \item In this case we would use the same procedure but with a different initial theorem
            \end{itemize}
        \end{itemize}
    \end{statement}
\end{summary}

% Problem 1
\begin{problem}{Problem 1}
    \begin{statement}{Problem Statement}
        Prove: For all positive integers, $A$, $B$, if $A$ and $B$ are odd, $A \cdot B = C$ is odd.
    \end{statement}

    \begin{Highlight}[Solution]
        \textbf{Theorem:} For all positive integers, $A$, $B$, if $A$ and $B$ are odd, $A \cdot B = C$ is odd. \vspace*{1em}

        \textbf{Direct Proof:} \newline
        \horizontalline{-1.5}{-1.5}
        \begin{align*}
            \text{There is $k$ such that } A & = 2k + 1 & \text{(By definition of odd)} \\
            \text{There is $l$ such that } B & = 2l + 1 & \text{(By definition of odd)} \\
            C & = A \cdot B & \text{(Premise)} \\
            & = (2k + 1)(2l + 1) & \text{(Substitution)} \\
            & = (2k + 1)(2l) + (2k + 1)(1) & \text{(Distributivity)} \\
            & = (2l)(2k + 1) + (1)(2k + 1) & \text{(Commutativity)} \\
            & = (2l)(2k) + (2k)(1) + (1)(2k) + (1)(1) & \text{(Distributivity)} \\
            & = (2)(2)(k)(l) + (2k)(1) + (2k)(1) + (1)(1) & \text{(Commutativity)} \\
            & = (4)(k)(l) + 2k + 2k + 1 & \text{(Simplification by multiplication)} \\
            & = 4kl + 2k + 2k + 1 & \text{(Associativity)} \\
            & = 4kl + 4k + 1 & \text{(Simplification by addition)} \\
            & = 4(kl + k) + 1 & \text{(Factoring)} \\
            & \textcolor{blue}{\text{Let } \alpha \text{ integer } = kl + k} & \text{(Closure by renaming $kl + k$)} \\
            & = 4\alpha + 1 & \text{(Substitution)} \\
            & = 2(2\alpha) + 1 & \text{(Factoring)} \\
            & \textcolor{blue}{\text{Let } \beta \text{ integer } = 2\alpha} & \text{(Closure by renaming $2\alpha$)} \\
            & = 2\beta + 1 & \text{(Substitution)} \\
        \end{align*}
        \horizontalline{-2.5}{0.5}
        Consequently, $C$ (the quantity $A\cdot B$) with $\beta$ is an integer, and thus by the \textbf{definition of odd} $C$ is \textbf{odd}. \qed
    \end{Highlight}
\end{problem}

% Problem 1 Summary
\begin{summary}{Problem 1 Summary}
    \begin{statement}{Procedure}
        \begin{itemize}
            \item Use the direct proof format
            \item Use the definitions of odd and even along with field axioms of algebra to reach the conclusion
        \end{itemize}
    \end{statement}
    \begin{statement}{Key Concepts}
        \begin{itemize}
            \item This problem uses the direct proof format to prove that a quantity is odd based off of the hypothesis' being odd
        \end{itemize}
    \end{statement}
    \begin{statement}{Variations}
        \begin{itemize}
            \item The hypothesis' could change
            \begin{itemize}
                \item In this case we would have to work through the problem as normal with the new hypothesis'
            \end{itemize}
            \item The conclusion could change
            \begin{itemize}
                \item In this case we would have to show if the conclusion is true or false
            \end{itemize}
        \end{itemize}
    \end{statement}
\end{summary}

% Problem 2
\begin{problem}{Problem 2}
    \begin{statement}{Problem Statement}
        Prove that if $x$ is odd, then $5x + 3$ is even two different ways.

        \begin{enumerate}[label=(\alph*)]
            \item A direct proof by \textbf{applying the Warm Up THMs} - this means just USE the RESULTS of what you have posted above.
            \item Using the \textit{techniques} of the warm up THM and the formal definitions of even and odd. (This will look a lot like your WUT proofs)
        \end{enumerate}
    \end{statement}

    \begin{Highlight}[Solution - Part (a)]
        \textbf{Theorem:} If $x$ is odd, then $5x + 3$ is even. \vspace*{1em}

        \textbf{Direct Proof:} Using the axioms from the warm up theorems. \newline
        \horizontalline{-1.5}{-1.5}
        \begin{align*}
            \text{There exists an integer $k$ such that } 5 & = 2k + 1 & \text{(By definition of odd)} \\
            \text{There exists an integer $l$ such that } x & = 2l + 1 & \text{(By definition of odd)} \\
            \text{There exists an integer $m$ such that } 3 & = 2m + 1 & \text{(By definition of odd)} \\
            & \textcolor{blue}{\text{Let } A = 5 \text { and } B = x \text{ integer}} & \text{(Closure by renaming $5$ and $x$)} \\
            & \textcolor{blue}{\text{Let } C = A \cdot B \text{ integer}} & \text{(Closure by renaming $A \cdot B$)} \\
            \text{Then $C$ is } & \text{\textbf{odd}} & \text{(By WUT \# 6)} \\
            & \textcolor{blue}{\text{Let } D = 3} & \text{(Closure by renaming $3$)} \\
            & \textcolor{blue}{\text{Let } E = C + D \text{ integer}} & \text{(Closure by renaming C + D)} \\
            \text{Then $E$ is } & \text{\textbf{even}} & \text{(By WUT \# 3)} \\
        \end{align*}
        \horizontalline{-2.5}{0.5}
        Consequently, $E$ (the quantity $5x + 3$) an integer, and thus by the \textbf{WUT's} $E$ is \textbf{even}. \qed
    \end{Highlight}

    \begin{Highlight}[Solution - Part (b)]
        \textbf{Theorem:} If $x$ is odd, then $5x + 3$ is even. \vspace*{1em}

        \textbf{Direct Proof:} \newline
        \horizontalline{-1.5}{-1.5}
        \begin{align*}
            \text{There exists an integer $k$ such that } x & = 2k + 1 & \text{(By definition of odd)} \\
            & \textcolor{blue}{\text{Let } C = 5x + 3 \text{ integer}} & \text{(Closure by renaming 5x + 3)} \\
            C & = 5x + 3 & \text{(Premise)} \\
            & = 5(2k + 1) + 3 & \text{(Substitution)} \\
            & = (5)(2k) + (5)(1) + 3 & \text{(Distributivity)} \\
            & = (5)(2)k + 5(1) + 3 & \text{(Associativity)} \\
            & = 10k + 5 + 3 & \text{(Simplification by multiplication)} \\
            & = 10k + 8 & \text{(Simplification by addition)} \\
            & = 2(5k + 4) & \text{(Factoring)} \\
            & \textcolor{blue}{\text{Let } \alpha = 5k + 4 \text{ integer}} & \text{(Closure by renaming $5k + 4$)} \\
            & = 2\alpha & \text{(Substitution)} \\
        \end{align*}
        \horizontalline{-2.5}{0.5}
        Consequently, $C$ (the quantity $5x + 3$) with $\alpha$ is an integer, and thus by the \textbf{definition of even} $C$ is \textbf{even}. \qed
    \end{Highlight}
\end{problem}

% Problem 2
\begin{summary}{Problem 2 Summary}
    \begin{statement}{Procedure}
        \begin{enumerate}[label = (\alph*)]
            \item Part (a)
            \begin{itemize}
                \item Use the results from the warm up theorems to show that the theorem is true
            \end{itemize}
            \item Part (b)
            \begin{itemize}
                \item Use the same procedure as the warm up theorems with the field axioms to show that the theorem is true
            \end{itemize}
        \end{enumerate}
    \end{statement}
    \begin{statement}{Key Concepts}
        \begin{itemize}
            \item This problem uses the direct proof format and field axioms to prove that a theorem is true
        \end{itemize}
    \end{statement}
    \begin{statement}{Variations}
        \begin{itemize}
            \item We could be given a different theorem to prove
            \begin{itemize}
                \item We would use the same procedure as the warm up theorems to show that the theorem is true
            \end{itemize}
            \item The hypothesis could change
            \begin{itemize}
                \item In this case we would still use the same procedure as the warm up theorems to prove or disprove the theorem
            \end{itemize}
        \end{itemize}
    \end{statement}
\end{summary}

% Problem 3
\begin{problem}{Problem 3}
    \begin{statement}{Problem Statement}
        Prove using a proof by contrapositive. (Be sure to use the right format) \vspace*{1em}

        If $3n + 7$ is odd, then $n$ is even ($n$ is an integer). Why does this work, when a direct proof did not?
    \end{statement}

    \begin{Highlight}[Solution]
        \textbf{Theorem:} If $3n + 7$ is odd, then $n$ is even ($n$ is an integer) \vspace*{1em}

        \textbf{Contrapositive Theorem:} If $n$ is odd, then $3n + 7$ is even. \vspace*{1em}

        \textbf{Contrapositive Proof:} \newline
        \horizontalline{-1.5}{-1.5}
        \begin{align*}
            \text{There exists an integer $k$ such that } n & = 2k + 1 & \text{(By definition of odd)} \\
            & \textcolor{blue}{\text{Let } C = 3n + 7 \text{ integer}} & \text{(Closure by renaming $3n + 7$)} \\
            C & = 3n + 7 & \text{(Premise)} \\
            & = 3(2k + 1) + 7 & \text{(Substitution)} \\
            & = (3)(2k) + (3)(1) + 7 & \text{(Distributivity)} \\
            & = (3)(2)k + 3(1) + 7 & \text{(Associativity)} \\
            & = 6k + 3 + 7 & \text{(Simplification by multiplication)} \\
            & = 6k + 10 & \text{(Simplification by addition)} \\
            & = 2(k + 5) & \text{(Factoring)} \\
            & \textcolor{blue}{\text{Let } \alpha = k + 5 \text{ integer}} & \text{(Closure by renaming $k + 5$)} \\
            & = 2\alpha & \text{(Substitution)} \\
        \end{align*}
        \horizontalline{-2.5}{0.5}
        Consequently, $C$ (the quantity $3n + 7$) with $\alpha$ is an integer, and thus by the \textbf{definition of even} $C$ is \textbf{even}. \qed
    \end{Highlight}

    \begin{Highlight}[Insights]
        Proving this theorem by using the \textit{contrapositive} statement is easier than doing a direct proof because of the algebra that will happen at some point while attempting a direct proof.
        There will come a time in the direct proof process where we are not currently equipped with the tools to proceed. Since the \textit{contrapositive} statement is equivalent in truth value to
        the original statement, we know that proving the \textit{contrapositive} will suffice.
    \end{Highlight}
\end{problem}

\begin{summary}{Problem 3 Summary}
    \begin{statement}{Procedure}
        \begin{itemize}
            \item Because the theorem is written in a difficult way, we would need to use a contrapositive statement to simplify the statement
            \item State the contrapositive of the statement
            \item Use the same procedure from here as the warm up theorems to prove the contrapositive statement
        \end{itemize}
    \end{statement}
    \begin{statement}{Key Concepts}
        \begin{itemize}
            \item The theorem is written in a way that makes proving it difficult with a direct proof
            \item To prove this theorem, we use a contrapositive proof by first stating the contrapositive
            \item Once the contrapositive statement has been stated, we then apply a direct proof on the contrapositive statement
            \item Contrapositive statements have the same truth value as the original statement
            \item Because the contrapositive statement has the same truth value as the original statement, we consequently prove the original statement by proving the contrapositive
        \end{itemize}
    \end{statement}
    \begin{statement}{Variations}
        \begin{itemize}
            \item We could be given a different theorem that requires a contrapositive proof
            \begin{itemize}
                \item In this case we would use the same procedure but with a different theorem
            \end{itemize}
        \end{itemize}
    \end{statement}
\end{summary}

% Problem 4
\begin{problem}{Problem 4}
    \begin{statement}{Problem Statement}
        Prove for positive integers, if $n^{2}$ is even, then $n$ is even.
    \end{statement}

    \begin{Highlight}[Solution]
        \textbf{Theorem:} For positive integers, if $n^{2}$ is even, then $n$ is even. \vspace{1em}

        \textbf{Contrapositive Theorem:} For positive integers, if $n$ is odd, then $n^{2}$ is odd. \vspace*{1em}

        \textbf{Contrapositive Proof:} \newline
        \horizontalline{-1.5}{-1.5}
        \begin{align*}
            \text{There exists an integer } k & \text{ such that } n = 2k + 1 & \text{(By definition of odd)} \\
            & \textcolor{blue}{\text{Let } C = n^{2} \text{ integer}} & \text{(Closure by renaming $n^{2}$)} \\
            C & = n^{2} & \text{(Premise)} \\
            & = (2k + 1)^{2} & \text{(Substitution)} \\
            & = (2k + 1)(2k + 1) & \text{(By definition of exponents)} \\
            & = (2k + 1)(2k) + (2k + 1)(1) & \text{(Distributivity)} \\
            & = (2k)(2k) + (1)(2k) + (2k)(1) + (1)(1) & \text{(Distributivity)} \\
            & = (2)(2)(k)(k) + (2k)(1) + (2k)(1) + (1)(1) & \text{(Commutativity)} \\
            & = 4(k)(k) + 2k + 2k + 1 & \text{(Simplification by multiplication)} \\
            & = 4(k)(k) + 4k + 1 & \text{(Simplification by addition)} \\
            & = 4k^{2} + 4k + 1 & \text{(By definition of exponents)} \\
            & = 2(2k^{2} + 2k) + 1 & \text{(Factoring)} \\
            & \textcolor{blue}{\text{Let } \alpha = 2k^{2} + 2k \text{ integer}} & \text{(Closure by renaming $2k^{2} + 2k$)} \\
            & = 2\alpha + 1 & \text{(Substitution)} \\
        \end{align*}
        \horizontalline{-2.5}{0.5}
        Consequently, $C$ (the quantity $n^{2}$) with $\alpha$ is an integer, and thus by the \textbf{definition of odd} $C$ is \textbf{odd}. \qed 

        \vspace*{1em}

        This proof employs a contrapositive strategy to demonstrate that if $n$ is an odd positive integer, then $2n^2$ is also odd. It begins with the assumption that $n$ is odd, represented as 
        $n = 2k + 1$ for some integer $k$. The goal is to prove that $n^2$ is odd. By renaming variables (letting $C = n^2$ and $\alpha = 2k^2 + 2k$), the proof simplifies the algebraic expressions. 
        The critical step is to show that $C = 2\alpha + 1$, indicating that $C$ is an odd integer by definition. In conclusion, this contrapositive proof establishes the original theorem's validity, 
        affirming that when $n$ is an odd positive integer, $2n^2$ is also odd.
    \end{Highlight}
\end{problem}

% Problem 4 Summary
\begin{summary}{Problem 4 Summary}
    \begin{statement}{Procedure}
        \begin{itemize}
            \item Employ a contrapositive proof for this theorem
            \item Use the field axioms and the warm up theorem procedure for proving the contrapositive statement
        \end{itemize}
    \end{statement}
    \begin{statement}{Key Concepts}
        \begin{itemize}
            \item Similar to problem (3), this theorem is stated in a way that cannot be proven easily with a direct proof
            \item We use a contrapositive proof to prove the theorem because it is easier than a direct proof
            \item We use field axioms and other properties to prove the theorem
        \end{itemize}
    \end{statement}
    \begin{statement}{Variations}
        \begin{itemize}
            \item We could be given a different theorem of the same format
            \begin{itemize}
                \item This would require the same procedure as the original theorem but with a new theorem
            \end{itemize}
        \end{itemize}
    \end{statement}
\end{summary}

% Problem 5
\begin{problem}{Problem 5}
    \begin{statement}{Problem Statement}
        Prove that $13n + 3$ is even if and only if $n$ is odd. $n$ is an integer. (Must prove both directions, why?)
    \end{statement}

    \begin{Highlight}[Solution]
        \textbf{Theorem:} $13n + 3$ is even if and only if $n$ is odd. $n$ is an integer. \vspace*{1em}

        \textbf{Contrapositive Theorem:} If $n$ is even, then $13n + 3$ is odd. \vspace*{1em}

        \textbf{Contrapositive Proof:} \newline
        \horizontalline{-1.5}{-1.5}
        \begin{align*}
            \text{There exists an integer } k & \text{ such that } n = 2k & \text{(By definition of even)} \\
            & \textcolor{blue}{\text{Let } C = 13n + 3 \text{ integer}} & \text{(Closure by renaming $13n + 3$)} \\
            C & = 13n + 3 & \text{(Premise)} \\
            & = 13(2k) + 3 & \text{(Substitution)} \\
            & = (13)(2)k + 3 & \text{(Associativity)} \\
            & = 26k + 3 & \text{(Simplification by multiplication)} \\
            & = 26k + 2 + 1 & \text{(Addition)} \\
            & = 2(13k + 1) + 1 & \text{(Factoring)} \\
            & \textcolor{blue}{\text{Let } \alpha = 13k + 1 \text{ integer}} & \text{(Closure by renaming $13k + 1$)} \\
            & = 2\alpha + 1 & \text{(Substitution)}
        \end{align*}
        \horizontalline{-1}{0.5}
        Consequently, $C$ (the quantity $13n + 3$) with $\alpha$ is an integer, and thus by the \textbf{definition of odd} $C$ is \textbf{odd}. \qed

        \vspace*{1em}
    
        \textbf{Theorem:} If $n$ is odd then $13n + 3$ is even. \vspace*{1em}
    
        \textbf{Direct Proof:} \newline
        \horizontalline{-1.5}{-1.5}
        \begin{align*}
            \text{There exists an integer } k & \text{ such that } n = 2k + 1 & \text{(By definition of even)} \\
            & \textcolor{blue}{\text{Let } C = 13n + 3 \text{ integer}} & \text{(Closure by renaming 13n + 3)} \\
            C & = 13n + 3 & \text{(Premise)} \\
            & = 13(2k + 1) + 3 & \text{(Substitution)} \\
            & = (13)(2)(k) + (13)(1) + 3 & \text{(Distributivity)} \\
            & = (13)(2)k + (13)(1) + 3 & \text{(Associativity)} \\
            & = 26k + 13 + 3 & \text{(Simplification by multiplication)} \\
            & = 26k + 16 & \text{(Simplification by addition)} \\
            & = 2(k + 13) & \text{(Factoring)} \\
            & \textcolor{blue}{\text{Let } \alpha = k + 13 \text{ integer}} & \text{(Closure by renaming $k + 13$)} \\
            & = 2\alpha & \text{(Substitution)} \\
        \end{align*}
        \horizontalline{-1}{0.5}
        Consequently, $C$ (the quantity $13n + 3$) with $\alpha$ is an integer, and thus by \textbf{definition of even} $C$ is \textbf{even}. \qed \vspace*{1em}

        This proof must be done both ways because the theorem that is presented in the problem statement is a bi-conditional statement.
    \end{Highlight}
\end{problem}

% Problem 5 Summary
\begin{summary}{Problem 5 Summary}
    \begin{statement}{Procedure}
        \begin{itemize}
            \item State the contrapositive of the original statement
            \item Prove the contrapositive statement for both directions to prove the parity of the statement
        \end{itemize}
    \end{statement}
    \begin{statement}{Key Concepts}
        \begin{itemize}
            \item This is a biconditional statement which means that it must be proven in both ways
            \item Biconditional statements that are written in this way must be proven in both directions
            \item This particular statement uses a contrapositive proof that makes the statement easier to prove
        \end{itemize}
    \end{statement}
    \begin{statement}{Variations}
        \begin{itemize}
            \item We could be given a different theorem
            \begin{itemize}
                \item In this case we would use the same procedure but with the new theorem
            \end{itemize}
        \end{itemize}
    \end{statement}
\end{summary}

% Problem 6
\begin{problem}{Problem 6}
    \begin{statement}{Problem Statement}
        Assume the domain of positive integers. Prove if $n = ab$, then $a \leq \sqrt{n}$ or $b \leq \sqrt{n}$. Prove by contradiction (using required steps).
    \end{statement}

    \begin{Highlight}[Solution]
        \textbf{Theorem:} If $n = ab$, then $a \leq \sqrt{n}$ or $b \leq \sqrt{n}$ \vspace*{1em}

        \textbf{Negation:} There exists a positive integer $a$ and there exists a positive integer $b$ such that $n = ab$ and $a > \sqrt{n}$ and $b > \sqrt{n}$. \vspace*{1em}

        \textbf{Proof By Contradiction:} \newline
        \horizontalline{-1.5}{-1.5}
        \setcounter{equation}{0}
        \begin{align}
            aa & = a^{2} & \text{(Definition of exponents)} \\
            bb & = b^{2} & \text{(Definition of exponents)} \\
            (n^{\frac{1}{2}})^2 & = n & \text{(Law of exponents)} \\
            a & > \sqrt{n} & \text{(Premise)} \\
            a^{2} & > n & \text{(Squaring both sides)} \\
            a^{2} & > ab & \text{(Substitution)} \\
            a & > b & \text{(Simplification by division)} \\
            b & > \sqrt{n} & \text{(Premise)} \\
            b^{2} & > n & \text{(Squaring both sides)} \\
            b^{2} & > ab & \text{(Substitution)} \\
            b & > a & \text{(Simplification by division)}
        \end{align}
        \horizontalline{-1}{0.5}
        From lines (7) and (11) we have a contradiction of $a > b$ and $b > a$, therefore if $n = ab$, then $a \leq \sqrt{n}$ or $b \leq \sqrt{n}$. \qed
    \end{Highlight}
\end{problem}

% Problem 6 Summary
\begin{summary}{Problem 6 Summary}
    \begin{statement}{Procedure}
        \begin{itemize}
            \item State the negation of the original statement
            \item Perform a proof by contradiction by attempting a direct proof
            \item Arrive at a contradiction and state the contradiction
        \end{itemize}
    \end{statement}
    \begin{statement}{Key Concepts}
        \begin{itemize}
            \item Proofs by contradiction require us to state the negation
            \item We then attempt to prove the negation with a direct proof and arrive at a contradiction
            \item The fact that the negation is false implicates that the original statement is true
        \end{itemize}
    \end{statement}
    \begin{statement}{Variations}
        \begin{itemize}
            \item We could be given a different theorem and be asked to prove it by contradiction
            \begin{itemize}
                \item We would then have to go through the same machinery of how to employ a proof by contradiction
            \end{itemize}
        \end{itemize}
    \end{statement}
\end{summary}

% Problem 7
\begin{problem}{Problem 7}
    \begin{statement}{Problem Statement}
        We define an exponent $b$ of $a$, ($a$ is an element of the positive real numbers, and $b$ is an element of the natural numbers) as:

        \begin{align*}
            a^{b} & \text{ is } a \text{ multiplied by itself } b \text{ times. Example } a^{4} = aaaa \\
            a^{0} & = 1.
        \end{align*}

        We will use only the basic principles of algebra (associativity, commutativity, etc), and the definition of exponents given above to prove the rules of exponents. \vspace*{1em}

        Assume a and b are elements of the positive real numbers. You must explain each step. \vspace*{1em}

        Notice the format of this example proof and follow this structure for \#8 and \#9. Do not `work both sides of the equation.' \vspace*{1em}

        Notice how this proof does not use the rule it is trying to prove. \textbf{For \#7, just fill in the blanks with the correct justification of each step.} \vspace*{1em}

        Prove that $a^{3} \cdot a^{5} = a^{3+5}$ \vspace*{1em}

        \begin{align*}
            a^{3} \cdot a^{5} & = (aaa) \cdot (aaaaa) & \text{Def. of exponents} \\
            & = (aaaaaaa) & \text{`' law} \\
            & = a^{8} & \text{Definition of `'} \\
            & = a^{(3 + 5)} & \text{Substitution using } 8 = 3 + 5
        \end{align*}
    \end{statement}

    \begin{Highlight}[Solution]
        \textbf{Theorem:} $a^{3}\cdot a^{5} = a^{3 + 5}$ \vspace*{1em}

        \textbf{Direct Proof:} \vspace*{1em}

        \horizontalline{-1}{-2}
        \begin{align*}
            a^{3} \cdot a^{5} & = (aaa) \cdot (aaaaa) & \text{Def. of exponents} \\
            & = (aaaaaaa) & \text{\textcolor{blue}{Associativity} law \textcolor{blue}{of multiplication}} \\
            & = a^{8} & \text{Def. of \textcolor{blue}{exponents}} \\
            & = a^{(3 + 5)} & \text{Substitution using } 8 = 3 + 5
        \end{align*}
        \horizontalline{-1}{0}
        Consequently, the quantity $a^{3}\cdot a^{5}$ is equal to that of $a^{3 + 5}$. \qed
    \end{Highlight}
\end{problem}

% Problem 7 Summary
\begin{summary}{Problem 7 Summary}
    \begin{statement}{Procedure}
        \begin{itemize}
            \item Fill in the blanks with their respective properties
        \end{itemize}
    \end{statement}
    \begin{statement}{Key Concepts}
        \begin{itemize}
            \item This direct proof uses rules of exponents and the field axioms to prove a theorem
        \end{itemize}
    \end{statement}
    \begin{statement}{Variations}
        \begin{itemize}
            \item We could be given a different proof to fill in the blanks
            \begin{itemize}
                \item We would then fill in the correct properties for each blank
            \end{itemize}
        \end{itemize}
    \end{statement}
\end{summary}

% Problem 8
\begin{problem}{Problem 8}
    \begin{statement}{Problem Statement}
        Prove that $(a^{(3)})^{5} = a^{3\cdot5}$.
    \end{statement}

    \begin{Highlight}[Solution]
        \textbf{Theorem:} $(a^{3})^{5} = a^{3\cdot 5}$ \vspace*{1em}

        \textbf{Direct Proof:} \vspace*{1em}

        \horizontalline{-1}{-2}
        \begin{align*}
            (a^{(3)})^{5} & = (aaa)^{5} & \text{(Definition of exponents)} \\
            & \textcolor{blue}{\text{Let } \alpha = aaa} & \text{(Closure by renaming $aaa$)} \\
            & = (\alpha)^{5} & \text{(Substitution)} \\
            & = \alpha \cdot \alpha \cdot \alpha \cdot \alpha \cdot \alpha & \text{(Definition of exponents)} \\
            & = (aaa) \cdot (aaa) \cdot (aaa) \cdot (aaa) \cdot (aaa) & \text{(Substitution)} \\
            & = aaaaaaaaaaaaaaa & \text{(Associativity)} \\
            & = a^{15} & \text{(Definition of exponents)} \\
            & = a^{3\cdot 5} & \text{(Substitution using $15 = 3 \cdot 5$)}
        \end{align*}
        \horizontalline{-1}{0}
        Consequently, the quantity $(a^{(3)})^{5}$ is equal to that of $a^{3 \cdot 5}$. \qed
    \end{Highlight}
\end{problem}

% Problem 8 Summary
\begin{summary}{Problem 8 Summary}
    \begin{statement}{Procedure}
        \begin{itemize}
            \item Use the definition of exponents to simplify the premise
            \item Use a direct proof
            \item Use closure, field axioms, and definition of exponents to prove the theorem
        \end{itemize}
    \end{statement}
    \begin{statement}{Key Concepts}
        \begin{itemize}
            \item This problem uses the rules of exponents to prove an exponent property
            \item Exponent rules are a consequence of the field axioms
        \end{itemize}
    \end{statement}
    \begin{statement}{Variations}
        \begin{itemize}
            \item We could be given a different theorem with using exponent rules
            \begin{itemize}
                \item We would then use exponent rules and field axioms to prove the new theorem
            \end{itemize}
        \end{itemize}
    \end{statement}
\end{summary}

% Problem 9
\begin{problem}{Problem 9}
    \begin{statement}{Problem Statement}
        Prove that $a^{5}b^{5} = (ab)^{5}$.
    \end{statement}

    \begin{Highlight}[Solution]
        \textbf{Theorem:} $a^{5}b^{5} = (ab)^{5}$ \vspace*{1em}

        \textbf{Direct Proof:} \vspace*{1em}

        \horizontalline{-1}{-2}
        \begin{align*}
            a^{5}b^{5} & = (aaaaa)(bbbbb) & \text{(Definition of exponents)} \\
            & = aaaaabbbbb & \text{(Associativity)} \\
            & = abaaaabbbb & \text{(Commutativity)} \\
            & = ababaaabbb & \text{(Commutativity)} \\
            & = abababaabb & \text{(Commutativity)} \\
            & = ababababab & \text{(Commutativity)} \\
            & \textcolor{blue}{\text{Let } \alpha = ab} & \text{(Closure by renaming $ab$)} \\
            & = \alpha \alpha \alpha \alpha \alpha & \text{(Substitution)} \\
            & = \alpha^{5} & \text{(Definition of exponents)} \\
            & = (ab)^{5} & \text{(Substitution)}
        \end{align*}
        \horizontalline{-1}{0}
        Consequently, the quantity $a^{5}b^{5}$ is equal to that of $(ab)^{5}$. \qed
    \end{Highlight}
\end{problem}

% Problem 9 Summary
\begin{summary}{Problem 9 Summary}
    \begin{statement}{Procedure}
        \begin{itemize}
            \item Use the definition of exponents to simplify the premise
            \item Use a direct proof
            \item Use closure, field axioms, and definition of exponents to prove the theorem
        \end{itemize}
    \end{statement}
    \begin{statement}{Key Concepts}
        \begin{itemize}
            \item This problem uses the rules of exponents to prove an exponent property
            \item Exponent rules are a consequence of the field axioms
        \end{itemize}
    \end{statement}
    \begin{statement}{Variations}
        \begin{itemize}
            \item We could be given a different theorem with using exponent rules
            \begin{itemize}
                \item We would then use exponent rules and field axioms to prove the new theorem
            \end{itemize}
        \end{itemize}
    \end{statement}
\end{summary}

% Problem 10
\begin{problem}{Problem 10}
    \begin{statement}{Problem Statement}
        Considering the above, when we add the definition of exponents to the algebra, are the `“rules”' of exponents really a new idea or just a consequence of the existing structure of algebra? Give 
        a thoughtful and complete answer.
    \end{statement}

    \begin{Highlight}[Solution]
        The rules of exponents are just a consequence of the existing structure of algebra. Since, in some senses, math is a pattern, these rules emerge from finding a more concise way of representing
        repeated multiplication. The rules of exponents also obey the field axioms, this can be seen below

        \begin{align*}
            \alpha & = a^{n} & \text{(Closure)} \\
            a^{0} & = 1 & \text{(Identity)} \\
            a^{-n} & = \frac{1}{a^{n}} & \text{(Inverse)} \\
            (a^{m})^{n} & = a^{mn} & \text{(Associativity)} \\
            a^{m}\cdot a^{n} & = a^{n}\cdot a^{m} & \text{(Commutativity)} \\
            a^{m}\cdot (a^{n} + b^{n}) & = a^{m} \cdot a^{n} + a^{m} \cdot b^{n} & \text{(Distributivity)}.
        \end{align*}
        The rules of exponents adhere to the foundation of mathematical structure and adhere to fundamental properties such as the field axioms. The rules of exponents, like stated previously, are a mere
        consequence of recognizing patterns in mathematics to then formally state equivalences in scenarios. One scenario is that of repeated multiplication, this pattern is then recognized and hence the
        rules of exponents have been born.
    \end{Highlight}
\end{problem}

% Problem 10 Summary
\begin{summary}{Problem 10 Summary}
    \begin{statement}{Procedure}
        \begin{itemize}
            \item Answer the prompt
        \end{itemize}
    \end{statement}
    \begin{statement}{Key Concepts}
        \begin{itemize}
            \item Exponent rules are a consequence of the field axioms
        \end{itemize}
    \end{statement}
    \begin{statement}{Variations}
        \begin{itemize}
            \item We could be given a different prompt to respond to
            \begin{itemize}
                \item We then would answer the prompt
            \end{itemize}
        \end{itemize}
    \end{statement}
\end{summary}