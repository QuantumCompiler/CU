\clearpage
\chapter{Practice Mastery Workbook}

% Chapter page
\section{Workbook Description}

\horizontalline

The purpose of this workbook is to get practice with submitting an assignment to grade scope. I will be taking the \LaTeX \hspace{1pt} route of doing assignments, as you might see I am
a huge fan of it :).

% Question 1
\problem{Question 1}{
    \begin{Highlight}[What would you like me to know about you? Jobs? Hobbies? Family?]
        My name is Taylor Larrechea and this is my second semester in CU's Post Bacc Applied Computer Science program. I obtained my first bachelors degree from Colorado Mesa University
        in the spring of 2020 where I earned a Bachelors of Science in Physics and a minor in Mathematics. I had a brief stint in graduate school at the University of Oklahoma where I was
        a PhD student who was aiming to be a theoretical physicist in the concentration of Atomic, Molecular, and Optical physics. In my one year of grad school I quickly realized that I 
        did not want to spend the rest of my life studying physics, but rather, programming (most of my research was numerical and the coding was the only thing I actually enjoyed). After
        my graduate school experiment, I managed to land a job with a semi conductor company called Applied Materials where I am a customer engineer who works on defect metrology equipment
        that is currently being used at Intel's research and development fabrication center here in Hillsboro Oregon.

        I have a strong mathematical background, ranging from calculus and differential equations to numerical analysis and mathematical modeling. I have actually already taken Discrete 
        Structures from CMU but my credit did not transfer when I entered this program. But I have heard amazing things about professor Stade and I welcome the opportunity to build a relationship
        with you and hopefully learn a lot in the process.

        At the completion of this program, I am hoping to potentially land a programming / developing position that will allow me to move back home to Colorado so that I can be closer to my
        family and lifelong friends. And if I cannot find a job easily after completing this program, I plan to potentially return to graduate school where I hope to get a PhD in Computer 
        Science and also potentially finish my masters degree in Physics.
    \end{Highlight}
}

% Question 2
\problem{Question 2} {
    \begin{Highlight}[When did you last take a math class? How did it go?]
        The last formal math class that I took was Numerical Analysis at CMU. To be candid, the class was not difficult for me at all. Beyond this class, I have taken multiple math classes from
        Calculus 1 up to Calculus 3, Differential Equations and Linear Algebra, Fourier Analysis, Mathematical Modeling, Applied Mathematics courses, and Discrete Structures as well. My senior 
        research that I did while at CMU was focused on Quantum Parameter Estimation which was littered with Linear Algebra and a tinge of Calculus. In my one year of graduate school in Physics, 
        I took a Methods of Applied Mathematics in Physics course, which was absolutely hell (like most graduate level physics courses are). Surprisingly, I was able to obtain in A in that course 
        and I look forward to expanding my mathematical capabilities from what I will potentially learn in this course.
    \end{Highlight}
}

% Question 3
\problem{Question 3} {
    \begin{Highlight}[Have you ever spent over an hour on one math problem...and enjoyed it?]
        There have been countless problems that I have spent doing a math problem that exceeded an hour. I distinctly remember doing an electron transmission problem in Quantum that took about 13 
        hours to do on its own. The rest of the assignment took me about 30 hours to complete as it was the most difficult module in the course that semester.

        In my one year of graduate school, I can't think of a single problem that I completed that took me under an hour. There were several exam problems that took me more than an hour to do (and 
        that was without a doubt not enjoyable, mainly because of the stress that was involved). Electromagnetism is my favorite academic subject as it is very mathematically rigorous and conceptually
        easy. When doing E\&M problems, the time taken to complete a problem made me feel very proud with the effort that was put in to solve said problem. So to answer if I enjoyed it; there have been
        times when I did enjoy it and others where I absolutely loathed it.
    \end{Highlight}
}

% Question 4
\problem{Question 4} {
    \begin{Highlight}[To you, is solving math problem more like following a recipe or solving a puzzle?]
        Doing math problems is like being able to differentiate a tree from a forest. You first have to observe the forest and what types of trees that you are working with. Once this has been established
        you can continue to observer the smaller details (like what type of trees you are looking at) and then decide what tools you need to perform whatever action you are working on. Once the tools are
        in your hand, using the tools becomes algorithmic and repetitive, much like following a recipe.

        So to summarize, it is a bit of both. The solving a puzzle aspect comes into play when you are trying to figure out what it even is that you are trying to solve. Once you know what you are trying to
        solve, you use tools much like you would follow a recipe to work out the smaller details.
    \end{Highlight}
}

% Question 5
\problem{Question 5} {
    \begin{Highlight}[If you win the lottery, what will you do? Do you have a plan?]
        If I won the lottery I would absolutely quit my job the instant that I found out. I would pay off all of my immediate families debts and I would fund my sisters education expenses as she went to Law
        School. As for myself, I would probably become a life long academic in the sense that I would proceed to acquire as many degrees as possible because my favorite thing to do is to learn new things.
    \end{Highlight}
}