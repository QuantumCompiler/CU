\clearpage

\renewcommand{\ChapTitle}{Combinatorics And Binomial}
\renewcommand{\SectionTitle}{Combinatorics And Binomial}

\chapter{\ChapTitle}
\section{\SectionTitle}
\horizontalline{0}{0}

\subsection{Assigned Reading}

The reading assignments for this week can be found below:

\begin{itemize}
    \item \textbf{Sections 6.1, 6.2, 6.3, 6.4, 6.5}
\end{itemize}

\subsection{Piazza}

Must post / respond to at least \textbf{two} Piazza posts this week. \due{(11/10/23)} \cbox{piazza-week11}

\subsection{Lectures}

The lectures for this week and their links can be found below:

\begin{itemize}
    \item \href{https://applied.cs.colorado.edu/mod/hvp/view.php?id=51784}{Product Rule} $\approx$ 10 min.
    \item \href{https://applied.cs.colorado.edu/mod/hvp/view.php?id=51785}{Product Rule Examples} $\approx$ 14 min.
    \item \href{https://applied.cs.colorado.edu/mod/hvp/view.php?id=51786}{Sum Rule} $\approx$ 7 min.
    \item \href{https://applied.cs.colorado.edu/mod/hvp/view.php?id=51787}{Sum + Product Rule Examples} $\approx$ 14 min.
    \item \href{https://applied.cs.colorado.edu/mod/hvp/view.php?id=51788}{Inclusion-Exclusion Principle} $\approx$ 6 min.
    \item \href{https://applied.cs.colorado.edu/mod/hvp/view.php?id=51789}{Division Rule} $\approx$ 16 min.
    \item \href{https://applied.cs.colorado.edu/mod/hvp/view.php?id=51791}{Pigeon Hole Principle} $\approx$ 11 min.
    \item \href{https://applied.cs.colorado.edu/mod/hvp/view.php?id=51792}{Generalized PHP} $\approx$ 7 min.
    \item \href{https://applied.cs.colorado.edu/mod/hvp/view.php?id=51794}{Permutations And Combinations: An Introduction} $\approx$ 14 min.
    \item \href{https://applied.cs.colorado.edu/mod/hvp/view.php?id=51795}{Counting Permutations} $\approx$ 10 min.
    \item \href{https://applied.cs.colorado.edu/mod/hvp/view.php?id=51796}{Counting Combinations} $\approx$ 13 min.
    \item \href{https://applied.cs.colorado.edu/mod/hvp/view.php?id=51797}{Permutations Of Repeated Objects} $\approx$ 11 min.
    \item \href{https://applied.cs.colorado.edu/mod/hvp/view.php?id=51799}{Binomial Theorem} $\approx$ 6 min.
    \item \href{https://applied.cs.colorado.edu/mod/hvp/view.php?id=51800}{Pascal's Triangle} $\approx$ 9 min.
    \item \href{https://applied.cs.colorado.edu/mod/hvp/view.php?id=51801}{Sum Of Binomial Coefficients} $\approx$ 6 min.
    \item \href{https://applied.cs.colorado.edu/mod/hvp/view.php?id=51803}{Proofs Using Pigeon-Hole Principle} $\approx$ 7 min.
    \item \href{https://applied.cs.colorado.edu/mod/hvp/view.php?id=51804}{Slick Proofs Using Pigeon Hole Principle} $\approx$ 9 min.
    \item \href{https://applied.cs.colorado.edu/mod/hvp/view.php?id=51805}{Ramsey's Theorem} $\approx$ 11 min.
    \item \href{https://applied.cs.colorado.edu/mod/hvp/view.php?id=51806}{Proof Of Binomial Theorem} $\approx$ 8 min.
    \item \href{https://applied.cs.colorado.edu/mod/hvp/view.php?id=51807}{Counting Solutions To Sum Of Variables - Part 1} $\approx$ 6 min.
    \item \href{https://applied.cs.colorado.edu/mod/hvp/view.php?id=51808}{Counting With Sum Of Variables - Part 2} $\approx$ 3 min.
    \item \href{https://applied.cs.colorado.edu/mod/hvp/view.php?id=51809}{Counting With Repetitions} $\approx$ 8 min.
    \item \href{https://applied.cs.colorado.edu/mod/hvp/view.php?id=51810}{Counting With Repetitions : Unordered} $\approx$ 11 min.
    \item \href{https://applied.cs.colorado.edu/mod/hvp/view.php?id=51811}{Further Properties Of Binomial Coefficients} $\approx$ 7 min.
\end{itemize}

\subsection{Assignments}

The assignment for this week is \pdflink{\AssDir Assignment 9 - Counting And Binomials.pdf}{Assignment 9 - Counting And Binomials} \due{(11/13/23)} \cbox{assignment-week11}

\subsection{Quiz}

The quiz's for this week can be found at \href{https://applied.cs.colorado.edu/mod/quiz/view.php?id=51813}{Counting Quiz} \textbullet \pdflink{\QuizDir Quiz 10 - Counting.pdf}{Quiz 10 - Counting} \due{(11/13/23)} \cbox{quiz-week11}

\subsection{Chapter Summary}

The first section that we are covering this week is \textbf{Section 6.1 - The Basics Of Counting}.

\begin{notes}{Section 6.1 - The Basics Of Counting}
    \subsubsection*{Overview}

    This section introduces the fundamental principles of counting which are essential in discrete mathematics and computer science for analyzing algorithms, computing probabilities, and solving 
    enumeration problems. The chapter is divided into various subtopics, each explaining different counting techniques and their applications. \vspace*{1em}
    
    \subsubsection*{Basic Counting Principles}
    The section starts with the basic counting principles:
    \begin{itemize}
        \item The \textit{product rule} explains that if a task can be divided into a sequence of two subtasks that are independent, the total number of ways to perform this task is the product of 
        the number of ways to perform each subtask.
        \item The \textit{sum rule} applies to situations where a task can be performed by one of two mutually exclusive methods.
        \item The \textit{subtraction rule} or inclusion-exclusion principle corrects for overcounting when two or more ways to do a task overlap.
        \item The \textit{division rule} states that if a task can be done in $ n $ ways, and there are $ d $ indistinguishable ways to do it, then there are $ n/d $ distinct ways to do the task.
    \end{itemize}
    
    \subsubsection*{Applications and Examples}
    Numerous examples illustrate the application of these rules, such as assigning offices to employees, creating license plates, counting functions from one set to another, and constructing variable 
    names in programming languages.
    
    \subsubsection*{More Complex Counting Problems}
    The chapter also delves into more complex counting problems that require a combination of the above rules, such as:
    \begin{itemize}
        \item Counting the number of different passwords that meet specific criteria.
        \item Determining the number of possible IPv4 addresses.
        \item Avoiding overcounting when sets have elements in common.
    \end{itemize}
    
    \subsubsection*{Key Takeaways}
    \begin{itemize}
        \item The principles of counting are essential for solving a wide range of problems in discrete mathematics and computer science.
        \item The product rule, sum rule, subtraction rule, and division rule are the foundational techniques for enumerative combinatorics.
        \item Examples and applications provided in the section highlight the practical significance of these counting principles.
    \end{itemize}
\end{notes}

The next section that we will be covering this week is \textbf{Section 6.2 - The Pigeonhole Principle}.

\begin{notes}{Section 6.2 - The Pigeonhole Principle}
    \subsubsection*{Overview}

    The section discusses the Pigeonhole Principle and its generalized version, which are fundamental concepts in combinatorics. The basic Pigeonhole Principle is introduced with the classic example 
    of pigeons and pigeonholes: If $k+1$ objects are placed into $k$ boxes, then at least one box contains two or more objects.
    
    This principle is formalized in \textbf{Theorem 1: The Pigeonhole Principle}, which states: If $k$ is a positive integer and $k+1$ or more objects are placed into $k$ boxes, then there is at least 
    one box containing two or more of the objects.
    
    The section provides several illustrative examples, demonstrating the principle's application in various scenarios, such as proving that among any group of 367 people, there must be at least two 
    with the same birthday, due to there being only 366 possible birthdays. \vspace*{1em}
    
    \subsubsection*{The Generalized Pigeonhole Principle}
    The Generalized Pigeonhole Principle allows us to determine a minimum threshold that guarantees a certain distribution of objects across containers: If $N$ objects are placed into $k$ boxes, then 
    there is at least one box containing at least $\left\lceil \frac{N}{k} \right\rceil$ objects.
    
    This principle is encapsulated in \textbf{Theorem 2: The Generalized Pigeonhole Principle}. Examples provided in the section apply this theorem to practical problems, such as determining the 
    minimum number of students required in a class to guarantee that at least a certain number receive the same grade. \vspace*{1em}
    
    \subsubsection*{Applications and Elegant Examples}
    The section also explores elegant applications of the Pigeonhole Principle, such as proving the existence of a subsequence within a sequence of real numbers: \textbf{Theorem 3:} Every sequence 
    of $n^2 + 1$ distinct real numbers contains a subsequence of length $n + 1$ that is either strictly increasing or strictly decreasing.
    
    Furthermore, the section touches on Ramsey theory and the existence of certain structures within a graph, as illustrated by the problem of friends and enemies at a party. \vspace*{1em}
    
    \subsubsection*{Key Takeaways}
    \begin{itemize}
        \item The Pigeonhole Principle is a simple yet powerful tool in proving the existence of certain properties in sets and distributions.
        \item The generalized version extends the principle to scenarios where objects outnumber containers by larger multiples.
        \item These principles find practical applications in various fields, such as data transmission, coding theory, and graph theory.
    \end{itemize}
\end{notes}

The next section that we will cover this week is \textbf{Section 6.3 - Permutations And Combinations}.

\begin{notes}{Section 6.3 - Permutations And Combinations}
    \subsubsection*{Overview}

    The section provides a comprehensive discussion on permutations and combinations, fundamental concepts in combinatorics that deal with the arrangement and selection of objects. The section begins 
    by defining permutations and demonstrating how to calculate them using the product rule. \vspace*{1em}
    
    \begin{highlight}[Permutations]
        Permutations are introduced as ordered arrangements of objects, with specific notation $ P(n, r) $ denoting the number of permutations of $ r $ objects from a set of $ n $ distinct 
        objects. The formula for permutations is derived as follows:
        \begin{equation*}
            P(n, r) = n(n - 1)(n - 2) \cdots (n - r + 1)
        \end{equation*}
        This formula is proven using the product rule, which states that the number of ways to perform a sequence of tasks is the product of the number of ways to perform each individual task.
    \end{highlight}

    \begin{highlight}[Combinations]
        Combinations, unlike permutations, are defined as selections where order does not matter. The notation $ C(n, r) $, or $ \binom{n}{r} $, represents the number of ways to choose $ r $ 
        objects from a set of $ n $ distinct objects without considering the order. The formula for combinations is given by:
        \begin{equation*}
            C(n, r) = \frac{n!}{r!(n - r)!}
        \end{equation*}
        This is explained by dividing the number of $ r $-permutations by the $ r! $ ways to order the $ r $ objects, eliminating the consideration of sequence.
    \end{highlight}
    
    \subsubsection*{Key Theorems and Corollaries}
    Important results discussed in the section include:
    \begin{itemize}
        \item \textbf{Theorem 1} establishes the formula for the number of permutations.
        \item \textbf{Theorem 2} presents the formula for combinations, also known as the binomial coefficient.
        \item \textbf{Corollary 2} indicates the symmetry in combinations: $ C(n, r) = C(n, n - r) $.
    \end{itemize}
    
    \subsubsection*{Applications}
    Several applications of permutations and combinations are provided through examples, demonstrating the utility of these concepts in solving practical problems involving arrangement and selection.
    
    \subsubsection*{Key Takeaways}
    \begin{itemize}
        \item Permutations and combinations are crucial for counting the number of possible arrangements or selections in a set.
        \item The difference between permutations and combinations lies in whether the order of selection is important.
        \item These concepts have significant applications in probability, statistics, and various fields where combinatorial optimization is required.
    \end{itemize}
\end{notes}

The last section that we will cover this week is \textbf{Section 6.4 - Binomial Coefficients and Identities}.

\begin{notes}{Section 6.4 - Binomial Coefficients and Identities}
    \subsubsection*{Overview}

    The section covers the fundamental concepts of binomial coefficients and various identities associated with them. Binomial coefficients, denoted by $\binom{n}{r}$, are the coefficients in the 
    expansion of powers of binomial expressions such as $(a + b)^n$. \vspace*{1em}
    
    \begin{highlight}[The Binomial Theorem]
        The Binomial Theorem is introduced as a way to express the expansion of powers of binomial expressions. It states that for any nonnegative integer $ n $ and any variables $ x $ and $ y $, 
        the expansion is given by:
        
        \begin{equation*}
            (x + y)^n = \sum_{j=0}^{n} \binom{n}{j} x^{n-j} y^j
        \end{equation*}
        
        This theorem is proved using a combinatorial argument, which demonstrates the relationship between algebraic expansion and combinatorial counting.
    \end{highlight}
    
    \begin{highlight}[Identities Involving Binomial Coefficients]
        Several identities involving binomial coefficients are presented, including Pascal's Identity, which relates the coefficients in the context of Pascal's Triangle:
        
        \begin{equation*}
            \binom{n+1}{k} = \binom{n}{k-1} + \binom{n}{k}
        \end{equation*}
        
        Other identities include Vandermonde's Identity, which provides a combinatorial method for calculating certain sums of binomial coefficients:
        \begin{equation*}
            \binom{m+n}{r} = \sum_{k=0}^{r} \binom{m}{r-k}\binom{n}{k}
        \end{equation*}
    \end{highlight}
    
    \subsubsection*{Applications and Examples}
    Applications and examples provided in the section include using the Binomial Theorem for expanding algebraic expressions and calculating specific coefficients, demonstrating the practical utility 
    of these mathematical tools.
    
    \subsection*{Key Takeaways}
    \begin{itemize}
        \item Binomial coefficients play a critical role in combinatorics, probability, and algebra.
        \item The Binomial Theorem and associated identities offer a systematic way to work with polynomials and combinatorial quantities.
        \item These concepts provide a deep understanding of the structural properties of binomial expressions and their coefficients.
    \end{itemize}    
\end{notes}