\clearpage

\renewcommand{\ChapTitle}{Relations}
\renewcommand{\SectionTitle}{Relations}

\chapter{\ChapTitle}
\section{\SectionTitle}
\horizontalline{0}{0}

\subsection{Assigned Reading}

The reading assignments for this week can be found below:

\begin{itemize}
    \item \textbf{Sections 9.1, 9.2, 9.3, 9.4, 9.5}
\end{itemize}

\subsection{Piazza}

Must post / respond to at least \textbf{two} Piazza posts this week. \due{(12/8/23)} \cbox{piazza-week15}

\subsection{Lectures}

The lectures for this week and their links can be found below:

\begin{itemize}
    \item \href{https://applied.cs.colorado.edu/mod/hvp/view.php?id=51853}{Basic Relations} $\approx$ 17 min.
    \item \href{https://applied.cs.colorado.edu/mod/hvp/view.php?id=51854}{Properties Of Relations} $\approx$ 20 min.
    \item \href{https://applied.cs.colorado.edu/mod/hvp/view.php?id=51855}{Closures Of Relations} $\approx$ 26 min.
    \item \href{https://applied.cs.colorado.edu/mod/hvp/view.php?id=51856}{Equivalence Relations: Part 1} $\approx$ 16 min.
    \item \href{https://applied.cs.colorado.edu/mod/hvp/view.php?id=51857}{Equivalence Classes} $\approx$ 17 min.
    \item \href{https://applied.cs.colorado.edu/mod/hvp/view.php?id=51858}{Partial Orders} $\approx$ 18 min.
    \item \href{https://applied.cs.colorado.edu/mod/hvp/view.php?id=51859}{Posets And Total Orders} $\approx$ 16 min.
    \item \href{https://applied.cs.colorado.edu/mod/hvp/view.php?id=51860}{Relations And Databases} $\approx$ 12 min.
    \item \href{https://applied.cs.colorado.edu/mod/hvp/view.php?id=51861}{Operations On Relations And Querying Databases} $\approx$ 25 min.
\end{itemize}

\noindent Below is a list of lecture notes for this week:

\begin{itemize}
    \item \pdflink{\LectureNotesDir Basic Relations And Their Representations Lecture Notes.pdf}{Basic Relations And Their Representations Lecture Notes}
    \item \pdflink{\LectureNotesDir Properties Of Relations Lecture Notes.pdf}{Properties Of Relations Lecture Notes}
    \item \pdflink{\LectureNotesDir Closures Of Relations Lecture Notes.pdf}{Closures Of Relations Lecture Notes}
\end{itemize}

\subsection{Assignments}

The assignment for this week is \pdflink{\AssDir Assignment 11 - Relations.pdf}{Assignment 11 - Relations} \due{(12/11/23)} \cbox{assignment-week15}

\subsection{Quiz}

The quiz's for this week can be found at \href{https://applied.cs.colorado.edu/mod/quiz/view.php?id=51863}{Relations Quiz} \textbullet \pdflink{\QuizDir Quiz 13 - Relations.pdf}{Quiz 13 - Relations} \due{(12/11/23)} \cbox{quiz-week15}

\subsection{Chapter Summary}

The first section that we are covering this week is \textbf{Section 9.1 - Relations And Their Properties}.

\begin{notes}{Section 9.1 - Relations And Their Properties}
    \subsubsection*{Overview}
    Binary relations are foundational concepts in discrete mathematics, representing complex relationships in a simple, structured form. They are defined as sets of ordered pairs, where each pair consists 
    of elements from two different sets. This section introduces the basic idea of binary relations through practical examples, such as the relationships between students and courses they are enrolled in, 
    or cities and their corresponding states.
    \begin{itemize}
        \item Binary relations are sets of ordered pairs representing relationships between elements of two sets.
        \item Real-world examples, such as student-course enrollments, illustrate various binary relations.
    \end{itemize}
    
    \subsubsection*{Functions as Relations}
    Functions are a special case of binary relations with unique characteristics. In this subsection, the textbook explains how every function is a relation but not every relation is a function. This 
    distinction is critical and is illustrated through the concept of mapping, where each element in the first set is uniquely paired with an element in the second set.
    \begin{itemize}
        \item Functions are special cases of relations where each element of one set is paired with exactly one element of another set.
    \end{itemize}
    
    \subsubsection*{Properties of Relations}
    This part of the textbook explores key properties of relations: reflexivity, symmetry, antisymmetry, and transitivity. These properties are important for understanding how different types of relations 
    behave and interact. The subsection provides various examples to illustrate these properties, helping to clarify complex abstract concepts with tangible scenarios.
    \begin{itemize}
        \item Key properties of relations include reflexivity, symmetry, antisymmetry, and transitivity.
        \item Examples demonstrate the application of these properties in different contexts.
    \end{itemize}
    
    \subsubsection*{Combining Relations}
    Relations can be combined or modified through various operations like union, intersection, and composition. This subsection explains these operations and their significance in the study of relations. 
    It uses practical examples to demonstrate how these operations are applied, showing how complex relations can be formed from simpler ones.
    \begin{itemize}
        \item Relations can be combined using operations like union, intersection, and composition.
        \item Practical examples demonstrate the application of these operations.
    \end{itemize}
    
    \subsubsection*{Theorem and Exercises}
    The final part of this section includes a theorem regarding the powers of transitive relations, a vital concept in understanding relation properties. To reinforce learning, the subsection offers 
    exercises that provide practical application of the theorem and the concepts discussed in the chapter.
    \begin{itemize}
        \item Includes a theorem related to transitive relations, reinforcing the concepts with exercises for practical understanding.
    \end{itemize}
\end{notes}

The next section that we will be covering this week is \textbf{Section 9.2 - $n$-ary Relations And Their Applications}.

\begin{notes}{Section 9.2 - $n$-ary Relations And Their Applications}
    \subsubsection*{Overview}
    $n$-ary relations extend binary relations to sets involving more than two elements. This section introduces these relations with practical examples like relationships in student databases and 
    geometric points. These concepts are foundational in understanding complex relationships in various fields including computer science.
    \begin{itemize}
        \item Discusses relationships involving elements from more than two sets, such as student name, major, and GPA.
        \item $n$-ary relations are used in databases and examples include relationships in mathematics and geometry.
    \end{itemize}
    
    \subsubsection*{Databases and Relations}
    This subsection delves into how databases utilize $n$-ary relations for efficient data management. Relations are represented as tables, where each tuple corresponds to a record. Examples include 
    student databases, where records are tuples of student details. This part emphasizes the practical application of $n$-ary relations in storing and managing complex data.
    \begin{itemize}
        \item Explains the role of $n$-ary relations in databases, with a focus on their efficiency and utility.
        \item Relations are represented as tables with attributes, illustrated with student record examples.
    \end{itemize}
    
    \subsubsection*{Operations on $n$-ary Relations}
    Various operations such as selection and projection are explored, highlighting how they can be used to extract and manipulate data from $n$-ary relations. Definitions are provided along with examples 
    showing the application of these operations in database queries. This part is crucial for understanding how complex data is processed and retrieved in relational databases.
    \begin{itemize}
        \item Covers operations like selection and projection to manipulate $n$-ary relations for specific queries.
        \item Includes definitions and examples to demonstrate these operations on databases.
    \end{itemize}
    
    \subsubsection*{SQL and $n$-ary Relations}
    The final subsection connects the concepts of $n$-ary relations with SQL, a standard database query language. It demonstrates how SQL commands are used to execute operations on relations, with 
    practical examples showing the translation of theoretical concepts into real-world database queries. This part bridges the gap between theoretical understanding and practical application in database management.
    \begin{itemize}
        \item Demonstrates the use of SQL in implementing operations on $n$-ary relations.
        \item Provides examples to show how SQL commands correspond to relational database operations.
    \end{itemize}
\end{notes}

The next section that we will be covering this week is \textbf{Section 9.3 - Representing Relations}.

\begin{notes}{Section 9.3 - Representing Relations}
    \subsubsection*{Overview}
    This subsection introduces zero-one matrices to represent binary relations. It explains how to construct a matrix for a relation $R$ between finite sets $A$ and $B$. It includes examples to 
    illustrate this representation method.
    \begin{itemize}
        \item Discusses representing binary relations using zero-one matrices.
        \item For a relation $R$ from set $A$ to $B$, matrix $M_R$ has 1 in position $(i,j)$ if $(a_i, b_j) \in R$ and 0 otherwise.
    \end{itemize}
    
    \subsubsection*{Properties of Relations Represented by Matrices}
    This part delves into how properties of relations like reflexivity, symmetry, and antisymmetry are represented in matrices. It details the criteria for identifying these properties using matrix 
    entries.
    \begin{itemize}
        \item Explores matrix characteristics representing reflexivity, symmetry, and antisymmetry in relations.
        \item Relation $R$ is reflexive if all diagonal elements of $M_R$ are 1, symmetric if $M_R$ is symmetric, and antisymmetric if $m_{ij} = 1$ implies $m_{ji} = 0$ for $i \neq j$.
    \end{itemize}
    
    \subsubsection*{Representing Relations Using Digraphs}
    The final subsection discusses representing relations using directed graphs or digraphs. It describes how elements of a set are represented as vertices and relations as directed edges, providing a 
    visual method for understanding binary relations.    
    \begin{itemize}
        \item Introduces directed graphs (digraphs) as a method for representing relations.
        \item In digraphs, elements are vertices and relations are represented as directed edges.
    \end{itemize}
\end{notes}

The next section that we will be covering this week is \textbf{Section 9.4 - Closures Of Relations}.

\begin{notes}{Section 9.4 - Closures Of Relations}
    \subsubsection*{Overview}
    This subsection discusses how to construct closures of relations, which are the smallest relations containing a given relation and having specific properties. The concept of transitive closure is 
    particularly emphasized, with examples showing how to find all pairs in a relation that are indirectly connected.
    \begin{itemize}
        \item Explores how to extend a relation to include certain properties like reflexivity, symmetry, and transitivity.
        \item Concepts like transitive closure and symmetric closure are introduced with examples.
    \end{itemize}
    
    \subsubsection*{Reflexive Closures}
    The reflexive closure of a relation involves adding pairs so that every element is related to itself. The section illustrates this concept with examples, showing how the reflexive closure can be 
    represented mathematically.
    \begin{itemize}
        \item Details the method of adding pairs to make a relation reflexive.
        \item Reflexive closure is represented as $R \cup \Delta$, where $\Delta$ is the diagonal relation.
    \end{itemize}
    
    \subsubsection*{Symmetric Closures}
    In symmetric closures, the textbook explains how to add pairs to a relation to ensure that if $(a, b)$ is in the relation, then $(b, a)$ is also included. The symmetric closure of a 
    relation is illustrated with clear examples.
    \begin{itemize}
        \item Discusses adding pairs to a relation to make it symmetric.
        \item Symmetric closure is represented as $R \cup R^{-1}$, where $R^{-1}$ is the inverse of $R$.
    \end{itemize}
    
    \subsubsection*{Transitive Closures}
    The transitive closure subsection is particularly detailed, explaining how to find all pairs that can be connected through a series of steps within a relation. It provides a comprehensive view of 
    how to construct a transitive closure using paths in directed graphs and matrix representation.
    \begin{itemize}
        \item Addresses the concept of adding pairs to make a relation transitive.
        \item Transitive closure involves finding all pairs that can be connected through intermediate steps in a relation.
    \end{itemize}
\end{notes}

The last section that we will be covering this week is \textbf{Section 9.5 - Equivalence Relations}.

\begin{notes}{Section 9.5 - Equivalence Relations}
    \subsubsection*{Overview}
    This subsection defines equivalence relations and illustrates them with practical examples. It emphasizes their foundational role in mathematics and computer science, particularly in grouping 
    elements based on relationships.
    \begin{itemize}
        \item Introduces equivalence relations as reflexive, symmetric, and transitive.
        \item Provides examples like congruence modulo and relational concepts in programming.
    \end{itemize}
    
    \subsubsection*{Properties of Equivalence Relations}
    The properties subsection delves into the critical aspects that define equivalence relations. It uses examples to show how these properties manifest in real-world scenarios and mathematical 
    contexts.
    \begin{itemize}
        \item Discusses the defining properties: reflexivity, symmetry, and transitivity.
        \item Examples demonstrate how these properties are applied in various contexts.
    \end{itemize}
    
    \subsubsection*{Equivalence Classes}
    This part explains the concept of equivalence classes, which are formed when a set is partitioned based on an equivalence relation. It discusses the significance of these classes in understanding 
    relationships within a set.
    \begin{itemize}
        \item Describes how equivalence relations partition a set into distinct classes.
        \item Focuses on the concept of equivalence classes in relation to set theory.
    \end{itemize}
    
    \subsubsection*{Practical Applications}
    The final subsection explores practical applications of equivalence relations. It highlights their importance in computer science, particularly in programming, and discusses their use in mathematical 
    fields like modular arithmetic.
    \begin{itemize}
        \item Applies equivalence relations to real-world situations, including computer programming.
        \item Discusses the use in modular arithmetic and other mathematical applications.
    \end{itemize}
\end{notes}