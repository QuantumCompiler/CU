\clearpage
\chapter{Week 14: (11/27 - 12/3)}

\section{Probability}
\horizontalline{0}{0}

\subsection{Assigned Reading}

The reading assignments for this week can be found below:

\begin{itemize}
    \item \textbf{Sections 7.1, 7.2, 7.3, 7.4}
\end{itemize}

\subsection{Piazza}

Must post / respond to at least \textbf{two} Piazza posts this week. \due{(12/1/23)} \cbox{piazza-week14}

\subsection{Lectures}

The lectures for this week and their links can be found below:

\begin{itemize}
    \item \href{https://applied.cs.colorado.edu/mod/hvp/view.php?id=51840}{Intro To Discrete Probability} $\approx$ 32 min.
    \item \href{https://applied.cs.colorado.edu/mod/hvp/view.php?id=51841}{Probability Theory 2} $\approx$ 22 min.
    \item \href{https://applied.cs.colorado.edu/mod/hvp/view.php?id=51842}{Conditional Probability And Independence} $\approx$ 44 min.
    \item \href{https://applied.cs.colorado.edu/mod/hvp/view.php?id=51843}{Bayes' Theorem And The LTP} $\approx$ 41 min.
    \item \href{https://applied.cs.colorado.edu/mod/hvp/view.php?id=51844}{Bayesian Spam Filters} $\approx$ 20 min.
\end{itemize}

\noindent Below is a list of lecture notes for this week:

\begin{itemize}
    \item \pdflink{./Notes/Bayes Lecture Notes.pdf}{Bayes}
    \item \pdflink{./Notes/Introduction To Discrete Probability Lecture Notes.pdf}{Introduction To Discrete Probability}
    \item \pdflink{./Notes/Introduction To Probability Theory Lecture Notes.pdf}{Introduction To Probability Theory}
    \item \pdflink{./Notes/Conditional Probability And Independence Lecture Notes.pdf}{Conditional Probability And Independence}
    \item \pdflink{./Notes/Bayes' Theorem And The Law Of Total Probability Lecture Notes.pdf}{Bayes' Theorem And The Law Of Total Probability}
\end{itemize}

\subsection{Assignments}

The assignment for this week is \pdflink{./Assignments/Probability Mastery Workbook.pdf}{Probability Mastery Workbook} \due{(12/4/23)} \cbox{assignment-week14}

\subsection{Quiz}

The quiz's for this week can be found at \href{https://applied.cs.colorado.edu/mod/quiz/view.php?id=51846}{Probability Quiz} \due{(12/4/23)} \cbox{quiz-week14}

\subsection{Chapter Summary}

The first section that we are covering this week is \textbf{Section 7.1 - An Introduction To Discrete Probability}.

\begin{notes}{Section 7.1 - An Introduction To Discrete Probability}
    \subsubsection*{Overview}

    This section introduces the basic concepts of discrete probability. \vspace*{1em}

    \subsubsection*{Key Concepts}

    \begin{itemize}
        \item \textbf{Probability Definition}: Probability is defined for experiments with a finite number of equally likely outcomes. The probability of an event $E$ is given by the formula:
            \begin{equation*}
            p(E) = \frac{|E|}{|S|}
            \end{equation*}
        where $|E|$ represents the number of outcomes in event $E$ and $|S|$ is the total number of outcomes in the sample space.
        \item \textbf{Probability Range}: The probability of any event is a value between 0 and 1, inclusive.
        \item \textbf{Illustrative Examples and Theorems}: The section includes various examples and theorems to illustrate probability calculations. Classic problems like drawing cards and rolling 
        dice are used to demonstrate these concepts.
        \item \textbf{Monty Hall Problem}: A famous problem in probability, demonstrating counterintuitive results in probability theory, is discussed to elucidate the concepts.
    \end{itemize}
\end{notes}

The next section that we will cover this week is \textbf{Section 7.2 - Probability Theory}.

\begin{notes}{Section 7.2 - Probability Theory}
    \subsubsection*{Overview}
    Probability theory is a branch of mathematics concerned with analyzing random events. The probabilities of different outcomes are calculated to understand the likelihood of various events occurring. \vspace*{1em}

    \subsubsection*{Basic Probability Assignment}
    \begin{itemize}
        \item The formula $p(E) = \frac{|E|}{|S|}$ is used to calculate the probability of an event $E$ occurring within a finite sample space $S$. This formula assumes each outcome is 
        equally likely.
        \item Probabilities are always between 0 and 1, inclusive, where 0 indicates impossibility and 1 indicates certainty.
    \end{itemize}

    \subsubsection*{Conditional Probability and Independence}
    \begin{itemize}
        \item Conditional probability is used when the probability of an event depends on another event. The formula $p(E | F) = \frac{p(E \cap F)}{p(F)}$ quantifies this dependence.
        \item Two events are independent if the occurrence of one does not affect the probability of the other. This is mathematically expressed as $p(E \cap F) = p(E)p(F)$.
    \end{itemize}

    \subsubsection*{Random Variables}
    \begin{itemize}
        \item A random variable translates outcomes of random processes into numerical values, facilitating probability calculations.
        \item The distribution of a random variable describes the probabilities of its different possible values.
    \end{itemize}

    \subsubsection*{Bernoulli Trials and Binomial Distribution}
    \begin{itemize}
        \item Bernoulli trials are experiments with two possible outcomes, typically "success" and "failure". The binomial distribution calculates the probability of a fixed number of successes in 
        a set number of these trials.
    \end{itemize}

    \subsubsection*{Monte Carlo Algorithms}
    \begin{itemize}
        \item Monte Carlo algorithms use random sampling to obtain numerical results, often used in scenarios where deterministic methods are impractical.
    \end{itemize}

    \subsubsection*{The Probabilistic Method}
    \begin{itemize}
        \item This method uses probability to prove the existence of certain structures or properties when direct construction is difficult. It's a powerful tool in combinatorics and theoretical 
        computer science.
    \end{itemize}
\end{notes}

The next section that we will cover this week is \textbf{Section 7.3 - Bayes' Theorem}.

\begin{notes}{Section 7.3 - Bayes' Theorem}
    \subsubsection*{Overview}
    Bayes' Theorem is a crucial result in probability theory and is used for calculating the likelihood of an event based on prior knowledge of conditions that might be related to the event. \vspace*{1em}

    \subsubsection*{Introduction to Bayes' Theorem}
    \begin{itemize}
        \item The theorem is named after Thomas Bayes and allows for the revision of predictions or hypotheses based on new evidence.
        \item It is widely used in various fields like medicine, law, and machine learning for decision making under uncertainty.
    \end{itemize}

    \subsubsection*{The Mathematical Formulation}
    \begin{itemize}
        \item The theorem mathematically expresses how a subjective degree of belief should rationally change to account for evidence: $P(A|B) = \frac{P(B|A)P(A)}{P(B)}$.
        \item Here, $P(A|B)$ is the probability of event A given that B is true, $P(B|A)$ is the probability of event B given that A is true, $P(A)$ is the independent probability of A, and $P(B)$ 
        is the independent probability of B.
    \end{itemize}

    \subsubsection*{Applications of Bayes' Theorem}
    \begin{itemize}
        \item The theorem is used in various real-world applications, such as medical diagnosis, where it helps in determining the probability of a disease given a specific test result.
        \item Another application is in the field of spam filtering in emails, where it helps to determine the probability of an email being spam based on its content.
    \end{itemize}

    \subsubsection*{Generalized Bayes' Theorem}
    \begin{itemize}
        \item The generalized form of Bayes' Theorem can handle situations with multiple events and is used in more complex scenarios.
        \item This generalization allows the incorporation of multiple pieces of evidence to refine the probability estimates of an event.
    \end{itemize}
\end{notes}

The last section that we will cover this week is \textbf{Section 7.4 - Expected Value And Variance}.

\begin{notes}{Section 7.4 - Expected Value And Variance}
    \subsubsection*{Overview}
    This section delves into the concepts of expected value and variance, which are pivotal in understanding the behavior of random variables in probability and statistics. \vspace*{1em}

    \subsubsection*{Expected Value}
    \begin{itemize}
        \item The expected value, often denoted as $E(X)$, represents the average or mean value of outcomes for a random variable $X$. 
        \item It is calculated as the sum of all possible values of $X$, each weighted by its probability of occurrence.
        \item The concept is crucial in various applications, including decision-making and risk assessment.
    \end{itemize}

    \subsubsection*{Variance}
    \begin{itemize}
        \item Variance measures the spread or dispersion of the random variable's values around the expected value. 
        \item Denoted as $V(X)$ or $\sigma^2$, it is the average of the squared differences from the mean.
        \item A high variance indicates a wide spread of values, whereas a low variance suggests a tight clustering around the mean.
    \end{itemize}
\end{notes}