\section*{BTree}

\subsection*{Overview}

A B-tree is a self-balancing tree data structure that maintains sorted data and is designed to efficiently handle large amounts of data and support fast insertions, deletions, and searches. It consists 
of a root node, internal nodes with multiple child pointers, and leaf nodes that store the actual data. The key feature of a B-tree is its ability to keep the tree balanced by ensuring that all leaf 
nodes are at the same level, which optimizes search operations. B-trees are commonly used in database systems and file systems where they provide efficient access and management of data in blocks or 
pages, reducing the number of disk reads and writes, and thereby improving overall performance.

\subsection*{Terminology}

A B-tree is a self-balancing tree data structure that uses several key terms to describe its properties and operations. The B-tree is characterized by its `order,' denoted as `m,' which specifies the 
maximum number of keys a node can hold. Nodes in a B-tree can be of various types, such as the `root node' representing the topmost node, `internal nodes' having keys and corresponding child pointers, 
and `leaf nodes' containing the actual data. The B-tree maintains the sorted order of keys within each node, ensuring that keys from left to right are in non-decreasing order. To handle insertions and 
deletions efficiently, the B-tree employs `node splitting' to create two nodes when a node is full and `node merging' when nodes become underflowed. The B-tree guarantees balance, where all leaf nodes 
have the same depth, and provides essential functions like search, insertion, deletion, and traversal to manage and manipulate data efficiently. With its balanced structure and logarithmic time 
complexity for search and modification operations, the B-tree is widely used in various applications involving large datasets, such as databases and file systems.

\subsection*{Structure}

A B-tree is a self-balancing tree data structure that maintains a sorted collection of keys and efficiently organizes large amounts of data. The B-tree consists of nodes, including the root node, internal 
nodes, and leaf nodes. Each node has a fixed capacity and holds a certain number of keys, arranged in non-decreasing order from left to right. Internal nodes have corresponding child pointers, which 
direct to their child subtrees. All leaf nodes, which store the actual data, are at the same level, ensuring a balanced tree structure. When a node becomes full upon inserting a new key, it splits into two 
nodes, and the middle key moves up to the parent node. Conversely, when a node becomes empty upon deleting a key, it may merge with its adjacent sibling node. B-trees guarantee that all leaf nodes have the 
same depth, resulting in efficient search, insertion, and deletion operations. These properties make B-trees well-suited for applications like database systems and file systems, where they provide rapid 
data access and management, particularly when dealing with large datasets.

The following is a list of functions that are typically required for building a B-tree:

\begin{itemize}
    \item \textbf{Deletion} - A function to remove a key from the B-tree, ensuring that the tree remains balanced and adheres to the B-tree rules. This function is usually recursive as it involves traversing 
    the B-tree recursively to find the key's location and then handle node merging and underflow cases.
    \item \textbf{Initialization} - A function to create and initialize an empty B-tree.
    \item \textbf{Insertion} - A function to insert a new key into the B-tree while maintaining the B-tree properties, such as balanced nodes and sorted keys. This function is often recursive as it 
    requires traversing the B-tree recursively to find the appropriate position for insertion and handle node splitting if necessary.
    \item \textbf{Merge Nodes} - A function to merge two nodes into one when they become underflowed (have less than $(m/2)$ keys) after deletion. This function can be recursive if the underflow propagates up 
    the tree. The function may recursively call itself to merge parent nodes until the tree is balanced again.
    \item \textbf{Minimum \& Maximum Key Retrieval} - Functions to find the minimum and maximum keys present in the B-tree.
    \item \textbf{Printing} - Functions to print the B-tree in a visually understandable format, helping with debugging and visualization. Printing functions can be recursive as they involve traversing the 
    tree to display its structure and keys.
    \item \textbf{Search} - A function to search for a specific key in the B-tree, returning either a pointer to the node containing the key or indicating that the key is not present. This function is 
    typically recursive as it involves traversing the B-tree recursively by calling itself on child nodes until the desired key is found or deemed not present.
    \item \textbf{Split Node} - A function to split a node into two when it becomes full (has $(m-1)$ keys) during insertion. This function can be recursive if the parent node also needs to split due to 
    overflow. The function may recursively call itself to handle the cascading split of parent nodes.
    \item \textbf{Successor \& Predecessor} - Functions to find the successor (the smallest key greater than a given key) and predecessor (the largest key smaller than a given key) of a given key. These 
    functions may be recursive, especially if the current node does not have the requested key, and they need to traverse the tree recursively.
    \item \textbf{Traversal} - Functions to traverse the B-tree in various orders, such as in-order, pre-order, and post-order, for processing and displaying the keys. Traversal functions are usually 
    implemented using recursion, as they involve visiting nodes and calling themselves on child nodes.
\end{itemize}

\subsection*{Efficiency}

B-trees offer an efficient runtime complexity for various operations due to their balanced structure. The average and worst-case time complexities for search, insertion, and deletion operations in a B-tree 
are all $\mathcal{O}(\log{(n)})$, where $n$ is the number of keys in the B-tree. This logarithmic complexity arises from the B-tree's ability to maintain a balanced height, ensuring that the number of keys 
per node remains within a constant range. As a result, B-trees are well-suited for managing large datasets, as their logarithmic time complexities enable fast data retrieval and manipulation even with 
substantial amounts of data. The B-tree's balanced nature makes it a popular data structure in database systems, file systems, and other applications that require efficient storage and retrieval of sorted data.

\begin{center}
    \begin{tabular}[ht]{|c|c|c|c|}
        \hline \textbf{Operation} & \textbf{Best Case $\Omega(n)$} & \textbf{Average Case $\Theta(n)$} & \textbf{Worst Case $\mathcal{O}(n)$} \\ \hline
        \textbf{Deletion} & $\Omega(1)$ & $\Theta(\log{(n)})$ & $\mathcal{O}(\log{(n)})$ \\ \hline
        \textbf{Insertion} & $\Omega(1)$ & $\Theta(\log{(n)})$ & $\mathcal{O}(\log{(n)})$ \\ \hline
        \textbf{Search} & $\Omega(1)$ & $\Theta(\log{(n)})$ & $\mathcal{O}(\log{(n)})$ \\ \hline
        \textbf{Traversal} & $\Omega(n)$ & $\Theta(n)$ & $\mathcal{O}(n)$ \\ \hline
    \end{tabular}
\end{center}

\subsection*{Deletion}

Deletion in a B-tree is the process of removing a key from the B-tree while maintaining its balanced and sorted properties. When deleting a key, the function first searches for the key in the B-tree. If the key 
is found in an internal node, it is replaced by its in-order predecessor or successor from the left or right subtree, respectively. If the key is found in a leaf node, it can be directly removed. After deletion, 
the B-tree might experience underflow in nodes, where a node has fewer keys than the minimum required. To restore balance, the function may perform node merging or redistribution with adjacent sibling nodes. 
Deletion in a B-tree ensures that the tree remains balanced and efficient in managing data.

\begin{highlight}[Deletion Algorithm]

Below is an example of the deletion algorithm.

\horizontalline

\begin{verbatim}
procedure delete_key(B-tree node, key_to_delete)
    // Search for the key to delete
    position <- find_position_in_node(node, key_to_delete)

    if node is a leaf node
        // Key is found in a leaf node, directly remove it
        remove_key_from_node(node, position)

    else
        // Key is found in an internal node
        predecessor <- find_predecessor(node, position)
        predecessor_key <- predecessor's rightmost key
        replace key_to_delete with predecessor_key

        // Recursively delete the predecessor_key from the child node
        delete_key(predecessor, predecessor_key)

    end if

    // Check if underflow occurs in the node
    if node is underflowed
        sibling <- find_sibling(node)
        if sibling can spare a key
            redistribute_keys(node, sibling)
        else
            merge_nodes(node, sibling)
            delete_key(node's parent, key that caused the merge)
        end if
    end if
end procedure
\end{verbatim}

\end{highlight}

\subsection*{Initialization}

The initialization algorithm for a B-tree involves creating and setting up the data structure for an empty B-tree. It typically includes allocating memory for the root node, initializing its properties 
(e.g., setting the number of keys to 0 and marking it as a leaf node), and initializing any other necessary variables or data structures used to manage the B-tree. This algorithm is essential to ensure 
the B-tree starts with a valid state and can be ready to store and manage keys efficiently when further operations are performed.

\begin{highlight}[Initialization Algorithm]

Below is an example of the initialization algorithm.

\horizontalline

\begin{verbatim}
procedure initialize_B_tree(order)
    root <- create_new_node()  // Create a new node
    root.is_leaf <- true       // Mark the root node as a leaf node
    root.num_keys <- 0         // Set the number of keys in the root node to 0
    root.order <- order        // Set the order of the B-tree

    return root                // Return the initialized root node
end procedure
\end{verbatim}
    
\end{highlight}

\subsection*{Insertion}

The insertion algorithm for a B-tree involves inserting a new key into the B-tree while preserving its balanced and sorted properties. The algorithm starts by searching for the appropriate leaf node where the 
key should be inserted. If the leaf node has space for the new key, it is simply added in a sorted manner. However, if the node is full, it is split into two nodes, and the middle key is moved up to the parent 
node. The process may cascade up the tree, potentially causing further splits until a node with available space is found or creating a new root node if the current root was split. The insertion algorithm 
ensures that the B-tree remains balanced and efficiently manages data even with frequent insertions.

\begin{highlight}[Insertion Algorithm]

Below is an example of the insertion algorithm.

\horizontalline

\begin{verbatim}
procedure insert_key(B-tree node, key_to_insert)
    if node is a leaf node
        // Insert the key into the current leaf node
        position <- find_position_in_node(node, key_to_insert)
        insert_key_at_position(node, key_to_insert, position)

        // Check if the node is full and handle splitting if necessary
        if node has (order - 1) keys
            split_node(node)
        end if
    else
        // Traverse to the appropriate child node
        child_node <- find_child_node(node, key_to_insert)
        insert_key(child_node, key_to_insert)

        // Check if the child node split and update parent node if necessary
        if child_node has (order - 1) keys
            split_node(child_node)
        end if
    end if
end procedure
\end{verbatim}

\end{highlight}

\subsection*{Merge Nodes}

The Merge Nodes algorithm in a B-tree is employed when a node becomes underflowed during a deletion operation, meaning it has fewer keys than the minimum required. When a node underflows, it may merge with its 
adjacent sibling node, effectively reducing the tree's height and maintaining the B-tree's balanced property. The Merge Nodes algorithm redistributes the keys and child pointers from the sibling node into the 
underflowed node, creating a single merged node. The key in the parent node, which was in between the two merged nodes, is removed, and the Merge Nodes algorithm may recursively propagate up the tree if the 
parent node also becomes underflowed. By merging nodes, this algorithm helps to prevent further underflow and maintain the B-tree's efficient data management and search capabilities.

\begin{highlight}[Merge Nodes Algorithm]

Below is an example of the merge nodes algorithm.

\horizontalline

\begin{verbatim}
procedure merge_nodes(B-tree parent, int left_index)
    left_child <- parent.children[left_index]
    right_child <- parent.children[left_index + 1]

    // Move the key from the parent into the left_child node
    move_key_from_parent(parent, left_index)

    // Move all keys and child pointers from the right_child into the left_child
    move_keys_and_pointers(right_child, left_child)

    // Remove the right_child node from the parent and free its memory
    remove_child_from_parent(parent, left_index + 1)
    free_memory(right_child)

    // Update the parent's key count and child pointers
    parent.num_keys <- parent.num_keys - 1
    update_parent_pointers(parent, left_index + 1)

    // If the parent becomes underflowed, recursively handle further merging
    if parent is underflowed
        merge_nodes(parent's parent, position of parent in its parent)
    end if
end procedure
\end{verbatim}

\end{highlight}

\subsection*{Minimum \& Maximum Key Retrieval}

The Minimum \& Maximum Key Retrieval algorithm in a B-tree is employed to find the smallest and largest keys present in the B-tree, respectively. To retrieve the minimum key, the algorithm traverses the leftmost 
path of the tree until it reaches a leaf node, where the smallest key resides. Similarly, to find the maximum key, the algorithm traverses the rightmost path of the tree until it reaches a leaf node, where the 
largest key exists. This algorithm does not require recursion and efficiently locates the minimum and maximum keys, providing quick access to the boundary values stored in the B-tree.

\begin{highlight}[Minimum \& Maximum Key Retrieval Algorithm]

Below is an example of the minimum \& maximum key retrieval algorithm.

\horizontalline

\begin{verbatim}
procedure find_minimum_key(B-tree node)
    while node is not a leaf node
        node <- node.children[0]  // Traverse to the leftmost child
    end while

    return node.keys[0]          // Return the smallest key in the leaf node
end procedure

procedure find_maximum_key(B-tree node)
    while node is not a leaf node
        node <- node.children[node.num_keys]  // Traverse to the rightmost child
    end while

    return node.keys[node.num_keys - 1]      // Return the largest key in the leaf node
end procedure
\end{verbatim}
    
\end{highlight}

\subsection*{Printing}

The Printing algorithm for a B-tree involves traversing the entire tree and displaying its structure and keys in a visually understandable format. Various traversal methods, such as in-order, pre-order, or 
post-order, can be used to achieve this. During the traversal, the algorithm visits each node, displaying its keys and child pointers. Proper indentation is used to represent the hierarchy of nodes in the 
B-tree. The Printing algorithm helps with debugging, visualization, and understanding the organization of the B-tree, making it a valuable tool for analyzing the structure and contents of the tree.

\begin{highlight}[Printing Algorithm]

Below is an example of the printing algorithm.

\horizontalline

\begin{verbatim}
procedure print_B_tree(B-tree node, indentation)
    if node is not null
        // Print keys and child pointers in the node
        for i <- 0 to node.num_keys - 1
            print indentation, node.keys[i]

        // Print the last child pointer
        print indentation, node.keys[node.num_keys]

        // Recursive call for each child node
        for i <- 0 to node.num_keys
            print_B_tree(node.children[i], indentation + "    ")
    end if
end procedure
\end{verbatim}
    
\end{highlight}

\subsection*{Search}

The Search algorithm for a B-tree involves finding a specific key within the tree efficiently. The algorithm starts from the root node and compares the target key with the keys in the current node. If the 
key is found, the algorithm returns a pointer to the node containing the key. If the key is not in the current node, the algorithm narrows down the search to the appropriate child subtree by comparing the 
key with the node's keys and follows the corresponding child pointer. This process continues until the key is found in a leaf node or determined to be absent. The Search algorithm's balanced nature and 
logarithmic time complexity enable rapid retrieval of keys in large datasets, making B-trees ideal for database systems and file systems.

\begin{highlight}[Search Algorithm]

Below is an example of the search algorithm.

\horizontalline

\begin{verbatim}
procedure search_key(B-tree node, key_to_find)
    if node is not null
        position <- find_position_in_node(node, key_to_find)

        if position < node.num_keys and node.keys[position] = key_to_find
            // Key found in the current node
            return node
        end if

        if node is a leaf node
            // Key not found
            return NULL
        else
            // Recursively search in the appropriate child subtree
            return search_key(node.children[position], key_to_find)
        end if
    end if
end procedure
\end{verbatim}
    
\end{highlight}

\subsection*{Split Node}

The Split Node algorithm in a B-tree is employed when a node becomes full (contains $(m-1)$ keys) during an insertion operation. To maintain the B-tree's balanced structure and sorted keys, the Split Node 
algorithm divides the full node into two separate nodes. The middle key is moved up to the parent node, creating a space for the new key to be inserted. The child pointers of the split node are 
appropriately adjusted, and the new key is inserted into the parent node. The Split Node algorithm may trigger further splits up the tree if the parent node also becomes full, ensuring that the B-tree 
remains balanced and efficient in managing data.

\begin{highlight}[Split Node Algorithm]

Below is an example of the split node algorithm.

\horizontalline

\begin{verbatim}
procedure split_node(B-tree parent, int child_index)
    full_node <- parent.children[child_index]  // Get the full node to split
    new_node <- create_new_node()             // Create a new node for splitting
    middle_key <- full_node.keys[(m-1)/2]      // Find the middle key to move up

    // Move keys and child pointers to the new node
    move_keys_and_pointers(full_node, new_node, (m-1)/2 + 1, m-1)

    // Adjust the number of keys in the nodes
    full_node.num_keys <- (m-1)/2
    new_node.num_keys <- (m-1)/2

    // Update the parent node with the middle key and new child pointer
    insert_key_in_node(parent, child_index, middle_key)
    insert_child_in_node(parent, child_index + 1, new_node)
end procedure
\end{verbatim}

\end{highlight}

\subsection*{Successor \& Predecessor}

The Successor \& Predecessor algorithm in a B-tree is used to find the key that comes immediately after or before a given key in sorted order within the B-tree. To find the successor of a key, the 
algorithm starts by searching for the key in the B-tree. If the key is present in an internal node, the successor is the leftmost key in its right subtree. If the key is found in a leaf node, the 
successor is the next key in the same node or the leftmost key in the right sibling node if the current node has no more keys. Similarly, to find the predecessor of a key, the algorithm follows a 
similar approach but looks for the rightmost key in the left subtree if the key is found in an internal node, or the previous key in the same node or the rightmost key in the left sibling node if 
the current node has no more keys. The Successor \& Predecessor algorithm efficiently navigates the B-tree to identify the next or previous key for a given key, making it useful for range queries 
and data traversal in sorted order.

\begin{highlight}[Successor \& Predecessor Algorithm]

Below is an example of the successor and predecessor algorithm.

\horizontalline

\begin{verbatim}
procedure find_successor(B-tree node, key_to_find)
    position <- find_position_in_node(node, key_to_find)

    if node is a leaf node
        // Key not found in the current node
        if position < node.num_keys
            // Successor is the next key in the same node
            return node.keys[position + 1]
        else
            // Successor is the leftmost key in the right sibling node
            return find_leftmost_key_in_right_sibling(node.parent, position + 1)
        end if
    else
        // Recursively search in the right subtree
        return find_leftmost_key(node.children[position + 1])
    end if
end procedure

procedure find_leftmost_key(B-tree node)
    while node is not a leaf node
        // Continue traversing to the leftmost child
        node <- node.children[0]
    end while

    return node.keys[0]  // Return the leftmost key in the leaf node
end procedure

procedure find_leftmost_key_in_right_sibling(B-tree parent, int right_sibling_index)
    right_sibling <- parent.children[right_sibling_index]

    // Traverse to the leftmost key in the right sibling node
    return find_leftmost_key(right_sibling)
end procedure

procedure find_predecessor(B-tree node, key_to_find)
    position <- find_position_in_node(node, key_to_find)

    if node is a leaf node
        // Key not found in the current node
        if position > 0
            // Predecessor is the previous key in the same node
            return node.keys[position - 1]
        else
            // Predecessor is the rightmost key in the left sibling node
            return find_rightmost_key_in_left_sibling(node.parent, position)
        end if
    else
        // Recursively search in the left subtree
        return find_rightmost_key(node.children[position])
    end if
end procedure

procedure find_rightmost_key(B-tree node)
    while node is not a leaf node
        // Continue traversing to the rightmost child
        node <- node.children[node.num_keys]
    end while

    return node.keys[node.num_keys - 1]  // Return the rightmost key in the leaf node
end procedure

procedure find_rightmost_key_in_left_sibling(B-tree parent, int left_sibling_index)
    left_sibling <- parent.children[left_sibling_index]

    // Traverse to the rightmost key in the left sibling node
    return find_rightmost_key(left_sibling)
end procedure
\end{verbatim}

\end{highlight}

\subsection*{Traversal}

The Traversal algorithm in a B-tree is used to visit and process all the keys in the tree in a specific order. Various traversal methods, such as in-order, pre-order, and post-order, allow 
accessing the keys in different sequences. In an in-order traversal, the algorithm visits the keys in ascending order, starting from the smallest key. In a pre-order traversal, the algorithm 
visits the current node's key before traversing its children. In a post-order traversal, the algorithm visits the children nodes before the current node's key. Traversal is often implemented 
recursively, efficiently navigating the B-tree's hierarchical structure and providing a valuable mechanism for processing and displaying the keys in the desired order.

\begin{highlight}[Traversal Algorithm]

Below is an example of the traversal algorithm.

\horizontalline

\begin{verbatim}
procedure in_order_traversal(B-tree node)
    if node is not null
        // Traverse left subtree
        in_order_traversal(node.children[0])

        // Process current node's keys
        for i <- 0 to node.num_keys - 1
            process_key(node.keys[i])

        // Traverse right subtree
        in_order_traversal(node.children[node.num_keys])

    end if
end procedure
\end{verbatim}
    
\end{highlight}