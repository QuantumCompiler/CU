\clearpage

\chapter{Week 8}

\section{Priority Queue, Heap, Treap}

\horizontalline

\subsection{Activities}

The following are the activities that are planned for Week 8 of this course.

\begin{itemize}
    \item Read Chapter 13 in zyBooks and work on book chapter activities. (Due Next Monday)
    \item BTree homework due Tuesday (Extra Credit)
    \item Watch video lectures
    \item Work on Priority Queue homework (Due Next Tuesday)
\end{itemize}

\subsection{Lectures}

Here are the lectures that can be found for this week:

\begin{itemize}
    \item \href{https://applied.cs.colorado.edu/mod/hvp/view.php?id=45994}{Intro to Abstract Data Types}
    \item \href{https://applied.cs.colorado.edu/mod/hvp/view.php?id=45995}{ADT Examples}
    \item \href{https://applied.cs.colorado.edu/mod/hvp/view.php?id=45996}{Commond ADT Operations}
    \item \href{https://applied.cs.colorado.edu/mod/hvp/view.php?id=45997}{Priority Queues}
    \item \href{https://applied.cs.colorado.edu/mod/hvp/view.php?id=45998}{Heaps}
    \item \href{https://applied.cs.colorado.edu/mod/hvp/view.php?id=45999}{Treaps}
\end{itemize}

\subsection{Programming Assignment}

The programming assignment for Week 8 - \href{https://github.com/cu-cspb-2270-Summer-2023/pa6-RelativiBit}{Priority Queue}

\subsection{Chapter 13 - Heaps \& Treaps}

The chapter of this week is Chapter 13 - Heaps \& Treaps.

\subsection*{Sec. 13.1 - Heaps}

\subsubsection{Overview}

Heaps are a fundamental data structure in computer science used for efficient management of priority queues. A heap is a binary tree where each node satisfies the heap property, which means that the 
value of each node is greater (or smaller) than or equal to the values of its children. Heaps can be implemented as arrays, allowing constant time access to the maximum (or minimum) element. They are 
commonly used in algorithms such as heap sort, Dijkstra's algorithm, and priority queue implementations, providing efficient operations for inserting, deleting, and retrieving the highest (or lowest) 
priority element.

\subsubsection{Max Heap}

Max heaps are a type of heap data structure where the value of each node is greater than or equal to the values of its children. They are typically implemented using an array, with the root node being 
the largest element. The insert operation in a max heap involves adding a new element to the bottom of the heap and then percolating it up by comparing it with its parent and swapping if necessary until 
the heap property is satisfied. The remove operation removes the maximum element from the heap, which is always the root node, and replaces it with the last element in the array. The element is then 
percolated down by comparing it with its children and swapping if necessary to maintain the heap property. Percolating refers to the process of moving an element up or down the heap to maintain the heap 
property after an insert or remove operation.

\subsubsection{Min Heap}

Min heaps are a type of heap data structure where the value of each node is smaller than or equal to the values of its children. Similar to max heaps, min heaps are often implemented using an array, with 
the root node being the smallest element. The insert operation in a min heap involves adding a new element to the bottom of the heap and percolating it up by comparing it with its parent and swapping if 
necessary until the heap property is satisfied. The remove operation removes the minimum element from the heap, which is always the root node, and replaces it with the last element in the array. The element 
is then percolated down by comparing it with its children and swapping if necessary to maintain the heap property. Percolating refers to the process of moving an element up or down the heap to maintain the 
heap property after an insert or remove operation.

\subsection*{Sec. 13.2 - Heaps Using Arrays}

\subsubsection{Overview}

Heaps can be efficiently stored in arrays, providing a compact representation and enabling constant time access to elements. In a binary heap, the elements are stored in a complete binary tree structure, 
where each level is filled from left to right, except for the last level which may be partially filled from the left. The heap property is maintained by carefully arranging the elements in the array based 
on their relationships with their parent and children nodes. The parent-child relationship is determined by the indices of the array, where for any given index $i$, the left child is located at index $2i+1$ 
and the right child is at index $2i+2$. Conversely, for any child node at index $j$, its parent can be found at index \textbf{floor}$((j-1)/2)$. This array representation allows for efficient heap operations 
such as insert, remove, and percolating, by performing simple arithmetic computations on indices to navigate and manipulate the array elements.

\subsubsection{Percolate Algorithm}

The percolate algorithm in heaps refers to the process of moving an element up or down the heap to maintain the heap property after an insert or remove operation. When an element is inserted or removed, it may 
violate the heap property, which states that the value of each node should be greater (in the case of a max heap) or smaller (in the case of a min heap) than or equal to the values of its children. Percolation 
involves comparing the element with its parent (in the case of percolating up) or with its children (in the case of percolating down) and swapping them if necessary. This process is repeated iteratively until 
the heap property is satisfied throughout the entire heap, ensuring the integrity and efficiency of heap operations.

\begin{solution}[Percolate Algorithm Example]

\subsubsection{Percolate Up}

Below is an example of the percolate up algorithm in a max heap in the context of C++:

\horizontalline

\begin{verbatim}
void maxHeapPercolateUp(int arr[], int index) {
    int parent = (index - 1) / 2;
    while (index > 0 && arr[index] > arr[parent]) {
        std::swap(arr[index], arr[parent]);
        index = parent;
        parent = (index - 1) / 2;
    }
}    
\end{verbatim}

\horizontalline

\subsubsection{Percolate Down}

Conversely, below is an example of the percolate down algorithm in a max heap in the context of C++:

\horizontalline

\begin{verbatim}
void maxHeapPercolateDown(int arr[], int index, int size) {
    int maxIndex = index;
    int leftChild = 2 * index + 1;
    int rightChild = 2 * index + 2;

    if (leftChild < size && arr[leftChild] > arr[maxIndex]) {
        maxIndex = leftChild;
    }

    if (rightChild < size && arr[rightChild] > arr[maxIndex]) {
        maxIndex = rightChild;
    }

    if (maxIndex != index) {
        std::swap(arr[index], arr[maxIndex]);
        maxHeapPercolateDown(arr, maxIndex, size);
    }
}    
\end{verbatim}

\horizontalline

In max heaps, the percolate up algorithm compares an element with its parent and swaps them if the element is greater, ensuring it moves up to its correct position. This process is repeated until the element is 
in its proper place or reaches the root. Conversely, the percolate down algorithm compares an element with its children, selecting the larger child and swapping them if the child is greater, allowing the element 
to move down to its appropriate position. This continues until the element is in its correct place or has no greater children. These algorithms maintain the max heap property, facilitating efficient operations 
like insertions, removals, and maintaining the proper ordering of elements based on their values.

\end{solution}

\subsection*{Sec. 13.3 - Heap Sort}

\subsubsection{Heapify}

The heapify operation is used to convert an array into a heap, either max heap or min heap, in linear time. It rearranges the elements of the array, transforming it into a complete binary tree that satisfies the 
heap property. The operation begins by selecting the last non-leaf node in the array and performing a percolate down operation on each node, starting from this node and moving upwards towards the root. This ensures 
that every subtree rooted at a particular node is a valid heap. By repeating this process for all non-leaf nodes, the entire array is transformed into a heap. The heapify operation is efficient and commonly employed 
to build a heap from an unordered array or to restore the heap property after an element is modified.

\begin{solution}[Heapify Algorithm Example]

Below is an example of the heapify algorithm in the context of C++:

\horizontalline

\begin{verbatim}
void maxHeapify(int arr[], int n, int i) {
    int largest = i;
    int leftChild = 2 * i + 1;
    int rightChild = 2 * i + 2;

    if (leftChild < n && arr[leftChild] > arr[largest])
        largest = leftChild;

    if (rightChild < n && arr[rightChild] > arr[largest])
        largest = rightChild;

    if (largest != i) {
        std::swap(arr[i], arr[largest]);
        maxHeapify(arr, n, largest);
    }
}
\end{verbatim}

\horizontalline

The above is an example of the heapify algorithm implementation in C++.

\end{solution}

\subsubsection{Heap Sort}

Heap sort is an efficient comparison-based sorting algorithm that utilizes the properties of a heap data structure. The algorithm begins by building a max heap from the given array, transforming it into a partially 
sorted structure. Next, it repeatedly extracts the maximum element from the heap (which is always the root), swaps it with the last element of the heap, and reduces the heap size. This process ensures that the maximum 
element "bubbles down" to its correct position. By iteratively applying this step, the array becomes fully sorted in ascending order. Heap sort has a time complexity of $\mathcal{O}(n\log{(n)})$, where $n$ is the number 
of elements in the array, making it an optimal choice for large datasets.

\begin{solution}[Heap Sort Algorithm Example]

Below is an example of the heap sort algorithm in the context of C++:

\horizontalline

\begin{verbatim}
void heapSort(int arr[], int n) {
    // Build max heap
    for (int i = n / 2 - 1; i >= 0; i--)
        maxHeapify(arr, n, i);

    // Extract elements from the heap one by one
    for (int i = n - 1; i > 0; i--) {
        std::swap(arr[0], arr[i]);
        maxHeapify(arr, i, 0);
    }
}
\end{verbatim}

\horizontalline

The above is an example of the heap sort algorithm implementation in C++.

\end{solution}

\subsection*{Sec. - 13.4 Priority Queue Abstract Data Type (ADT)}

\subsubsection{Overview}

The Priority Queue Abstract Data Type (ADT) is a fundamental data structure that allows elements to be stored with associated priority values. It operates on the principle of priority, where elements 
with higher priority are given precedence over elements with lower priority. The key operations supported by a priority queue include inserting elements with their respective priorities, accessing or 
retrieving the element with the highest priority, and removing the element with the highest priority. Priority queues can be implemented using various data structures such as heaps, arrays, or linked 
lists, and they find applications in algorithms where efficient management of priorities is crucial, such as task scheduling, graph algorithms, and event-driven simulations. The priority queue ADT 
provides a flexible and efficient way to handle elements based on their priorities, allowing for efficient and ordered processing of tasks or events.

\subsubsection{Common Priority Queue Operations}

Common priority queue operations include inserting an element with its associated priority, accessing or retrieving the element with the highest priority, and removing the element with the highest priority. 
Insertion involves adding an element to the priority queue while maintaining the order based on its priority. Accessing the highest priority element allows retrieving the element without removing it from the 
queue. Removal of the highest priority element removes and returns the element with the highest priority from the priority queue, ensuring that the next highest priority element becomes the new highest priority. 
These operations are the key building blocks for managing priorities and efficiently processing elements in a priority queue data structure.

\begin{center}
    \begin{tabular}[ht]{|c|c|}
        \hline \textbf{Operation} & \textbf{Description} \\ \hline
        Dequeue(PQueue) & Returns and removes the item at the front of PQueue. \\ \hline
        Enqueue(PQueue, x) & Inserts x after all equal or higher priority items. \\ \hline
        GetLength(PQueue) & Returns the number of items in PQueue. \\ \hline
        IsEmpty(PQueue) & Returns true if PQueue has no items. \\ \hline
        Peek(PQueue) & Returns but does not remove the item at the front of PQueue. \\ \hline
    \end{tabular}
\end{center}

\subsubsection{Enqueueing With Priority}

Enqueueing items with priority refers to the process of adding elements to a data structure, such as a priority queue, while associating each element with a specific priority value. This operation involves 
inserting the element into the data structure in a way that preserves the order based on the assigned priorities. Elements with higher priorities are typically placed ahead of elements with lower priorities, 
ensuring that the highest priority element is readily accessible. Enqueueing items with priority is crucial in scenarios where elements need to be processed or accessed based on their importance or urgency, 
allowing for efficient handling and retrieval of items according to their assigned priorities.

\begin{center}
    \begin{tabular}[ht]{|c|c|c|}
        \hline \textbf{Priority Queue Operation} & \textbf{Heap Functionality} & \textbf{Worst Case Runtime} \\ \hline
        Enqueue & Insert & $\mathcal{O}(\log{(N)})$ \\ \hline
        Dequeue & Remove & $\mathcal{O}(\log{(N)})$ \\ \hline
        GetLength & Return number of nodes & $\mathcal{O}(1)$ \\ \hline
        IsEmpty & Return true if no nodes in heap & $\mathcal{O}(1)$ \\ \hline
        Peek & Return value of root node & $\mathcal{O}(1)$ \\ \hline
    \end{tabular}
\end{center}

\subsection*{Sec. 13.5 - Treaps}

\subsubsection{Overview}

Treaps are a combination of binary search trees (BSTs) and binary heaps, merging the advantages of both data structures. Each node in a treap holds a key-value pair, with keys following the BST 
property and priorities assigned randomly. The priority determines the heap property, where the priority of any node is greater than or equal to the priorities of its children. Treaps maintain 
the probabilistic balance between the BST and heap properties, resulting in a balanced binary search tree with expected logarithmic time complexity for key-based operations. Insertion and deletion 
in treaps involve performing rotations to maintain the BST and heap properties. Treaps find applications in various scenarios, including priority queues, range queries, and randomized binary search 
tree implementations.

\subsubsection{Search}

Search in treaps follows the principles of binary search trees, leveraging the ordering of keys to efficiently locate specific elements. Starting from the root, the search algorithm compares the target 
key with the key at the current node. If the keys match, the desired element is found. If the target key is smaller, the search continues in the left subtree. If the target key is larger, the search 
proceeds in the right subtree. This process continues recursively until either the target key is found or a leaf node is reached, indicating that the element does not exist in the treap. The randomized 
nature of treaps ensures that the search operation maintains an expected logarithmic time complexity, making it an efficient data structure for retrieval operations.

\subsubsection{Insert}

Insertion in treaps involves adding a new element, represented by a key-value pair, to the treap data structure. The algorithm begins by performing a standard binary search tree insertion, placing the new 
element in its appropriate position based on the key values. After the element is inserted, the heap property of the treap may be violated since the priorities are assigned randomly. To restore the heap 
property, rotations are performed to move the newly inserted element upwards until its priority satisfies the heap property. The rotations adjust the tree structure while maintaining the order of the keys. 
The insertion process in treaps ensures that the resulting data structure remains balanced and offers an expected logarithmic time complexity for subsequent search, delete, and other operations. Treaps provide 
an efficient way to insert elements while combining the benefits of binary search trees and binary heaps.

\subsubsection{Delete}

Deletion in treaps involves removing a specific element from the data structure while maintaining the treap's binary search tree and heap properties. The algorithm begins by searching for the element to 
be deleted, following the same process as the search operation. Once the element is found, it is removed from the treap by performing rotations to restore the binary search tree and heap properties. The 
rotations reorganize the tree while ensuring that the priorities and keys remain consistent. The exact rotations performed depend on the relative priorities of the nodes involved. The deletion process 
ensures that the resulting treap remains balanced and preserves the expected logarithmic time complexity for subsequent search and other operations. Treaps provide an efficient way to remove elements from 
a data structure while maintaining the benefits of both binary search trees and binary heaps.