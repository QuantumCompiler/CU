\clearpage
\chapter{Week 2}

\section{Object Orientation in C++ \& ADT}
\horizontalline

\subsection{Activities}
The following are the activities that are planned for Week 2 of this course.
\begin{itemize}
    \item Read the zyBook chapter(s) assigned and complete the reading quiz(s) by next Tuesday (usually Monday but it's a holiday).
    \item Read the C++ refresher or access other resources to improve your skills.
    \item Watch the videos on C++ Classes and Abstract Data Types.
    \item Watch the videos on Object-Oriented Thinking and Debugging your Assignments.
    \item Implement the examples In week videos for yourself on Jupytherhub machine.
    \item Access the GitHub Classroom to get your Assignment-1 repository  (assignment due next Tuesday).
\end{itemize}

\subsection{Lectures}
Here are the lectures that can be found for this week:
\begin{itemize}
    \item \href{https://applied.cs.colorado.edu/mod/hvp/view.php?id=45872}{C++ Classes Basics}
    \begin{itemize}
        \item \href{https://applied.cs.colorado.edu/mod/folder/view.php?id=45873}{Source Files}
    \end{itemize}
    \item \href{https://applied.cs.colorado.edu/mod/hvp/view.php?id=45874}{Abstract Data Type (ADT)}
    \item \href{https://applied.cs.colorado.edu/mod/hvp/view.php?id=45875}{Notes for Assignment 1 - Vector10}
    \item \href{https://applied.cs.colorado.edu/mod/hvp/view.php?id=45877}{Objected Oriented Thinking}
    \item \href{https://applied.cs.colorado.edu/mod/hvp/view.php?id=45878}{Object Lifestyle}
    \item \href{https://applied.cs.colorado.edu/mod/hvp/view.php?id=45879}{My Code is Not Working}
\end{itemize}

\subsection{Programming Assignment}
The programming assignment for Week 2 - \href{https://github.com/cu-cspb-2270-Summer-2023/pa1-RelativiBit}{Vector10}

\subsection{Chapter 2 - Objects and Classess} 
The first chapter of this week is Chapter 2 - Objects and Classes.

% The chapters of this week is Chapter 2 - Objects and Classes and Chapter 3 - Introduction to Algorithms.

\subsection*{Sec. 2.1 - Objects: Introduction}
\subsubsection*{Objects}

Objects are fundamental concepts in object-oriented programming (OOP) that represent real-world entities or abstract concepts. They encapsulate both data (attributes) and behavior (methods), allowing for modular and reusable
code, enhanced code oganization, and modeling of complex systems. Objects promote the principles of encapsulation, inheritence, and polymorphism, facilitating efficient and modular software development.

\begin{solution}[Objects Example]
    Here is a simple example of objects in C++: \\
    \horizontalline
    \begin{verbatim}
    class Person {
    private:
        std::string name;
        int age;
    
    public:
        Person(const std::string& name, int age) : name(name), age(age) {}

        void displayInfo() {
            std::cout << "Name: " << name << ", Age: " << age << std::endl;
        }
    };

    int main() {
        Person person("John", 25);
        person.displayInfo();
        return 0;
    }
    \end{verbatim}
    
    \horizontalline
    
    In this example, the Car class represents a car with a make, model, and year. An object of type Car named car is created in the main function, and its displayInfo method is called to print the car's make, model, and year.
\end{solution}

\subsubsection*{Abstraction}

Abstraction is a core concept in computer programming, helping to simplify complex systems by focusing on important aspects and hiding unnecessary details. It involves representing real-world objects or systems in a generalized 
way using classes and objects. Through abstraction, we create abstract classes that define a common interface and behavior for related objects while hiding implementation specifics. This allows us to manage system complexity, 
improve code organization, and promote reusability. Abstraction is crucial for creating modular, scalable, and maintainable software systems, allowing us to work at higher levels of abstraction without getting caught up in implementation intricacies.

\begin{solution}[Abstraction Example]
    An example of abstraction can be seen below: \\
    \horizontalline
    \begin{verbatim}
    #include <iostream>

    // Abstract class
    class Shape {
    public:
        virtual void draw() = 0;  // Pure virtual function
        
        void printName() {
            std::cout << "Shape" << std::endl;
        }
    };

    // Concrete class
    class Circle : public Shape {
    public:
        void draw() override {
            std::cout << "Drawing a circle." << std::endl;
        }
    };

    // Concrete class
    class Rectangle : public Shape {
    public:
        void draw() override {
            std::cout << "Drawing a rectangle." << std::endl;
        }
    };

    int main() {
        // Creating objects of concrete classes
        Circle circle;
        Rectangle rectangle;
        
        // Using the abstract class pointer to achieve abstraction
        Shape* shapePtr = nullptr;
        
        // Polymorphic behavior
        shapePtr = &circle;
        shapePtr->draw();  // Calls draw() of Circle class
        
        shapePtr = &rectangle;
        shapePtr->draw();  // Calls draw() of Rectangle class
        
        shapePtr->printName();  // Calls printName() of Shape class
        
        return 0;
    }
    \end{verbatim}
    
    \horizontalline

    This code demonstrates the concept of abstraction and polymorphism using an example of shapes. It defines an abstract class called "Shape" with a pure virtual function "draw()" and a non-virtual function "printName()". Two concrete classes, 
    "Circle" and "Rectangle", inherit from the abstract class and provide their own implementations of the "draw()" function.

    \noindent In the main function, objects of the concrete classes are created. An abstract class pointer, "shapePtr", is used to achieve abstraction. The pointer is assigned the address of the "Circle" object, and the "draw()" function is called, resulting
    in the message "Drawing a circle." Similarly, the pointer is assigned the address of the "Rectangle" object, and the "draw()" function is called, resulting in the message "Drawing a rectangle." This demonstrates polymorphic behavior, where the
    appropriate "draw()" function is called based on the object type.

    \noindent Additionally, the "printName()" function of the abstract class is called using the abstract class pointer. This function is not overridden in the concrete classes, so the implementation in the abstract class is invoked, printing the message 
    "Shape".

    \noindent Overall, this code illustrates the use of abstract classes, pure virtual functions, inheritance, and polymorphism to achieve abstraction and enable the flexible handling of different objects through a common interface. It showcases the power 
    of using abstract classes and polymorphism to create modular and extensible code for working with related objects in a data structures context.
\end{solution}

\subsection*{Sec. 2.2 - Using a Class}
\subsubsection*{Public Member Functions}

Public member functions in object-oriented programming allow objects to interact with each other and provide functionality to the outside world. They define the behavior and operations that objects of a class can perform, encapsulating the logic and operations
related to the class. Public member functions serve as an interface through which users can interact with objects, accessing and utilizing the functionality provided without exposing the internal implementation details. They promote code reusability, encapsulation, 
and maintainability, ensuring controlled access to object behavior and enabling modular design in object-oriented programming.

\begin{solution}[Public Member Functions Example]
    Below is an example of Public Member Functions in C++: \\
    \horizontalline
    \begin{verbatim}
    #include <iostream>

    class Rectangle {
    private:
        int width;
        int height;
        
    public:
        void setDimensions(int w, int h) {
            width = w;
            height = h;
        }
        
        int calculateArea() {
            return width * height;
        }
        
        void printInfo() {
            std::cout << "Width: " << width << ", Height: " << height << std::endl;
        }
    };
    
    int main() {
        Rectangle rect;
        
        rect.setDimensions(5, 3);
        int area = rect.calculateArea();
        std::cout << "Area: " << area << std::endl;
        
        rect.printInfo();
        
        return 0;
    }
    \end{verbatim}
    
    \horizontalline

    This code example demonstrates the concept of public member functions in C++. It defines a Rectangle class with private member variables for width and height. The class provides three public member functions: setDimensions(), calculateArea(), and printInfo().

    \noindent The setDimensions() function allows users to set the width and height of the rectangle by passing the values as parameters. The calculateArea() function performs the calculation of the rectangle's area by multiplying the width and height and returns the 
    result. Finally, the printInfo() function prints the width and height of the rectangle to the console.

    \noindent In the main() function, an object of the Rectangle class is created, and the public member functions are utilized to set the dimensions of the rectangle, calculate its area, and print the information. This example showcases how public member functions 
    provide an interface for interacting with objects, allowing users to manipulate data, perform computations, and retrieve information in a controlled manner, promoting encapsulation and modular design in C++ programming.
\end{solution}

\subsection*{Sec. 2.3 - Defining a Class}
\subsubsection*{Private Data Members}

In object-oriented programming (OOP), private data members are a fundamental concept that allows for encapsulation and data hiding. Private data members are variables declared within a class that can only be accessed or modified by member functions within the same class. 
By designating data members as private, they are shielded from direct access by code outside the class, ensuring that the internal state and implementation details of an object are protected. This encapsulation promotes data integrity, enhances code maintainability, and 
prevents external code from inadvertently modifying or corrupting the object's data. Private data members facilitate information hiding and abstraction, allowing objects to maintain their integrity while providing controlled access to their functionality through public 
member functions.

\begin{solution}[Private Data Members Example]
    Here is an example of private data members in C++: \\
    \horizontalline
    \begin{verbatim}
    #include <iostream>

    class BankAccount {
    private:
        std::string accountNumber;
        double balance;
        
    public:
        void deposit(double amount) {
            balance += amount;
        }
        
        void withdraw(double amount) {
            if (amount <= balance) {
                balance -= amount;
            } else {
                std::cout << "Insufficient balance." << std::endl;
            }
        }
        
        void displayBalance() {
            std::cout << "Account balance: " << balance << std::endl;
        }
    };
    
    int main() {
        BankAccount myAccount;
        
        myAccount.deposit(1000.0);
        myAccount.displayBalance();
        
        myAccount.withdraw(500.0);
        myAccount.displayBalance();
        
        myAccount.withdraw(800.0);
        myAccount.displayBalance();
        
        return 0;
    }
    \end{verbatim}
    
    \horizontalline

    In the main() function, an object of the BankAccount class is created. Public member functions are used to deposit an amount, display the balance, withdraw amounts, and display the updated balance.

    \noindent By making the accountNumber and balance private, they cannot be directly accessed or modified from outside the class. This ensures the encapsulation and data hiding of sensitive information. 
    Users can interact with the bank account object through the public member functions, maintaining data integrity and preventing unauthorized access to or modification of the private data members.

    \noindent This example illustrates how private data members in C++ provide encapsulation and data hiding. By hiding the internal implementation details, the class enforces controlled access to the data 
    and protects it from unauthorized manipulation. Private data members facilitate proper data management and security within an object, ensuring that only the intended interface, defined by public member 
    functions, is used to interact with and modify the object's state.
\end{solution}

\subsection*{Sec. 2.4 - Inline Member Functions}
\subsubsection*{Inline Member Functions}

An inline member function in C++ is a function that is defined within a class declaration and is marked with the inline keyword. When a member function is declared as inline, it suggests to the compiler that 
the function should be expanded at the point of its call instead of being invoked through a function call. This expansion replaces the function call with the actual code of the function, eliminating the overhead 
of the function call itself. Inline member functions are typically used for small and frequently used functions to improve performance by reducing the function call overhead. They provide a mechanism for code 
optimization and are especially useful when the function body is simple, making it more efficient to replace the function call with the actual code.

\begin{solution}[Inline Member Function Example]
    Below is an example of inline member functions: \\
    \horizontalline
    \begin{verbatim}
    #include <iostream>

    class Rectangle {
    private:
        int width;
        int height;
        
    public:
        void setDimensions(int w, int h) {
            width = w;
            height = h;
        }
        
        // Inline member function
        inline int calculateArea() {
            return width * height;
        }
    };
    
    int main() {
        Rectangle rect;
        
        rect.setDimensions(5, 3);
        int area = rect.calculateArea();
        
        std::cout << "Area: " << area << std::endl;
        
        return 0;
    }        
    \end{verbatim}
    
    \horizontalline

    In this example, we have a Rectangle class with private data members width and height. The class provides two member functions: setDimensions() and calculateArea(). The setDimensions() function sets the width 
    and height of the rectangle, while the calculateArea() function calculates the area of the rectangle by multiplying its width and height.

    \noindent The calculateArea() function is declared as an inline member function by using the inline keyword before the function declaration. This suggests to the compiler that the function should be expanded 
    at the point of its call. In this case, when calculateArea() is called, the compiler replaces the function call with the actual code of the function, eliminating the overhead of the function call.

    \noindent In the main() function, an object of the Rectangle class is created. The setDimensions() function is called to set the width and height of the rectangle. Then, the calculateArea() function is invoked 
    to calculate the area of the rectangle, and the result is printed to the console.

    \noindent This example demonstrates how inline member functions can be used to optimize code performance by reducing the overhead of function calls. By marking the calculateArea() function as inline, the compiler 
    expands the function call at the point of invocation, avoiding the function call overhead and providing direct access to the function's code. Inline member functions are particularly useful for small and frequently 
    used functions, where the expansion at the call site can lead to performance improvements.
\end{solution}

\subsection*{Sec. 2.5 - Mutators, Accessors, \& Private Helpers}
\subsubsection*{Mutators \& Accessors}

Mutators and accessors are two types of member functions commonly used in object-oriented programming to manipulate and retrieve the values of private data members of a class. Mutators, also known as setter functions or 
modifiers, are used to modify the values of private data members by accepting parameters and updating the internal state of the object. They provide a controlled way to change the values of the object's attributes while 
enforcing any necessary validation or business rules. Accessors, also known as getter functions or inspectors, are used to retrieve the values of private data members without allowing direct access to them. They return 
the values of private data members, allowing users to access the object's attributes in a read-only manner. Mutators and accessors play a crucial role in encapsulation, providing an interface to manipulate and retrieve 
the object's state while maintaining data integrity, encapsulation, and abstraction. They allow for controlled interaction with the object's data and facilitate modular design and code maintainability by separating the 
implementation details from the external interface of the class.

\begin{solution}[Mutators \& Accessors Example]
    Below is an example of mutators \& accessors in C++: \\
    \horizontalline
    \begin{verbatim}
    #include <iostream>

    class Circle {
    private:
        double radius;
        
    public:
        // Mutator
        void setRadius(double r) {
            if (r >= 0) {
                radius = r;
            }
        }
        
        // Accessor
        double getRadius() const {
            return radius;
        }
        
        double calculateArea() const {
            return 3.14 * radius * radius;
        }
    };
    
    int main() {
        Circle myCircle;
        
        myCircle.setRadius(5.0);
        double radius = myCircle.getRadius();
        double area = myCircle.calculateArea();
        
        std::cout << "Radius: " << radius << std::endl;
        std::cout << "Area: " << area << std::endl;
        
        return 0;
    }        
    \end{verbatim}
    
    \horizontalline

    In this example, we have a Circle class with a private data member radius. The class provides two member functions: setRadius() and getRadius().

    \noindent The setRadius() function is a mutator that allows users to set the value of the radius data member. It accepts a parameter r and updates the radius only if the value is non-negative.

    \noindent The getRadius() function is an accessor that returns the value of the radius data member. It allows users to retrieve the value of radius without directly accessing the private data member.

    \noindent In the main() function, an object of the Circle class is created. The setRadius() mutator is called to set the radius of the circle to 5.0. The getRadius() accessor is then used to retrieve the value of the 
    radius, and the calculateArea() function is invoked to calculate the area of the circle. Finally, the radius and area are printed to the console.

    \noindent This example demonstrates how mutators and accessors provide a controlled interface for manipulating and retrieving the values of private data members. The mutator setRadius() allows users to set the radius 
    of the circle, while the accessor getRadius() allows them to retrieve the radius. By encapsulating the private data member and providing these member functions, the class ensures data integrity and abstraction. Users 
    can interact with the object through the mutators and accessors without direct access to the private data member, promoting encapsulation and modular design in C++ programming.
\end{solution}

\subsection*{Sec. 2.6 - Initialization \& Constructors}
\subsubsection*{Data Member Initialization}

Data member initialization in C++ allows you to assign initial values to the data members of a class when objects are created. It provides a convenient way to ensure that data members have valid initial values and avoids 
the need for separate initialization steps. Data member initialization can be done using two approaches: member initialization list and default member initializer. Member initialization list initializes data members directly 
in the constructor's initialization list, while default member initializer assigns values to data members directly in the class declaration. By initializing data members during object creation, you can ensure that the object 
starts in a consistent state and avoid potential bugs or undefined behavior caused by uninitialized data. Data member initialization enhances code readability, simplifies object construction, and promotes good programming 
practices in C++.

\begin{solution}[Data Member Initialization Example]
    Here is an example of data member initialization in C++.
    \horizontalline
    \begin{verbatim}
    #include <iostream>

    class Rectangle {
    private:
        int width;
        int height;
    
    public:
        // Constructor with member initialization list
        Rectangle(int w, int h) : width(w), height(h) {
            // Additional constructor code, if needed
        }
    
        // Default member initializer
        int area = width * height;
    
        void printArea() {
            std::cout << "Area: " << area << std::endl;
        }
    };
    
    int main() {
        Rectangle rect(5, 3);
        rect.printArea();
    
        return 0;
    }        
    \end{verbatim}
    
    \horizontalline

    In this example, we have a Rectangle class with private data members width and height. There are two ways to initialize these data members.

    \noindent Firstly, in the constructor declaration, we use a member initialization list to initialize the width and height data members directly. The constructor takes two parameters w and h, and the member initialization 
    list assigns these values to the corresponding data members.

    \noindent Secondly, we can use default member initializer directly in the class declaration. In this case, we initialize the area data member using a default member initializer, which calculates the area as the product 
    of width and height.

    \noindent In the main() function, we create an object of the Rectangle class named rect with width 5 and height 3. The constructor initializes the width and height data members using the member initialization list. The 
    printArea() function is called, which displays the calculated area of the rectangle.

    \noindent This example demonstrates how data member initialization can be done using member initialization list in the constructor or default member initializer in the class declaration. It ensures that the data members 
    have valid initial values when objects are created, simplifies object construction, and promotes code readability. Data member initialization is a useful feature in C++ that helps ensure the consistency and integrity of 
    objects' initial states.
\end{solution}

\subsubsection*{Constructors}

Constructors in object-oriented programming (OOP) are special member functions that are responsible for initializing objects of a class. They are called automatically when an object is created and allow you to set the initial 
state of the object. Constructors have the same name as the class and can have parameters to receive values required for initialization. They can perform various tasks, such as allocating memory, initializing data members, 
setting default values, and executing other necessary initialization logic. Constructors play a crucial role in object creation and ensure that objects start in a valid and consistent state. They promote encapsulation, as 
they provide a controlled way to initialize objects and enforce any necessary validation or business rules during the creation process. Constructors contribute to code readability, reusability, and maintainability by encapsulating 
the object initialization logic within the class itself.

\begin{solution}[Constructors Example]
    Here is an example of constructors in C++: \\
    \horizontalline
    \begin{verbatim}
    #include <iostream>

    class Rectangle {
    private:
        int width;
        int height;
    
    public:
        // Default constructor
        Rectangle() {
            width = 0;
            height = 0;
        }
    
        // Parameterized constructor
        Rectangle(int w, int h) {
            width = w;
            height = h;
        }
    
        void printDimensions() {
            std::cout << "Width: " << width << ", Height: " << height << std::endl;
        }
    };
    
    int main() {
        // Creating objects using constructors
        Rectangle rect1; // Default constructor called
        Rectangle rect2(5, 3); // Parameterized constructor called
    
        // Printing dimensions
        rect1.printDimensions(); // Output: Width: 0, Height: 0
        rect2.printDimensions(); // Output: Width: 5, Height: 3
    
        return 0;
    }        
    \end{verbatim}
    
    \horizontalline

    In this example, we have a Rectangle class with private data members width and height. The class provides two constructors: a default constructor and a parameterized constructor.

    \noindent The default constructor initializes the width and height to 0. It is called automatically when an object is created without any arguments, as in the case of rect1.

    \noindent The parameterized constructor takes two arguments w and h and initializes the width and height using the provided values. It is called when an object is created with specific values, 
    as in the case of rect2.

    \noindent In the main() function, we create two objects of the Rectangle class, rect1 and rect2, using the constructors. We then call the printDimensions() function to display the dimensions 
    of the rectangles.

    \noindent This example demonstrates how constructors are used to initialize objects of a class. The default constructor allows objects to be created with default values, while the parameterized constructor 
    allows objects to be created with custom values. Constructors enable proper initialization of objects, ensuring they start in a valid state. They provide flexibility and encapsulation in object 
    creation, enhancing code readability and maintainability.
\end{solution}

\subsection*{Sec. 2.7 - Classes and Vectors / Classes}
\subsubsection*{Vectors}

The `std::vector' class in C++ is a dynamic array container that provides a flexible and convenient way to store and manipulate a sequence of elements. It allows for dynamic resizing, efficient element access, insertion, and deletion 
at both ends, and provides various member functions to perform common operations on the elements. Vectors are templated, which means they can store elements of any type, providing great flexibility. They offer automatic memory management, 
handling memory allocation and deallocation internally. Vectors are widely used in C++ programming due to their versatility, efficiency, and ease of use, making them a fundamental data structure for managing collections of elements.

\begin{solution}[Vectors Example]
    Here is an example of the `vectors' class in C++: \\
    \horizontalline
    \begin{verbatim}
    #include <iostream>
    #include <vector>
    
    int main() {
        // Create a vector of integers
        std::vector<int> numbers;
    
        // Add elements to the vector
        numbers.push_back(10);
        numbers.push_back(20);
        numbers.push_back(30);
    
        // Access elements using indexing
        std::cout << "First element: " << numbers[0] << std::endl;
        std::cout << "Second element: " << numbers[1] << std::endl;
        std::cout << "Third element: " << numbers[2] << std::endl;
    
        // Iterate over the vector using a loop
        std::cout << "All elements: ";
        for (int i = 0; i < numbers.size(); i++) {
            std::cout << numbers[i] << " ";
        }
        std::cout << std::endl;
    
        // Remove the last element
        numbers.pop_back();
    
        // Check the size of the vector
        std::cout << "Size of vector: " << numbers.size() << std::endl;
    
        return 0;
    }        
    \end{verbatim}
    
    \horizontalline

    In this example, we include the necessary header files for using std::vector. We create a vector named numbers that stores integers.

    \noindent We use the push\_back() function to add elements to the vector. In this case, we add the integers 10, 20, and 30 to the vector. We access elements of the vector using indexing, such as numbers[0] to access the first element. We iterate 
    over the vector using a loop and print all the elements. We use the pop\_back() function to remove the last element from the vector. Finally, we check the size of the vector using the size() function.

    \noindent This example demonstrates the basic usage of std::vector in C++. It shows how to create a vector, add elements, access elements using indexing, iterate over the vector, remove elements, and check the size. The std::vector class 
    provides a convenient and flexible way to work with dynamic arrays, making it a powerful data structure in C++ for managing collections of elements.
\end{solution}

\subsection*{Sec. 2.8 - Separate Files for Classes}
\subsubsection*{Two Files Per Class}

Separate files for classes in C++ programs provide a modular approach to organizing code. Each class is defined in its own header file, containing the class declaration, and the member function implementations are placed in a corresponding source 
file. This practice enhances code organization, readability, and reusability. It simplifies navigation, allowing developers to quickly locate and modify code related to specific classes. Separating classes into individual files also promotes code 
reuse by facilitating their inclusion in other projects. Additionally, it aids in managing dependencies, prevents name conflicts, and simplifies maintenance and debugging. Overall, utilizing separate files for classes in C++ programs improves code 
structure and facilitates the development and management of complex projects.

\subsection*{Sec. 2.9 - Choosing Classes to Create}
\subsubsection*{Decomposing Into Classes}

When creating classes in Object-Oriented Programming (OOP), it is important to follow certain guidelines to ensure a well-designed and effective class structure. Start by identifying the attributes and behaviors that define the class's purpose and 
responsibilities. Encapsulate the data by declaring private data members and provide public access through member functions. Design intuitive and descriptive names for the class and its members. Establish clear and meaningful relationships between 
classes, using inheritance and composition when appropriate. Implement appropriate constructors, destructors, and assignment operators to manage the lifecycle of objects. Strive for cohesive and focused classes with single responsibilities. Apply 
principles like encapsulation, abstraction, inheritance, and polymorphism to achieve modularity, code reusability, and maintainability. Document the class with clear comments and adhere to coding style conventions for consistency. Regularly review 
and refine the class design as needed to ensure a well-structured and efficient implementation.

\subsection*{Sec. 2.10 - Unit Testing (Classes)}
\subsubsection*{Testbenches}

In Object-Oriented Programming (OOP), a test bench refers to a dedicated component or code module designed to test and validate the functionality of other classes or modules in a system. It serves as an environment for conducting systematic and comprehensive 
testing of software components. A test bench provides a controlled setting to simulate different scenarios and input conditions, allowing developers to verify the correctness and robustness of their code. It typically includes test cases, input data, and expected 
output values, along with mechanisms to execute the tests and compare the actual results against the expected ones. By using test benches, developers can identify and rectify issues early in the development process, ensuring the quality and reliability of the 
software. Test benches play a crucial role in achieving effective testing and debugging practices, enabling thorough assessment and validation of object-oriented systems.

\subsubsection*{Regression Testing}

In Object-Oriented Programming (OOP), regression testing refers to the process of retesting previously tested code to ensure that any modifications or enhancements to the system do not introduce new defects or regressions. It involves rerunning existing test cases 
on the modified code to verify that the changes made to the system have not adversely affected its existing functionality. Regression testing is crucial for maintaining the stability and reliability of software systems, especially in complex object-oriented projects 
where changes in one module can have unintended consequences on other interconnected modules. By performing regression testing, developers can identify and fix any regressions or unintended side effects caused by code modifications, ensuring that the system continues 
to function as expected and previous functionalities are not compromised. Regression testing is an integral part of the software development lifecycle, providing confidence in the system's integrity and minimizing the risk of introducing new defects during the development 
and maintenance phases.

\subsubsection*{Erroneous Unit Tests}

Erroneous unit tests refer to test cases or test code that are flawed or incorrect, resulting in inaccurate or misleading test results. These tests may have various issues, such as incorrect assumptions, flawed logic, inadequate coverage, or improper assertions. Erroneous 
unit tests can lead to false positives or false negatives, where passing tests falsely indicate correct functionality or failing tests erroneously indicate defects. Such tests can be problematic as they can give a false sense of security or create confusion during the development 
process. It is important to identify and rectify erroneous unit tests promptly to ensure the reliability and effectiveness of the testing process. Conducting regular code reviews, employing static analysis tools, and encouraging collaboration and knowledge sharing within the 
development team can help in identifying and addressing erroneous unit tests, resulting in more accurate and reliable testing outcomes.

\begin{solution}[Unit Test Example]
    Here is an example of unit testing classes in C++: \\
    \horizontalline
    \begin{verbatim}
    // File: MyClass.h
    #ifndef MYCLASS_H
    #define MYCLASS_H
    
    class MyClass {
    private:
        int value;
    
    public:
        MyClass(int val);
    
        int getValue() const;
        void setValue(int val);
    };
    
    #endif
    
    // File: MyClass.cpp
    #include "MyClass.h"
    
    MyClass::MyClass(int val) : value(val) {}
    
    int MyClass::getValue() const {
        return value;
    }
    
    void MyClass::setValue(int val) {
        value = val;
    }
    
    // File: MyClassTest.cpp
    #include <gtest/gtest.h>
    #include "MyClass.h"
    
    TEST(MyClassTest, ConstructorSetsInitialValue) {
        MyClass obj(42);
        EXPECT_EQ(obj.getValue(), 42);
    }
    
    TEST(MyClassTest, SettingNewValueUpdatesValue) {
        MyClass obj(0);
        obj.setValue(100);
        EXPECT_EQ(obj.getValue(), 100);
    }
    
    int main(int argc, char** argv) {
        testing::InitGoogleTest(&argc, argv);
        return RUN_ALL_TESTS();
    }        
    \end{verbatim}
    
    \horizontalline

    In this example, we have a class called MyClass with a private member variable value and public member functions getValue() and setValue(). We write unit tests for this class using the Google Test framework.

    \noindent In the MyClassTest.cpp file, we define two test cases using the TEST macro provided by Google Test. Each test case focuses on testing a specific aspect of the MyClass class. For example, one test case checks if the constructor sets the initial value correctly, and another 
    test case verifies that setting a new value updates the value correctly.

    \noindent The main function initializes the Google Test framework using testing::InitGoogleTest and runs all the defined tests using RUN\_ALL\_TESTS(). To compile and run the tests, you would need to include the Google Test framework and compile the test files along with it.

    \noindent This example demonstrates how you can use unit testing to verify the behavior and correctness of your class implementation. Each test case focuses on a specific aspect of the class, ensuring that it behaves as expected in different scenarios. By running these tests, you can 
    identify and address any issues or regressions in your class implementation, leading to more reliable and robust code.
\end{solution}

\subsection*{Sec. 2.11 - Constructor Overloading}
\subsubsection*{Basics}

Constructor overloading in Object Oriented Programming (OOP) refers to the ability to define multiple constructors for a class, each with different set of parameters. This allows objects to be created with different initial states or configurations, providing flexibility and customization 
during object instantiation. By overloading constructors, developers can conveniently initialize objects with different combinations of values or provide default values for certain parameters. Constructor overloading enables the creation of objects that meet specific requirements or use cases, 
making the class more versatile and adaptable to different scenarios. It promotes code reuse and enhances the usability of the class by accommodating various ways of object initialization.

\begin{solution}[Constructor Overloading Example]
    Below is an example of constructor overloading in C++: \\
    \horizontalline
    \begin{verbatim}
    #include <iostream>
    
    class MyClass {
    private:
        int value;
    
    public:
        // Default constructor
        MyClass() {
            value = 0;
        }
    
        // Constructor with one parameter
        MyClass(int val) {
            value = val;
        }
    
        // Constructor with two parameters
        MyClass(int val1, int val2) {
            value = val1 + val2;
        }
    
        int getValue() const {
            return value;
        }
    };
    
    int main() {
        MyClass obj1;               // Calls the default constructor
        MyClass obj2(42);           // Calls the constructor with one parameter
        MyClass obj3(10, 20);       // Calls the constructor with two parameters
    
        std::cout << obj1.getValue() << std::endl;    // Output: 0
        std::cout << obj2.getValue() << std::endl;    // Output: 42
        std::cout << obj3.getValue() << std::endl;    // Output: 30
    
        return 0;
    }
    \end{verbatim}
    
    \horizontalline

    In the above example, the MyClass class demonstrates constructor overloading. It has three constructors: a default constructor, a constructor with one parameter, and a constructor with two parameters. Each constructor initializes the value member variable based on the provided arguments 
    or default values.

    \noindent In the main function, we create three objects of the MyClass class using different constructor calls. The first object obj1 is created using the default constructor, which sets the value to 0. The second object obj2 is created by invoking the constructor with one parameter, 
    setting the value to 42. The third object obj3 is created using the constructor with two parameters, where the value is the sum of the two provided values (10 and 20).

    \noindent By overloading the constructors, we can instantiate objects with different initial states or configurations depending on the parameters provided. This enhances the flexibility and usability of the class, allowing developers to create objects with specific values or default values 
    conveniently. Constructor overloading promotes code reuse and simplifies the process of object creation, making the class more versatile and adaptable to different use cases.
\end{solution}

\subsection*{Sec. 2.12 - Constructor Initializer List}

Constructor initializer lists in Object-Oriented Programming (OOP) provide a way to initialize class member variables directly in the constructor declaration, rather than assigning values to them within the body of the constructor. By using the initializer list syntax, constructors can efficiently 
initialize member variables, especially for cases involving const variables or reference variables that need to be initialized upon object creation. Constructor initializer lists offer several benefits, including improved performance, the ability to initialize const and reference variables, and the 
initialization of base class subobjects. They enhance code readability and maintainability by clearly expressing the initialization process and ensuring that member variables are properly initialized before the constructor body executes. Overall, constructor initializer lists are a powerful feature 
in C++ that enable efficient and proper initialization of class member variables during object construction.

\begin{solution}[Constructor Initializer List Example]
    Below is an example of constructor initializer list in C++: \\
    \horizontalline
    \begin{verbatim}
    #include <iostream>

    class MyClass {
    private:
        int value;
        const int constantValue;
        int& refValue;
    
    public:
        MyClass(int val, int& ref) : value(val), constantValue(42), refValue(ref) {
            // Constructor body
        }
    
        int getValue() const {
            return value;
        }
    
        int getConstantValue() const {
            return constantValue;
        }
    
        int& getRefValue() const {
            return refValue;
        }
    };
    
    int main() {
        int ref = 100;
        MyClass obj(42, ref);
    
        std::cout << obj.getValue() << std::endl;          // Output: 42
        std::cout << obj.getConstantValue() << std::endl;  // Output: 42
        std::cout << obj.getRefValue() << std::endl;       // Output: 100
    
        return 0;
    }
    \end{verbatim}
    
    \horizontalline

    In the above example, the MyClass class demonstrates the use of constructor initializer lists. It has three member variables: value, constantValue, and refValue, representing an integer, a constant integer, and a reference to an integer, respectively.

    \noindent In the MyClassTest.cpp file, we define two test cases using the TEST macro provided by Google Test. Each test case focuses on testing a specific aspect of the MyClass class. For example, one test case checks if the constructor sets the initial value correctly, and another test case 
    verifies that setting a new value updates the value correctly.

    \noindent By using the constructor initializer list, we can efficiently and directly initialize these member variables, including the initialization of const and reference variables, which cannot be assigned values inside the constructor body.

    \noindent The main function creates an object of the MyClass class, passing in the values 42 and ref as arguments. We can then access the member variables using the appropriate getter functions to verify their values.

    \noindent Constructor initializer lists enhance code readability by explicitly and efficiently initializing member variables during object construction. They ensure proper initialization of const and reference variables, making the code more robust and maintainable. By using initializer lists, 
    we can initialize member variables directly, avoiding unnecessary assignment statements within the constructor body.
\end{solution}

\subsection*{Sec. 2.13 - The `this' Implicit Parameter}
\subsubsection*{Implicit Parameter}

In C++, the `this' implicit parameter is a pointer that is automatically passed to member functions of a class. It refers to the object on which the member function is being called. The `this' pointer allows access to the member variables and member functions of the object within its own scope, 
distinguishing them from local variables or function parameters with the same name. It is particularly useful in scenarios where there is a need to differentiate between the object's member variables and function parameters that have the same names. The `this' pointer enables efficient and unambiguous 
access to the object's data and behavior, promoting encapsulation and facilitating object-oriented programming principles.

\subsubsection*{Using `this' In Class Member Functions and Constructors}

In C++, the `this' pointer is used in class member functions and constructors to refer to the object on which the function is being invoked. Within member functions, `this' allows direct access to the member variables and member functions of the current object, differentiating them from local variables 
or function parameters. It is particularly useful when there is a need to disambiguate between class members and local variables with the same name. `this' can also be used in constructors to initialize member variables, especially in cases where the parameter names clash with the member variable names. 
By using `this' in member functions and constructors, developers can ensure accurate and unambiguous access to the object's data and behavior, promoting clarity, readability, and maintainability of the code.

\begin{solution}[`this' Implicit Parameter Example]
    Here is an example of the use of `this' implicit parameter in C++: \\
    \horizontalline
    \begin{verbatim}
    #include <iostream>

    class MyClass {
    private:
        int value;
    
    public:
        MyClass(int value) {
            this->value = value;
        }
    
        void printValue() {
            std::cout << "Value: " << this->value << std::endl;
        }
    };
    
    int main() {
        MyClass obj(42);
        obj.printValue();  // Output: Value: 42
    
        return 0;
    }
    \end{verbatim}
    
    \horizontalline

    In the above example, we have a class called MyClass with a private member variable value and a constructor that takes an integer parameter. Inside the constructor, we use the `this' pointer to differentiate between the parameter value and the member variable value. By using this->value, we explicitly 
    refer to the member variable and assign the value of the parameter to it.

    \noindent The printValue() member function of MyClass also uses the `this' pointer. Within the function, we use this->value to access the member variable and print its value. In the main() function, we create an object of MyClass called obj and pass the value 42 to the constructor. We then call the 
    printValue() member function on the obj object, which outputs the value of the value member variable.

    \noindent By using the `this' pointer, we can differentiate between local variables and member variables within the class scope, ensuring the correct variable is accessed or modified. It promotes clarity and avoids naming conflicts between function parameters and member variables.
\end{solution}

\subsection*{Sec. 2.14 - Operator Overloading}
\subsubsection*{Overview}

Operator overloading in Object-Oriented Programming (OOP) allows the customization of the behavior of predefined operators for user-defined classes. It enables objects of a class to exhibit intuitive and meaningful behavior when used with operators such as +, -, *, /, ==, and so on. By overloading operators, 
developers can define how objects of a class interact with operators, making code more expressive and natural. Operator overloading enables the use of familiar syntax and semantics for user-defined types, enhancing code readability and maintainability. It allows objects to participate in operations that are 
consistent with their intended purpose, leading to more concise and intuitive code.

\subsubsection*{Overloading Same Operator}

Overloading the same operator more than once in a single class in C++ allows different behaviors to be defined for the same operator depending on the types of the operands. This feature is known as operator overloading with different argument types. By providing multiple implementations of an operator, each 
with distinct parameter types, the class can handle different scenarios and provide appropriate behavior for each case. This enables flexibility in how the class interacts with the operator, accommodating various operand combinations and ensuring consistent and meaningful operations. Overloading the same operator 
multiple times in a class allows for versatile and specialized behavior, enhancing the usability and adaptability of the class within different contexts.

\begin{solution}[Operator Overloading Example]
    Below is an example of operator overloading in C++: \\
    \horizontalline
    \begin{verbatim}
    #include <iostream>
    
    class Vector2D {
    private:
        double x, y;
    
    public:
        Vector2D(double x = 0.0, double y = 0.0) : x(x), y(y) {}
    
        Vector2D operator+(const Vector2D& other) const {
            return Vector2D(x + other.x, y + other.y);
        }
    
        Vector2D operator-(const Vector2D& other) const {
            return Vector2D(x - other.x, y - other.y);
        }
    
        Vector2D operator*(double scalar) const {
            return Vector2D(x * scalar, y * scalar);
        }
    };
    
    int main() {
        Vector2D v1(2.0, 3.0);
        Vector2D v2(1.0, 2.0);
    
        Vector2D sum = v1 + v2;           // Operator+ overload
        Vector2D difference = v1 - v2;     // Operator- overload
        Vector2D scaled = v1 * 2.5;        // Operator* overload
    
        std::cout << "Sum: (" << sum.x << ", " << sum.y << ")" << std::endl;
        std::cout << "Difference: (" << difference.x << ", " << difference.y << ")" 
        << std::endl;
        std::cout << "Scaled: (" << scaled.x << ", " << scaled.y << ")" 
        << std::endl;
    
        return 0;
    }
    \end{verbatim}
    
    \horizontalline

    In the above example, we have a class called Vector2D representing a 2D vector. The class overloads the +, -, and * operators to perform vector addition, subtraction, and scalar multiplication, respectively.

    \noindent By providing multiple implementations of the same operator, each with different parameter types (Vector2D and double in this case), the class can handle different scenarios. The operator+ overload performs element-wise addition of the coordinates, the operator- overload performs element-wise subtraction, 
    and the operator* overload performs scalar multiplication.

    \noindent In the main() function, we create two Vector2D objects, v1 and v2. We then use the overloaded operators to perform vector addition, subtraction, and scalar multiplication. The results are stored in sum, difference, and scaled variables, respectively.

    \noindent The program outputs the results, demonstrating how the overloaded operators provide intuitive and meaningful behavior for the Vector2D class. Overloading the same operator multiple times in the class allows for flexible and specialized operations, enhancing the usability and expressiveness of the class.
\end{solution}

\subsection*{Sec. 2.15 - Overloading Comparison Operators}

Overloading comparison operators in Object-Oriented Programming (OOP) allows custom behavior to be defined for comparing objects of user-defined classes. By overloading operators such as ==, !=, <, >, <=, and >=, developers can specify how objects should be compared based on their internal data or specific criteria. This 
enables objects of a class to be compared in a way that is meaningful and appropriate for the class's concept and purpose. Overloading comparison operators allows for more natural and intuitive code, as objects can be compared using familiar syntax and semantics. It enhances the readability and clarity of code by providing 
consistent and logical comparisons for user-defined types, making it easier to reason about the behavior of objects in comparison operations.

\begin{solution}[Comparison Operator Overloading Example]
    Below is an example of comparison operator overloading in C++: \\
    \horizontalline
    \begin{verbatim}
    #include <iostream>

    class Fraction {
    private:
        int numerator;
        int denominator;
    
    public:
        Fraction(int numerator = 0, int denominator = 1) 
        : numerator(numerator), denominator(denominator) {}
    
        bool operator==(const Fraction& other) const {
            return (numerator == other.numerator) 
            && (denominator == other.denominator);
        }
    
        bool operator!=(const Fraction& other) const {
            return !(*this == other);
        }
    
        bool operator<(const Fraction& other) const {
            return (numerator * other.denominator) 
            < (other.numerator * denominator);
        }
    
        bool operator>(const Fraction& other) const {
            return (numerator * other.denominator) 
            > (other.numerator * denominator);
        }
    };
    
    int main() {
        Fraction f1(3, 4);
        Fraction f2(2, 3);
        Fraction f3(3, 4);
    
        if (f1 == f2) {
            std::cout << "f1 and f2 are equal." << std::endl;
        } else {
            std::cout << "f1 and f2 are not equal." << std::endl;
        }
    
        if (f1 != f3) {
            std::cout << "f1 and f3 are not equal." << std::endl;
        } else {
            std::cout << "f1 and f3 are equal." << std::endl;
        }
    
        if (f2 < f1) {
            std::cout << "f2 is less than f1." << std::endl;
        } else {
            std::cout << "f2 is not less than f1." << std::endl;
        }
    
        if (f1 > f2) {
            std::cout << "f1 is greater than f2." << std::endl;
        } else {
            std::cout << "f1 is not greater than f2." << std::endl;
        }
    
        return 0;
    }
    \end{verbatim}
    
    \horizontalline

    In the above example, we have a Fraction class representing a fraction with a numerator and denominator. We overload the comparison operators ==, !=, <, and > to compare fractions.

    \noindent The operator== compares two fractions for equality, checking if both the numerator and denominator are the same. The operator!= is implemented in terms of operator==, negating the result. The operator< compares fractions based on their relative values, using cross multiplication to compare the numerators and 
    denominators. Similarly, the operator> is implemented based on operator<, but with the operands swapped.

    \noindent In the main() function, we create three Fraction objects, f1, f2, and f3, and perform comparison operations using the overloaded operators. We check for equality, inequality, less than, and greater than relationships between fractions and print the corresponding messages.

    \noindent The program outputs the results of the comparisons, demonstrating the custom behavior defined by overloading the comparison operators. By overloading these operators, we can compare fractions using intuitive syntax and obtain meaningful results based on their numerical values.
\end{solution}

\subsection*{Sec. 2.16 - Vector ADT}

The Vector Abstract Data Type (ADT) is a versatile and efficient dynamic array-like structure that allows for the flexible storage and manipulation of elements. It offers constant-time access by index, efficient appending and removal of elements, and automatic resizing when needed. Vectors are widely used in programming for 
their ability to adapt to changing collection sizes, making them suitable for a variety of applications. They provide a contiguous block of memory, allowing for efficient traversal and sequential access. With their ability to store elements of any type, vectors serve as a fundamental data structure in algorithms, data structures, 
and applications that require dynamic and efficient element storage. Understanding the capabilities and operations of the Vector ADT is crucial for effectively managing and manipulating collections of elements in programming tasks.

\begin{solution}[Vector ADT Example]
    Here is an example of a vector ADT in C++: \\
    \horizontalline
    \begin{verbatim}
    #include <iostream>
    #include <vector>
    
    int main() {
        // Creating a vector to store integers
        std::vector<int> numbers;
    
        // Adding elements to the vector
        numbers.push_back(10);
        numbers.push_back(20);
        numbers.push_back(30);
    
        // Accessing elements by index
        std::cout << "First element: " << numbers[0] << std::endl;
        std::cout << "Second element: " << numbers[1] << std::endl;
        std::cout << "Third element: " << numbers[2] << std::endl;
    
        // Iterating over the vector
        std::cout << "Elements in the vector: ";
        for (int i = 0; i < numbers.size(); i++) {
            std::cout << numbers[i] << " ";
        }
        std::cout << std::endl;
    
        // Removing an element from the vector
        numbers.pop_back();
    
        // Querying the size and capacity of the vector
        std::cout << "Size of the vector: " << numbers.size() << std::endl;
        std::cout << "Capacity of the vector: " << numbers.capacity() << std::endl;
    
        return 0;
    }
    \end{verbatim}
    
    \horizontalline

    In this example, we include the <vector> header to use the Vector ADT provided by the C++ Standard Library. We create a vector called numbers to store integers. We use the push\_back() function to add elements to the vector, and the [] operator to access elements by index. We iterate over the vector using a loop and print the 
    elements. Then, we remove an element using the pop\_back() function. Finally, we query the size of the vector using the size() function and the capacity using the capacity() function.

    \noindent When you run this program, it will output the elements of the vector, the size, and the capacity. This example demonstrates the basic usage of the Vector ADT in C++ for dynamic storage and manipulation of elements.
\end{solution}

\subsection*{Sec. 2.17 - Namespaces}

Namespaces in C++ provide a way to group related code elements and prevent naming conflicts. They act as a container for identifiers such as variables, functions, and classes, allowing them to be organized and accessed in a structured manner. By enclosing code within a namespace, we can avoid naming collisions between entities with 
the same name but defined in different contexts. Namespaces enhance code modularity, readability, and maintainability by providing a hierarchical structure to the codebase. They enable developers to create separate logical units and manage the scope of identifiers more effectively. With namespaces, it becomes easier to differentiate 
and reference code elements, making the codebase more manageable and reducing the risk of naming conflicts when integrating different libraries or modules.

\begin{solution}[Namespaces Example]
    Here is an example of namespaces in C++: \\
    \horizontalline
    \begin{verbatim}
    #include <iostream>

    // First namespace
    namespace First {
        void greet() {
            std::cout << "Hello from First namespace!" << std::endl;
        }
    }
    
    // Second namespace
    namespace Second {
        void greet() {
            std::cout << "Hello from Second namespace!" << std::endl;
        }
    }
    
    int main() {
        First::greet();   // Calling greet() from the First namespace
        Second::greet();  // Calling greet() from the Second namespace
    
        return 0;
    }
    \end{verbatim}
    
    \horizontalline

    In this example, we define two namespaces: First and Second. Each namespace has its own greet() function that outputs a greeting message. In the main() function, we explicitly specify the namespace when calling the greet() function to differentiate between the two implementations.

    \noindent This example demonstrates how namespaces in C++ allow us to organize code elements into separate logical units. By enclosing code within namespaces, we can prevent naming conflicts and explicitly specify which version of a function or variable to use. Namespaces help improve code readability and maintainability, 
    especially in larger projects where different libraries or modules may have overlapping identifiers.
\end{solution}

\subsection*{Sec. 2.18 - Static Data Members \& Functions}

Static data members and functions in C++ are associated with the class itself rather than specific instances of the class. A static data member is shared among all objects of the class and has a single instance regardless of the number of objects created. Similarly, a static member function is not bound to any specific object 
and can be called directly using the class name. Static members are useful for storing and accessing shared data or performing operations that are independent of individual objects. They can be accessed without creating an instance of the class and are commonly used for maintaining counts, global variables, utility functions, 
or class-wide properties. Static members provide a way to encapsulate data or functionality that is not tied to a specific object but belongs to the class as a whole.

\begin{solution}[Static Data Members \& Functions Example]
    Below is an example of static data members \& functions in C++: \\
    \horizontalline
    \begin{verbatim}
    #include <iostream>
    
    class MyClass {
    public:
        static int count;  // Static data member
    
        static void incrementCount() {  // Static member function
            count++;
        }
    
        void displayCount() {
            std::cout << "Count: " << count << std::endl;
        }
    };
    
    int MyClass::count = 0;  // Initializing static data member
    
    int main() {
        MyClass::incrementCount();  // Calling static member function
        MyClass obj1;
        obj1.displayCount();  // Output: Count: 1
    
        MyClass::incrementCount();
        MyClass obj2;
        obj2.displayCount();  // Output: Count: 2
    
        MyClass::count = 10;  // Modifying static data member directly
    
        MyClass obj3;
        obj3.displayCount();  // Output: Count: 10
    
        return 0;
    }
    \end{verbatim}
    
    \horizontalline

    In this example, we have a class called MyClass with a static data member count and a static member function incrementCount(). The count variable is shared among all objects of the class and is initialized to 0. The incrementCount() function increments the count by one. In the main() function, we call the static member 
    function incrementCount() using the class name MyClass::incrementCount(). We also create multiple objects of MyClass and call the member function displayCount() to display the current value of count. We can directly access and modify the static data member count using the class name as shown. The output demonstrates how 
    the static data member is shared among all objects and how the static member function can be used to manipulate it.
\end{solution}

\subsection{Chapter 3 - Introduction to Algorithms}

The second chapter of this week is Chapter 3 - Introduction to Algorithms.

\subsection*{Sec. 3.1 - Introduction to Algorithms}
\subsubsection*{Algorithms}

In object-oriented programming (OOP), an algorithm refers to a set of step-by-step instructions or procedures designed to solve a specific problem or perform a particular task. It is a logical sequence of operations that can be implemented in code to achieve a desired outcome. In OOP, algorithms are often encapsulated within 
methods or functions of classes, enabling reusability and modularity. Algorithms in OOP can involve various operations such as data manipulation, conditional statements, loops, and function calls. They play a crucial role in implementing the logic and functionality of programs by providing a systematic approach to solving problems 
and achieving specific objectives. Well-designed algorithms are efficient, correct, and maintainable, contributing to the overall effectiveness and quality of the software.

\subsubsection*{Algorithm Efficiency}

Algorithm efficiency in object-oriented programming (OOP) refers to the measure of how well an algorithm utilizes computational resources such as time and memory. It involves analyzing the performance characteristics of an algorithm and understanding its scalability as the input size increases. Efficiency is crucial in OOP as it 
directly impacts the program's overall performance and resource utilization. By designing and implementing efficient algorithms, developers can optimize the execution time and memory usage of their programs, leading to faster and more responsive software. Techniques like algorithmic complexity analysis, Big O notation, and data 
structure selection are employed to evaluate and improve algorithm efficiency. Striving for efficient algorithms is essential for developing high-performance applications that can handle large-scale data and complex computations effectively.

\begin{solution}[Big $\mathcal{O}$ Notation]
    Below are the different Big $\mathcal{O}$ Notations for algorithms with a simple explanation of each: \\
    \begin{center}
        \begin{tabular}{|c|p{12cm}|}
            \hline \textbf{Big $\mathcal{O}$ Notation} & \multicolumn{1}{|c|}{\textbf{Explanation}} \\ \hline
            $\mathcal{O}(1)$ & Constant time complexity - The algorithm's execution time is constant regardless of the input size. \\ \hline
            $\mathcal{O}(\log{(n)})$ & Logarithmic time complexity - The algorithm's execution time increases logarithmically with the input size. \\ \hline
            $\mathcal{O}(n)$ & Linear time complexity - The algorithm's execution time increases linearly with the input size. \\ \hline
            $\mathcal{O}(n\hspace{1pt}\log{(n)})$ & Linearithmic time complexity - The algorithm's execution time grows in proportion to the product of the input size and its logarithm. \\ \hline
            $\mathcal{O}(n^{2})$ & Quadratic time complexity - The algorithm's execution time increases quadratically with the input size. \\ \hline
            $\mathcal{O}(2^{n})$ & Exponential time complexity - The algorithm's execution time grows exponentially with the input size. \\ \hline
            $\mathcal{O}(n!)$ & 	Factorial time complexity - The algorithm's execution time increases factorially with the input size. \\ \hline
        \end{tabular}
    \end{center}
    These notations provide a way to express the scalability and efficiency of algorithms, allowing developers to compare and analyze different algorithms based on their time complexity and make informed decisions when designing and optimizing their programs.
\end{solution}

To further demonstrate what an algorithm is, we take a look at a couple of examples.

\begin{solution}[Algorithms Example]
    Below are some examples of algorithms in C++: \\
    \horizontalline
    \begin{verbatim}
    #include <iostream>
    #include <vector>
    #include <algorithm>

    int main() {
        std::vector<int> numbers = {4, 2, 7, 5, 1, 3, 6};

        // O(1) - Accessing an element in a vector using index
        int element = numbers[2];  // Accessing the third element

        // O(log n) - Binary search algorithm
        std::sort(numbers.begin(), numbers.end());
        bool found = std::binary_search(numbers.begin(), numbers.end(), 5);

        // O(n) - Linear search algorithm
        bool exists = std::find(numbers.begin(), numbers.end(), 8) != numbers.end();

        // O(n log n) - Sorting algorithm (e.g., Quick Sort)
        std::sort(numbers.begin(), numbers.end());

        // O(n^2) - Bubble sort algorithm
        for (int i = 0; i < numbers.size() - 1; i++) {
            for (int j = 0; j < numbers.size() - i - 1; j++) {
                if (numbers[j] > numbers[j + 1]) {
                    std::swap(numbers[j], numbers[j + 1]);
                }
            }
        }

        // O(2^n) - Recursive Fibonacci sequence calculation
        int fibonacci(int n) {
            if (n <= 1)
                return n;
            return fibonacci(n - 1) + fibonacci(n - 2);
        }

        // O(n!) - Permutation generation using recursion
        void generatePermutations(std::vector<int>& arr, int start, int end) {
            if (start == end) {
                for (int num : arr) {
                    std::cout << num << " ";
                }
                std::cout << std::endl;
            } else {
                for (int i = start; i <= end; i++) {
                    std::swap(arr[start], arr[i]);
                    generatePermutations(arr, start + 1, end);
                    std::swap(arr[start], arr[i]);
                }
            }
        }

        // Example usage of the functions
        std::cout << "Element: " << element << std::endl;
        std::cout << "Binary search found: " << found << std::endl;
        std::cout << "Linear search exists: " << exists << std::endl;

        std::cout << "Sorted numbers: ";
        for (int num : numbers) {
            std::cout << num << " ";
        }
        std::cout << std::endl;

        int fibResult = fibonacci(5);
        std::cout << "Fibonacci(5): " << fibResult << std::endl;

        std::vector<int> permutationArr = {1, 2, 3};
        generatePermutations(permutationArr, 0, permutationArr.size() - 1);

        return 0;
    }
    \end{verbatim}
    \horizontalline

    \noindent This code demonstrates the use of different algorithms corresponding to various Big $\mathcal{O}$ notations. It includes examples such as accessing an element in a vector with $\mathcal{O}(1)$, binary search with $\mathcal{O}(\log{(n)})$, linear search with $\mathcal{O}(n)$, sorting algorithms with $\mathcal{O}(n\hspace{1pt}\log{(n)})$ and $\mathcal{O}(n^2)$, recursive Fibonacci sequence calculation with $\mathcal{O}(2^n)$, 
    and permutation generation using recursion with $\mathcal{O}(n!)$. The output of the code showcases the results of each algorithm. This example provides a practical illustration of how different algorithms perform in terms of time complexity and highlights their corresponding efficiency characteristics.
\end{solution}

\subsection*{Sec. 3.2 - Relation Between Data Structures and Algorithms}
\subsubsection*{Algorithms for Data Structures}

Algorithms for data structures refer to the set of procedures or methods designed to operate on specific data structures efficiently. These algorithms encompass a wide range of operations, including insertion, deletion, searching, sorting, and traversal, among others. The goal is to devise algorithms that leverage the underlying properties and organization of the data structure to optimize time and space complexity. For example, data structures 
like arrays, linked lists, stacks, queues, trees, and graphs each have their own set of algorithms tailored to their unique characteristics and usage scenarios. Efficient algorithms for data structures are essential for achieving optimal performance and scalability in various applications, enabling efficient data manipulation and retrieval operations. By employing appropriate algorithms for specific data structures, developers can harness the 
full potential of these structures and unlock efficient solutions for a wide range of computational problems.

\subsubsection*{Algorithms Using Data Structures}

Algorithms using data structures refer to the utilization of specific data structures in combination with well-designed procedures to solve computational problems efficiently. These algorithms leverage the properties and functionality of data structures to store, organize, and manipulate data in a way that optimizes performance and resource utilization. By selecting the appropriate data structure for a given problem and implementing efficient 
algorithms, it is possible to achieve faster execution times, reduced memory consumption, and improved overall efficiency. Algorithms using data structures encompass a broad range of applications, including searching, sorting, graph traversal, pathfinding, data compression, and more. The synergy between algorithms and data structures is fundamental in computer science, enabling the development of powerful and efficient solutions to complex 
problems across various domains.

\begin{solution}[Data Structures Algorithm Example]
    Here is an example of an algorithm that is using a data structure in C++: \\
    \horizontalline
    \begin{verbatim}
    #include <iostream>
    #include <vector>
    #include <algorithm>
    
    int main() {
        std::vector<int> numbers = {5, 2, 7, 1, 3};
        
        // Sorting the numbers using the std::sort algorithm
        std::sort(numbers.begin(), numbers.end());
        
        // Searching for a specific number using the std::binary_search algorithm
        int target = 7;
        bool found = std::binary_search(numbers.begin(), numbers.end(), target);
        
        // Displaying the result
        if (found) {
            std::cout << "The number " << target 
            << " is found in the vector." << std::endl;
        } else {
            std::cout << "The number " << target 
            << " is not found in the vector." << std::endl;
        }
        
        return 0;
    }        
    \end{verbatim}
    
    \horizontalline

    \noindent In this example, a std::vector is used as the data structure to store a collection of numbers. The std::sort algorithm is employed to sort the numbers in ascending order. Then, the std::binary\_search algorithm is utilized to search for a specific number (target) within the sorted vector. The result is displayed based on whether the number is found or not. This example showcases the combination of algorithms 
    (std::sort and std::binary\_search) with the data structure (std::vector) to efficiently manipulate and search data, providing a concise and practical illustration of algorithms using data structures in C++.
\end{solution}

\subsection*{Sec. 3.3 - Algorithm Efficiency}
\subsubsection*{Algorithm Efficiency}

Algorithm efficiency refers to the measure of how well an algorithm performs in terms of time and space usage. It is crucial to assess and analyze the efficiency of algorithms as it directly impacts the overall performance and scalability of a program. Efficiency is commonly evaluated by considering the time complexity, which measures how the algorithm's execution time grows with the input size, and the space complexity, 
which determines the amount of memory required by the algorithm. The goal is to design and select algorithms that exhibit favorable efficiency characteristics, such as lower time and space complexities, to ensure optimal performance and resource utilization. By employing efficient algorithms, developers can significantly improve program efficiency, reduce computational costs, and enable the handling of larger datasets and 
more complex problem instances. Evaluating and optimizing algorithm efficiency is a fundamental aspect of algorithm design and analysis, enabling the development of faster and more scalable solutions in various domains.

\subsubsection*{Runtime Complexity, Best Case, \& Worst Case}

Runtime complexity refers to the measure of how the performance of an algorithm scales with the size of the input. It provides insights into the efficiency of an algorithm in terms of time and space usage. The best case runtime complexity represents the lowest possible amount of time an algorithm can take to complete, usually occurring when the input is in the most favorable configuration. On the other hand, the worst case 
runtime complexity represents the maximum amount of time an algorithm can take to complete, typically occurring when the input is in the least favorable configuration. Analyzing the best and worst case scenarios helps in understanding the upper and lower bounds of an algorithm's performance. By considering both the best and worst case runtime complexities, developers can make informed decisions about the algorithm's efficiency 
and choose the most suitable algorithm for a given problem, balancing trade-offs between time and space requirements.

\subsubsection*{Space Complexity}

Space complexity refers to the measure of the amount of memory or storage space required by an algorithm to solve a problem. It assesses how the space usage of an algorithm grows with the size of the input. The space complexity of an algorithm is influenced by factors such as the data structures used, the number of variables and their sizes, and any auxiliary space required during the execution. It is commonly expressed in terms 
of the maximum space used by the algorithm relative to the input size. Analyzing the space complexity helps in understanding the memory requirements of an algorithm and enables the estimation of how much space will be consumed during its execution. By considering the space complexity, developers can optimize memory utilization, minimize unnecessary storage allocation, and ensure the algorithm can handle larger inputs without 
exhausting available memory resources.

\begin{solution}[Algorithm Efficiency Example]
    Below is an example of algorithm efficiency in C++: \\
    \horizontalline
    \begin{verbatim}
    #include <iostream>
    #include <vector>
    
    // Function to find the maximum element in a vector
    int findMax(const std::vector<int>& nums) {
        int max = nums[0];
        for (int i = 1; i < nums.size(); ++i) {
            if (nums[i] > max) {
                max = nums[i];
            }
        }
        return max;
    }
    
    int main() {
        std::vector<int> numbers = {5, 2, 8, 3, 1};
    
        // Find the maximum element in the vector
        int maxNum = findMax(numbers);
        std::cout << "Maximum number: " << maxNum << std::endl;
    
        return 0;
    }        
    \end{verbatim}
    
    \horizontalline

    \noindent In this example, the findMax function takes a vector of integers as input and returns the maximum element in the vector. It uses a simple linear search algorithm to iterate through the vector and update the maximum element as it encounters larger values. The runtime complexity of this algorithm is $\mathcal{O}(n)$, where $n$ is the size of the input vector. In the best case, when the maximum element is located at the beginning 
    of the vector, the algorithm will terminate early, resulting in a lower execution time. In the worst case, when the maximum element is located at the end of the vector or when all elements are the same, the algorithm will perform the maximum number of comparisons, leading to a higher execution time. As for space complexity, this algorithm requires a constant amount of additional space to store the maximum element and loop variables, 
    resulting in $\mathcal{O}(1)$ space complexity.

    \noindent By analyzing the runtime complexity, best case, worst case, and space complexity of this example, we can understand the performance characteristics of the algorithm and make informed decisions about its efficiency and suitability for different input scenarios.
\end{solution}