\newcommand{\CSPBDataStruct}{CSPB 2270 - Computer Science 2: Data Structures}
\makeatletter
    \startcontents[sections]
    \thispagestyle{fancy}
\makeatother
\chapter{CSPB 2270 Details}
Below is the class description for \CSPBDataStruct. The class description for when the class was offered may be slightly different from the description of the course website. The program website is found at \href{https://www.colorado.edu/program/cspb/}{\textbf{CU Boulder Applied Computer Science}}. The program curriculum can be found at \href{https://www.colorado.edu/program/cspb/academics/curriculum-course-list}{\textbf{B.S. ACS Curriculum}} and the \CSPBDataStruct course description can be found at \href{https://www.colorado.edu/program/cspb/cspb-2270-computer-science-2-data-structures}{\textbf{CSPB 2270 Course Description}}.

\section{Class Description}
\horizontalline
\noindent \href{https://www.colorado.edu/program/cspb/cspb-2270-computer-science-2-data-structures}{\textbf{\CSPBDataStruct}} - Prerequisites: \href{https://www.colorado.edu/program/cspb/cspb-1300-computer-science-1-starting-computing}{\textbf{CSPB 1300}} Credits: \textbf{4}
\subsection{Brief Description of Course Content}
Studies data abstractions (e.g., stacks, queues, lists, trees) and their representation techniques (e.g., linking, arrays). Introduces concepts used in algorithm design and analysis including criteria for selecting data structures to fit their applications. \\ 

\noindent Topics include data and program representations, computer organization effect on performance and mechanisms used for program isolation and memory management. \\

\subsection{Specific Goals}
Below are some specific goals of \CSPBDataStruct. The first specific goal pertains to Specific Outcomes of Instruction.
\begin{solution}[Specific Outcomes of Instruction]
\noindent The following are the Specific Outcomes of Instruction for \CSPBDataStruct.
    \begin{itemize}
        \item Document code including precondition/postcondition contracts for functions and invariants for classes.
        \item Determine quadratic, linear and logarithmic running time behavior in simple algorithms, write big-O expressions to describe this behavior, and state the running time behaviors for all basic operations on the data structures presented in the course.
        \item Create and recognize appropriate test data for simple problems, including testing boundary conditions and creating/running test cases, and writing simple interactive test programs to test any newly implemented class.
        \item Define basic data types (vector, stack, queue, priority queue, map, list).
        \item Specify, design and test new classes using the principle of information hiding for the following data structures: array-based collections (including dynamic arrays), list-based collections (singly-linked lists, doubly-linked lists, circular-linked lists), stacks, queues, priority queues, binary search trees, heaps, hash tables, graphs (e.g. for depth-first and breadth-first search), and at least one balanced search tree.
        \item Be able to describe how basic data types are stored in memory (sequential or distributed), predict what may happen when they exceed those bounds.
        \item Correctly use and manipulate pointer variables to change variables and build dynamic data structures.
        \item Determine an appropriate data structure for given problems.
        \item Follow, explain, trace, and be able to implement standard computer science algorithms using standard data types, such as a stack-based evaluation of arithmetic expressions or a traversal of a graph.
        \item Recognize situations in which a subtask is nothing more than a simpler version of the larger problem and design recursive solutions for these problems.
        \item Follow, explain, trace, and be able to implement binary search and a variety of quadratic sorting algorithms including mergesort, quicksort and heapsort.
    \end{itemize}
\end{solution}

Next is a Brief List of Topics to be Covered for \CSPBDataStruct.
\begin{solution}[Brief List of Topics to be Covered]
\noindent The following is a Brief List of Topics to be Covered for \CSPBDataStruct.
    \begin{itemize}
        \item Cost of algorithms and Big O notation.
        \item Memory and pointers, structs, and dynamic memory allocation.
        \item Linked lists, stacks and queues.
        \item Trees: Binary trees, binary search trees, tree traversal, recursion.
        \item Tree balancing: red-black trees.
        \item Graphs: graph traversal algorithms, depth-first and breadth-first search.
        \item Hash tables, hash functions, collision resolution algorithms.
        \item Algorithms for sorting, such as insertion sort, bubble sort, quick sort, and merge sort.
    \end{itemize}
\end{solution}

Lastly, the following is a list of Mathematical Concepts Used for \CSPBDataStruct.
\begin{solution}[Mathematical Concepts Used]
\noindent The following is a brief list of Mathematical Concepts Used in \CSPBDataStruct.
    \begin{itemize}
        \item Logarithms
        \item Big O
        \item Recursion
        \item Trees
        \item Graphs
    \end{itemize}
\end{solution}

\subsection{Instructor Information}
The following are the details of this courses instructor. This course was given for the Summer term of 2023.
\begin{itemize}
    \item Name: Dr. Frank Jones
    \item Email: francis.jones@colorado.edu
    \item Office Hours:
    \begin{itemize}
        \item Moddays: 7:00 PM - 8:00 PM MT
        \item Wednesdays: 1:00 PM - 2:00 PM MT
        \item By Appointment
    \end{itemize}
\end{itemize}

\subsection{Important Dates}
The following are important dates for this course. This course runs from May 22, 2023 - August 18, 2023.
\begin{table}[ht]
    \centering
    \begin{tabular}{c c}
        \textbf{Assessment} & \textbf{Date} \\
        Exam 1 & July 7th, 10 AM - 10 PM MT \\
        Exam 2 & August 4th, 10 AM - 10 PM MT \\
        Interview Grade for Linked List Assignment & June 14-16 \\
        Interview Grade for Sorting Assignment & June 5-7 \\
        Interview Grade for Graph Assignment & July 26-28 \\
        Final Project & Aug 14-15 \\
        Quizzes & Usually Due on Mondays \\
        Programming Assignments & Usually Due on Tuesdays \\
        Assignment Interviews & Usually Held on Wednesdays \& Thursdays \\
    \end{tabular}
\end{table}
\par \noindent

\subsection{Grade Breakdown}
The following consists of a grade breakdown for this class.
\begin{table}[ht]
    \centering
    \begin{tabular}{c c c}
        \textbf{Item} & \textbf{Percent of Grade} & \textbf{Notes} \\
        Reading Quizzes & 10 & Assignments From Textbook \\
        Programming Assignments (10) & 25 & Autograded \\
        Assignment Interviews (3) & 15 & Interviews Asking About Assignments \\
        Exams (2) & 30 & Myriad of Types of Questions \\
        Final Project & 20 & Entire Grade is Interview Based \\
    \end{tabular}
\end{table}
\par \noindent

\subsection{Grading Scale}
The following is how grades will be assigned for this class.
\begin{table}[ht]
    \centering
    \begin{tabular}{c c}
        \textbf{Score \%} & \textbf{Grade} \\
        93 - 100 & A \\
        90 - 93 & A- \\
        87 - 90 & B+ \\
        83 - 87 & B \\
        80 - 83 & B- \\
        77 - 80 & C+ \\
        73 - 77 & C \\
        70 - 73 & C- \\
        67 - 70 & D+ \\
        63 - 67 & D \\
        60 - 63 & D- \\
    \end{tabular}
\end{table}
\par \noindent
% \horizontalline