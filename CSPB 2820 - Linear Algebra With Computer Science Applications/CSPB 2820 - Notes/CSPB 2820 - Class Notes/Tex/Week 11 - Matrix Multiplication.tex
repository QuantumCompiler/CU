\clearpage

\renewcommand{\ChapTitle}{Matrix Multiplication}
\renewcommand{\SectionTitle}{Matrix Multiplication}

\chapter{\ChapTitle}
\section{\SectionTitle}
\horizontalline{0}{0}

\subsection{Assigned Reading}

The reading assignments for this week can be found below:

\begin{itemize}
    \item \textbf{Sections 10.1, 10.2, 10.3, 10.4} from VMLS.
\end{itemize}

\subsection{Piazza}

Must post / respond to at least \textbf{four} Piazza posts this week. \due{(11/10/23)} \cbox{piazza-week11}

\subsection{Lectures}

The lectures for this week and their links can be found below:

\begin{itemize}
    \item \href{https://applied.cs.colorado.edu/mod/hvp/view.php?id=50788}{10.1 Matrix Multiply} $\approx$ 24 min.
    \item \href{https://applied.cs.colorado.edu/mod/hvp/view.php?id=50789}{Transpose Proof} $\approx$ 14 min.
    \item \href{https://applied.cs.colorado.edu/mod/hvp/view.php?id=50790}{10.1 Matrix Multiplication Identities} $\approx$ 19 min.
    \item \href{https://applied.cs.colorado.edu/mod/hvp/view.php?id=50791}{10.2 Linear Function Composition} $\approx$ 14 min.
    \item \href{https://applied.cs.colorado.edu/mod/hvp/view.php?id=50792}{10.3 Matrix Power} $\approx$ 12 min.
    \item \href{https://applied.cs.colorado.edu/mod/hvp/view.php?id=50793}{10.4 QR Factorization} $\approx$ 22 min.
    \item \href{https://applied.cs.colorado.edu/mod/hvp/view.php?id=50794}{Random Exercise: Chapter 10} $\approx$ 16 min.
\end{itemize}

\subsection{Assignments}

The assignment for this week is \pdflink{\AssDir Assignment 10 - Matrix Multiplication.pdf}{Assignment 10 - Matrix Multiplication} \due{(11/13/23)} \cbox{assignment-week11}

\subsection{Quiz}

The quiz for this week is \href{https://applied.cs.colorado.edu/mod/quiz/view.php?id=50797}{Chapter 10} \textbullet \pdflink{\QuizDir Quiz 10 - Matrix Multiplication.pdf}{Quiz 10 - Matrix Multiplication} \due{(11/13/23)} \cbox{quiz-week11}

\subsection{Chapter Summary}

The chapter that we will review this week is \textbf{VMLS Chapter 10 - Matrix Multiplication}. The first section that we will cover this week is \textbf{VMLS Section 10.1 - Matrix-Matrix Multiplication}.

\begin{notes}{VMLS Section 10.1 - Matrix-Matrix Multiplication}
    \subsubsection*{Overview}

    Matrix multiplication is a central operation in linear algebra that finds application in various disciplines such as computer science, physics, engineering, and economics. It involves the 
    multiplication of two matrices to yield a third matrix, known as the product matrix. \vspace*{1em}

    \subsubsection*{Concept}
    Matrix-matrix multiplication, also referred to as the matrix product, is an operation that is distinct from element-wise multiplication. It involves a row-by-column multiplication of two 
    matrices. \vspace*{1em}
    
    \subsubsection*{Conditions}
    The number of columns in the first matrix must equal the number of rows in the second matrix for multiplication to be defined. If we have a first matrix $A$ of size $m \times n$ 
    and a second matrix $B$ of size $n \times p$, then the resulting product matrix $C$ will have a size of $m \times p$. \vspace*{1em}

    \begin{highlight}[Mathematical Formula]
        The element at the $i$-th row and $j$-th column of the product matrix $C$ is calculated as follows:
        \begin{equation*}
            c_{ij} = \sum_{k=1}^{n} a_{ik} \cdot b_{kj}
        \end{equation*}
        where $c_{ij}$ is computed by summing the products of the corresponding elements of the $i$-th row of matrix $A$ and the $j$-th column of matrix $B$.
        \subsubsection*{Steps}
        \begin{enumerate}
            \item Select the $i$-th row from matrix $A$.
            \item Select the $j$-th column from matrix $B$.
            \item Multiply the corresponding elements and sum up the products.
            \item Place the result in the $i$-th row and $j$-th column of the product matrix $C$.
            \item Repeat the process for all rows and columns of the resulting matrix.
        \end{enumerate}
    \end{highlight}
    
    \subsubsection*{Properties}
    \begin{itemize}
        \item Matrix multiplication is associative, i.e., $(AB)C = A(BC)$, but not commutative since $AB \neq BA$ generally.
        \item An identity matrix $I$ serves as a multiplicative identity for matrix multiplication: $AI = IA = A$.
        \item The distributive property is applicable: $A(B + C) = AB + AC$ and $(B + C)A = BA + CA$.
    \end{itemize}
    
    \subsubsection*{Context}
    In computer science, matrix multiplication is utilized in algorithms, graphics, optimization problems, among other areas. Proficiency in executing matrix multiplication efficiently is critical for 
    enhancing the performance of many computational tasks.
\end{notes}

The next section that we will be covering this week is \textbf{VMLS Section 10.2 - Composition Of Linear Functions}.

\begin{notes}{VMLS Section 10.2 - Composition Of Linear Functions}
    \subsubsection*{Overview}

    Matrix multiplication is intimately linked with the composition of linear functions. Each matrix represents a linear transformation, and multiplying two matrices corresponds to composing two such 
    transformations. \vspace*{1em}

    \subsubsection*{Composition of Linear Functions}
    
    A linear transformation can be seen as a function that maps vectors from one vector space to another while preserving vector addition and scalar multiplication. If we represent linear transformations 
    $T: U \rightarrow V$ and $S: V \rightarrow W$ by matrices $A$ and $B$, the composition of $S$ and $T$, denoted as $S \circ T$, is the function that applies $T$ to a vector, followed by $S$. \vspace*{1em}

    \begin{highlight}[Mathematical Representation]
        The matrix representation of the composition $S \circ T$ is obtained by the matrix product $BA$, where $A$ and $B$ represent $T$ and $S$, respectively. For a vector $\vec{x}$ in $U$, the composition 
        is defined as:
        \begin{equation*}
            (S \circ T)(\vec{x}) = S(T(\vec{x})) = B(A\vec{x})
        \end{equation*}
    \end{highlight}
    
    \subsubsection*{Context}
    
    The concept of composing linear functions is essential in many mathematical and computational applications. In computer graphics, transformations applied to objects are represented by matrices. The 
    cumulative effect of multiple transformations is achieved by the product of the corresponding matrices, illustrating the practical significance of matrix multiplication in the composition of linear functions.
\end{notes}

The next section that we will be covering this week is \textbf{VMLS Section 10.3 - Matrix Power}.

\begin{notes}{VMLS Section 10.3 - Matrix Power}
    \subsubsection*{Overview}

    Matrix-matrix products provide a foundation for understanding the composition of linear functions. Given matrices $A$ of size $m \times p$ and $B$ of size $p \times n$, we can associate them 
    with linear functions $f: \mathbb{R}^p \to \mathbb{R}^m$ and $g: \mathbb{R}^n \to \mathbb{R}^p$, respectively. The composition of these functions is $h(x) = f(g(x)) = A(Bx) = (AB)x$, where 
    $h: \mathbb{R}^n \to \mathbb{R}^m$. This composition is a linear function, which can be written as $h(x) = Cx$ where $C = AB$.
    
    Matrix multiplication, when viewed in this way, clarifies why $AB \neq BA$ in general, as the composition order of linear functions matters. The section uses $2 \times 2$ matrices as an example 
    to illustrate this point. It further discusses the concept of a second difference matrix, which is a product of two consecutive difference matrices, and how it represents the second difference 
    function in terms of linear functions.
    
    The composition of affine functions is also an affine function. If $f(x) = Ax + b$ and $g(x) = Cx + d$ are affine functions, then their composition $h$ is $h(x) = A(Cx + d) + b = (AC)x + (Ad + b)$.
    
    Moreover, the chain rule of differentiation is explored, showing that the derivative matrix of a composed function $h$ can be expressed as the product of the derivative matrices of the individual 
    functions, which can be interpreted in terms of matrix multiplication.
    
    This also touches upon linear dynamical systems with state feedback, explaining that the state $x_t$ is affected by the input $u_t$, and with state feedback, the input is a linear function 
    of the state. This relationship can be captured using a state-feedback gain matrix $K$, and the closed-loop dynamics matrix is $A + BK$.
    
    Matrix powers are explained with their properties, including the effect of multiplying a matrix by itself multiple times and how this relates to paths in a directed graph represented by the matrix. The 
    section uses the adjacency matrix of a directed graph to exemplify how matrix powers can enumerate paths of a certain length within the graph.
    
    Lastly, the section discusses the application of matrix powers in linear dynamical systems, describing how the power of a matrix $A$ can propagate the system state forward in time. \vspace*{1em}
    
    \subsubsection*{Key Points}
    \begin{itemize}
        \item Matrix multiplication corresponds to the composition of linear functions.
        \item The order of multiplication reflects the order in which functions are applied.
        \item Affine functions compose to form another affine function, with the matrix of the composition being the product of the individual matrices.
        \item The chain rule in differentiation can be expressed as a matrix-matrix product.
        \item Matrix powers are useful in understanding the dynamics of systems and enumerating paths in graphs.
    \end{itemize}    
\end{notes}

The last section that we will cover this week is \textbf{VMLS Section 10.4 - QR Factorization}.

\begin{notes}{VMLS Section 10.4 - QR Factorization}
    \subsubsection*{Overview}

    \begin{highlight}[System State]
        The section discusses the concept of matrix powers in the context of time-invariant linear dynamical systems with inputs. The evolution of the system's state over time is described by the equation:
        \begin{equation*}
            x_{t+\tau} = A^\tau x_t + \sum_{j=0}^{\tau-1} A^j B u_{t+\tau-j-1}
        \end{equation*}
        where $x_t$ is the state at time $t$, $A$ is the system matrix, $B$ is the input matrix, and $u_t$ is the input at time $t$. This demonstrates how the state at a future time point is influenced 
        by the powers of $A$ and the inputs up to that time.
    \end{highlight}

    
    \subsubsection*{Properties of Matrices with Orthonormal Columns}
    \begin{itemize}
        \item The condition for $n$-vectors $a_1, \ldots, a_k$ to be orthonormal is given by the matrix equation $A^T A = I$, where $A$ is the matrix with columns $a_1, \ldots, a_k$.
        \item For an $m \times n$ matrix $A$ with orthonormal columns and $n$-vectors $x$ and $y$, we have the following properties:
        \begin{itemize}
        \item Norm preservation: $\| Ax \| = \| x \|$.
        \item Inner product preservation: $(Ax)^T (Ay) = x^T y$.
        \item Angle preservation: The angle between $Ax$ and $Ay$ is the same as the angle between $x$ and $y$.
        \end{itemize}
    \end{itemize}

    \begin{highlight}[QR Factorization]
        QR factorization is a method to decompose a matrix $A$ into a product of two matrices $Q$ and $R$, where $Q$ has orthonormal columns and $R$ is upper triangular. If $A$ is an $n \times k$ matrix 
        with linearly independent columns, the QR factorization is expressed as:
        \begin{equation*}
            A = QR
        \end{equation*}
        This factorization is derived from the Gram-Schmidt process, which is used to obtain the orthonormal vectors that form the columns of $Q$. The matrix $R$ is then constructed to reflect the 
        coefficients used to express the original columns of $A$ as linear combinations of the columns of $Q$.
    \end{highlight}
    
    The section also mentions alternative algorithms for QR factorization that are more stable in the presence of round-off errors and are efficient for sparse matrices.
    
    \subsubsection*{Key Takeaways}
    \begin{itemize}
        \item The QR factorization is useful for expressing a matrix as the product of an orthogonal (or orthonormal) matrix and an upper triangular matrix.
        \item The properties of norm, inner product, and angle preservation by orthonormal matrices have significant implications in various applications.
        \item Understanding the QR factorization and matrix powers is essential for analyzing linear dynamical systems and their behavior over time.
    \end{itemize}
\end{notes}