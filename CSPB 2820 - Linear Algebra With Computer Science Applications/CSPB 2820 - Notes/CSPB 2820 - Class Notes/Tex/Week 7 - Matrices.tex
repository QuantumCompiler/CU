\clearpage

\renewcommand{\ChapTitle}{Matrices}
\renewcommand{\SectionTitle}{Matrices}

\chapter{\ChapTitle}
\section{\SectionTitle}
\horizontalline{0}{0}

\subsection{Assigned Reading}

The reading assignments for this week can be found below:

\begin{itemize}
    \item \textbf{Sections 6.1, 6.2, 6.3, 6.4} from VMLS.
\end{itemize}

\subsection{Piazza}

Must post / respond to at least \textbf{four} Piazza posts this week. \due{(10/13/23)} \cbox{piazza-week7}

\subsection{Lectures}

The lectures for this week and their links can be found below:

\begin{itemize}
    \item \href{https://applied.cs.colorado.edu/mod/hvp/view.php?id=50733}{6.1 - Matrices} $\approx$ 16 min.
    \item \href{https://applied.cs.colorado.edu/mod/hvp/view.php?id=50734}{6.2 - Matrix Operations} $\approx$ 12 min.
    \item \href{https://applied.cs.colorado.edu/mod/hvp/view.php?id=50735}{6.3 - Matrix-Vector Multiply} $\approx$ 22 min.
    \item \href{https://applied.cs.colorado.edu/mod/hvp/view.php?id=50736}{Random Exercise: Chapter 6} $\approx$ 4 min.
\end{itemize}

\subsection{Assignments}

The assignment for this week is \pdflink{\AssDir Assignment 6 - Matrices.pdf}{Assignment 6 - Matrices} \due{(10/16/23)} \cbox{assignment-week7}

\subsection{Quiz}

The quiz for this week is \href{https://applied.cs.colorado.edu/mod/quiz/view.php?id=50739}{Chapter 6} \textbullet \pdflink{\QuizDir Quiz 6 - Matrices.pdf}{Quiz 6 - Matrices} \due{(10/16/23)} \cbox{quiz-week7}

\subsection{Chapter Summary}

The chapter that we will review this week is \textbf{VMLS Chapter 6 - Matrices}. The first section that we will cover this week is \textbf{VMLS Section 6.1 - Matrices}.

\begin{notes}{VMLS Section 6.1 - Matrices}
    \subsubsection*{Overview}

    Matrices are fundamental mathematical structures used to represent and manipulate data in various fields, including linear algebra, statistics, and computer science. A matrix is essentially a 
    two-dimensional array of numbers, symbols, or expressions arranged in rows and columns. It provides a compact and organized way to store and perform operations on data.

    Matrices have several key characteristics and properties:

    \begin{enumerate}
        \item Dimensions: A matrix is defined by its dimensions, which specify the number of rows and columns. For example, an "m x n" matrix has m rows and n columns.

        \item Elements: Each entry in a matrix is called an element, and it is identified by its position using row and column indices. Elements can be real numbers, complex numbers, or symbols.

        \item Types: Matrices come in various types, such as square matrices (equal number of rows and columns), rectangular matrices (unequal number of rows and columns), row matrices (1 row and 
        multiple columns), and column matrices (1 column and multiple rows).

        \item Operations: Matrices support various operations, including addition, subtraction, scalar multiplication, and matrix multiplication. Matrix multiplication is a fundamental operation in 
        linear algebra, often used for transformations and solving systems of linear equations.

        \item Transposition: The transpose of a matrix involves switching its rows and columns. It's denoted by "$A^{T}$" and is useful for various mathematical operations and applications.

        \item Inverse: Not all matrices have inverses, but those that do can be inverted to find a matrix that, when multiplied by the original matrix, results in the identity matrix.

        \item Applications: Matrices find applications in diverse fields, including physics, engineering, computer graphics, data analysis, and machine learning. They are used for solving linear 
        equations, representing transformations, and handling multi-dimensional data.
    \end{enumerate}

    Matrices are versatile mathematical structures that play a crucial role in representing, organizing, and manipulating data in various mathematical and practical contexts. Their properties and 
    operations make them invaluable tools in fields that deal with complex data and equations.
\end{notes}

The next section that we will cover this week is \textbf{VMLS Section 6.2 - Zero And Identity Matrices}.

\begin{notes}{VMLS Section 6.2 - Zero And Identity Matrices}
    \subsubsection*{Overview}

    Zero and Identity matrices are special types of matrices with unique properties and applications in linear algebra and mathematics.

    \subsubsection*{Zero Matrix}

    A Zero Matrix, denoted as $\mathbf{0}$ or $\mathbf{0}_{m \times n}$, is a matrix in which all of its elements are equal to zero. It has the following characteristics:

    \begin{itemize}
        \item All elements are zeros: $a_{ij} = 0$ for all $i$ and $j$.

        \item Dimensions: It can have any dimensions, such as $m \times n$, where $m$ represents the number of rows, and $n$ represents the number of columns.

        \item Properties: When a matrix is added to a zero matrix, it remains unchanged ($A + \mathbf{0} = A$), and when multiplied by a zero matrix, the result is also a zero matrix ($A \cdot \mathbf{0}
        = \mathbf{0}$).

        \item Application: Zero matrices are often used in matrix operations to represent additive identity elements and for various algebraic manipulations.
    \end{itemize}

    \subsubsection*{Identity Matrix}

    An Identity Matrix, denoted as $\mathbf{I}$ or $\mathbf{I}_n$ (or simply $\mathbf{I}$ when the size is clear), is a square matrix in which all elements on the main diagonal (from the top-left to 
    the bottom-right) are equal to 1, and all other elements are zero. It has the following characteristics:

    \begin{itemize}
        \item Diagonal elements are 1: $a_{ii} = 1$ for all $i$, and $a_{ij} = 0$ for $i \neq j$.

        \item Dimensions: It is always a square matrix, represented as $n \times n$.

        \item Properties: When a matrix $A$ is multiplied by an identity matrix, it remains unchanged ($A \cdot \mathbf{I} = A$). The identity matrix serves as a multiplicative identity element in 
        matrix multiplication.

        \item Application: Identity matrices are fundamental in linear transformations and solving systems of linear equations. They represent transformations that leave vectors unchanged in magnitude 
        and direction.
    \end{itemize}

    Zero and Identity matrices are essential mathematical constructs in linear algebra. The Zero Matrix contains only zero elements and serves as an additive identity, while the Identity Matrix contains 
    diagonal elements of 1 and serves as a multiplicative identity in matrix operations. These matrices play crucial roles in various mathematical operations and applications.
\end{notes}

The next section that we will cover this week is \textbf{VMLS Section 6.3 - Transpose, Addition, And Norm}.

\begin{notes}{VMLS Section 6.3 - Transpose, Addition, And Norm}
    \subsubsection*{Overview}

    Matrix operations play a fundamental role in linear algebra, offering powerful tools for manipulating and analyzing data. The transpose of a matrix, denoted as \(A^T\), involves swapping rows and 
    columns and has properties such as self-inversion and distribution over addition. Matrix addition, \(A + B\), combines two matrices element-wise, obeying commutative and associative properties. 
    The norm of a matrix, like the Frobenius norm (\(\|A\|_F\)), quantifies the matrix's magnitude and is useful in measuring data size. These operations serve as building blocks for solving various 
    mathematical problems, and their properties provide a solid foundation for understanding and working with matrices in diverse applications.

    \subsubsection*{Transpose of a Matrix}

    The transpose of a matrix $A$, denoted as $A^T$, is a new matrix obtained by swapping its rows and columns. The transpose operation has the following properties:
    
    \begin{itemize}
        \item If $A$ is an $m \times n$ matrix, then $A^T$ is an $n \times m$ matrix.
        \item The transpose of a transpose matrix is the original matrix: $(A^T)^T = A$.
        \item Transpose distributes over addition: $(A + B)^T = A^T + B^T$.
        \item Transpose of a product is the product of transposes in reverse order: $(AB)^T = B^TA^T$.
    \end{itemize}
    
    \subsubsection*{Matrix Addition}
    
    Matrix addition is an operation that combines two matrices of the same dimensions element-wise. Let $A$ and $B$ be two $m \times n$ matrices. Their sum, denoted as $A + B$, is calculated by adding 
    corresponding elements:
    
    \[
    (A + B)_{ij} = A_{ij} + B_{ij} \text{ for } 1 \leq i \leq m, \; 1 \leq j \leq n.
    \]
    
    Matrix addition has the following properties:
    
    \begin{itemize}
        \item Commutative: $A + B = B + A$.
        \item Associative: $(A + B) + C = A + (B + C)$.
        \item Identity element: There exists a zero matrix $\mathbf{0}$ such that $A + \mathbf{0} = A$ for any matrix $A$.
        \item Inverse element: For every matrix $A$, there exists an additive inverse matrix $(-A)$ such that $A + (-A) = \mathbf{0}$.
    \end{itemize}
    
    \subsubsection*{Norm of a Matrix}
    
    The norm of a matrix is a scalar value that quantifies the "size" or magnitude of the matrix. There are different ways to define the norm of a matrix, but one common norm is the Frobenius norm, denoted 
    as $\|A\|_F$, for an $m \times n$ matrix $A$. The Frobenius norm is calculated as:
    
    \[
    \|A\|_F = \sqrt{\sum_{i=1}^{m} \sum_{j=1}^{n} |a_{ij}|^2}.
    \]
    
    The Frobenius norm satisfies the following properties:
    
    \begin{itemize}
        \item Non-negativity: $\|A\|_F \geq 0$, and $\|A\|_F = 0$ if and only if $A$ is the zero matrix.
        \item Scalar multiplication: $\|\alpha A\|_F = |\alpha| \cdot \|A\|_F$ for any scalar $\alpha$.
        \item Triangle inequality: $\|A + B\|_F \leq \|A\|_F + \|B\|_F$.
    \end{itemize}
    
    These properties make the Frobenius norm a useful tool for measuring the size of matrices in various applications, including linear algebra and optimization.    
\end{notes}

The last section that we will cover this week is \textbf{VMLS Section 6.4 - Matrix-Vector Multiplication}.

\begin{notes}{VMLS Section 6.4 - Matrix-Vector Multiplication}
    \subsubsection*{Overview}

    Matrix-vector multiplication is a fundamental operation in linear algebra where a matrix is multiplied by a vector to produce a new vector. This operation combines the linear transformations represented 
    by the matrix with the vector's coordinates, resulting in a transformed vector. The process involves taking the dot product of each row of the matrix with the vector, and the resulting components of the 
    new vector represent linear combinations of the matrix's rows. Matrix-vector multiplication is a cornerstone in solving systems of linear equations, representing linear transformations, and finding 
    eigenvectors and eigenvalues, making it a crucial concept in various mathematical and scientific fields.

    \subsubsection*{Matrix-Vector Multiplication}

    Consider an $m \times n$ matrix $A$ and an $n$-dimensional column vector $\mathbf{v}$. The result of the matrix-vector multiplication $A\mathbf{v}$ is an $m$-dimensional column vector, denoted as $\mathbf{w}$. 
    Each component of $\mathbf{w}$ is obtained by taking a linear combination of the columns of matrix $A$, where the coefficients are provided by the components of vector $\mathbf{v}$. In other words, 
    $\mathbf{w}$ is computed as follows:

    \[
    \mathbf{w} = A\mathbf{v} =
    \begin{bmatrix}
    a_{11} & a_{12} & \cdots & a_{1n} \\
    a_{21} & a_{22} & \cdots & a_{2n} \\
    \vdots & \vdots & \ddots & \vdots \\
    a_{m1} & a_{m2} & \cdots & a_{mn}
    \end{bmatrix}
    \begin{bmatrix}
    v_1 \\
    v_2 \\
    \vdots \\
    v_n
    \end{bmatrix}
    =
    \begin{bmatrix}
    \sum_{i=1}^{n} a_{1i}v_i \\
    \sum_{i=1}^{n} a_{2i}v_i \\
    \vdots \\
    \sum_{i=1}^{n} a_{mi}v_i
    \end{bmatrix}
    \]

    Key points about matrix-vector multiplication include:

    \begin{enumerate}
        \item \textbf{Linear Combination:} This operation computes a linear combination of the columns of matrix $A$, with the weights determined by the components of vector $\mathbf{v}$.
        
        \item \textbf{Dimensions:} The dimensions of the resulting vector $\mathbf{w}$ depend on the dimensions of matrix $A$ and vector $\mathbf{v}$. If $A$ is of size $m \times n$, then $\mathbf{v}$ 
        must be an $n$-dimensional vector, resulting in an $m$-dimensional vector $\mathbf{w}$.
        
        \item \textbf{Applications:} Matrix-vector multiplication is used for solving linear equations, representing linear transformations, and finding solutions to optimization problems. It plays a 
        vital role in computer graphics, data analysis, and scientific simulations.
        
        \item \textbf{Computational Efficiency:} Efficient algorithms exist for matrix-vector multiplication, making it suitable for large matrices and enabling complex computations in various fields.
    \end{enumerate}

    Matrix-vector multiplication serves as a fundamental building block for advanced linear algebraic operations and is a cornerstone of scientific and engineering applications.
\end{notes}