\clearpage

\renewcommand{\ChapTitle}{SIR/Modeling/Dynamical Systems}
\renewcommand{\SectionTitle}{SIR/Modeling/Dynamical Systems}

\chapter{\ChapTitle}
\section{\SectionTitle}
\horizontalline{0}{0}

\subsection{Assigned Reading}

The reading assignments for this week are from, \VMLS \hspace*{1pt} and \PyCap:

\begin{itemize}
    \item \textbf{VMLS Chapter 9.1 - Linear Dynamical Systems}
    \item \textbf{VMLS Chapter 9.2 - Population Dynamics}
    \item \textbf{VMLS Chapter 9.3 - Epidemic Dynamics}
    \item \textbf{VMLS Chapter 9.4 - Motion Of A Mass}
    \item \textbf{VMLS Chapter 9.5 - Supply Chain Dynamics}
    \item \textbf{Python Companion Chapter 9.1 - Linear Dynamical Systems}
    \item \textbf{Python Companion Chapter 9.2 - Population Dynamics}
    \item \textbf{Python Companion Chapter 9.3 - Epidemic Dynamics}
    \item \textbf{Python Companion Chapter 9.4 - Motion Of A Mass}
    \item \textbf{Python Companion Chapter 9.5 - Supply Chain Dynamics}
\end{itemize}

\subsection{Piazza}

Must post / respond to at least \textbf{four} Piazza posts this week.

\subsection{Lectures}

The lectures for this week and their links can be found below:

\begin{itemize}
    \item \href{https://applied.cs.colorado.edu/mod/hvp/view.php?id=50773}{Introduction To Modelling With Systems Of Non-linear Differential Equations} $\approx$ 13 min.
    \item \href{https://scholarworks.smith.edu/textbooks/2/}{Calculus in Context}
    \item \href{https://www.youtube.com/watch?v=NS0jvPLtgGc}{VMSL - Dynamical Systems Overview - Chapter 9} $\approx$ 39 min.
\end{itemize}

\subsection{Assignments}

The assignment for this week is:

\begin{itemize}
    \item \href{https://github.com/QuantumCompiler/CU/tree/main/CSPB%202820%20-%20Linear%20Algebra%20With%20Computer%20Science%20Applications/Assignments/Assignment%209%20-%20SIR%20Model}{Assignment 9 - SIR Model}
\end{itemize}

\subsection{Quiz}

The quiz for this week is:

\begin{itemize}
    \item \href{https://applied.cs.colorado.edu/mod/quiz/view.php?id=50777}{Chapter 9}
\end{itemize}

\newpage

\subsection{Chapter Summary}

The chapter that we will review this week is \textbf{VMLS Chapter 9 - Linear Dynamical Systems}. The first section that we will cover this week is \textbf{VMLS Section 9.1 - Linear Dynamical Systems}.

\begin{notes}{VMLS Section 9.1 - Linear Dynamical Systems}
    \subsubsection*{Overview}

    Linear dynamical systems are mathematical models used to describe the time-evolution of variables in various fields such as physics, engineering, economics, and biology. They are particularly useful 
    for studying systems where linear relationships between variables exist. Key points about linear dynamical systems include:

    \begin{itemize}
        \item \textbf{Linearity:} LDS assume that the relationships governing the system's behavior are linear. This means that the evolution of each variable is a linear combination of its current value 
        and other variables' values, weighted by constant coefficients.
        \item \textbf{State-Space Representation:} LDS are often represented in a state-space form. The state vector contains all relevant variables, and the dynamics are described by a set of linear 
        differential or discrete equations. This representation simplifies the analysis and control of complex systems.
        \item \textbf{Time-Invariance:} Linear dynamical systems are time-invariant, meaning that system behavior does not change over time. The coefficients in the equations governing the system remain 
        constant.
        \item \textbf{Stability and Control:} Stability analysis plays a crucial role in LDS. Engineers and scientists use stability criteria to determine if a system's behavior remains bounded or diverges 
        over time. Control theory leverages LDS to design controllers that regulate system behavior.
        \item \textbf{Eigenvalues and Eigenvectors:} Eigenvalues and eigenvectors of the system matrix are essential for characterizing the behavior of LDS. Eigenvalues indicate stability, while 
        eigenvectors determine the system's response to different initial conditions.
        \item \textbf{Applications:} Linear dynamical systems find applications in a wide range of domains, including mechanical systems, electrical circuits, chemical reactions, population dynamics, 
        and financial modeling. They are instrumental in predicting system responses and designing control strategies.
        \item \textbf{Matrix Exponentiation:} Solving linear dynamical systems often involves matrix exponentiation. The system matrix is exponentiated to obtain the time evolution operator, which allows 
        us to predict future states of the system.
    \end{itemize}
    
    Linear dynamical systems provide a foundational framework for understanding and predicting the behavior of dynamic systems. Their simplicity, analytical tractability, and wide applicability make them 
    indispensable tools for engineers, scientists, and researchers in various fields.

    \subsubsection*{Markov Model}

    A Markov model, named after Russian mathematician Andrey Markov, is a mathematical framework used to describe stochastic processes where future events or states depend solely on the current state 
    and are independent of previous states. Markov models are widely applied in various fields, including physics, engineering, economics, and biology. Key points about Markov models include:

    \begin{itemize}
        \item \textbf{Markov Property:} The central assumption in a Markov model is the Markov property, which states that the probability of transitioning to a future state depends only on the current 
        state and not on the sequence of past states.
        \item \textbf{State Space:} Markov models are defined by a finite or countable set of states, representing possible conditions or configurations of the system under study. Transitions between 
        states are governed by transition probabilities.
        \item \textbf{Transition Probabilities:} Transition probabilities specify the likelihood of moving from one state to another in a single time step. These probabilities are often represented 
        in a transition matrix, where each entry corresponds to the probability of transitioning between specific states.
        \item \textbf{Homogeneous and Non-Homogeneous Models:} Markov models can be homogeneous, where transition probabilities remain constant over time, or non-homogeneous, allowing transition 
        probabilities to change over time.
        \item \textbf{Applications:} Markov models find applications in various fields, including weather forecasting, stock market analysis, natural language processing, epidemiology (e.g., disease 
        spread modeling), and machine learning (e.g., Hidden Markov Models in speech recognition).
        \item \textbf{Steady-State Analysis:} In many Markov models, the long-term behavior is of interest. Steady-state analysis involves finding the equilibrium distribution of states, which 
        represents the probabilities of being in each state over time.
        \item \textbf{Limitations:} Markov models assume that future events are conditionally independent of the past given the current state. While this simplifying assumption works well in many 
        scenarios, it may not capture complex dependencies in some real-world systems.
    \end{itemize}
    
    Markov models provide a powerful framework for modeling and analyzing systems that exhibit random and sequential behavior. Their simplicity, along with their ability to capture dynamic processes, 
    makes them valuable tools for understanding and predicting various phenomena.
\end{notes}

The next section that we will cover this week is \textbf{VMLS Section 9.2 - Population Dynamics}.

\begin{notes}{VMLS Section 9.2 - Population Dynamics}
    \subsubsection*{Overview}

    Population dynamics refers to the study of how the size and composition of populations change over time. In the context of linear dynamical systems, population dynamics are often modeled using 
    differential equations and matrices to describe the interactions and evolution of populations. Key points about population dynamics in linear systems include:

    \begin{itemize}
        \item \textbf{State Variables:} In population dynamics, state variables represent characteristics of a population, such as the number of individuals in different age groups or species. These 
        state variables are typically organized into vectors or matrices.
        \item \textbf{Linear Differential Equations:} Linear dynamical systems describe how state variables change over time using linear differential equations. These equations capture birth, death, 
        migration, and other population processes as linear transformations.
        \item \textbf{Matrix Representation:} Matrices play a crucial role in linear dynamical systems, with each matrix representing the rates of change or interactions between different subpopulations. 
        Matrix multiplication allows for efficient modeling of complex population structures.
        \item \textbf{Eigenvalues and Eigenvectors:} Eigenvalues and eigenvectors of population matrices provide insights into the long-term behavior of populations. Stable eigenvalues indicate 
        population equilibrium, while unstable eigenvalues suggest population instability.
        \item \textbf{Steady States:} Steady states represent population configurations where population sizes remain constant over time. Finding steady states involves solving linear systems of 
        equations and often requires considering external factors like resource availability.
        \item \textbf{Limitations:} Linear dynamical models assume linear relationships between population variables and may not capture all nuances of real-world population dynamics, such as nonlinear 
        interactions or demographic stochasticity.
        \item \textbf{Applications:} Linear dynamical models are applied in ecology, epidemiology, economics, and other fields to understand and predict population trends, disease spread, economic 
        growth, and more.
    \end{itemize}
    
    Population dynamics within linear dynamical systems provide a valuable framework for studying and managing populations in various contexts. By mathematically modeling population interactions and 
    changes, researchers gain insights into the stability and sustainability of ecosystems, economies, and communities.
\end{notes}

The next section that we will cover this week is \textbf{VMLS Section 9.3 - Epidemic Dynamics}.

\begin{notes}{VMLS Section 9.3 - Epidemic Dynamics}
    \subsubsection*{Overview}

    Epidemic dynamics is a crucial field of study that focuses on understanding the spread and control of infectious diseases within populations. In the context of linear dynamical systems, epidemic 
    dynamics involves modeling the transmission of diseases using mathematical equations and matrices. Key points about epidemic dynamics in linear systems include:

    \begin{itemize}
        \item \textbf{State Variables:} In epidemic models, state variables represent different populations or groups involved in the disease transmission process. These may include susceptible 
        individuals, infected individuals, and recovered individuals, often organized into vectors or matrices.
        \item \textbf{Compartmental Models:} Epidemic models are often compartmental, where individuals move between compartments based on their disease status. Common compartments include 
        Susceptible (S), Infected (I), and Recovered (R), giving rise to the SIR model.
        \item \textbf{Linear Differential Equations:} Linear dynamical systems are used to describe how the number of individuals in each compartment changes over time. These systems capture 
        disease transmission rates, recovery rates, and other factors as linear transformations.
        \item \textbf{Matrix Representation:} Matrices represent the interactions and transitions between compartments. The basic reproduction number (\(R_0\)) is determined by the eigenvalues of 
        these matrices, providing insights into epidemic outcomes.
        \item \textbf{Epidemic Threshold:} Linear models often identify a critical threshold value of \(R_0\) below which an epidemic cannot sustain itself, leading to disease extinction. Above 
        this threshold, epidemics can occur and may reach an endemic equilibrium.
        \item \textbf{Control Strategies:} Linear models help evaluate the impact of various control strategies, such as vaccination, quarantine, and social distancing. These strategies can be 
        represented as changes in model parameters.
        \item \textbf{Limitations:} Linear dynamical models assume constant transmission rates and homogeneous mixing, which may not capture all real-world complexities. More advanced models, 
        including nonlinear and stochastic approaches, are often used for greater accuracy.
        \item \textbf{Applications:} Epidemic dynamics models have been instrumental in understanding and managing infectious diseases, including the study of COVID-19, HIV, influenza, and many 
        others. They inform public health policies and interventions.
    \end{itemize}

    Epidemic dynamics within linear dynamical systems provide a valuable framework for predicting disease outbreaks, assessing the effectiveness of interventions, and guiding public health decisions. 
    These models offer insights into the complex interplay between disease spread, population behavior, and control measures.
\end{notes}

The next section that we will cover this week is \textbf{VMLS Section 9.4 - Motion Of A Mass}.

\begin{notes}{VMLS Section 9.4 - Motion Of A Mass}
    \subsubsection*{Overview}

    The motion of a mass within linear dynamical systems is a fundamental concept in physics and engineering. It involves describing the behavior of a physical mass (or object) under the influence of 
    external forces using linear equations and matrices. Key points about the motion of a mass in linear dynamical systems include:
    
    \begin{itemize}
        \item \textbf{State Variables:} In this context, the state variables represent the position, velocity, and acceleration of the mass. These variables are often organized into vectors or matrices 
        to describe the system comprehensively.
        \item \textbf{Newton's Second Law:} The motion of a mass is governed by Newton's second law of motion, which states that the force acting on an object is equal to the mass of the object times 
        its acceleration (\(F = ma\)). Linear systems translate this law into differential equations and matrix equations.
        \item \textbf{Linear Differential Equations:} Linear dynamical systems model the motion of a mass using linear differential equations that relate position, velocity, and acceleration. These 
        equations may include external forces, such as springs or dampers, which are represented as linear transformations.
        \item \textbf{Matrix Representation:} Matrices are used to represent the linear relationships between state variables. For example, in a mechanical system, matrices describe how forces, 
        displacements, and velocities are related.
        \item \textbf{Eigenvalues and Stability:} The eigenvalues of the system matrices provide insights into the stability and behavior of the mass's motion. Stable systems lead to bounded and 
        predictable motion, while unstable systems may exhibit unbounded behavior.
        \item \textbf{Control and Feedback:} Linear dynamical systems allow for the application of control strategies to modify the mass's motion. Feedback control, where system states are measured 
        and adjusted in real-time, plays a significant role in engineering applications.
        \item \textbf{Applications:} Understanding the motion of a mass is essential in various fields, including mechanical engineering, robotics, aerospace engineering, and physics. It is used in 
        designing structures, optimizing control systems, and analyzing the behavior of physical systems.
        \item \textbf{Limitations:} Linear dynamical models assume linearity and may not capture nonlinear effects or extreme conditions. In such cases, more advanced nonlinear dynamical systems are 
        employed.
    \end{itemize}
    
    The study of the motion of a mass in linear dynamical systems provides a foundation for analyzing and controlling the behavior of mechanical and physical systems. It serves as a fundamental concept 
    in engineering and physics, enabling the design and optimization of a wide range of systems and structures.
\end{notes}

The last section that we will cover this week is \textbf{VMLS Section 9.5 - Supply Chain Dynamics}.

\begin{notes}{VMLS Section 9.5 - Supply Chain Dynamics}
    \subsubsection*{Overview}

    In the context of linear dynamical systems, supply chain dynamics refer to the analysis of how goods, materials, and information flow through a supply chain network over time. These systems are 
    typically modeled using linear equations and matrices to understand and optimize supply chain behavior. Key aspects of supply chain dynamics in linear dynamical systems include:
    
    \begin{itemize}
        \item \textbf{Network Representation:} Supply chains are represented as networks with nodes representing entities like suppliers, manufacturers, and retailers. Links denote the flow of products 
        and information.
        \item \textbf{Demand and Supply:} Linear models incorporate supply and demand equations to describe production, distribution, and consumption within the supply chain.        
        \item \textbf{Lead Times:} Lead times, representing product transit times, affect inventory and order quantities.        
        \item \textbf{Inventory Management:} Linear models optimize inventory levels and reorder points for efficient operations.        
        \item \textbf{Production Planning:} Planning production quantities considers demand forecasts and capacity constraints.        
        \item \textbf{Transportation and Logistics:} Optimization includes route planning, carrier selection, and logistics to minimize costs.        
        \item \textbf{Uncertainty and Variability:} Linear models incorporate stochastic elements for probabilistic analysis.        
        \item \textbf{Performance Metrics:} Metrics like order fill rates and inventory turnover assess supply chain effectiveness.        
        \item \textbf{Sustainability:} Models evaluate environmental impact and sustainability in supply chain decisions.        
        \item \textbf{Real-Time Optimization:} Linear systems aid real-time decisions for adapting to changing market conditions.        
        \item \textbf{Integration:} Data integration from various sources enhances model accuracy and responsiveness.
    \end{itemize}
    
    Supply chain dynamics in linear dynamical systems offer a structured approach to managing modern supply chains, optimizing efficiency, reducing costs, and adapting to market changes.
\end{notes}