\clearpage
\chapter{Study Guide 8}

\section{Linear Equations}
\horizontalline{0}{0}

\begin{center}
    \large{\textbf{Study Guide Instructions}}
\end{center}

\horizontalline{-1}{0}

\begin{itemize}
    \item Submit your work in Gradescope as a PDF - you will identify where your “questions are.”
    \item Identify the question number as you submit.  Since we grade "blind" if the questions are NOT identified, the work WILL NOT BE GRADED and a 0 will be recorded. Always leave enough time to 
    identify the questions when submitting.
    \item One section per page (if a page or less) - We prefer to grade the main solution in a single page, extra work can be included on the following page.
    \item Long instructions may be removed to fit on a single page.
    \item \textbf{Do not start a new question in the middle of a page.}
    \item Solutions to book questions are provided for reference.
    \item You may NOT submit given solutions - this includes minor modifications - as your own.
    \item Solutions that do not show individual engagement with the solutions will be marked as no credit and can be considered a violation of honor code.
    \item If you use the given solutions you must reference or explain how you used them, in particular...
\end{itemize}

\horizontalline{-1}{0}

\begin{center}
    \large{\textbf{Method Selection}}
\end{center}

\horizontalline{-1}{0}

\textbf{For full credit,  EACH book exercise in the Study Guides must use one or more of the following methods and FOR EACH QUESTION.  Identify the number the method by number to ensure full credit.}

\begin{itemize}
    \item \textbf{Method 1} - Provide original examples which demonstrate the ideas of the exercise in addition to your solution.
    \item \textbf{Method 2} - Include and discuss the specific topics needed from the chapter and how they relate to the question.
    \item \textbf{Method 3} - Include original Python code, of reasonable length (as screenshot or text)  to show how the topic or concept was explored.
    \item \textbf{Method 4} - Expand the given solution in a significant way, with additional steps and comments. All steps are justified. This is a good method for a proof for which you are only given a basic outline.
    \item \textbf{Method 5} - Attempt the exercise without looking at the solution and then the solution is used to check work. Words are used to describe the results.
    \item \textbf{Method 6} - Provide an analysis of the strategies used to understand the exercise, describing in detail what was challenging, who helped you or what resources were used. The process of understanding is
    described.
\end{itemize}

% Problem 1
\begin{problem}{Problem 1}
    \begin{statement}{Problem Statement}
        Solve and explain the solution to 8.1  here in your own words. (Since you are given a solution, you will be graded on your ability to explain). \vspace*{1em}

        \noindent \textbf{Original Question} \vspace*{1em}

        \textit{Sum of linear functions}. Suppose $f : \mathbf{R}^{n} \rightarrow \mathbf{R}^{m}$ and $g : \mathbf{R}^{n} \rightarrow \mathbf{R}^{m}$ are linear functions. Their sum is the function 
        $h : \mathbf{R}^{n} \rightarrow \mathbf{R}^{m}$ , defined as $h(x) = f (x) + g(x)$ for any $n$-vector $x$. The sum function is often denoted as $h = f + g$. (This is another case of overloading
        the + symbol, in this case to the sum of functions.) If $f$ has matrix representation $f(x) = Fx$, and $g$ has matrix representation $f(x) = Gx$, where $F$ and $G$ are $m \times n$ matrices, 
        what is the matrix representation of the sum function $h = f + g$? Be sure to identify any + symbols appearing in your justification.
    \end{statement}

    \begin{highlight}[Solution]
        For this problem I will be using \textbf{Method 4}. \vspace*{1em}

        \noindent \textbf{Explanation} \vspace*{1em}

        We know from the problem statement that the matrix representation of $f$ is

        \begin{equation}
            f(x) = Fx
        \end{equation}
        and the matrix representation of $g$ is

        \begin{equation}
            g(x) = Gx.
        \end{equation}
        This in turn means that the sum of these matrices will be of the form

        \begin{equation}
            h(x) = f(x) + g(x).
        \end{equation}
        Filling in the gaps of (3) we can then say that the matrix representation of their sum is going to be

        \begin{equation}
            h(x) = Fx + Gx = (F + G)x.
        \end{equation}
        In the context of (4) we know that the matrices $F$ and $G$ are $m \times n$ matrices, this in turn means that the final sum of them $(F + G)$ is also going to be an $m \times n$ matrix. Now,
        we know that the vector $x$ is an $n$-vector. Because the terms $Fx$ and $Gx$ are an $m \times n$ matrix that is being multiplied by an $n$-vector, this means that when we add $f(x)$ and $g(x)$
        together we are adding $m$-vectors. Conversely, the sum of $F + G$ is the addition of $m \times n$ matrices.
    \end{highlight}
\end{problem}

% Problem 1 Summary
\begin{summary}{Problem 1 Summary}
    \begin{statement}{Procedure}
        \begin{itemize}
            \item Perform simple matrix addition of the two matrices in the problem.
        \end{itemize}
    \end{statement}
    \begin{statement}{Key Concepts}
        \begin{itemize}
            \item \textbf{Problem Statement:}
            \begin{itemize}
                \item The problem involves understanding the sum of two linear functions $f: \mathbb{R}^n \to \mathbb{R}^m$ and $g: \mathbb{R}^n \to \mathbb{R}^m$.
                \item The sum function $h$ is defined as $h(x) = f(x) + g(x)$ for any $n$-vector $x$.
                \item The task is to find the matrix representation of the sum function $h = f + g$.
            \end{itemize}
            \item \textbf{Matrix Representations of Linear Functions:}
            \begin{itemize}
                \item The linear function $f$ has a matrix representation $f(x) = Fx$.
                \item The linear function $g$ has a matrix representation $g(x) = Gx$.
                \item Both $F$ and $G$ are $m \times n$ matrices.
            \end{itemize}
            \item \textbf{Matrix Representation of the Sum Function:}
            \begin{itemize}
                \item The matrix representation of their sum $h$ is $h(x) = Fx + Gx$.
                \item By the properties of matrix addition and multiplication, this simplifies to $h(x) = (F + G)x$.
                \item The sum of the matrices $F + G$ is also an $m \times n$ matrix.
            \end{itemize}
            \item \textbf{Explanation:}
            \begin{itemize}
                \item The solution demonstrates that the sum of two linear functions is itself a linear function.
                \item The matrix representation of this sum function is obtained by simply adding the matrices corresponding to the individual functions.
                \item The explanation focuses on the properties of matrix addition and how it applies in the context of linear functions.
            \end{itemize}
            \item \textbf{Linear Algebra Concepts:}
            \begin{itemize}
                \item This problem showcases the application of basic linear algebra concepts, such as matrix addition and multiplication, in the context of linear transformations and functions.
                \item It also illustrates the concept of overloading the $+$ symbol to denote the sum of functions, emphasizing the relationship between function operations and matrix operations.
            \end{itemize}
        \end{itemize}
    \end{statement}
    \begin{statement}{Variations}
        \begin{itemize}
            \item We could be asked to demonstrate a different matrix operation.
            \begin{itemize}
                \item In this case we would define matrices in a similar context of this problem and then perform the matrix operation.
            \end{itemize}
        \end{itemize}
    \end{statement}
\end{summary}

% Problem 2
\begin{problem}{Problem 2}
    \begin{statement}{Problem Statement}
        Solve and Explain the solution to 8.4 here in your own words. (Since you are given a solution, you will be graded on your ability to explain). Use multiple pages. \vspace*{1em}

        \noindent \textbf{Original Question} \vspace*{1em}

        \textit{Linear functions of images}. In this problem we consider several linear functions of a monochrome image with $N \times N$ pixels. To keep the matrices small enough to work out by hand, we will 
        consider the case with $N = 3$ (which would hardly qualify as an image). We represent a $3 \times 3$ image as a 9-vector using the ordering of pixels shown below.

        \renewcommand{\arraystretch}{1.5}
        \begin{center}
            \begin{tabular}{|@{\hspace{10pt}}c@{\hspace{10pt}}|@{\hspace{10pt}}c@{\hspace{10pt}}|@{\hspace{10pt}}c@{\hspace{10pt}}|}
                \hline 1 & 4 & 7 \\ \hline
                2 & 5 & 8 \\ \hline
                3 & 6 & 9 \\ \hline
            \end{tabular}
        \end{center}
        (This ordering is called column-major.) Each of the operations or transformations below defines a function $y = f(x)$, where the 9-vector $x$ represents the original image, and the 9-vector 
        $y$ represents the resulting or transformed image. For each of these operations, give the $9 \times 9$ matrix $A$ for which $y = Ax$.

        \begin{enumerate}[label = (\alph*)]
            \item Turn the original image $x$ upside-down.
            \item Rotate the original image $x$ clockwise $90^{\circ}$.
            \item Translate the image up by 1 pixel and to the right by 1 pixel. In the translated image, assign the value $y_{i} = 0$ to the pixels in the first column and the last row.
            \item Set each pixel value $y_{i}$ to be the average of the neighbors of pixel $i$ in the original image. By neighbors, we mean the pixels immediately above and below, and immediately to 
            the left and right. The center pixel has 4 neighbors; corner pixels have 2 neighbors, and the remaining pixels have 3 neighbors.
        \end{enumerate}
    \end{statement}

    \begin{highlight}[Stratgey]
        \noindent \textbf{Strategy} \vspace*{1em}

        In these problems, the transformed image is going to be represented by a vector $y$. We calculate $y$ by

        \setcounter{equation}{0}
        \begin{equation}
            y = Ax
        \end{equation}
        where $A$ is a $9 \times 9$ matrix that operates on $x$ to generate $y$. We can view this transformation as something like

        \renewcommand{\arraystretch}{1.5}
        \begin{center}
            \begin{tabular}{|@{\hspace{10pt}}c@{\hspace{10pt}}|@{\hspace{10pt}}c@{\hspace{10pt}}|@{\hspace{10pt}}c@{\hspace{10pt}}|}
                \hline $x_{1}$ & $x_{4}$ & $x_{7}$ \\ \hline
                $x_{2}$ & $x_{5}$ & $x_{8}$ \\ \hline
                $x_{3}$ & $x_{6}$ & $x_{9}$ \\ \hline
            \end{tabular}
            \hspace{20pt} $\rightarrow$ \hspace{20pt}
            \begin{tabular}{|@{\hspace{10pt}}c@{\hspace{10pt}}|@{\hspace{10pt}}c@{\hspace{10pt}}|@{\hspace{10pt}}c@{\hspace{10pt}}|}
                \hline $y_{1}$ & $y_{4}$ & $y_{7}$ \\ \hline
                $y_{2}$ & $y_{5}$ & $y_{8}$ \\ \hline
                $y_{3}$ & $y_{6}$ & $y_{9}$ \\ \hline
            \end{tabular}
        \end{center}
        where the entries of $x$ and $y$ are stored in a $9 \times 9$ column vector. Precisely we can represent the original image $x$ and final image $y$ and matrix $A$ as

        \begin{equation}
            x = (1,2,3,4,5,6,7,8,9)^{T}
            \hspace{10pt}
            ,
            \hspace{10pt}
            y = (y_{1},y_{2},y_{3},y_{4},y_{5},y_{6},y_{7},y_{8},y_{9})^{T}
            \hspace{10pt}
            ,
            \hspace{10pt}
            A = 
            \begin{bmatrix}
                a_{11} & \dots & a_{19} \\
                \vdots & \ddots & \vdots \\
                a_{91} & \dots & a_{99} \\
            \end{bmatrix}.
        \end{equation}
        We now need to find $A$ for a given rotation of $x$ that is represented by $y$.
    \end{highlight}

    \begin{highlight}[Solution - Part (a)]
        For this problem I will be using \textbf{Method 4}. \vspace*{1em}

        \noindent \textbf{Explanation} \vspace*{1em}

        For this first transformation we want to flip the image upside down. To do this, we essentially have to swap the top and bottom rows of pixels in the image $x$. This will correspond to an image
        that looks something like
        
        \renewcommand{\arraystretch}{1.5}
        \begin{center}
            \begin{tabular}{|@{\hspace{10pt}}c@{\hspace{10pt}}|@{\hspace{10pt}}c@{\hspace{10pt}}|@{\hspace{10pt}}c@{\hspace{10pt}}|}
                \hline 3 & 6 & 9 \\ \hline
                2 & 5 & 8 \\ \hline
                1 & 4 & 7 \\ \hline
            \end{tabular}
        \end{center}
        where the center row of pixels is left intact. To achieve this, we need to strategically place 1s inside the matrix $A$ for a given column that will produce the vector $y$ such that

        \setcounter{equation}{0}
        \begin{equation}
            y = (3,2,1,6,5,4,9,8,7)^{T}.
        \end{equation}
        This in turn means that we can produce the flipped image $y$ with

        \begin{equation}
            y = Ax = 
            \begin{bmatrix}
                0 & 0 & 1 & 0 & 0 & 0 & 0 & 0 & 0 \\
                0 & 1 & 0 & 0 & 0 & 0 & 0 & 0 & 0 \\
                1 & 0 & 0 & 0 & 0 & 0 & 0 & 0 & 0 \\
                0 & 0 & 0 & 0 & 0 & 1 & 0 & 0 & 0 \\
                0 & 0 & 0 & 0 & 1 & 0 & 0 & 0 & 0 \\
                0 & 0 & 0 & 1 & 0 & 0 & 0 & 0 & 0 \\
                0 & 0 & 0 & 0 & 0 & 0 & 0 & 0 & 1 \\
                0 & 0 & 0 & 0 & 0 & 0 & 0 & 1 & 0 \\
                0 & 0 & 0 & 0 & 0 & 0 & 1 & 0 & 0 \\
            \end{bmatrix}
            \begin{bmatrix}
                1 \\
                2 \\
                3 \\
                4 \\
                5 \\
                6 \\
                7 \\
                8 \\
                9 \\
            \end{bmatrix}
            = 
            \begin{bmatrix}
                3 & 2 & 1 & 6 & 5 & 4 & 9 & 8 & 7 \\
            \end{bmatrix}^{T}.
        \end{equation}
        This then means that the matrix $A$ for this example is then

        \begin{equation}
            A =
            \begin{bmatrix}
                0 & 0 & 1 & 0 & 0 & 0 & 0 & 0 & 0 \\
                0 & 1 & 0 & 0 & 0 & 0 & 0 & 0 & 0 \\
                1 & 0 & 0 & 0 & 0 & 0 & 0 & 0 & 0 \\
                0 & 0 & 0 & 0 & 0 & 1 & 0 & 0 & 0 \\
                0 & 0 & 0 & 0 & 1 & 0 & 0 & 0 & 0 \\
                0 & 0 & 0 & 1 & 0 & 0 & 0 & 0 & 0 \\
                0 & 0 & 0 & 0 & 0 & 0 & 0 & 0 & 1 \\
                0 & 0 & 0 & 0 & 0 & 0 & 0 & 1 & 0 \\
                0 & 0 & 0 & 0 & 0 & 0 & 1 & 0 & 0 \\
            \end{bmatrix}.
        \end{equation}
        This in turn means that the function $f(x)$ that maps $x$ to this representation is 

        \begin{equation}
            f(x) = (x_{3},x_{2},x_{1},x_{6},x_{5},x_{4},x_{9},x_{8},x_{7}).
        \end{equation}
        The results from (3) and (4) illustrate how this transformation can be achieved.
    \end{highlight}

    \begin{highlight}[Solution - Part (b)]
        For this problem I will be using \textbf{Method 4}. \vspace*{1em}

        \noindent \textbf{Explanation} \vspace*{1em}

        This problem requests that we rotate the original image $x$ by $90^{\circ}$. This then means that our final image $y$ will look like

        \renewcommand{\arraystretch}{1.5}
        \begin{center}
            \begin{tabular}{|@{\hspace{10pt}}c@{\hspace{10pt}}|@{\hspace{10pt}}c@{\hspace{10pt}}|@{\hspace{10pt}}c@{\hspace{10pt}}|}
                \hline 3 & 2 & 1 \\ \hline
                6 & 5 & 4 \\ \hline
                9 & 8 & 7 \\ \hline
            \end{tabular} \hspace{2pt}.
        \end{center}
        Similar to part (a) we need to find a matrix $A$ that is constructed strategically of 1s and 0s where the final image can be produced. This means that the final image in vector form is going
        to be

        \setcounter{equation}{0}
        \begin{equation}
            y = (3,6,9,2,5,8,1,4,7)^{T}.
        \end{equation}
        Proceeding to determine $A$ we can achieve the vector in (1) by the calculation

        \begin{equation}
            y = Ax = 
            \begin{bmatrix}
                0 & 0 & 1 & 0 & 0 & 0 & 0 & 0 & 0 \\
                0 & 0 & 0 & 0 & 0 & 1 & 0 & 0 & 0 \\
                0 & 0 & 0 & 0 & 0 & 0 & 0 & 0 & 1 \\
                0 & 1 & 0 & 0 & 0 & 0 & 0 & 0 & 0 \\
                0 & 0 & 0 & 0 & 1 & 0 & 0 & 0 & 0 \\
                0 & 0 & 0 & 0 & 0 & 0 & 0 & 1 & 0 \\
                1 & 0 & 0 & 0 & 0 & 0 & 0 & 0 & 0 \\
                0 & 0 & 0 & 1 & 0 & 0 & 0 & 0 & 0 \\
                0 & 0 & 0 & 0 & 0 & 0 & 1 & 0 & 0 \\
            \end{bmatrix}
            \begin{bmatrix}
                1 \\
                2 \\
                3 \\
                4 \\
                5 \\
                6 \\
                7 \\
                8 \\
                9 \\
            \end{bmatrix}
            = 
            \begin{bmatrix}
                3 & 6 & 9 & 2 & 5 & 8 & 1 & 4 & 7 \\
            \end{bmatrix}^{T}.
        \end{equation}
        This in turn means that our matrix $A$ for this rotation will be
        
        \begin{equation}
            A = 
            \begin{bmatrix}
                0 & 0 & 1 & 0 & 0 & 0 & 0 & 0 & 0 \\
                0 & 0 & 0 & 0 & 0 & 1 & 0 & 0 & 0 \\
                0 & 0 & 0 & 0 & 0 & 0 & 0 & 0 & 1 \\
                0 & 1 & 0 & 0 & 0 & 0 & 0 & 0 & 0 \\
                0 & 0 & 0 & 0 & 1 & 0 & 0 & 0 & 0 \\
                0 & 0 & 0 & 0 & 0 & 0 & 0 & 1 & 0 \\
                1 & 0 & 0 & 0 & 0 & 0 & 0 & 0 & 0 \\
                0 & 0 & 0 & 1 & 0 & 0 & 0 & 0 & 0 \\
                0 & 0 & 0 & 0 & 0 & 0 & 1 & 0 & 0 \\
            \end{bmatrix}.
        \end{equation}
        The function $f(x)$ that will map $x$ to this final image is then
        
        \begin{equation}
            f(x) = (x_{3},x_{6},x_{9},x_{2},x_{5},x_{8},x_{1},x_{4},x_{7}).
        \end{equation}
        The results from (3) and (4) illustrate how this transformation can be achieved.
    \end{highlight}

    \begin{highlight}[Solution - Part (c)]
        For this problem I will be using \textbf{Method 4}. \vspace*{1em}

        \noindent \textbf{Explanation} \vspace*{1em}

        For this problem we are needing to shift the image essentially up and to the right. The final image will look like

        \renewcommand{\arraystretch}{1.5}
        \begin{center}
            \begin{tabular}{|@{\hspace{10pt}}c@{\hspace{10pt}}|@{\hspace{10pt}}c@{\hspace{10pt}}|@{\hspace{10pt}}c@{\hspace{10pt}}|}
                \hline 0 & 2 & 5 \\ \hline
                0 & 3 & 6 \\ \hline
                0 & 0 & 0 \\ \hline
            \end{tabular} \hspace{2pt}.
        \end{center}
        This means that the final image $y$ in vector form will look like

        \setcounter{equation}{0}
        \begin{equation}
            y = (0,0,0,2,3,0,5,6,0)^{T}.
        \end{equation}
        Applying the same general strategy from parts (a) and (b) we can represent the calculation of $y$ as

        \begin{equation}
            y = Ax = 
            \begin{bmatrix}
                0 & 0 & 0 & 0 & 0 & 0 & 0 & 0 & 0 \\
                0 & 0 & 0 & 0 & 0 & 0 & 0 & 0 & 0 \\
                0 & 0 & 0 & 0 & 0 & 0 & 0 & 0 & 0 \\
                0 & 1 & 0 & 0 & 0 & 0 & 0 & 0 & 0 \\
                0 & 0 & 1 & 0 & 0 & 0 & 0 & 0 & 0 \\
                0 & 0 & 0 & 0 & 0 & 0 & 0 & 0 & 0 \\
                0 & 0 & 0 & 0 & 1 & 0 & 0 & 0 & 0 \\
                0 & 0 & 0 & 0 & 0 & 1 & 0 & 0 & 0 \\
                0 & 0 & 0 & 0 & 0 & 0 & 0 & 0 & 0 \\
            \end{bmatrix}
            \begin{bmatrix}
                1 \\
                2 \\
                3 \\
                4 \\
                5 \\
                6 \\
                7 \\
                8 \\
                9 \\
            \end{bmatrix}
            = 
            \begin{bmatrix}
                0 & 0 & 0 & 2 & 3 & 0 & 5 & 6 & 0 \\
            \end{bmatrix}^{T}.
        \end{equation}
        This of course means that the matrix $A$ is going to be

        \begin{equation}
            A =
            \begin{bmatrix}
                0 & 0 & 0 & 0 & 0 & 0 & 0 & 0 & 0 \\
                0 & 0 & 0 & 0 & 0 & 0 & 0 & 0 & 0 \\
                0 & 0 & 0 & 0 & 0 & 0 & 0 & 0 & 0 \\
                0 & 1 & 0 & 0 & 0 & 0 & 0 & 0 & 0 \\
                0 & 0 & 1 & 0 & 0 & 0 & 0 & 0 & 0 \\
                0 & 0 & 0 & 0 & 0 & 0 & 0 & 0 & 0 \\
                0 & 0 & 0 & 0 & 1 & 0 & 0 & 0 & 0 \\
                0 & 0 & 0 & 0 & 0 & 1 & 0 & 0 & 0 \\
                0 & 0 & 0 & 0 & 0 & 0 & 0 & 0 & 0 \\
            \end{bmatrix}.
        \end{equation}
        The function $f(x)$ that will map $x$ to this final image $y$ is then

        \begin{equation}
            f(x) = (0,0,0,x_{2},x_{3},0,x_{5},x_{6},0)
        \end{equation}
        The results from (3) and (4) illustrate how this transformation can be achieved.
    \end{highlight}

    \begin{highlight}[Solution - Part (d)]
        For this problem I will be using \textbf{Method 4}. \vspace*{1em}

        \noindent \textbf{Explanation} \vspace*{1em}

        For this last transformation we want to calculate the average of the pixels that are neighbors for a given pixel. This final image will look like

        \renewcommand{\arraystretch}{1.5}
        \begin{center}
            \begin{tabular}{|@{\hspace{10pt}}c@{\hspace{10pt}}|@{\hspace{10pt}}c@{\hspace{10pt}}|@{\hspace{10pt}}c@{\hspace{10pt}}|}
                \hline $\frac{1}{2}(x_{2} + x_{4})$ & $\frac{1}{3}(x_{1} + x_{5} + x_{7})$ & $\frac{1}{2}(x_{4} + x_{8})$ \\ \hline
                $\frac{1}{3}(x_{1} + x_{3} + x_{5})$ & $\frac{1}{4}(x_{2} + x_{4} + x_{6} + x_{8})$ & $\frac{1}{3}(x_{5} + x_{7} + x_{9})$ \\ \hline
                $\frac{1}{2}(x_{2} + x_{6})$ & $\frac{1}{3}(x_{3} + x_{5} + x_{9})$ & $\frac{1}{2}(x_{6} + x_{8})$ \\ \hline
            \end{tabular} \hspace{2pt}.
        \end{center}
        This means that our final image $y$ in vector form will be 


        \setcounter{equation}{0}
        \tiny{
            \begin{equation}
                y = (\frac{1}{2}(x_{2} + x_{4}),\frac{1}{3}(x_{1} + x_{3} + x_{5}),\frac{1}{2}(x_{2} + x_{6}),\frac{1}{3}(x_{1} + x_{5} + x_{7}),\frac{1}{4}(x_{2} + x_{4} + x_{6} + x_{8}),\frac{1}{3}(x_{3} + x_{5} + x_{9}),\frac{1}{2}(x_{4} + x_{8}),\frac{1}{3}(x_{5} + x_{7} + x_{9}),\frac{1}{2}(x_{6} + x_{8})).
            \end{equation}
        }
        \normalsize
        This means that when we go to construct $A$, instead of 1s and 0s in our rows and columns we will include the coefficients in front of the terms for the trans formation. The calculation of 
        $y$ is then

        \begin{equation}
            \tiny{
                y = Ax =
                \begin{bmatrix}
                    0 & 1/2 & 0 & 1/2 & 0 & 0 & 0 & 0 & 0 \\
                    1/3 & 0 & 1/3 & 0 & 1/3 & 0 & 0 & 0 & 0 \\
                    0 & 1/2 & 0 & 0 & 0 & 1/2 & 0 & 0 & 0 \\
                    1/3 & 0 & 0 & 0 & 1/3 & 0 & 1/3 & 0 & 0 \\
                    0 & 1/4 & 0 & 1/4 & 0 & 1/4 & 0 & 1/4 & 0 \\
                    0 & 0 & 1/3 & 0 & 1/3 & 0 & 0 & 0 & 1/3 \\
                    0 & 0 & 0 & 1/2 & 0 & 0 & 0 & 1/2 & 0 \\
                    0 & 0 & 0 & 0 & 1/3 & 0 & 1/3 & 0 & 1/3 \\
                    0 & 0 & 0 & 0 & 0 & 1/2 & 0 & 1/2 & 0 \\
                \end{bmatrix}
                \begin{bmatrix}
                    1 \\
                    2 \\
                    3 \\
                    4 \\
                    5 \\
                    6 \\
                    7 \\
                    8 \\
                    9 \\
                \end{bmatrix}
                = 
                \begin{bmatrix}
                    3 & 2.97 & 4 & 4.29 & 5 & 5.61 & 6 & 6.93 & 7 \\
                \end{bmatrix}^{T}.
            }
            \normalsize
        \end{equation}
        This in turn means that the matrix $A$ for this rotation is

        \begin{equation}
            A = 
            \begin{bmatrix}
                0 & 1/2 & 0 & 1/2 & 0 & 0 & 0 & 0 & 0 \\
                1/3 & 0 & 1/3 & 0 & 1/3 & 0 & 0 & 0 & 0 \\
                0 & 1/2 & 0 & 0 & 0 & 1/2 & 0 & 0 & 0 \\
                1/3 & 0 & 0 & 0 & 1/3 & 0 & 1/3 & 0 & 0 \\
                0 & 1/4 & 0 & 1/4 & 0 & 1/4 & 0 & 1/4 & 0 \\
                0 & 0 & 1/3 & 0 & 1/3 & 0 & 0 & 0 & 1/3 \\
                0 & 0 & 0 & 1/2 & 0 & 0 & 0 & 1/2 & 0 \\
                0 & 0 & 0 & 0 & 1/3 & 0 & 1/3 & 0 & 1/3 \\
                0 & 0 & 0 & 0 & 0 & 1/2 & 0 & 1/2 & 0 \\
            \end{bmatrix}
        \end{equation}
        The function $f(x)$ that will map $x$ for this rotation is then

        \tiny{
            \begin{equation}
                f(x) = (\frac{1}{2}(x_{2} + x_{4}),\frac{1}{3}(x_{1} + x_{3} + x_{5}),\frac{1}{2}(x_{2} + x_{6}),\frac{1}{3}(x_{1} + x_{5} + x_{7}),\frac{1}{4}(x_{2} + x_{4} + x_{6} + x_{8}),\frac{1}{3}(x_{3} + x_{5} + x_{9}),\frac{1}{2}(x_{4} + x_{8}),\frac{1}{3}(x_{5} + x_{7} + x_{9}),\frac{1}{2}(x_{6} + x_{8})).
            \end{equation}
        }
        \normalsize
        The results from (3) and (4) illustrate how this transformation can be achieved.
    \end{highlight}
\end{problem}

% Problem 2 Summary
\begin{summary}{Problem 2 Summary}
    \begin{statement}{Procedure}
        \begin{itemize}
            \item For part (a), first switch the top row with the bottom row and vice versa.
            \item For part (b), rotate the image $90^{\circ}$ clockwise.
            \item For part (c), move the image up and to the right and add one to the non zero values.
            \item For part (d), set each pixel value as the average of their neighbors.
            \item For each part of the problem:
            \begin{itemize}
                \item Generate a vector where the elements are the elements found in the image from first row, through the columns, and then repeat the process for the following column.
                \item Construct a matrix that is filled with 1s and 0s where the matrix and vector multiplication will produce the resultant vector.
            \end{itemize}
        \end{itemize}
    \end{statement}
    \begin{statement}{Key Concepts}
        \begin{itemize}
            \item \textbf{Problem Context:}
            \begin{itemize}
                \item The problem is set in the context of applying linear algebra to image processing.
                \item It involves analyzing linear transformations applied to a monochrome image with $N \times N$ pixels.
                \item The specific case considered is $N = 3$, with the image represented as a 9-vector using a column-major ordering.
            \end{itemize}
            \item \textbf{Image Transformations:}
            \begin{itemize}
                \item Several operations or transformations are defined as functions $y = f(x)$, where $x$ represents the original image and $y$ the transformed image.
                \item Each operation requires finding a $9 \times 9$ transformation matrix $A$ such that $y = Ax$.
                \item The transformations include:
                \begin{enumerate}
                    \item Turning the image upside-down.
                    \item Rotating the image clockwise by 90 degrees.
                    \item Translating the image up by 1 pixel and to the right by 1 pixel, with specific pixel assignments.
                    \item Setting each pixel value to the average of its neighbors.
                \end{enumerate}
            \end{itemize}
            \item \textbf{Matrix Representation:}
            \begin{itemize}
                \item The matrix $A$ for each transformation is constructed based on the desired operation.
                \item The matrices are designed to manipulate the 9-vector representing the image to achieve the specified transformations.
            \end{itemize}
            \item \textbf{Calculation Methodology:}
            \begin{itemize}
                \item The process of constructing each transformation matrix involves strategically placing 1s (and other coefficients) in the matrix to map the original image vector $x$ to the 
                transformed image vector $y$.
                \item The problem demonstrates how different image transformations can be achieved by applying specific linear transformations.
            \end{itemize}
            \item \textbf{Linear Algebra Application:}
            \begin{itemize}
                \item This problem showcases the application of matrix operations in the field of image processing.
                \item It highlights the use of linear transformations to achieve various effects on images, emphasizing the role of matrix-vector multiplication in these processes.
            \end{itemize}
        \end{itemize}
    \end{statement}
    \begin{statement}{Variations}
        \begin{itemize}
            \item We could be given a different original image.
            \begin{itemize}
                \item We would then perform the same operations as found in this problem.
            \end{itemize}
            \item We could be given different operations to perform on the image.
            \begin{itemize}
                \item We would then go through the same general process depicted in this problem but with the new operation.
            \end{itemize}
        \end{itemize}
    \end{statement}
\end{summary}

% Problem 3
\begin{problem}{Problem 3}
    \begin{statement}{Problem Statement}
        Solve and Explain the solution to 8.7 here in your own words. (Since you are given a solution, you will be graded on your ability to explain). \vspace*{1em}

        \noindent \textbf{Original Question} \vspace*{1em}

        \textit{Interpolation of polynomial values and derivatives}. The 5-vector $c$ represents the coefficients of a quartic polynomial $p(x) = c_{1} + c_{2}x + c_{3}x^{2} + c_{4}x^{3} + c_{5}x^{4}$. Express the conditions

        \begin{equation*}
            p(0) = 0, \hspace{5pt} p'(0) = 0, \hspace{5pt} p(1) = 1, \hspace{5pt} p'(1) = 0,
        \end{equation*}
        as a set of linear equations of the form $Ac = b$. Is the system of equations under-determined, over-determined, or square?
    \end{statement}

    \begin{highlight}[Solution]
        For this problem I will be using \textbf{Method 4}. \vspace*{1em}

        \noindent \textbf{Explanation} \vspace*{1em}

        We first need to calculate the derivative of polynomial that was given to us. The derivative of $p(x)$ is of course

        \setcounter{equation}{0}
        \begin{equation}
            p'(x) = c_{2} + 2c_{3}x + 3c_{4}x^{2} + 4c_{5}x^{3}.
        \end{equation}
        Taking the result from (1) we can then say the conditions then evaluate to

        \begin{align}
            p(0) & = c_{1} + c_{2}(0) + c_{3}(0)^{2} + c_{4}(0)^{3} + c_{5}(0)^{4} = c_{1} + 0 + \dots + 0 = c_{1} = 0 \\
            p'(0) & = c_{2} + 2c_{3}(0) + 3c_{4}(0)^{2} + 4c_{5}(0)^{3} = c_{2} + 0 + \dots + 0 = c_{2} = 0 \\
            p(1) & = c_{1} + c_{2}(1) + c_{3}(1)^{2} + c_{4}(1)^{3} + c_{5}(1)^{4} = c_{1} + c_{2} + c_{3} + c_{4} + c_{5} = 1 \\
            p'(1) & = c_{2} + 2c_{3}(1) + 3c_{4}(1)^{2} + 4c_{5}(1)^{3} = c_{2} + 2c_{3} + 3c_{4} + 4c_{5} = 0.
        \end{align}
        We then need to find a way to formulate the expression $Ac = b$. $A$ is going to be a matrix that is consisted of the coefficients found in our expressions. $c$ is a column vector that has the
        terms of $c_{i}$ in their elements. $b$ is a column vector that represents what our systems of equation evaluate to for a specific condition. Piecing this all together we then have

        \begin{equation}
            A = 
            \begin{bmatrix}
                1 & 0 & 0 & 0 & 0 \\
                0 & 1 & 0 & 0 & 0 \\
                1 & 1 & 1 & 1 & 1 \\
                0 & 1 & 2 & 3 & 4 \\
            \end{bmatrix}
            \hspace{10pt} , \hspace{10pt}
            c = 
            \begin{bmatrix}
                c_{1} \\
                c_{2} \\
                c_{3} \\
                c_{4} \\
                c_{5} \\
            \end{bmatrix}
            \hspace{10pt} , \hspace{10pt}
            b = 
            \begin{bmatrix}
                0 \\
                0 \\
                1 \\
                0 \\
            \end{bmatrix}
            \hspace{5pt} , \hspace{5pt}
            Ac = 
            \begin{bmatrix}
                1 & 0 & 0 & 0 & 0 \\
                0 & 1 & 0 & 0 & 0 \\
                1 & 1 & 1 & 1 & 1 \\
                0 & 1 & 2 & 3 & 4 \\
            \end{bmatrix}
            \begin{bmatrix}
                c_{1} \\
                c_{2} \\
                c_{3} \\
                c_{4} \\
                c_{5} \\
            \end{bmatrix}
            = 
            \begin{bmatrix}
                0 \\
                0 \\
                1 \\
                0 \\
            \end{bmatrix}
            = b.
        \end{equation}
        Because $A$ is $4 \times 5$ and $c$ is of length $5$, when we execute the multiplication of $Ac$ we will get a column vector of length 4 which matches the length of $b$.

        From our results of (2)-(6) we can see that our system of equations consists of 4 equations with 5 variables. This is mathematically impossible to solve because we must have the same number or 
        more equations as the number of variables in our system of equations. This in turn means that our system of equations is \textbf{under-determined}. Indicating that we need at least one more 
        condition for it to be \textbf{square}. As mentioned previously, this system is impossible to solve.
    \end{highlight}
\end{problem}

% Problem 3 Summary
\begin{summary}{Problem 3 Summary}
    \begin{statement}{Procedure}
        \begin{itemize}
            \item Calculate the derivative of the polynomial.
            \item Calculate the value of the polynomial and value of the derivative of the polynomial.
            \item Create a system of equations.
            \item Create a matrix that will produce the system of equations.
            \item Show the system of equations that are produced with the matrix / vector multiplication.
        \end{itemize}
    \end{statement}
    \begin{statement}{Key Concepts}
        \begin{itemize}
            \item \textbf{Problem Context:}
            \begin{itemize}
                \item The problem involves interpolating polynomial values and derivatives.
                \item A 5-vector $c$ represents the coefficients of a quartic polynomial $p(x) = c_1 + c_2x + c_3x^2 + c_4x^3 + c_5x^4$.
                \item The task is to express certain conditions as a set of linear equations of the form $Ac = b$ and determine whether the system is under-determined, over-determined, or square.
            \end{itemize}
            \item \textbf{Conditions and Equations:}
            \begin{itemize}
                \item The conditions given are $p(0) = 0$, $p'(0) = 0$, $p(1) = 1$, and $p'(1) = 0$.
                \item These conditions are translated into linear equations involving the coefficients $c_i$.
            \end{itemize}
            \item \textbf{Derivative of Polynomial}
            \begin{itemize}
                \item The derivative of $p(x)$ is calculated as $p'(x) = c_2 + 2c_3x + 3c_4x^2 + 4c_5x^3$.
                \item This derivative is used to formulate conditions at $x = 0$ and $x = 1$.
            \end{itemize}
            \item \textbf{Matrix Formulation:}
            \begin{itemize}
                \item The matrix $A$ is formed from the coefficients in the expressions derived from the conditions.
                \item The vector $c$ is the column vector of the polynomial's coefficients.
                \item The vector $b$ represents the results of the system of equations for each condition.
                \item The final matrix equation is $Ac = b$.
            \end{itemize}
            \item \textbf{System Analysis:}
            \begin{itemize}
                \item The system of equations consists of 4 equations (from conditions) with 5 variables (coefficients $c_i$).
                \item It is concluded that the system is under-determined, indicating that it is impossible to solve without additional conditions.
                \item The under-determined nature suggests that the matrix $A$ is not square and has more columns than rows.
            \end{itemize}
            \item \textbf{Linear Algebra Concepts:}
            \begin{itemize}
                \item This problem illustrates the use of polynomial derivatives and evaluations in constructing linear systems.
                \item It highlights the importance of matrix dimensions and the relationship between the number of equations and variables in determining the solvability of a system.
            \end{itemize}
        \end{itemize}
    \end{statement}
    \begin{statement}{Variations}
        \begin{itemize}
            \item We could be given a different polynomial.
            \begin{itemize}
                \item We would then go through the same procedure but generate matrices and vectors that produce the system of equations.
            \end{itemize}
        \end{itemize}
    \end{statement}
\end{summary}

% Problem 4
\begin{problem}{Problem 4}
    \begin{statement}{Problem Statement}
        Solve and Explain the solution to 8.9 here in your own words. (Since you are given a solution, you will be graded on your ability to explain). \vspace*{1em}

        \noindent \textbf{Original Question} \vspace*{1em}

        Required nutrients. We consider a set of $n$ basic foods (such as rice, beans, apples) and a set of $m$ nutrients or components (such as protein, fat, sugar, vitamin C). Food $j$ has a cost 
        given by $c_{j}$ (say, in dollars per gram), and contains an amount $N_{ij}$ of nutrient $i$ (per gram). (The nutrients are given in some appropriate units, which can depend on the particular 
        nutrient.) A daily diet is represented by an $n$-vector $d$, with $d_{i}$ the daily intake (in grams) of food $i$. Express the condition that a diet $d$ contains the total nutrient amounts given 
        by the $m$-vector $n^{\text{des}}$ , and has a total cost $B$ (the budget) as a set of linear equations in the variables $d_{1}, \dots , d_{n}$. (The entries of $d$ must be nonnegative, but we 
        ignore this issue here.)
    \end{statement}

    \begin{highlight}[Solution]
        For this problem I will be using \textbf{Method 4}. \vspace*{1em}

        \noindent \textbf{Explanation} \vspace*{1em}

        The simplest way to address this problem is to first look at the dimensional analysis of what we are looking to find. We are looking for the total nutrients for a given set of basic foods $n$ as
        well as the cost for a given food. In this context, we can represent the amount of nutrients for a given food in $\frac{\text{nutrients}}{\text{gram}}$ and the cost of a given food in 
        $\frac{\text{\$}}{\text{gram}}$. Our matrix in this example is going to be a stacked matrix.

        To represent $N$ we will designate $N_{ij}$ as the amount of nutrients per gram for a given food and $c_{j}$ as the cost in dollars per gram of said food. We then can represent this matrix as

        \setcounter{equation}{0}
        \begin{equation}
            \eta = 
            \begin{bmatrix}
                N_{11} & \dots & N_{1n} \\
                \vdots & \ddots & \vdots \\
                N_{m1} & \dots & N_{mn} \\
                c_{1} & \dots & c_{n} \\
            \end{bmatrix}
        \end{equation}
        where $\eta$ is of the dimension $(m + 1) \times n$. If we then wish to represent the amount of food in the column vector $d$ we would have

        \begin{equation}
            d =
            \begin{bmatrix}
                d_{1} \\
                \vdots \\
                \vdots \\
                d_{n}
            \end{bmatrix}
        \end{equation}
        where $d$ is of length $n$. The multiplication of $\eta$ and $d$ will result in the final units of nutrients for the first $m$ rows and $n$ columns and \$ in the last row and $n$ columns.\
        Multiplying this matrix $\eta$ by vector $d$ we will then have

        \begin{equation}
            \eta d = 
            \begin{bmatrix}
                N_{11} & \dots & N_{1n} \\
                \vdots & \ddots & \vdots \\
                N_{m1} & \dots & N_{mn} \\
                c_{1} & \dots & c_{n} \\
            \end{bmatrix}
            \begin{bmatrix}
                d_{1} \\
                \vdots \\
                \vdots \\
                d_{n}
            \end{bmatrix}
            = 
            \begin{bmatrix}
                n_{1}^{\text{des}} \\
                \vdots \\
                n_{m}^{\text{des}} \\
                B \\
            \end{bmatrix}
        \end{equation}
        where $B$ in (3) represents the budget for the food used in this diet. $n_{i}^{\text{des}}$ represents the nutrients in the diet for a given food $j$. The result from (3) is in turn our system
        equations.
    \end{highlight}
\end{problem}

% Problem 4 Summary
\begin{summary}{Problem 4 Summary}
    \begin{statement}{Procedure}
        \begin{itemize}
            \item Generate a matrix that will produce the desired result, with the correct entries in the matrix.
        \end{itemize}
    \end{statement}
    \begin{statement}{Key Concepts}
        \begin{itemize}
            \item \textbf{Problem Context and Statement:}
            \begin{itemize}
                \item The problem involves nutritional diet planning using linear algebra concepts. It considers a set of $n$ basic foods and a set of $m$ nutrients.
                \item Each food has an associated cost per gram, $c_j$, and contains an amount $N_{ij}$ of nutrient $i$ per gram.
                \item A daily diet is represented by an $n$-vector $d$, with $d_i$ indicating the daily intake of food $i$.
                \item The task is to express the conditions for a diet $d$ to contain the total nutrient amounts given by the $m$-vector $\mathbf{ndes}$ and have a total cost $B$ as a 
                set of linear equations.
            \end{itemize}
            \item \textbf{Matrix and Vector Formulation:}
            \begin{itemize}
                \item The matrix $S$ represents the nutrient and cost information. It is a stacked matrix combining nutrient content and food cost.
                \item The dimensions and entries of $S$ are such that the first $m$ rows represent nutrient contents $N_{ij}$ and the last row represents the cost per gram $c_j$.
                \item The vector $d$ represents the quantity of each food item in the diet.
            \end{itemize}
            \item \textbf{Linear System for Diet Planning:}
            \begin{itemize}
                \item The condition for the diet is formulated as a linear system $Sd = \mathbf{b}$.
                \item The vector $\mathbf{b}$ includes the desired nutrient amounts $\mathbf{ndes}$ and the budget $B$.
                \item This system of equations captures both the nutritional and cost requirements of the diet.
            \end{itemize}
            \item \textbf{Solution Approach:}
            \begin{itemize}
                \item The problem involves solving for the diet vector $d$ that satisfies the nutritional and cost requirements.
                \item The solution is dependent on the properties of the matrix $S$ and the feasibility of the system $Sd = \mathbf{b}$.
            \end{itemize}
            \item \textbf{Linear Algebra Application:}
            \begin{itemize}
                \item This problem illustrates the application of linear algebra in a practical scenario of diet planning.
                \item It demonstrates how matrix-vector multiplication can be used to model and solve real-world problems involving constraints and requirements.
            \end{itemize}
        \end{itemize}
    \end{statement}
    \begin{statement}{Variations}
        \begin{itemize}
            \item We could be asked to model a different system for this.
            \begin{itemize}
                \item We would then use the same procedure that was depicted in this problem for the new system.
            \end{itemize}
        \end{itemize}
    \end{statement}
\end{summary}