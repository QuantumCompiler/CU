\clearpage

\section*{Priority Queue}

\subsection*{Overview}

A priority queue is an abstract data type that allows for efficient insertion and retrieval of elements based on their priority. In a priority queue, each element is associated with a priority value, 
and the elements are arranged in a way that ensures the element with the highest (or lowest, depending on the implementation) priority is always at the front of the queue. Priority queues are useful 
in scenarios where elements need to be processed based on their importance, urgency, or other priority criteria. Common operations supported by a priority queue include inserting elements, extracting 
the element with the highest priority, and peeking at the highest-priority element without removing it. Priority queues can be implemented using various data structures, such as binary heaps, binomial 
heaps, Fibonacci heaps, or self-balancing binary search trees, each offering different trade-offs in terms of time complexity for different operations. Priority queues find applications in various 
domains, including task scheduling, network routing, Huffman coding, and Dijkstra's algorithm for finding the shortest path in a graph.

\subsection*{Heaps}

A heap is a specialized binary tree-based data structure that satisfies the heap property, which determines the order of elements within the tree. In a binary heap, each node has a key, and the key of 
each node is either greater than or equal to (in a max heap) or less than or equal to (in a min heap) the keys of its children. This property ensures that the root node always contains the highest 
(max heap) or lowest (min heap) key in the heap, making it easy to access and remove the element with the highest or lowest priority in constant time. Binary heaps are usually implemented as arrays, 
where the element at index $i$ has its left child at index $2i+1$ and its right child at index $2i+2$. Binary heaps are commonly used to implement priority queues, which efficiently manage tasks or 
elements based on their priority, and they find applications in various algorithms like Dijkstra's algorithm, heap sort, and Huffman coding. The balanced nature and logarithmic time complexity of heap 
operations make them a powerful tool for efficient data management and processing.

\subsection*{Structure}

Priority queues can be constructed in a number of different ways. Typically, a priority queue requires some sort of container to hold the data types of the queue. This container can be a vector, array,
list, or something similar. Beyond this container, priority queues typically require at least three types of operations. These operations usually include, inserting, removing, and peeking (looking at
the element that is of highest or lowest priority). Priority queues also require other operations such as percolating up or down, depending on whether or not the heap is a min or max heap. In the context
of the assignment for this course, the structure of the priority queue consists a structure and a class. We first look at the structure \textbf{pq}:

\begin{itemize}
    \item \textbf{heapPriority} - This is a vector of float values that represent a nodes priority in the queue.
    \item \textbf{heapString} - This is a vector of string objects that are associated with a specific node that exists in the priority queue.
    \item \textbf{percolateUp} - This is a function that percolates elements up a priority queue to be inserted in the correct index of the queue. Since this priority queue incorporates a max
    heap, percolate up is needed instead of percolate down.
\end{itemize}

\noindent Along with the previous structure \textbf{pq}, there is a class that encapsulates the implementation of the priority queue. The class \textbf{PriorityQueue} consists of the following member 
functions:

\begin{itemize}
    \item \textbf{InitPriorityQueue} - This function initializes a \textbf{pq} structure and sets the values inside the structure to a default value.
    \item \textbf{Insert} - This function inserts a node into a priority queue at its correct location within the queue.
    \item \textbf{Peek} - This function will retrieve the node with the highest element in a priority queue and return it without modifying the queue itself.
    \item \textbf{Remove} - This function removes a node with the highest priority in a queue.
\end{itemize}

\subsection*{Efficiency}

The runtime complexity of a binary heap, a type of heap data structure, is efficient for essential operations. Insertion and deletion of elements in the heap have a worst-case time complexity of 
$\mathcal{O}(\log{(n)})$ and a best-case time complexity of $\mathcal{O}(1)$ when dealing with the root element. The average time complexity for these operations is also $\mathcal{O}(\log{(n)})$, 
considering the balanced nature of binary heaps. Peeking at the minimum or maximum element can be done in constant time ($\mathcal{O}(1)$) as it requires accessing the root directly. Building a binary 
heap from an unordered array has a time complexity of $\mathcal{O}(n)$ in both average and worst cases. The logarithmic and linear time complexities of heap operations make it an effective data structure 
for managing priorities and finding extremum values efficiently. The runtime complexities of the common operations in a heap can be seen below:

\begin{center}
    \begin{tabular}[ht]{|c|c|c|c|}
        \hline \textbf{Operation} & \textbf{Best Case} $\Omega(n)$ & \textbf{Average Case} $\Theta(n)$ & \textbf{Worst Case} $\mathcal{O}(n)$ \\ \hline
        \textbf{Deletion} & $\Omega(1)$ & $\Theta(\log{(n)})$ & $\mathcal{O}(\log{(n)})$ \\ \hline
        \textbf{Heapify-ing} & $\Omega(n)$ & $\Theta(n)$ & $\mathcal{O}(n)$ \\ \hline
        \textbf{Insertion} & $\Omega(1)$ & $\Theta(\log{(n)})$ & $\mathcal{O}(\log{(n)})$ \\ \hline
        \textbf{Peeking} & $\Omega(1)$ & $\Theta(1)$ & $\mathcal{O}(1)$ \\ \hline
    \end{tabular}
\end{center}

\noindent Heapifying is the process of converting an array (or a portion of it) into a binary heap, satisfying the heap property. In the context of binary heaps, heapify involves `bubbling down' or 
`sifting down' elements starting from the non-leaf nodes to their correct positions in the heap. This process is essential for maintaining the binary heap's balanced and ordered structure, ensuring that 
the root node holds the maximum (in a max heap) or minimum (in a min heap) value, and that the heap property holds true for all the elements. Heapify is typically used to build a binary heap from an 
unordered array or to restore the heap property after a deletion operation.

\textbf{PQ Structure}

As mentioned above, the \textbf{pq} structure is a structure that is used in tandem with the \textbf{PriorityQueue} class. The definition of this structure can be seen below.

\begin{highlight}[PQ Structure]

Below is the definition of the \textbf{pq} structure:

\horizontalline

\begin{verbatim}
struct pq {
    vector<string> heapString;
    vector<float> heapPriority;

    void percolateUp(int index) {
        while (index > 0) {
            int parentIndex = (index - 1) / 2;
            if (this->heapPriority.at(index) <= this->heapPriority.at(parentIndex)) {
                return;
            }
            else {
                swap(this->heapString.at(index), this->heapString.at(parentIndex));
                swap(this->heapPriority.at(index), this->heapPriority.at(parentIndex));
                index = parentIndex;
            }
        }
    }
};
\end{verbatim}
    
\end{highlight}

\subsection*{InitPriorityQueue}

The function `InitPriorityQueue' is a member function of the `PriorityQueue' class that returns a shared pointer to a dynamically allocated `pq' object, representing a priority queue. Within the function, 
a new `pq' object is created and its data members, `heapString' and `heapPriority', are initialized to be empty by calling their `clear()' methods. The `pq' object is then wrapped in a shared pointer before 
being returned from the function. This function is responsible for initializing an empty priority queue, allowing it to be used for storing elements with associated priorities later in the program. The definition
of this function can be seen below.

\begin{highlight}[InitPriorityQueue Algorithm]

Below is the definition of the `InitPriorityQueue' function:

\horizontalline

\begin{verbatim}
shared_ptr<pq> PriorityQueue::InitPriorityQueue(){
    shared_ptr<pq> queue(new pq);
    queue->heapString.clear();
    queue->heapPriority.clear();
    return queue;
}
\end{verbatim}

\end{highlight}

\subsection*{Insert}

The function `Insert' is a member function of the `PriorityQueue' class that allows inserting a new element with its associated priority into an existing priority queue represented by the shared pointer `queue'. 
The function takes the `text' (a string representing the element) and `priority' (a float value representing its priority) as inputs. The element and its priority are added to the back of their respective vectors 
`heapString' and `heapPriority'. After insertion, the function calculates the index of the last element in the `heapString' vector and performs the "percolate up" operation, which ensures that the newly inserted 
element moves to its correct position in the heap to maintain the heap property. The `percolateUp' function is likely implemented to handle the upward movement of the element in the heap. This function enables 
efficient insertion and maintenance of the priority queue's ordered structure, ensuring that higher-priority elements are positioned closer to the root of the heap. The definition of this function can be seen
below.

\begin{highlight}[Insert Algorithm]

Below is the definition of the `Insert' function:

\horizontalline 

\begin{verbatim}
void PriorityQueue::Insert(shared_ptr<pq> queue, string text, float priority){
    queue->heapString.push_back(text);
    queue->heapPriority.push_back(priority);
    int currentIndex = queue->heapString.size() - 1;
    queue->percolateUp(currentIndex);
}
\end{verbatim}

\end{highlight}

\subsection*{Peek}

The function `Peek' is a member function of the `PriorityQueue' class that allows accessing the element with the highest priority in the priority queue represented by the shared pointer `queue' without removing it. 
The function begins by creating a string variable `headVal', which will hold the value of the element at the top of the priority queue. It then checks if the queue is empty, in which case it sets `headVal' to an empty 
string. If the queue is not empty, the function retrieves the element at the top of the queue (root of the binary heap) by accessing the first element in the `heapString' vector, which represents the elements in the 
binary heap. The value of the top element is then assigned to `headVal'. The function finally returns `headVal', which represents the element with the highest priority in the priority queue without removing it. This 
allows users to peek at the highest-priority element without altering the queue's contents. The definition of this function can be seen below.

\begin{highlight}[Peek Algorithm]

Below is the definition of the `Peek' function:

\horizontalline

\begin{verbatim}
string PriorityQueue::Peek(shared_ptr<pq> queue){
    string headVal;
    if (queue->heapString.size() == 0) {
        headVal = "";
    }
    else {
        headVal = queue->heapString.at(0);
    }
    return headVal;
}
\end{verbatim}

\end{highlight}

\subsection*{Remove}

The function `Remove' is a member function of the `PriorityQueue' class that removes and returns the element with the highest priority from the priority queue represented by the shared pointer `queue'. The function 
begins by creating a default string `returnVal', which is used to store the element that will be removed and returned. It first checks if the priority queue is empty, in which case it sets `returnVal' to an empty 
string. If the queue is not empty, it searches for the element with the highest priority by iterating through the `heapPriority' vector. Once the element with the highest priority is found, its corresponding string 
value is stored in `returnVal', and both the priority and text values are removed from their respective vectors using the `erase' method. The function then returns `returnVal', which represents the element that was 
removed from the priority queue with the highest priority. The definition of this function can be seen below.

\begin{highlight}[Remove Algorithm]

Below is the definition of the `Remove' function:

\horizontalline

\begin{verbatim}
string PriorityQueue::Remove(shared_ptr<pq> queue){
    string returnVal;
    if (queue->heapString.size() == 0) {
        returnVal = "";
    }
    else {
        float maxVal = 0;
        int index = 0;
        for (int i = 0; i < queue->heapString.size(); i++) {
            if (maxVal <= queue->heapPriority.at(i)) {
                maxVal = queue->heapPriority.at(i);
                index = i;
            }
        }
        returnVal = queue->heapString.at(index);
        queue->heapString.erase(queue->heapString.begin() + index);
        queue->heapPriority.erase(queue->heapPriority.begin() + index);
    }
    return returnVal;
}
\end{verbatim}

\end{highlight}